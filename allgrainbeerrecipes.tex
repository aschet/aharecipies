\documentclass[parskip=half,fontsize=9pt,oneside,toc=chapterentrydotfill]{scrbook}

\usepackage{scrlayer-scrpage}
\usepackage[utf8]{inputenc}
\usepackage[american]{babel}
\usepackage[pass]{geometry}
\usepackage[regular,condensed,sfdefault]{roboto}
\usepackage[T1]{fontenc}
\usepackage{fp}
\usepackage{xcolor}
\usepackage{mdframed}
\usepackage{colortbl}
\usepackage{graphicx}
\usepackage{tabu}
\usepackage{booktabs}
\usepackage[version=4]{mhchem}
\usepackage{xifthen}
\usepackage{imakeidx}
\usepackage[hidelinks,pdfencoding=auto,pdftex,
  pdfauthor={Thomas Ascher},
  pdfusetitle,
  pdfkeywords={beer, brewing, recipes}]{hyperref}
\usepackage{multicol}
\usepackage{microtype}

\KOMAoptions{toc=flat,toc=listof}

\addtokomafont{part}{\Huge}
\addtokomafont{chapter}{\Huge}
\addtokomafont{section}{\Huge}

\definecolor{softgray}{HTML}{F1F1F1}
\definecolor{hardgray}{HTML}{6A6A6A}
\chead{}
\setlength{\parindent}{0mm} 
\RedeclareSectionCommand[tocnumwidth=0.85cm]{part}

\newmdenv[linewidth=3pt,
linecolor=black,
backgroundcolor=softgray,
fontcolor=hardgray,
rightline=false,
leftline=false,
bottomline=false,
skipabove=1mm,
skipbelow=1.5mm]{recipeframe}

\newcommand{\stylecategory}[1]{\part{#1}}
\newcommand{\stylesection}[1]{\chapter{#1} \clearpage}

\newenvironment{recipe}[2][]{\begin{multicols}{2}[\section*{#2}]\ifthenelse{\equal{#1}{}}{\index{#2}}{\index{#1@#2}}}{\end{multicols} \newpage}

\newcommand{\recipebreak}{\vfill\null \columnbreak}

\newenvironment{recipeblock}[1]
{\uppercase{\textbf{#1}} \begin{recipeframe}}{\end{recipeframe}}

\newenvironment{aboutblock}
{\begin{recipeblock}{About This Recipe}}{\end{recipeblock}}

\newenvironment{ingredientsblock}
{\begin{recipeblock}{Ingredients}}{\end{recipeblock}}

\newcommand{\recipesection}[2]{
\begin{minipage}[t][1.2cm]{\textwidth}
	\begin{minipage}[c][1.2cm][c]{0.95cm}
		\includegraphics[width=0.8cm]{#1}
	\end{minipage}
	\begin{minipage}[c][1.2cm][c]{5cm}
		\large{\textcolor{black}{\uppercase{\textbf{#2}}}}
	\end{minipage}
	\hfill
\end{minipage}
}

\newcommand{\ibutoibu}[1]{\FPeval\amountg{round((#1):0)}\FPprint\amountg ~IBU}
\newcommand{\ftoc}[1]{\FPeval\amountg{round(((#1 - 32.0) / 1.8):0)}\FPprint\amountg ~°C}
\newcommand{\gpgaltogpl}[1]{\FPeval\amountg{round((#1 * 0.264172):1)}\FPprint\amountg ~g/l}
\newcommand{\ozpgaltogpl}[1]{\FPeval\amountg{round((#1 * 7.48915):1)}\FPprint\amountg ~g/l}
\newcommand{\tsptog}[1]{\FPeval\amountg{round((#1 * 4.92892):0)}\FPprint\amountg ~g}
\newcommand{\tbsptog}[1]{\FPeval\amountg{round((#1 * 15.0):0)}\FPprint\amountg ~g}
\newcommand{\oztog}[1]{\FPeval\amountg{round((#1 * 28.34952):0)}\FPprint\amountg ~g}
\newcommand{\oztokg}[1]{\FPeval\amountg{round((#1 * 0.02834952):2)}\FPprint\amountg ~kg}
\newcommand{\lbtokg}[1]{\FPeval\amountg{round((#1 * 0.4535924):2)}\FPprint\amountg ~kg}
\newcommand{\galtol}[1]{\FPeval\amountg{round((#1 * 3.78541):1)}\FPprint\amountg ~l}
\newcommand{\qttol}[1]{\FPeval\amountg{round((#1 * 0.94635295):1)}\FPprint\amountg ~l}
\newcommand{\cuptoml}[1]{\FPeval\amountg{round((#1 * 240.0):0)}\FPprint\amountg ~ml}
\newcommand{\tsptoml}[1]{\FPeval\amountg{round((#1 * 4.92892):0)}\FPprint\amountg ~ml}
\newcommand{\srmtoebc}[1]{\FPeval\amountg{round((#1 * 1.97):0)}\FPprint\amountg ~EBC}
\newcommand{\sgtop}[1]{\FPeval\amountg{round((135.997 * pow(3,#1) - 630.272 * pow(2,#1) + 1111.14 * #1 - 616.868):1)}\FPprint\amountg ~°P}
\newcommand{\voltogl}[1]{\FPeval\amountg{round((#1 * 1.96):1)}\FPprint\amountg ~g/l}
\newcommand{\lboztokg}[2]{\FPeval\amountg{round((#1 * 0.4535924 + #2 * 28.34952 / 1000.0):2)}\FPprint\amountg ~kg}


\newcommand{\specitem}[1]{\textcolor{black}{\uppercase{\textbf{#1}}}}

\newenvironment{malts} {\recipesection{images/malt.pdf}{Malt / Mash Additions}
\begin{tabu} to \textwidth {Xr}
\textbf{Name} & \textbf{Amount} \\ \midrule}{\end{tabu}}
\newcommand{\malt}[2]{#1 & #2 \\ \midrule}

\newenvironment{hops} {\recipesection{images/hop.pdf}{Hops / Boil Additions}
\begin{tabu} to \textwidth {Xcr}
\textbf{Name} & \textbf{Addition} & \textbf{Amount} \\ \midrule}{\end{tabu}
{\centering\small{FWP = first wort hopping, FO = flameout, WP = whirlpool, DHBT =
biotransformation, DH = dry hopping} \par}}
\newcommand{\hop}[4]{#1 \ifthenelse{\isempty{#2}}{}{(#2)} & \ifthenelse{\isempty{#3}}{--}{#3} & #4 \\ \midrule}

\newenvironment{yeastsx} {\recipesection{images/yeast.pdf}{Yeast}
\begin{tabu} to \textwidth {Xr}
\textbf{Name} & \textbf{Addition} \\ \midrule}{\end{tabu}}
\newcommand{\yeastx}[2]{#1 & #2 \\ \midrule}

\newenvironment{yeasts} {\recipesection{images/yeast.pdf}{Yeast}
\begin{tabu} to \textwidth {X}}{\end{tabu}}
\newcommand{\yeast}[1]{#1 \\ \midrule}

\newcommand{\singleyeast}[1]{
\begin{yeasts}
\yeast{#1}
\end{yeasts}
}

\newenvironment{twists} {\recipesection{images/star.pdf}{Fermentation Additons}
\begin{tabu} to \textwidth {Xcr}
\textbf{Name} & \textbf{Addition} & \textbf{Amount} \\ \midrule}{\end{tabu}}
\newcommand{\twist}[3]{#1 & \ifthenelse{\isempty{#2}}{--}{#2} & #3 \\ \midrule}

\newenvironment{mashsteps} {\recipesection{images/mash.pdf}{Mash}
\begin{tabu} to \textwidth {Xr}}{\end{tabu}}
\newcommand{\mashstep}[2]{#1 & #2 \\ \midrule}
\newcommand{\mashdecoctthin}[1]{\mashstep{}{Thin decoction #1}}
\newcommand{\mashdecoctthick}[1]{\mashstep{}{Thick decoction #1}}
\newcommand{\mashdecoctboil}[1]{\mashstep{\ftoc{212}}{Boil #1}}
\newcommand{\mashdecoctreturn}[2]{\mashstep{#1}{Return to main mash\ifthenelse{\isempty{#2}}{}{; #2}}}

\newenvironment{fermentationsteps} {\recipesection{images/ferment.pdf}{Fermentation} 
\begin{tabu} to \textwidth {Xr}}{\end{tabu}}
\newcommand{\fermentationstep}[2]{#1 & #2 \\ \midrule}

\newenvironment{directions} {\recipesection{images/bulp.pdf}{Directions}}{}

\newenvironment{methodandtiming} {\begin{recipeblock}{Method / Timings}}{\end{recipeblock}}

\newenvironment{yeastinfos}[1] {\begin{recipeblock}{#1} 
\begin{tabu} to \textwidth {lX}}{\end{tabu}
\end{recipeblock}}
\newcommand{\yeastinfo}[2]{#1 & #2 \\ \midrule}

\newcommand{\dryhbt}[2]{DHBT\ifthenelse{\isempty{#1}}{}{\textsubscript{#1}}\ifthenelse{\isempty{#2}}{}{ (#2)}}
\newcommand{\dryh}[2]{DH\ifthenelse{\isempty{#1}}{}{\textsubscript{#1}}\ifthenelse{\isempty{#2}}{}{ (#2)}}
\newcommand{\whirl}[2]{WP\ifthenelse{\isempty{#1}}{}{\textsubscript{#1}}\ifthenelse{\isempty{#2}}{}{ (#2)}}
\newcommand{\fwh}{FWH}
\newcommand{\foh}[1]{FO\ifthenelse{\isempty{#1}}{}{ (#1)}}

\newcommand{\hopahtanum}{Ahtanum}
\newcommand{\hopamarillo}{Amarillo}
\newcommand{\hopapollo}{Apollo}
\newcommand{\hopaurora}{Aurora}
\newcommand{\hopbravo}{Bravo}
\newcommand{\hopcalypso}{Calypso}
\newcommand{\hopcascade}{Cascade}
\newcommand{\hopceleia}{Celeia}
\newcommand{\hopcentennial}{Centennial}
\newcommand{\hopchallenger}{Challenger}
\newcommand{\hopchinook}{Chinook}
\newcommand{\hopcitra}{Citra}
\newcommand{\hopcolumbus}{Columbus}
\newcommand{\hopcomet}{Comet}
\newcommand{\hopcrystal}{Crystal}
\newcommand{\hopeastkentgolding}{East Kent Golding}
\newcommand{\hopeldorado}{El Dorado}
\newcommand{\hopeureka}{Eureka}
\newcommand{\hopfalconersflight}{Falconer's Flight}
\newcommand{\hopfuggle}{Fuggle}
\newcommand{\hopgalaxy}{Galaxy}
\newcommand{\hopgolding}{Golding}
\newcommand{\hophallertaumittelfruh}{Hallertau Mittelfrüh}
\newcommand{\hophallertautradition}{Hallertau Tradition}
\newcommand{\hopherkules}{Herkules}
\newcommand{\hophersbrucker}{Hersbrucker}
\newcommand{\hophorizon}{Horizon}
\newcommand{\hopidahoseven}{Idaho 7}
\newcommand{\hopkohatu}{Kohatu}
\newcommand{\hoplemondrop}{Lemondrop}
\newcommand{\hopliberty}{Liberty}
\newcommand{\hoploral}{Loral}
\newcommand{\hoplubelska}{Lubelska}
\newcommand{\hopmagnum}{Magnum}
\newcommand{\hopmandarinabavaria}{Mandarina Bavaria}
\newcommand{\hopmarynka}{Marynka}
\newcommand{\hopmosaic}{Mosaic}
\newcommand{\hopmotueka}{Motueka}
\newcommand{\hopmthood}{Mt. Hood}
\newcommand{\hopnelsonsauvin}{Nelson Sauvin}
\newcommand{\hopnorthernbrewer}{Northern Brewer}
\newcommand{\hopnugget}{Nugget}
\newcommand{\hoppacifica}{Pacifica}
\newcommand{\hoppacificgem}{Pacific Gem}
\newcommand{\hoppacificjade}{Pacific Jade}
\newcommand{\hoppalisade}{Palisade}
\newcommand{\hopperle}{Perle}
\newcommand{\hoppolaris}{Polaris}
\newcommand{\hoprakau}{Rakau}
\newcommand{\hopsaaz}{Saaz}
\newcommand{\hopsantiam}{Santiam}
\newcommand{\hopsimcoe}{Simcoe}
\newcommand{\hopspalt}{Spalt}
\newcommand{\hopspaltselect}{Spalt Select}
\newcommand{\hopsterling}{Sterling}
\newcommand{\hopstrisselspalt}{Strisselspalt}
\newcommand{\hopstyriangolding}{Styrian Golding}
\newcommand{\hopsummit}{Summit}
\newcommand{\hoptarget}{Target}
\newcommand{\hoptettnang}{Tettnang}
\newcommand{\hopvanguard}{Vanguard}
\newcommand{\hopvicsecret}{Vic Secret}
\newcommand{\hopwakatu}{Wakatu}
\newcommand{\hopwarrior}{Warrior}
\newcommand{\hopwillamette}{Willamette}
\newcommand{\hopzythos}{Zythos}

% Light Lager
\newcommand{\styleamericanlager}{American Lager}
\newcommand{\stylemunichhelles}{Munich Helles}
\newcommand{\stylefestbier}{Festbier}

% Pilsner
\newcommand{\stylegermanpils}{German Pils}
\newcommand{\styleczechpremiumpalelager}{Czech Premium Pale Lager}

% European Amber Lager
\newcommand{\styleviennalager}{Vienna Lager}

% Bock
\newcommand{\stylehellesbock}{Helles Bock}
\newcommand{\styledopplebock}{Dopplebock}

% Dark Lager
\newcommand{\stylemunichdunkel}{Munich Dunkel}
\newcommand{\styleschwarzbier}{Schwarzbier}

% Light Hybrid Beer
\newcommand{\stylecreamale}{Cream Ale}
\newcommand{\styleblondeale}{Blonde Ale}
\newcommand{\styleamericanwheat}{American Wheat}

% Amber Hybrid Beer
\newcommand{\styleinternationalamberlager}{International Amber Lager}
\newcommand{\stylecaliforniacommon}{California Common}

% English Pale Ale
\newcommand{\styleordinarybitter}{Ordinary Bitter}
\newcommand{\stylestrongbitter}{Strong Bitter}

% Scottish & Irish Ale
\newcommand{\stylescottishlight}{Scottish Light}
\newcommand{\stylescottishheavy}{Scottish Heavy}
\newcommand{\styleirishredale}{Irish Red Ale}

% American Pale Ale
\newcommand{\styleamericanpaleale}{American Pale Ale}

% Other American Ale
\newcommand{\styleamericanamberale}{American Amber Ale}

% English Brown Ale
\newcommand{\styledarkmild}{Dark Mild}
\newcommand{\stylebritishbrownale}{British Brown Ale}

% Porter
\newcommand{\styleenglishporter}{English Porter}
\newcommand{\styleamericanporter}{American Porter}
\newcommand{\stylebalticporter}{Baltic Porter}

% Stout
\newcommand{\styleoatmealstout}{Oatmeal Stout}
\newcommand{\stylesweetstout}{Sweet Stout}

% Strong Stout
\newcommand{\styleamericanstout}{American Stout}
\newcommand{\styleimperialstout}{Imperial Stout}

% American IPA
\newcommand{\styleamericanipa}{American IPA}

% India Pale Ale
\newcommand{\styleindiapaleale}{India Pale Ale}
\newcommand{\styleryeipa}{Specialty IPA: Rye IPA}

% German Wheat Rye Beer
\newcommand{\styleweissbier}{Weissbier}
\newcommand{\styleweizenbock}{Weizenbock}

% Belgian & French Al
\newcommand{\stylewitbier}{Witbier}
\newcommand{\stylebelgianpaleale}{Belgian Pale Ale}
\newcommand{\stylesaison}{Saison}

% Belgian Strong Ale
\newcommand{\stylebelgianblondale}{Belgian Blond Ale}
\newcommand{\stylebelgiandubbel}{Belgian Dubbel}
\newcommand{\stylebelgiangoldenstrongale}{Belgian Golden Strong Ale}
\newcommand{\stylebelgiandarkstrongale}{Belgian Dark Strong Ale}

% Strong Ale
\newcommand{\styleoldale}{Old Ale}
\newcommand{\styleenglishbarleywine}{English Barleywine}
\newcommand{\styleamericanbarleywine}{American Barleywine}

% Fruit Beer
\newcommand{\stylefruitbeer}{Fruit Beer}

% Spice / Herb / Vegetable Beer
\newcommand{\stylewinterseasonalbeer}{Winter Seasonal Beer}

% Specialty Beer
\newcommand{\stylealternativesugarbeer}{Alternative Sugar Beer}

\newcommand{\sourceaha}{Source: American Homebrewers Association.}
\newcommand{\sourcezymurgy}[1]{Source: Zymurgy #1.}
\newcommand{\waterprofile}[6]{\ce{Ca} #1~ppm, \ce{Mg} #2~ppm, \ce{SO4} #3~ppm, \ce{Na} #4~ppm, \ce{Cl} #5~ppm\ifthenelse{\isempty{#6}}{}{, \ce{HCO3} #6~ppm}}

\newcommand{\specifications}[9]{
\begin{recipeblock}{Specifications}
\begin{tabu} to \textwidth {Xr}
\specitem{Style} & #1 \\ \midrule
\specitem{Volume} & #2 \\ \midrule
\specitem{Original Gravity} & \ifthenelse{\isempty{#3}}{--}{\sgtop{#3} / #3} \\ \midrule
\specitem{Final Gravity} & \ifthenelse{\isempty{#4}}{--}{\sgtop{#4} / #4} \\ \midrule
\specitem{ABV} & \ifthenelse{\isempty{#5}}{--}{#5} \\ \midrule
\specitem{Bitterness} & \ifthenelse{\isempty{#6}}{--}{\ibutoibu{#6}} \\ \midrule
\specitem{Color} & \ifthenelse{\isempty{#7}}{--}{#7} \\ \midrule
\specitem{Boil Time} & \ifthenelse{\isempty{#8}}{--}{#8} \\ \midrule
\specitem{Carbonation} & \ifthenelse{\isempty{#9}}{--}{\voltogl{#9} / #9~vol} \\ \midrule
\end{tabu}
\end{recipeblock}
}

\newgeometry{left=2.5cm,right=2.5cm,top=2.5cm,bottom=3cm}

\RedeclareSectionCommand[beforeskip=0mm,afterindent=false,afterskip=3mm]{chapter}
\RedeclareSectionCommand[afterindent=false,afterskip=0.5mm]{section}
\renewcommand*{\partpagestyle}{empty} 

\makeindex[columns=2, title=Recipe Index, options= -s recipeindex.ist]

\begin{document}

\pdfbookmark[part]{Cover}{cover}
\title{All Grain Homebrew Beer Recipes}
\date{\today}
\publishers{Editor: Thomas Ascher}
\dedication{Dedicated to BrauCampus Graz Hobbybrauer Stammtisch}
\maketitle

\frontmatter

\chapter*{Cheers!}

This beer recipe book was created as an homage to the BrewDog Recipe
Guide and compiled from various sources. It is not meant for the faint hearted
since it will not teach you about beer styles, ingredients, the brewing
process itself or safety measures. Books like Randy Mosher's "Mastering Home Brew"
and John Palmer's "How To Brew" will provide you with the necessary information.

All recipes in this book are organized by the National Homebrew Competition
2016 Style Guidelines and reproduced in a simplified form as faithfully
as possible. However, some information was altered for the sake of consistency,
interpreted when missing or discarded when deemed unnecessary or hard to reproduce.

Some carbonation levels specified by the following recipes exceed the safety
limits of single use or standard glass bottles. This is especially true for some
German and Belgian beer styles. Always take all necessary precautions when
bottle conditioning.

Notice of Liability: Neither the editor or contributors assume any responsibility
for the use or misuse of the presented information. It is the responsibility of
the reader to exercise good judgement and to observe all local laws and ordinances
regarding the production and consumption of alcoholic beverages.

\clearpage
\pdfbookmark[part]{Contents}{contents}
\tableofcontents

\mainmatter

% Last source: Ginger Dinger 2017 Dark Lager 

% National Homebrew Competition 2016 Style Guidelines

% TODO: recheck water profiles for CaCO3 vs HCO3

% -----------------------------------------------------------------------------
\stylecategory{Light Lager}
\stylesection{\styleamericanlightlager}

% -----------------------------------------------------------------------------
\begin{recipe}{Kein Bier Light Lager} % rechecked
% -----------------------------------------------------------------------------

\begin{aboutblock}
Recipe by Gregory Strawser and Ryan Metzger of Rochester, NY. Silver medal in
Category 1: Pale American Lager during the 2018 National Homebrew Competition in
Portland, OR.
\sourceaha
\end{aboutblock}

\specifications{\styleamericanlightlager}{\galtol{5}}{1.034}{1.005}{3.8~\%}{13}{\srmtoebc{3}}{60~min}{2.5}

\begin{methodandtiming}

\begin{mashsteps}
\mashstep{\ftoc{148}}{60~min}
\mashstep{\ftoc{168}}{Mash out}
\end{mashsteps}

\begin{fermentationsteps}
\fermentationstep{\ftoc{50}}{10~days}
\fermentationstep{\ftoc{65}}{Slow raise; 3~days}
\end{fermentationsteps}

\begin{directions}
Cold crash for 2 days.
\end{directions}

\end{methodandtiming}

\recipebreak

\begin{ingredientsblock}

\begin{malts}
\malt{Pale}{\lbtokg{6}}
\malt{Flaked Rice}{\lbtokg{1}}
\malt{Weyermann CARAHELL}{\lbtokg{0.25}}
\end{malts}

\begin{hops}
\hop{\hophallertaumittelfruh}{5~\%}{60~min}{\oztog{0.75}}
\end{hops}

\singleyeast{White Labs WLP840 / White Labs WLP838}

\end{ingredientsblock}



\part{\styleamericanlager}

\begin{recipie}{Parleaux Beer Lab Lemony Sippit Lager}

\begin{aboutblock}
This hopped-up take on a classic American lager from Parleaux Beer Lab is whirlpooled
with lemongrass and then dry-hopped with Lemondrop hops, lending it a rich, bright aroma
with a smooth lemon candy-like finish.
\end{aboutblock}

\specifications{\styleamericanlager}{\galtol{5}}{1.052}{1.012}{5.2~\%}{21}{\srmtoebc{3.6}}{90~min}

\begin{methodandtiming}
  
\begin{mashsteps}
\mashstep{\ftoc{152}}{}
\end{mashsteps}

\begin{fermentationsteps}
\fermentationstep{\ftoc{50}}{Until \sgtop{1.025}}
\fermentationstep{\ftoc{68}}{Raise to by 3~°C per day}
\fermentationstep{\ftoc{68}}{48~hours}
\fermentationstep{\ftoc{40}}{3--4~weeks}
\end{fermentationsteps}

\begin{directions}
Fresh lemongrass is always preferred if available. If a heavy lemongrass flavor is preferred
try making a lemongrass tincture, and adding adding to taste at packaging. Simply combine
\oztog{6} of vodka with \oztog{2} of fresh lemongrass in an airtight container and let sit at
room temperature for 2--4 weeks.
\end{directions}

\end{methodandtiming}

\pagebreak

\begin{ingredientsblock}

\begin{malts}
\malt{Weyermann Bohemian Pilsner}{\lbtokg{7.5}}
\malt{Vienna}{\lbtokg{1.5}}
\malt{Flaked Rice}{\lbtokg{1.5}}
\malt{Carapils}{\lbtokg{1}}
\end{malts}

\begin{hops}
\hop{\hoplemondrop}{6~\%}{40~min}{\oztog{1}}
\hop{\hoplemondrop}{6~\%}{\whirl{}{20~min}}{\oztog{2}}
\hop{Lemongrass}{}{\whirl{}{20~min}}{\oztog{2}}
\hop{\hoplemondrop}{6~\%}{\dryh{}{}}{\oztog{2}}
\end{hops}

\begin{yeasts}
\yeast{Wyeast 2278}
\end{yeasts}

\end{ingredientsblock}

\end{recipie}

\begin{recipie}{Phil's Lager}

\begin{aboutblock}
Philip Blosser of Oregon, OH, member of the Glass City Mashers, won a gold medal in
Category \#1: Light Lager during the 2017 National Homebrew Competition Final Round in
Minneapolis, MN. Blosser’s Light Lager was chosen as the best among 89 entries in the
category.
\end{aboutblock}

\specifications{\styleamericanlager}{\galtol{5}}{1.035}{1.005}{3.9~\%}{}{}{60~min}

\begin{methodandtiming}
  
\begin{mashsteps}
\mashstep{\ftoc{152}}{90~min}
\end{mashsteps}

\begin{fermentationsteps}
\fermentationstep{\ftoc{54}}{15~days}
\fermentationstep{\ftoc{34}}{14~days}
\end{fermentationsteps}

\end{methodandtiming}

\pagebreak

\begin{ingredientsblock}

\begin{malts}
\malt{German Pilsner}{\lbtokg{2.72}}
\malt{Caramel 10 L}{\lbtokg{0.25}}
\malt{Caramel 20 L}{\lbtokg{0.25}}
\end{malts}

\begin{hops}
\hop{\hopcentennial}{8.4~\%}{60~min}{\oztog{0.3}}
\hop{Whirlfloc Tablet}{}{}{1}
\end{hops}

\begin{yeasts}
\yeast{Fermentis SafLager W-34/70}
\end{yeasts}

\end{ingredientsblock}

\end{recipie}
\stylesection{\styleinternationalpalelager}

% -----------------------------------------------------------------------------
\begin{recipe}{Homebrew Challenge International Pale Lager}
% -----------------------------------------------------------------------------

\begin{aboutblock}
Recipe by Martin Keen.
\sourcehomebrewchallenge
\end{aboutblock}

\specifications{\styleinternationalpalelager}{\galtol{5}}{}{}{}{25}{}{60~min}{}

\begin{methodandtiming}

\begin{mashsteps}
\mashstep{\ftoc{152}}{60~min}
\end{mashsteps}

\begin{fermentationsteps}
\fermentationstep{\ftoc{50}}{}
\end{fermentationsteps}

\end{methodandtiming}

\recipebreak

\begin{ingredientsblock}

\begin{malts}
\malt{\maltpilsner}{\lbtokg{11}}
\malt{\maltvienna}{\lbtokg{1}}
\malt{\maltacidulated}{\oztokg{5}}
\malt{Caramel / Crystal 10 L}{\oztokg{5}}
\end{malts}

\begin{hops}
\hop{\hopchinook}{}{45~min}{\oztog{0.4}}
\hop{\hopsaaz}{}{10~min}{\oztog{1}}
\hop{\hoptettnang}{}{10~min}{\oztog{1}}
\end{hops}

\singleyeast{White Labs WLP830}

\end{ingredientsblock}

\end{recipe}

\stylesection{\stylemunichhelles}

% -----------------------------------------------------------------------------
\begin{recipe}{Brewin' in Maillard Helles}
% -----------------------------------------------------------------------------

\begin{aboutblock}
Recipe by Russell Berger of Portland, OR. Bronze medal in Category \#2: Pale
European Beer during the 2019 National Homebrew Competition in Providence, RI.
\sourceaha
\end{aboutblock}

\specifications{\stylemunichhelles}{\galtol{6}}{1.054}{1.010}{6.5~\%}{22}{\srmtoebc{5}}{90~min}{2.7}

\begin{methodandtiming}
 
\begin{mashsteps}
\mashstep{\ftoc{131}}{15~min}
\mashstep{\ftoc{148}}{30~min}
\mashstep{\ftoc{160}}{30~min}
\mashstep{\ftoc{168}}{10~min}
\end{mashsteps}

\begin{fermentationsteps}
\fermentationstep{\ftoc{52}}{Until attenauted to \sgtop{1.030}}
\fermentationstep{\ftoc{57}}{Until attenuated to \sgtop{1.016}}
\fermentationstep{\ftoc{62}}{Until fully attenuated}
\fermentationstep{\ftoc{32}}{Slowly reduce to}
\end{fermentationsteps}

\begin{directions}
Bring mash pH to 5.2. Once the beer has reached \ftoc{32}, fine with
gelatin and lager for 4 weeks.
\end{directions}

\end{methodandtiming}

\recipebreak

\begin{ingredientsblock}

\begin{malts}
\malt{BEST Heidelberg}{\lbtokg{3.85}}
\malt{Weyermann Pilsner}{\lbtokg{3.85}}
\malt{Great Western Pure Idaho}{\lbtokg{3.75}}
\malt{Weyermann Melanoidin}{\lbtokg{0.25}}
\malt{Weyermann Munich I}{\lbtokg{0.65}}
\malt{Wyeast BrewTan B}{--}
\end{malts}

\begin{hops}
\hop{\hoptettnang}{4.4~\%}{\fwh}{\oztog{1.75}}
\hop{Whirlfloc Tablet}{}{15~min}{1}
\hop{Yeast Nutrient}{}{10~min}{--}
\end{hops}

\singleyeast{Imperial Yeast L17}

\end{ingredientsblock}

\end{recipe}

% -----------------------------------------------------------------------------
\begin{recipe}{Helicious Helles}
% -----------------------------------------------------------------------------

\begin{aboutblock}
Recipe by Kerry Martin. Best of show title in the 2010 Lunar Rendezbrew XVII
homebrew competition. \sourceaha
\end{aboutblock}

\specifications{\stylemunichhelles}{\galtol{7}}{1.050}{1.012}{5~\%}{19.1}{}{90~min}{}

\begin{methodandtiming}
 
\begin{mashsteps}
\mashstep{\ftoc{130}}{10~min}
\mashstep{\ftoc{148}}{60~min}
\mashstep{\ftoc{168}}{Mash out}
\end{mashsteps}

\begin{fermentationsteps}
\fermentationstep{\ftoc{50}}{2--3~weeks}
\end{fermentationsteps}

\begin{directions}
Add gelatin to clarify. Lager at \ftoc{40} for 3--6 weeks.
\end{directions}

\end{methodandtiming}

\recipebreak

\begin{ingredientsblock}

\begin{malts}
\malt{Pilsner}{\lbtokg{9.5}}
\malt{Torrified Wheat}{\lbtokg{1}}
\malt{Melanoidin}{\oztog{5}}
\malt{Acidulated}{\oztog{7}}
\end{malts}

\begin{hops}
\hop{\hoptettnang}{4~\%}{60~min}{\oztog{1.9}}
\hop{Whirlfloc Tablet}{}{10~min}{1}
\end{hops}

\singleyeast{Wyeast 2247-PC / White Labs 830 / White Labs 833}

\end{ingredientsblock}

\end{recipe}

% -----------------------------------------------------------------------------
\begin{recipe}{Hot Schnitzel}
% -----------------------------------------------------------------------------

\begin{aboutblock}
Recipe by Metts Potter of Browns Summit, NC. Silver medal in Category \#2: Pale
European during the 2019 National Homebrew Competition in Providence, RI.
\sourceaha
\end{aboutblock}

\specifications{\stylemunichhelles}{\galtol{5.5}}{1.050}{1.010}{5.2~\%}{18}{\srmtoebc{4.5}}{90~min}{2.5}

\begin{methodandtiming}
 
\begin{mashsteps}
\mashstep{\ftoc{151}}{60~min}
\mashstep{\ftoc{170}}{Mash out}
\end{mashsteps}

\begin{fermentationsteps}
\fermentationstep{\ftoc{50}}{5~days / fermentation start}
\fermentationstep{\ftoc{70}}{Raise to over 4~days; 5~days}
\end{fermentationsteps}

\begin{directions}
Mash at 5.3 pH. Clarify with gelatin.
\end{directions}

\end{methodandtiming}

\recipebreak

\begin{ingredientsblock}

\begin{malts}
\malt{Weyermann Pilsner}{\lbtokg{4}}
\malt{Weyermann Barke Pilsner}{\lbtokg{4}}
\malt{BEST Munich}{\lbtokg{1}}
\malt{Weyermann CARAFOAM}{\oztog{4}}
\malt{Weyermann CARAHELL}{\oztog{2}}
\end{malts}

\begin{hops}
\hop{\hoptettnang}{3.2~\%}{\fwh}{1}
\hop{\hoptettnang}{3.2~\%}{15~min}{1}
\end{hops}

\singleyeast{White Labs WLP838}

\end{ingredientsblock}

\end{recipe}

% -----------------------------------------------------------------------------
\begin{recipe}{Old Helen Helles Exportbier}
% -----------------------------------------------------------------------------

\begin{aboutblock}
Recipe by Gregory Irving of North Royalton, OH. Gold medal in Category \#2: Pale
European Beer during the 2019 National Homebrew Competition in Providence, RI.
\sourceaha
\end{aboutblock}

\specifications{\stylemunichhelles}{\galtol{10}}{1.052}{1.012}{5.3~\%}{28}{\srmtoebc{5}}{90~min}{2.5}

\begin{methodandtiming}
 
\begin{mashsteps}
\mashstep{\ftoc{152}}{60~min}
\mashstep{\ftoc{168}}{Mash out}
\end{mashsteps}

\begin{fermentationsteps}
\fermentationstep{\ftoc{50}}{5~days}
\fermentationstep{\ftoc{65}}{Raise to over 5~days; until fully attenuated}
\fermentationstep{\ftoc{45}}{Slowly reduce to}
\end{fermentationsteps}

\end{methodandtiming}

\recipebreak

\begin{ingredientsblock}

\begin{malts}
\malt{Pilsner}{\lbtokg{19}}
\malt{Carapils / Dextrin}{\lbtokg{1}}
\malt{Melanoidin}{\lbtokg{1}}
\end{malts}

\begin{hops}
\hop{\hophersbrucker ~Cones}{}{\fwh}{\oztog{1}}
\hop{\hopmagnum ~Cones}{}{60~min}{\oztog{1.2}}
\hop{\hophallertaumittelfruh}{}{12~min}{\oztog{2}}
\hop{\hophersbrucker ~Cones}{}{3~min}{\oztog{1.5}}
\end{hops}

\singleyeast{White Labs WLP833}

\end{ingredientsblock}

\end{recipe}

\stylesection{\stylefestbier}

% -----------------------------------------------------------------------------
\begin{recipe}{A Festbier For the Rest Of Us} % rechecked
% -----------------------------------------------------------------------------

\begin{aboutblock}
Recipe by John Watson of Antioch, CA. Gold medal in Category 3: European Amber Lager
during the 2013 National Homebrew Competition in Philadelphia, PA.
\sourceaha
\end{aboutblock}

\specifications{\stylefestbier}{\galtol{6}}{1.056}{1.014}{5.51~\%}{24.6}{\srmtoebc{7}}{}{}

\begin{methodandtiming}

\begin{mashsteps}
\mashstep{\ftoc{140}}{Mash in}
\mashdecoctthick{with \qttol{4} of mash}
\mashstep{\ftoc{154}}{15~min}
\mashdecoctboil{20~min}
\mashdecoctreturn{\ftoc{150}}{}
\mashstep{\ftoc{154}}{15~min}
\mashstep{\ftoc{156}}{15~min}
\mashstep{\ftoc{168}}{5~min}
\end{mashsteps}

\begin{fermentationsteps}
\fermentationstep{\ftoc{50}}{14~days}
\fermentationstep{\ftoc{68}}{2~days}
\fermentationstep{\ftoc{34}}{Reduction by \ctoc{1}/day; 3~months}
\end{fermentationsteps}

\begin{directions}
Water adjustment: carbon filtered Antioch, CA tap water.
\end{directions}

\end{methodandtiming}

\recipebreak

\begin{ingredientsblock}

\begin{malts}
\malt{Weyermann Bohemian Pilsner}{\lbtokg{5}}
\malt{Weyermann Munich I}{\lbtokg{4}}
\malt{Weyermann Vienna}{\lbtokg{3}}
\end{malts}

\begin{hops}
\hop{\hopmillennium}{16.8~\%}{\fwh}{\oztog{0.25}}
\hop{\hoptettnang}{4.8~\%}{30~min}{\oztog{0.75}}
\hop{\hoptettnang}{4.8~\%}{10~min}{\oztog{0.5}}
\end{hops}

\singleyeast{Wyeast 2633}

\end{ingredientsblock}

\end{recipe}

% -----------------------------------------------------------------------------
\begin{recipe}{Chris \& Ryan's Oktoberfest} % rechecked
% -----------------------------------------------------------------------------

\begin{aboutblock}
Recipe by Chris Bozzo and Ryan Reid of Livermore, CA. Silver medal in Category 4:
Amber European Beer during the 2018 National Homebrew Competition in Portland, OR.
\sourceaha
\end{aboutblock}

\specifications{\stylefestbier}{\galtol{10}}{1.058}{1.018}{5.3~\%}{21}{\srmtoebc{7.5}}{60~min}{2.3}

\begin{methodandtiming}

\begin{mashsteps}
\mashstep{\ftoc{152}}{90~min}
\mashstep{\ftoc{168}}{Mash out}
\end{mashsteps}

\begin{fermentationsteps}
\fermentationstep{\ftoc{50}}{14~days / attenuation of 80~\%}
\fermentationstep{\ftoc{60}}{Raise by \ctoc{2}/day; 5~days}
\end{fermentationsteps}

\begin{directions}
Water adjustment: \waterprofile{66}{30}{48}{56}{72}{112}, mash pH of 5.6, sparge
water pH of 6. Larger for 14--30 days.
\end{directions}

\end{methodandtiming}

\recipebreak

\begin{ingredientsblock}

\begin{malts}
\malt{Weyermann Pilsner}{\lboztokg{12}{3.2}}
\malt{Munich}{\lboztokg{9}{1}}
\malt{Vienna}{\oztog{13.9}}
\malt{Briess Caramel 60 L}{\oztog{8}}
\end{malts}

\begin{hops}
\hop{\hophallertaumittelfruh}{3.6~\%}{60~min}{\oztog{3.75}}
\end{hops}

\singleyeast{White Labs WLP820}

\end{ingredientsblock}

\end{recipe}

% -----------------------------------------------------------------------------
\begin{recipe}{Klang Freudenfest Oktoberfest} % rechecked
% -----------------------------------------------------------------------------

\begin{aboutblock}
One of three Big Brew 2017 official recipes.
\sourceaha
\end{aboutblock}

\specifications{\stylefestbier}{\galtol{5}}{1.055}{1.013}{5.6~\%}{27}{\srmtoebc{9}}{60~min}{2.5}

\begin{methodandtiming}

\begin{mashsteps}
\mashstep{\ftoc{129}}{5~min}
\mashstep{\ftoc{150}}{30~min}
\mashdecoctthick{with 40~\% of mash}
\mashstep{\ftoc{158}}{15~min}
\mashdecoctboil{10~min}
\mashdecoctreturn{\ftoc{167}}{5~min}
\end{mashsteps}

\begin{fermentationsteps}
\fermentationstep{\ftoc{52}}{14~days}
\fermentationstep{\ftoc{59}}{3~days}
\fermentationstep{\ftoc{48}}{Slow reduction; 8~days}
\fermentationstep{\ftoc{33}}{Slow reduction; 16~days}
\end{fermentationsteps}

\begin{directions}
Water adjustment: \waterprofile{75--125}{10}{--}{50--100}{100--150}{100--150}. % RA 0--50
\end{directions}

\end{methodandtiming}

\recipebreak

\begin{ingredientsblock}

\begin{malts}
\malt{Pilsner}{\lbtokg{3.5}}
\malt{Munich}{\lbtokg{3.5}}
\malt{Vienna}{\lbtokg{4}}
\malt{Aromatic Munich 20 L}{\lbtokg{1}}
\malt{Caravienne}{\lbtokg{0.33}}
\end{malts}

\begin{hops}
\hop{\hophallertautradition}{6~\%}{60~min}{\oztog{1}}
\hop{\hoptettnang}{4~\%}{20~min}{\oztog{1}}
\end{hops}

\singleyeast{German Lager}

\end{ingredientsblock}

\end{recipe}

% -----------------------------------------------------------------------------
\begin{recipe}{Jekyll Brewing Seven Bridges Oktoberfest Clone} % rechecked
% -----------------------------------------------------------------------------

\begin{aboutblock}
Malty sweetness is nicely balanced by German Tettnang hops.
\sourceaha
\end{aboutblock}

\specifications{\stylefestbier}{\galtol{10}}{1.054}{1.012}{}{23.2}{\srmtoebc{8.9}}{90~min}{}

\begin{methodandtiming}

\begin{mashsteps}
\mashstep{\ftoc{153}}{}
\end{mashsteps}

\begin{fermentationsteps}
\fermentationstep{\ftoc{65}}{Fermentation start}
\fermentationstep{\ftoc{54}}{Reduction over 1~day; almost full attenuation}
\fermentationstep{\ftoc{68}}{Raise by \ctoc{2}/day; 2~days}
\fermentationstep{\ftoc{55}}{Transfer to secondary}
\end{fermentationsteps}

\begin{directions}
After the transfer to secondary, reduce the temperature to lagering temperatures and
hold for 7--14 days.
\end{directions}

\end{methodandtiming}

\recipebreak

\begin{ingredientsblock}

\begin{malts}
\malt{Weyermann Munich I}{\lbtokg{9.75}}
\malt{Weyermann Vienna}{\lbtokg{4.9}}
\malt{Weyermann Pilsner}{\lbtokg{4}}
\malt{Weyermann Melanoidin}{\lbtokg{1}}
\malt{Weyermann Acidulated}{\oztog{2.9}}
\end{malts}

\begin{hops}
\hop{\hoptettnang}{4.5~\%}{60~min}{\oztog{2.2}}
\hop{\hoptettnang}{4.5~\%}{10~min}{\oztog{1.85}}
\end{hops}

\singleyeast{White Labs WLP830}

\end{ingredientsblock}

\end{recipe}

% -----------------------------------------------------------------------------
\begin{recipe}{Oktoberfest Hallertau} % rechecked
% -----------------------------------------------------------------------------

\begin{aboutblock}
Recipe by Stephen Quintana of Sheridan, WY. Gold medal in Category 3: European
Amber Lager during the 2009 National Homebrew Competition in Oakland, CA.
\sourceaha
\end{aboutblock}

\specifications{\stylefestbier}{\galtol{6}}{1.057}{1.015}{5.51~\%}{}{}{60~min}{2.4}

\begin{methodandtiming}

\begin{mashsteps}
\mashstep{\ftoc{151}}{80~min}
\mashstep{\ftoc{161}}{5~min}
\end{mashsteps}

\begin{fermentationsteps}
\fermentationstep{\ftoc{52}}{40~days}
\fermentationstep{\ftoc{32}}{21~days}
\end{fermentationsteps}

\end{methodandtiming}

\recipebreak

\begin{ingredientsblock}

\begin{malts}
\malt{Durst Pilsner}{\lbtokg{7}}
\malt{Weyermann Munich I}{\lbtokg{4}}
\malt{Weyermann Munich II}{\lbtokg{1}}
\malt{Weyermann CARAMUNICH I}{\lbtokg{0.5}}
\malt{Caramel / Crystal 40 L}{\lbtokg{0.5}}
\end{malts}

\begin{hops}
\hop{\hophallertaumittelfruh}{3~\%}{60~min}{\oztog{2}}
\hop{\hophallertaumittelfruh}{3~\%}{30~min}{\oztog{1}}
\hop{\hophallertaumittelfruh}{3~\%}{15~min}{\oztog{1}}
\end{hops}

\singleyeast{White Labs WLP820}

\end{ingredientsblock}

\end{recipe}

% -----------------------------------------------------------------------------
\begin{recipe}{Pickelhaube Festbier} % rechecked
% -----------------------------------------------------------------------------

\begin{aboutblock}
A Vienna-style Oktoberfest beer. \sourcezymurgy{January / February 2016}
\end{aboutblock}

\specifications{\stylefestbier}{\galtol{5.5}}{1.055}{1.010}{5.9~\%}{21}{\srmtoebc{4.1}}{90~min}{}

\begin{methodandtiming}
 
\begin{mashsteps}
\mashstep{\ftoc{122}}{20~min}
\mashdecoctthick{with \qttol{9} of mash}
\mashdecoctboil{15~min}
\mashdecoctreturn{\ftoc{150}}{60~min}
\mashstep{\ftoc{168}}{10~min}
\end{mashsteps}

\begin{fermentationsteps}
\fermentationstep{\ftoc{48}}{Fermentation slowdown}
\fermentationstep{\ftoc{55}}{3~days}
\fermentationstep{\ftoc{35}}{Attenuation to \sgtop{1.012}; 3~months}
\end{fermentationsteps}

\begin{directions}
Water adjustment: reverse osmosis water with \gpgaltogpl{1} calcium cloride.
\end{directions}

\end{methodandtiming}

\recipebreak

\begin{ingredientsblock}

\begin{malts}
\malt{Pilsner}{\lbtokg{7}}
\malt{Vienna}{\lbtokg{4}}
\end{malts}

\begin{hops}
\hop{\hophallertaumittelfruh}{4.8~\%}{\fwh}{\oztog{1}}
\hop{\hopspalt}{4.5~\%}{10~min}{\oztog{0.5}}
\end{hops}

\singleyeast{Wyeast 2206 / Wyeast 2308}

\end{ingredientsblock}

\end{recipe}

% -----------------------------------------------------------------------------
\begin{recipe}{Pickelhaube Märzen} % rechecked
% -----------------------------------------------------------------------------

\begin{aboutblock}
Classic German lager dedicated to Fred Eckhardt.
\sourceaha
\end{aboutblock}

\specifications{\stylefestbier}{\galtol{5.5}}{1.056}{1.010}{6~\%}{23.5}{\srmtoebc{10}}{90~min}{}

\begin{methodandtiming}

\begin{mashsteps}
\mashstep{\ftoc{122}}{20~min}
\mashdecoctthick{with \qttol{9} of mash}
\mashdecoctboil{15~min}
\mashdecoctreturn{\ftoc{150}}{60~min}
\mashstep{\ftoc{168}}{10~min}
\end{mashsteps}

\begin{fermentationsteps}
\fermentationstep{\ftoc{48}}{Pitch}
\fermentationstep{\ftoc{55}}{Fermentation slowdown; 3~days}
\fermentationstep{\ftoc{35}}{Attenuation to \sgtop{1.012}; 3~months}
\end{fermentationsteps}

\begin{directions}
Water adjustment: reverse osmosis water with \gpgaltogpl{1} calcium chloride.
\end{directions}

\end{methodandtiming}

\recipebreak

\begin{ingredientsblock}

\begin{malts}
\malt{Vienna}{\lbtokg{5.5}}
\malt{Briess Bonlander Munich 10 L}{\lbtokg{3.5}}
\malt{Briess Aromatic Munich 20 L}{\lbtokg{2.5}}
\end{malts}

\begin{hops}
\hop{\hoptettnang}{4.5~\%}{90~min}{\oztog{1.25}}
\hop{\hoptettnang}{4.5~\%}{15~min}{\oztog{0.25}}
\end{hops}

\singleyeast{Bavarian Lager / Munich Lager}

\end{ingredientsblock}

\end{recipe}

% -----------------------------------------------------------------------------
\begin{recipe}{Rodtoberfest} % rechecked
% -----------------------------------------------------------------------------

\begin{aboutblock}
Recipe by Nick Rodammer of Grand Rapids, MI. Gold medal in Category 3: European
Amber Lager during the 2014 National Homebrew Competition in Grand Rapids, MI.
\sourceaha
\end{aboutblock}

\specifications{\stylefestbier}{\galtol{5.5}}{1.056}{1.012}{5.78~\%}{}{}{60~min}{2.5}

\begin{methodandtiming}

\begin{mashsteps}
\mashstep{\ftoc{146}}{60~min}
\mashdecoctthick{with \qttol{10} of mash}
\mashdecoctboil{20~min}
\mashdecoctreturn{\ftoc{155}}{30~min}
\end{mashsteps}

\begin{fermentationsteps}
\fermentationstep{\ftoc{48}}{10~days}
\fermentationstep{\ftoc{58}}{7~days}
\fermentationstep{\ftoc{30}}{2~months}
\end{fermentationsteps}

\begin{directions}
Water adjustment: 75~\% reverse osmosis and 25~\% Plainfield tap water with
\gtog{3} calcium chloride and \gtog{1} calcium sulfate, acidified with \mltoml{2}
lactic acid. Heat up decoction in pressure cooker to \psitobar{15}.
\end{directions}

\end{methodandtiming}

\recipebreak

\begin{ingredientsblock}

\begin{malts}
\malt{Weyermann Vienna}{\lbtokg{10}}
\malt{Weyermann Munich II}{\lbtokg{3.5}}
\malt{Weyermann Acidulated}{\lbtokg{0.3}}
\end{malts}

\begin{hops}
\hop{\hopperle}{7.8~\%}{60~min}{\oztog{1}}
\end{hops}

\singleyeast{White Labs WLP833}

\end{ingredientsblock}

\end{recipe}

\stylesection{\stylegermanhellesexportbier}

% -----------------------------------------------------------------------------
\begin{recipe}{Homebrew Challenge German Helles Exportbier}
% -----------------------------------------------------------------------------

\begin{aboutblock}
Recipe by Martin Keen.
\sourcehomebrewchallenge
\end{aboutblock}

\specifications{\stylegermanhellesexportbier}{\galtol{5}}{}{}{6.4~\%}{}{}{60~min}{}

\begin{methodandtiming}

\begin{mashsteps}
\mashstep{\ftoc{148}}{60~min}
\end{mashsteps}

\begin{fermentationsteps}
\fermentationstep{\ftoc{50}}{}
\end{fermentationsteps}

\end{methodandtiming}

\recipebreak

\begin{ingredientsblock}

\begin{malts}
\malt{\maltpilsner}{\lbtokg{8}}
\malt{\maltvienna}{\lbtokg{2}}
\malt{\maltmunich}{\lbtokg{1}}
\end{malts}

\begin{hops}
\hop{\hophallertaumittelfruh}{}{60~min}{\oztog{2}}
\hop{\hophallertaumittelfruh}{}{10~min}{\oztog{1}}
\hop{\hoptettnang}{}{\foh{}{}}{\oztog{1}}
\end{hops}

\singleyeast{White Labs WLP830}

\end{ingredientsblock}

\end{recipe}

% -----------------------------------------------------------------------------
\begin{recipe}{Portlandmunder X}
% -----------------------------------------------------------------------------

\begin{aboutblock}
Ritchie Marvin of Portland, OR. Gold medal in Category 1: Light Lager during the
2014 National Homebrew Competition in Grand Rapids, MI.
\sourceaha
\end{aboutblock}

\specifications{\stylegermanhellesexportbier}{\galtol{4}}{1.053}{1.014}{5.12~\%}{}{}{60~min}{2.5}

\begin{methodandtiming}

\begin{mashsteps}
\mashstep{\ftoc{149}}{90~min}
\end{mashsteps}

\begin{fermentationsteps}
\fermentationstep{\ftoc{53}}{10~days}
\fermentationstep{\ftoc{35}}{38~days}
\end{fermentationsteps}

\end{methodandtiming}

\recipebreak

\begin{ingredientsblock}

\begin{malts}
\malt{\maltpilsner}{\lbtokg{5}}
\malt{\maltvienna}{\lbtokg{1.5}}
\malt{\maltpale}{\lbtokg{1.5}}
\malt{Flaked Wheat}{\lbtokg{0.5}}
\end{malts}

\begin{hops}
\hop{\hopsaaz}{4.2~\%}{60~min}{\oztog{0.75}}
\hop{\hopsaaz}{4.2~\%}{20~min}{\oztog{0.5}}
\hop{\hopultra}{5~\%}{10~min}{\oztog{0.75}}
\hop{\hopultra}{5~\%}{5~min}{\oztog{0.25}}
\hop{\hopsaaz}{4.2~\%}{5~min}{\oztog{0.5}}
\end{hops}

\singleyeast{White Labs WLP815}

\end{ingredientsblock}

\end{recipe}


% -----------------------------------------------------------------------------
\stylecategory{Pilsner}
\stylesection{\stylegermanleichtbier}

% -----------------------------------------------------------------------------
\begin{recipe}{Homebrew Challenge German Leichtbier}
% -----------------------------------------------------------------------------

\begin{aboutblock}
Recipe by Martin Keen.
\sourcehomebrewchallenge
\end{aboutblock}

\specifications{\stylegermanleichtbier}{\galtol{5}}{}{}{4~\%}{}{}{60~min}{}

\begin{methodandtiming}

\begin{mashsteps}
\mashstep{\ftoc{152}}{60~min}
\end{mashsteps}

\begin{fermentationsteps}
\fermentationstep{\ftoc{65}}{}
\end{fermentationsteps}

\end{methodandtiming}

\recipebreak

\begin{ingredientsblock}

\begin{malts}
\malt{\maltpilsner}{\lbtokg{6}}
\malt{Briess Victory}{\oztokg{8}}
\end{malts}

\begin{hops}
\hop{\hopstyriangolding}{}{60~min}{\oztog{1}}
\hop{\hophallertaumittelfruh}{}{5~min}{\oztog{0.5}}
\hop{\hopsaaz}{}{\foh{}{}}{\oztog{1}}
\end{hops}

\singleyeast{White Labs WLP029}

\end{ingredientsblock}

\end{recipe}

\stylesection{\styleczechpalelager}

% -----------------------------------------------------------------------------
\begin{recipe}{Homebrew Challenge Czech Pale Lager}
% -----------------------------------------------------------------------------

\begin{aboutblock}
Recipe by Martin Keen.
\sourcehomebrewchallenge
\end{aboutblock}

\specifications{\styleczechpalelager}{\galtol{5}}{}{}{3.8~\%}{24}{}{60~min}{}

\begin{methodandtiming}

\begin{mashsteps}
\mashstep{\ftoc{152}}{60~min}
\end{mashsteps}

\begin{fermentationsteps}
\fermentationstep{\ftoc{50}}{}
\end{fermentationsteps}

\end{methodandtiming}

\recipebreak

\begin{ingredientsblock}

\begin{malts}
\malt{\maltweyermannbohemianpilsner}{\lbtokg{6}}
\malt{Briess Victory}{\oztokg{12}}
\malt{\maltacidulated}{\oztokg{4}}
\end{malts}

\begin{hops}
\hop{\hopstyriangolding}{}{60~min}{\oztog{1.5}}
\hop{\hopsaaz}{}{15~min}{\oztog{1}}
\hop{\hopsaaz}{}{\foh{}{}}{\oztog{1}}
\end{hops}

\singleyeast{Wyeast 2308}

\end{ingredientsblock}

\end{recipe}

\stylesection{\stylegermanpils}

% -----------------------------------------------------------------------------
\begin{recipe}{Mitte Flüsse Pils} % rechecked
% -----------------------------------------------------------------------------

\begin{aboutblock}
Recipe by Jerry Mitchell of O'Fallon, MO. Silver medal in Category 3: Pilsner
during the 2019 National Homebrew Competition in Providence, RI. \sourceaha
\end{aboutblock}

\specifications{\stylegermanpils}{\galtol{5.7}}{1.049}{1.009}{4.8~\%}{35.8}{\srmtoebc{3.2}}{90~min}{}

\begin{methodandtiming}
 
\begin{mashsteps}
\mashstep{\ftoc{148}}{60~min}
\end{mashsteps}

\begin{fermentationsteps}
\fermentationstep{\ftoc{45}}{Pitch}
\fermentationstep{\ftoc{50}}{Free raise; 4~days}
\fermentationstep{\ftoc{60}}{7~days}
\fermentationstep{\ftoc{33}}{Reduction over 3~days; 1~month}
\end{fermentationsteps}

\begin{directions}
Water adjustment: reverse osmosis water with \gtog{3.5} each of calcium chloride
and calcium sulphate.
\end{directions}

\end{methodandtiming}

\recipebreak

\begin{ingredientsblock}

\begin{malts}
\malt{Weyermann Pilsner}{\lbtokg{8.7}}
\malt{Weyermann CARAFOAM}{\oztog{8}}
\malt{Acidulated}{\oztog{2}}
\malt{Melanoidin}{\oztog{2}}
\end{malts}

\begin{hops}
\hop{\hopperle}{7.2~\%}{60~min}{\oztog{0.8}}
\hop{\hophallertautradition}{5.9~\%}{30~min}{\oztog{0.5}}
\hop{\hopspaltselect}{4.6~\%}{30~min}{\oztog{0.5}}
\hop{\hophallertautradition}{5.9~\%}{5~min}{\oztog{0.4}}
\hop{\hopspaltselect}{4.6~\%}{5~min}{\oztog{0.4}}
\end{hops}

\singleyeast{Wyeast 2308}

\end{ingredientsblock}

\end{recipe}

% -----------------------------------------------------------------------------
\begin{recipe}{SunRaker German Pils} % rechecked
% -----------------------------------------------------------------------------

\begin{aboutblock}
Recipe by Philip Verdieck of Houston, TX. Gold medal in Category 3: Pilsner
during the 2019 National Homebrew Competition in Providence, RI. \sourceaha
\end{aboutblock}

\specifications{\stylegermanpils}{\galtol{5}}{1.050}{1.010}{5.1~\%}{48}{\srmtoebc{4.1}}{60~min}{}

\begin{methodandtiming}
 
\begin{mashsteps}
\mashstep{\ftoc{149}}{60~min}
\mashstep{\ftoc{169}}{10~min}
\end{mashsteps}

\begin{fermentationsteps}
\fermentationstep{\ftoc{50}}{14~days}
\fermentationstep{\ftoc{58}}{Full attenuation}
\fermentationstep{\ftoc{68}}{2~days}
\end{fermentationsteps}

\begin{directions}
Water adjustment: mash pH of 5.4. Add dry hops on day 6.
\end{directions}

\end{methodandtiming}

\recipebreak

\begin{ingredientsblock}

\begin{malts}
\malt{Avangard Pilsner}{\lbtokg{10}}
\malt{Briess Victory}{\oztog{4}}
\malt{Acidulated}{\oztog{2}}
\end{malts}

\begin{hops}
\hop{\hopmagnum}{13.2~\%}{60~min}{\oztog{1}}
\hop{\hophallertaumittelfruh}{3.6~\%}{10~min}{\oztog{1}}
\hop{\hopsaaz}{2.8~\%}{\foh{}{}}{\oztog{0.75}}
\hop{\hophersbrucker}{2.3~\%}{\dryh{}{}}{\oztog{1}}
\end{hops}

\singleyeast{Wyeast 2124}

\end{ingredientsblock}

\end{recipe}

\stylesection{\styleczechpremiumpalelager}

% -----------------------------------------------------------------------------
\begin{recipe}{Hobo BoHo Pilsner} % rechecked
% -----------------------------------------------------------------------------

\begin{aboutblock}
\sourcezymurgy{May / June 2019}
\end{aboutblock}

\specifications{\styleczechpremiumpalelager}{\galtol{5.5}}{1.051}{1.010}{5.2~\%}{40}{\srmtoebc{3}}{60~min}{}

\begin{methodandtiming}
 
\begin{mashsteps}
\mashstep{\ftoc{122}}{20~min}
\mashdecoctthick{with 1/3 of mash}
\mashstep{\ftoc{152}}{20~min}
\mashdecoctboil{}
\mashdecoctreturn{\ftoc{150}}{}
\mashdecoctthin{with 1/3 of mash}
\mashdecoctboil{}
\mashdecoctreturn{\ftoc{170}}{mash out}
\end{mashsteps}

\begin{fermentationsteps}
\fermentationstep{\ftoc{48}}{14~days}
\fermentationstep{\ftoc{64}}{1~day}
\fermentationstep{\ftoc{35}}{14~days}
\end{fermentationsteps}

\begin{directions}
Water adjustment: a minimum of \ppmtopptm{50} calcium and minimal levels of
other minerals. Instead of decoction, a single infusion mash at \ftoc{150} may
be performed.
\end{directions}

\end{methodandtiming}

\recipebreak

\begin{ingredientsblock}

\begin{malts}
\malt{Weyermann Bohemian Pilsner}{\lbtokg{11}}
\end{malts}

\begin{hops}
\hop{\hopsaaz}{4.5~\%}{60~min}{\oztog{2.25}}
\hop{\hopsaaz}{4.5~\%}{\foh{}{}}{\oztog{1}}
\end{hops}

\singleyeast{Wyeast 2124}

\end{ingredientsblock}

\end{recipe}

% -----------------------------------------------------------------------------
\begin{recipe}{Homebrew Challenge Czech Premium Pale Lager} % rechecked
% -----------------------------------------------------------------------------

\begin{aboutblock}
Recipe by Martin Keen.
\sourcehomebrewchallenge
\end{aboutblock}

\specifications{\styleczechpremiumpalelager}{\galtol{5}}{}{}{}{40}{}{60~min}{}

\begin{methodandtiming}

\begin{mashsteps}
\mashstep{\ftoc{154}}{60~min}
\end{mashsteps}

\begin{fermentationsteps}
\fermentationstep{\ftoc{50}}{}
\end{fermentationsteps}

\end{methodandtiming}

\recipebreak

\begin{ingredientsblock}

\begin{malts}
\malt{Pilsner}{\lbtokg{10}}
\malt{Carapils / Dextrin}{\oztokg{12}}
\malt{Weyermann Munich I}{\oztokg{12}}
\end{malts}

\begin{hops}
\hop{\hoptettnang}{}{60~min}{\oztog{2}}
\hop{\hopsaaz}{}{30~min}{\oztog{1}}
\hop{\hopsaaz}{}{10~min}{\oztog{1}}
\hop{\hopsaaz}{}{\foh{}{}}{\oztog{1}}
\end{hops}

\singleyeast{Wyeast 2001}

\end{ingredientsblock}

\end{recipe}

% -----------------------------------------------------------------------------
\begin{recipe}{Meister Groll Bohemian Pils} % rechecked
% -----------------------------------------------------------------------------

\begin{aboutblock}
Recipe by Jan Brücklmeier of Aurora, OH. \sourceaha
\end{aboutblock}

\specifications{\styleczechpremiumpalelager}{\galtol{5.28}}{1.048}{1.012}{4.8~\%}{40}{\srmtoebc{3}}{120~min}{}

\begin{methodandtiming}
 
\begin{mashsteps}
\mashstep{\ftoc{100}}{30~min}
\mashdecoctthick{with 1/3 of mash}
\mashstep{\ftoc{151}}{20~min}
\mashstep{\ftoc{162}}{10~min}
\mashdecoctboil{10~min}
\mashdecoctreturn{\ftoc{151}}{10~min}
\mashdecoctthick{with 1/3 of mash}
\mashstep{\ftoc{162}}{10~min}
\mashdecoctboil{10~min}
\mashdecoctreturn{\ftoc{162}}{10~min}
\mashdecoctthin{with 1/3 of mash}
\mashdecoctboil{10~min}
\mashdecoctreturn{\ftoc{172}}{10~min}
\end{mashsteps}

\begin{fermentationsteps}
\fermentationstep{\ftoc{52}}{Attenuation to \sgtop{1.016}}
\fermentationstep{\ftoc{52}}{Transfer to secondary; 7~days}
\fermentationstep{\ftoc{32}}{1~month}
\end{fermentationsteps}

\begin{directions}
Water adjustment: soft water like that in Pilsen, \waterprofile{7}{2}{2}{5}{5}{}.
Use a top pressure of \psitobar{16} in secondary.
\end{directions}

\end{methodandtiming}

\recipebreak

\begin{ingredientsblock}

\begin{malts}
\malt{Weyermann Bohemian Pilsner}{\lbtokg{9.2}}
\malt{Acidulated}{\oztog{8}}
\end{malts}

\begin{hops}
\hop{\hopsaaz}{3.5~\%}{110~min}{\oztog{0.7}}
\hop{\hopsaaz}{3.5~\%}{45~min}{\oztog{1.8}}
\hop{\hopsaaz}{3.5~\%}{15~min}{\oztog{0.9}}
\end{hops}

\singleyeast{Wyeast 2206}

\end{ingredientsblock}

\end{recipe}


% -----------------------------------------------------------------------------
\stylecategory{European Amber Lager}
\stylesection{\styleviennalager}

% -----------------------------------------------------------------------------
\begin{recipe}{American Vienna Lager} % checked
% -----------------------------------------------------------------------------

\begin{aboutblock}
Recipe by Kevin Lemme of Carmel, IN. Bronze medal in Category 4: Amber European
Beer during the 2019 National Homebrew Competition in Providence, RI. \sourceaha
\end{aboutblock}

\specifications{\styleviennalager}{\galtol{10}}{1.050}{1.012}{5.1~\%}{20}{\srmtoebc{10}}{90~min}{}

\begin{methodandtiming}
 
\begin{mashsteps}
\mashstep{\ftoc{125}}{30~min}
\mashstep{\ftoc{147}}{30~min}
\mashstep{\ftoc{162}}{30~min}
\end{mashsteps}

\begin{fermentationsteps}
\fermentationstep{\ftoc{52}}{5~days}
\fermentationstep{\ftoc{58}}{3~days}
\fermentationstep{\ftoc{62}}{3~days}
\fermentationstep{\ftoc{50}}{Reduce to by 1~°C/day; 3~months}
\end{fermentationsteps}

\end{methodandtiming}

\recipebreak

\begin{ingredientsblock}

\begin{malts}
\malt{Pilsner}{\lbtokg{8}}
\malt{Vienna}{\lbtokg{8}}
\malt{Weyermann Munich II}{\lbtokg{2.5}}
\malt{Weyermann CARAAMBER}{\lbtokg{2.5}}
\end{malts}

\begin{hops}
\hop{\hopnorthernbrewer}{}{60~min}{\oztog{7.5}}
\hop{\hopsaaz}{}{20~min}{\oztog{3.5}}
\end{hops}

\singleyeast{Imperial Yeast L17}

\end{ingredientsblock}

\end{recipe}

% -----------------------------------------------------------------------------
\begin{recipe}{Lovelace Vienna Lager} % checked
% -----------------------------------------------------------------------------

\begin{aboutblock}
Recipe by Robert Lovelace of Denver, NC. Gold medal in Category 4: Amber European
Beer during the 2018 National Homebrew Competition in Portland, OR. \sourceaha
\end{aboutblock}

\specifications{\styleviennalager}{\galtol{6.5}}{1.052}{1.013}{5.2~\%}{31}{\srmtoebc{28}}{60~min}{}

\begin{methodandtiming}
 
\begin{mashsteps}
\mashstep{\ftoc{144}}{30~min}
\mashstep{\ftoc{158}}{30~min}
\mashstep{\ftoc{170}}{10~min}
\end{mashsteps}

\begin{fermentationsteps}
\fermentationstep{\ftoc{48}}{10~days}
\fermentationstep{\ftoc{58}}{Raise to over 4~days}
\fermentationstep{\ftoc{36}}{14~days}
\end{fermentationsteps}

\begin{directions}
Water adjustment: use low-mineral target mesh pH of 5.2.
\end{directions}

\end{methodandtiming}

\recipebreak

\begin{ingredientsblock}

\begin{malts}
\malt{Weyermann Pilsner}{\lbtokg{6.5}}
\malt{Weyermann Vienna}{\lbtokg{5}}
\malt{Weyermann Munich II}{\lbtokg{1.5}}
\malt{Briess Victory}{\oztokg{8}}
\malt{Weyermann CARAAMBER}{\oztokg{8}}
\malt{Briess Blackprinz}{\lbtokg{0.15}}
\end{malts}

\begin{hops}
\hop{\hopmagnum}{14~\%}{60~min}{\oztog{0.63}}
\hop{\hoptettnang}{3.5~\%}{\foh{}}{\oztog{1}}
\end{hops}

\singleyeast{White Labs WLP830}

\end{ingredientsblock}

\end{recipe}

% -----------------------------------------------------------------------------
\begin{recipe}{Viva Vienna} % checked
% -----------------------------------------------------------------------------

\begin{aboutblock}
Recipe by Tim Gerbracht of Vienna, VA. Silver medal in Category 4: Amber European
Beer during the 2019 National Homebrew Competition in Providence, RI. \sourceaha
\end{aboutblock}

\specifications{\styleviennalager}{\galtol{5.5}}{1.048}{1.013}{4.7~\%}{22}{\srmtoebc{9}}{60~min}{}

\begin{methodandtiming}
 
\begin{mashsteps}
\mashstep{\ftoc{154}}{60~min}
\end{mashsteps}

\begin{fermentationsteps}
\fermentationstep{\ftoc{50}}{5~days}
\fermentationstep{\ftoc{65}}{Raise to by 2.5~°C/12~hours; 5~days}
\fermentationstep{\ftoc{34}}{Reduce to by 3~°C/12~hours; 5~days}
\end{fermentationsteps}

\begin{directions}
Water adjustment: target mesh pH of 5.3.
\end{directions}

\end{methodandtiming}

\recipebreak

\begin{ingredientsblock}

\begin{malts}
\malt{Avangard Vienna}{\lbtokg{8.5}}
\malt{Weyermann Barke Pilsner}{\lbtokg{3}}
\malt{Weyermann Melanoidin}{\lbtokg{0.25}}
\malt{Briess Midnight Wheat}{\oztog{1.5}}
\end{malts}

\begin{hops}
\hop{\hopmagnum}{12.5~\%}{60~min}{\oztog{0.5}}
\hop{\hopsaaz}{3.5~\%}{15~min}{\oztog{0.5}}
\end{hops}

\singleyeast{Imperial Yeast L13}

\end{ingredientsblock}

\end{recipe}

\stylesection{\styleczechamberlager}

% -----------------------------------------------------------------------------
\begin{recipe}{Homebrew Challenge Czech Amber Lager}
% -----------------------------------------------------------------------------

\begin{aboutblock}
Recipe by Martin Keen.
\sourcehomebrewchallenge
\end{aboutblock}

\specifications{\styleczechamberlager}{\galtol{5}}{}{}{}{}{}{60~min}{}

\begin{methodandtiming}

\begin{mashsteps}
\mashstep{\ftoc{152}}{60~min}
\end{mashsteps}

\begin{fermentationsteps}
\fermentationstep{\ftoc{50}}{}
\end{fermentationsteps}

\end{methodandtiming}

\recipebreak

\begin{ingredientsblock}

\begin{malts}
\malt{Weyermann Bohemian Pilsner}{\lbtokg{4}}
\malt{\maltmarisotter}{\lbtokg{3}}
\malt{Weyermann Munich I}{\lbtokg{2}}
\malt{Caramel / Crystal 80 L}{\oztokg{12}}
\malt{Aromatic}{\oztokg{8}}
\end{malts}

\begin{hops}
\hop{\hopsaaz}{}{60~min}{\oztog{2}}
\hop{\hopsaaz}{}{15~min}{\oztog{1}}
\end{hops}

\singleyeast{Wyeast 2124}

\end{ingredientsblock}

\end{recipe}

\stylesection{\stylemarzen}

% -----------------------------------------------------------------------------
\begin{recipe}{Pickelhaube Märzen} % rechecked
% -----------------------------------------------------------------------------

\begin{aboutblock}
Classic German lager dedicated to Fred Eckhardt.
\sourceaha
\end{aboutblock}

\specifications{\stylemarzen}{\galtol{5.5}}{1.056}{1.010}{6~\%}{23.5}{\srmtoebc{10}}{90~min}{}

\begin{methodandtiming}

\begin{mashsteps}
\mashstep{\ftoc{122}}{20~min}
\mashdecoctthick{with \qttol{9} of mash}
\mashdecoctboil{15~min}
\mashdecoctreturn{\ftoc{150}}{60~min}
\mashstep{\ftoc{168}}{10~min}
\end{mashsteps}

\begin{fermentationsteps}
\fermentationstep{\ftoc{48}}{Fermentation slowdown}
\fermentationstep{\ftoc{55}}{3~days}
\fermentationstep{\ftoc{35}}{Attenuation to \sgtop{1.012}; 3~months}
\end{fermentationsteps}

\begin{directions}
Water adjustment: reverse osmosis water with \gpgaltogpl{1} calcium chloride.
\end{directions}

\end{methodandtiming}

\recipebreak

\begin{ingredientsblock}

\begin{malts}
\malt{Vienna}{\lbtokg{5.5}}
\malt{Briess Bonlander Munich 10 L}{\lbtokg{3.5}}
\malt{Briess Aromatic Munich 20 L}{\lbtokg{2.5}}
\end{malts}

\begin{hops}
\hop{\hoptettnang}{4.5~\%}{90~min}{\oztog{1.25}}
\hop{\hoptettnang}{4.5~\%}{15~min}{\oztog{0.25}}
\end{hops}

\singleyeast{Bavarian Lager / Munich Lager}

\end{ingredientsblock}

\end{recipe}

\stylesection{\stylekellerbier}

% -----------------------------------------------------------------------------
\begin{recipe}{Homebrew Challenge Amber Kellerbier}
% -----------------------------------------------------------------------------

\begin{aboutblock}
Recipe by Martin Keen.
\sourcehomebrewchallenge
\end{aboutblock}

\specifications{\stylekellerbier}{\galtol{5}}{1.059}{1.013}{6~\%}{}{}{60~min}{}

\begin{methodandtiming}

\begin{mashsteps}
\mashstep{\ftoc{152}}{60~min}
\end{mashsteps}

\begin{fermentationsteps}
\fermentationstep{\ftoc{50}}{}
\end{fermentationsteps}

\begin{directions}
Age for 4 weeks.
\end{directions}

\end{methodandtiming}

\recipebreak

\begin{ingredientsblock}

\begin{malts}
\malt{\maltvienna}{\lbtokg{7}}
\malt{\maltpilsner}{\lboztokg{3}{8}}
\malt{\maltweyermanncarafatwo}{\oztokg{3}}
\malt{\maltweyermannmelanoidin}{\oztokg{3}}
\end{malts}

\begin{hops}
\hop{\hophallertaumittelfruh}{}{60~min}{\oztog{1}}
\hop{\hophallertaumittelfruh}{}{30~min}{\oztog{0.5}}
\hop{\hopnorthernbrewer}{}{30~min}{\oztog{0.5}}
\hop{\hophallertaumittelfruh}{}{10~min}{\oztog{0.5}}
\hop{\hopnorthernbrewer}{}{10~min}{\oztog{0.5}}
\end{hops}

\singleyeast{White Labs WLP820}

\end{ingredientsblock}

\end{recipe}

% -----------------------------------------------------------------------------
\begin{recipe}{Homebrew Challenge Pale Kellerbier}
% -----------------------------------------------------------------------------

\begin{aboutblock}
Recipe by Martin Keen.
\sourcehomebrewchallenge
\end{aboutblock}

\specifications{\stylekellerbier}{\galtol{5}}{1.049}{1.010}{5.1~\%}{}{}{60~min}{}

\begin{methodandtiming}

\begin{mashsteps}
\mashstep{\ftoc{152}}{60~min}
\end{mashsteps}

\begin{fermentationsteps}
\fermentationstep{\ftoc{50}}{}
\end{fermentationsteps}

\begin{directions}
Age for 3 weeks.
\end{directions}

\end{methodandtiming}

\recipebreak

\begin{ingredientsblock}

\begin{malts}
\malt{\maltpilsner}{\lbtokg{9}}
\malt{\maltcarapils}{\oztokg{8}}
\end{malts}

\begin{hops}
\hop{\hopperle}{}{60~min}{\oztog{1}}
\hop{\hophallertaumittelfruh}{}{15~min}{\oztog{0.5}}
\end{hops}

\singleyeast{White Labs WLP820}

\end{ingredientsblock}

\end{recipe}


% -----------------------------------------------------------------------------
\stylecategory{Dark Lager}
\stylesection{\styleinternationaldarklager}

% -----------------------------------------------------------------------------
\begin{recipe}{Homebrew Challenge International Dark Lager} % rechecked
% -----------------------------------------------------------------------------

\begin{aboutblock}
Recipe by Martin Keen.
\sourcehomebrewchallenge
\end{aboutblock}

\specifications{\styleinternationaldarklager}{\galtol{5}}{}{}{5.6~\%}{}{}{60~min}{}

\begin{methodandtiming}

\begin{mashsteps}
\mashstep{\ftoc{152}}{60~min}
\end{mashsteps}

\begin{fermentationsteps}
\fermentationstep{\ftoc{50}}{}
\end{fermentationsteps}

\end{methodandtiming}

\recipebreak

\begin{ingredientsblock}

\begin{malts}
\malt{Pilsner}{\lbtokg{5}}
\malt{Vienna}{\lbtokg{5}}
\malt{Weyermann CARAFA SPECIAL II}{\oztokg{8}}
\end{malts}

\begin{hops}
\hop{\hophallertaumittelfruh}{}{60~min}{\oztog{1}}
\end{hops}

\singleyeast{White Labs WLP830}

\end{ingredientsblock}

\end{recipe}

\stylesection{\stylemunichdunkel}

% -----------------------------------------------------------------------------
\begin{recipe}{Church Brew Works Pious Monk Dunkel Clone} % rechecked
% -----------------------------------------------------------------------------

\begin{aboutblock}
\sourceaha
\end{aboutblock}

\specifications{\stylemunichdunkel}{\galtol{5}}{1.054}{1.012}{5.5~\%}{26}{\srmtoebc{16.5}}{90~min}{}

\begin{methodandtiming}
 
\begin{mashsteps}
\mashstep{\ftoc{140}}{Start}
\mashdecoctthick{}
\mashdecoctboil{}
\mashdecoctreturn{\ftoc{154}}{}

\end{mashsteps}

\begin{fermentationsteps}
\fermentationstep{\ftoc{52}}{}
\fermentationstep{\ftoc{65}}{Diacetyl rest}
\fermentationstep{\ftoc{35}}{Reduction by 2.5~°C/day; 21~days}
\end{fermentationsteps}

\end{methodandtiming}

\recipebreak

\begin{ingredientsblock}

\begin{malts}
\malt{Pilsner}{\lbtokg{4.16}}
\malt{Munich}{\lbtokg{4.16}}
\malt{Melanoidin}{\oztog{12}}
\malt{Weyermann CARAPILS}{\oztog{10}}
\malt{Weyermann CARAAROMA}{\oztog{6}}
\malt{Weyermann CARAFA II}{\oztog{3}}
\end{malts}

\begin{hops}
\hop{\hopperle}{8.1~\%}{90~min}{\oztog{0.7}}
\hop{\hoptettnang}{4.7~\%}{20~min}{\oztog{0.7}}
\end{hops}

\singleyeast{Wyeast 2352}

\end{ingredientsblock}

\end{recipe}

\stylesection{\styleczechdarklager}

% -----------------------------------------------------------------------------
\begin{recipe}{Homebrew Challenge Czech Dark Lager}
% -----------------------------------------------------------------------------

\begin{aboutblock}
Recipe by Martin Keen.
\sourcehomebrewchallenge
\end{aboutblock}

\specifications{\styleczechdarklager}{\galtol{5}}{}{}{4~\%}{}{}{60~min}{}

\begin{methodandtiming}

\begin{mashsteps}
\mashstep{\ftoc{152}}{60~min}
\end{mashsteps}

\begin{fermentationsteps}
\fermentationstep{\ftoc{50}}{}
\end{fermentationsteps}

\end{methodandtiming}

\recipebreak

\begin{ingredientsblock}

\begin{malts}
\malt{Weyermann Bohemian Pilsner}{\lbtokg{6}}
\malt{Weyermann CARAMUNICH I}{\lbtokg{1}}
\malt{Briess Victory}{\lbtokg{1}}
\malt{\maltchocolate}{\oztokg{12}}
\end{malts}

\begin{hops}
\hop{\hopsaaz}{}{60~min}{\oztog{2}}
\hop{\hopsaaz}{}{15~min}{\oztog{1}}
\end{hops}

\singleyeast{White Labs WLP830}

\end{ingredientsblock}

\end{recipe}

\stylesection{\styleschwarzbier}

% -----------------------------------------------------------------------------
\begin{recipe}{May the Schwartz Be With You}
% -----------------------------------------------------------------------------

\begin{aboutblock}
Frank Taddeo of Royersford, PA, member of the Bruclear Homebrew Club, won a gold
medal in Category \#5: Dark European Lager with a Schwarzbier during the 2019
National Homebrew Competition Final Round in Providence, RI. Taddeo's Dark European
Lager was chosen as the best among 241 entries in the category. \sourceaha
\end{aboutblock}

\specifications{\styleschwarzbier}{\galtol{5.5}}{1.056}{1.015}{5.3~\%}{30}{\srmtoebc{28}}{90~min}{2}

\begin{methodandtiming}
 
\begin{mashsteps}
\mashstep{\ftoc{154}}{45~min}
\mashstep{\ftoc{170}}{Mashout}
\end{mashsteps}

\begin{fermentationsteps}
\fermentationstep{\ftoc{50}}{5~days}
\fermentationstep{\ftoc{65}}{Raise to over 2~days; 2~days}
\fermentationstep{\ftoc{50}}{Reduce to over 2~days}
\end{fermentationsteps}

\begin{directions}
Target mash pH of 5.3. Chill to \ftoc{170} before adding the whirlpool hops.
\end{directions}

\end{methodandtiming}

\recipebreak

\begin{ingredientsblock}

\begin{malts}
\malt{Dark Munich}{\lbtokg{6}}
\malt{Pilsner}{\lbtokg{6}}
\malt{Chocolate}{\oztog{6}}
\malt{Caramel / Crystal 40 L}{\oztog{6}}
\malt{Roast Barley}{\oztog{3.5}}
\malt{Weyerman CARAFA SPECIAL II}{\oztog{3.5}}
\malt{Acidulated}{\oztog{2}}
\end{malts}

\begin{hops}
\hop{\hophallertaumittelfruh}{4~\%}{60~min}{\oztog{1.4}}
\hop{\hophallertaumittelfruh}{4~\%}{20~min}{\oztog{0.5}}
\hop{\hophallertaumittelfruh}{4~\%}{\whirl{}{10~min}}{\oztog{0.5}}
\end{hops}

\singleyeast{Wyeast 2206}

\end{ingredientsblock}

\end{recipe}

% -----------------------------------------------------------------------------
\begin{recipe}{Red Rock Brewing Black Bier Clone}
% -----------------------------------------------------------------------------

\begin{aboutblock}
This beer recipe was taken from the book Session Beers: Brewing for Flavor and
Balance by Jennifer Talley. This German-style dark lager from Red Rock Brewery
is a classic schwarzbier that won a gold medal at the 2010 Great American Beer
Festival. It's unusually dark color is gained from four different malts and very
light hopping that make it a sessionable black lager. \sourceaha
\end{aboutblock}

\specifications{\styleschwarzbier}{\galtol{5}}{1.040}{1.009}{4~\%}{28}{}{60~min}{2.5}

\begin{methodandtiming}
 
\begin{mashsteps}
\mashstep{\ftoc{148}}{}
\end{mashsteps}

\begin{fermentationsteps}
\fermentationstep{\ftoc{48}}{}
\end{fermentationsteps}

\begin{directions}
Use reverse osmosis water pH with lactic acid, no calcium needed.
Age in the coldest place you have, \ftoc{33}--\ftoc{36} is optimal. Since it's
only 4~\% ABV do not cool below \ftoc{33}.
\end{directions}

\end{methodandtiming}

\recipebreak

\begin{ingredientsblock}

\begin{malts}
\malt{Pilsner}{\lbtokg{7}}
\malt{Munich}{\oztokg{12}}
\malt{Dehusked Black}{\oztokg{5}}
\malt{Roast}{\oztokg{1.5}}
\end{malts}

\begin{hops}
\hop{\hoptettnang}{4~\%}{60~min}{\oztog{0.75}}
\hop{\hoptettnang}{4~\%}{30~min}{\oztog{0.75}}
\hop{\hophallertaumittelfruh}{4~\%}{\whirl{}{}}{\oztog{1.5}}
\end{hops}

\singleyeast{Wyeast 2124}

\end{ingredientsblock}

\end{recipe}


% -----------------------------------------------------------------------------
\stylecategory{Bock}
\stylesection{\stylehellesbock}

% -----------------------------------------------------------------------------
\begin{recipe}{Homebrew Challenge Helles Bock} % rechecked
% -----------------------------------------------------------------------------

\begin{aboutblock}
Recipe by Martin Keen.
\sourcehomebrewchallenge
\end{aboutblock}

\specifications{\stylehellesbock}{\galtol{5}}{}{}{7.5~\%}{}{}{60~min}{}

\begin{methodandtiming}

\begin{mashsteps}
\mashstep{\ftoc{152}}{60~min}
\end{mashsteps}

\begin{fermentationsteps}
\fermentationstep{\ftoc{50}}{}
\end{fermentationsteps}

\end{methodandtiming}

\recipebreak

\begin{ingredientsblock}

\begin{malts}
\malt{Pilsner}{\lbtokg{6}}
\malt{Maris Otter}{\lbtokg{5}}
\malt{Munich}{\lbtokg{3}}
\end{malts}

\begin{hops}
\hop{\hophallertaumittelfruh}{}{60~min}{\oztog{2}}
\hop{\hophallertaumittelfruh}{}{15~min}{\oztog{1}}
\end{hops}

\singleyeast{White Labs WLP830}

\end{ingredientsblock}

\end{recipe}

% -----------------------------------------------------------------------------
\begin{recipe}{Malty Boye Helles Bock} % rechecked
% -----------------------------------------------------------------------------

\begin{aboutblock}
Recipe by Derek Springer of San Marcos, CA. Bronze medal in Category 6: Strong
European Lager during the 2019 National Homebrew Competition in Providence, RI.
\sourceaha
\end{aboutblock}

\specifications{\stylehellesbock}{\galtol{5.5}}{1.068}{1.016}{7.4~\%}{32.4}{\srmtoebc{6.4}}{60~min}{}

\begin{methodandtiming}
 
\begin{mashsteps}
\mashstep{\ftoc{145}}{40~min}
\mashstep{\ftoc{158}}{20~min}
\mashstep{\ftoc{168}}{10~min}
\end{mashsteps}

\begin{fermentationsteps}
\fermentationstep{\ftoc{50}}{2~days}
\fermentationstep{\ftoc{55}}{2~days}
\fermentationstep{\ftoc{60}}{2~days}
\fermentationstep{\ftoc{65}}{6~days}
\fermentationstep{\ftoc{32}}{2~days}
\end{fermentationsteps}

\end{methodandtiming}

\recipebreak

\begin{ingredientsblock}

\begin{malts}
\malt{Weyermann Pilsner}{\lbtokg{8}}
\malt{Weyermann Vienna}{\lbtokg{3}}
\malt{Weyermann Munich II}{\lbtokg{1.5}}
\malt{Dingemans Biscuit}{\oztokg{4}}
\malt{Weyermann Melanoidin}{\oztokg{4}}
\end{malts}

\begin{hops}
\hop{\hopsterling}{11.3~\%}{60~min}{\oztog{0.5}}
\hop{\hopsterling}{11.3~\%}{15~min}{\oztog{0.5}}
\hop{\hopsterling}{11.3~\%}{5~min}{\oztog{0.5}}
\end{hops}

\singleyeast{White Labs WLP830}

\end{ingredientsblock}

\end{recipe}

\stylesection{\styledunklesbock}

% -----------------------------------------------------------------------------
\begin{recipe}{Homebrew Challenge Dunkles Bock} % rechecked
% -----------------------------------------------------------------------------

\begin{aboutblock}
Recipe by Martin Keen.
\sourcehomebrewchallenge
\end{aboutblock}

\specifications{\styledunklesbock}{\galtol{5}}{}{}{7~\%}{22}{}{60~min}{}

\begin{methodandtiming}

\begin{mashsteps}
\mashstep{\ftoc{152}}{60~min}
\end{mashsteps}

\begin{fermentationsteps}
\fermentationstep{\ftoc{50}}{}
\end{fermentationsteps}

\end{methodandtiming}

\recipebreak

\begin{ingredientsblock}

\begin{malts}
\malt{Weyermann Munich I}{\lbtokg{10}}
\malt{Weyermann Bohemian Pilsner}{\lbtokg{3}}
\malt{Weyermann CARAMUNICH III}{\oztokg{12}}
\malt{Weyermann CARAFA SPECIAL II}{\oztokg{2}}
\end{malts}

\begin{hops}
\hop{\hopperle}{}{60~min}{\oztog{1}}
\end{hops}

\singleyeast{White Labs WLP830}

\end{ingredientsblock}

\end{recipe}

\stylesection{\styledoppelbock}

% -----------------------------------------------------------------------------
\begin{recipe}{Big \& Beautiful Bock}
% -----------------------------------------------------------------------------

\begin{aboutblock}
Recipe by Hank Keller of Cypress, TX. Silver medal in Category 6: Strong European
Lager during the 2019 National Homebrew Competition in Providence, RI.
\sourceaha
\end{aboutblock}

\specifications{\styledoppelbock}{\galtol{6}}{1.110}{1.032}{10.2~\%}{18}{}{60~min}{}

\begin{methodandtiming}
 
\begin{mashsteps}
\mashstep{\ftoc{154}}{}
\end{mashsteps}

\begin{fermentationsteps}
\fermentationstep{\ftoc{50}}{14~days}
\fermentationstep{\ftoc{65}}{3~days}
\fermentationstep{\ftoc{40}}{Slow reduction}
\end{fermentationsteps}

\begin{directions}
Caramelized wort: boil \galtol{2} of first runnings until reduced to \galtol{0.5}.
\end{directions}

\end{methodandtiming}

\recipebreak

\begin{ingredientsblock}

\begin{malts}
\malt{Weyermann Munich II}{\lbtokg{11}}
\malt{\maltpilsner}{\lbtokg{6}}
\malt{Weyermann CARAMUNICH I}{\lbtokg{2}}
\malt{Melanoidin}{\lbtokg{0.5}}
\end{malts}

\begin{hops}
\hop{\hopmagnum}{12.4~\%}{60~min}{\oztog{0.83}}
\hop{Caramelized Wort}{}{10~min}{\galtol{0.5}}
\end{hops}

\singleyeast{Wyeast 2206}

\end{ingredientsblock}

\end{recipe}

% -----------------------------------------------------------------------------
\begin{recipe}{Homebrew Challenge Doppelbock}
% -----------------------------------------------------------------------------

\begin{aboutblock}
Recipe by Martin Keen.
\sourcehomebrewchallenge
\end{aboutblock}

\specifications{\styledoppelbock}{\galtol{5}}{1.085}{1.020}{8.5~\%}{}{}{60~min}{}

\begin{methodandtiming}

\begin{mashsteps}
\mashstep{\ftoc{152}}{60~min}
\end{mashsteps}

\begin{fermentationsteps}
\fermentationstep{\ftoc{50}}{}
\end{fermentationsteps}

\begin{directions}
Age for 6 weeks.
\end{directions}

\end{methodandtiming}

\recipebreak

\begin{ingredientsblock}

\begin{malts}
\malt{Weyermann Munich II}{\lbtokg{12}}
\malt{\maltpilsner}{\lbtokg{2}}
\malt{Weyermann CARAMUNICH III}{\lbtokg{1}}
\end{malts}

\begin{hops}
\hop{\hopnorthernbrewer}{}{60~min}{\oztog{0.75}}
\hop{\hoptettnang}{}{20~min}{\oztog{0.5}}
\end{hops}

\singleyeast{Wyeast 2308}

\end{ingredientsblock}

\end{recipe}

\stylesection{\styleeisbock}

% -----------------------------------------------------------------------------
\begin{recipe}{Homebrew Challenge Eisbock} % rechecked
% -----------------------------------------------------------------------------

\begin{aboutblock}
Recipe by Martin Keen.
\sourcehomebrewchallenge
\end{aboutblock}

\specifications{\styleeisbock}{\galtol{5}}{1.086}{1.020}{11~\%}{30}{}{60~min}{}

\begin{methodandtiming}

\begin{mashsteps}
\mashstep{\ftoc{152}}{60~min}
\end{mashsteps}

\begin{fermentationsteps}
\fermentationstep{\ftoc{50}}{}
\end{fermentationsteps}

\begin{directions}
Age for 7.5 weeks. Store at \ftoc{-3} for 10 hours in a keg. Then separate the
remaining liquid from the ice.
\end{directions}

\end{methodandtiming}

\recipebreak

\begin{ingredientsblock}

\begin{malts}
\malt{Pilsner}{\lbtokg{8}}
\malt{Weyermann Munich II}{\lbtokg{7}}
\malt{Weyermann CARAMUNICH II}{\lbtokg{1}}
\malt{Weyermann CARAFA II}{\oztokg{4}}
\end{malts}

\begin{hops}
\hop{\hopperle}{}{60~min}{\oztog{1.25}}
\hop{\hoptettnang}{}{20~min}{\oztog{0.5}}
\end{hops}

\singleyeast{White Labs WLP830}

\end{ingredientsblock}

\end{recipe}


% -----------------------------------------------------------------------------
\stylecategory{Light Hybrid Beer}
\stylesection{\stylecreamale}

% -----------------------------------------------------------------------------
\begin{recipe}{\$12 Cream Ale} % rechecked
% -----------------------------------------------------------------------------

\begin{aboutblock}
Recipe by Nishan Derbabian of Murfreesboro, TN. Gold medal in Category 6: Light
Hybrid Beer during the 2011 National Homebrew Competition in San Diego, CA.
\sourceaha
\end{aboutblock}

\specifications{\stylecreamale}{\galtol{5.5}}{1.054}{1.015}{5.12~\%}{20.7}{\srmtoebc{3.6}}{60~min}{2.6}

\begin{methodandtiming}

\begin{mashsteps}
\mashstep{\ftoc{154}}{60~min}
\mashstep{\ftoc{168}}{10~min}
\end{mashsteps}

\begin{fermentationsteps}
\fermentationstep{\ftoc{60}}{21~days}
\end{fermentationsteps}

\end{methodandtiming}

\recipebreak

\begin{ingredientsblock}

\begin{malts}
\malt{Briess Brewers}{\lbtokg{11}}
\end{malts}

\begin{hops}
\hop{\hopwillamette}{4.7~\%}{60~min}{\oztog{0.75}}
\hop{\hopwillamette}{4.7~\%}{30~min}{\oztog{0.5}}
\hop{\hopwillamette}{4.7~\%}{\foh{}{}}{\oztog{0.25}}
\end{hops}

\singleyeast{White Labs WLP051}

\end{ingredientsblock}

\end{recipe}

% -----------------------------------------------------------------------------
\begin{recipe}{Aspen Ridge Cream Ale} % rechecked
% -----------------------------------------------------------------------------

\begin{aboutblock}
Recipe by Aspen Ridge Brew Crew. \sourcezymurgy{May / June 2019}
\end{aboutblock}

\specifications{\stylecreamale}{\galtol{10}}{1.051}{1.011}{5.4~\%}{20}{\srmtoebc{4}}{60~min}{2.3}

\begin{methodandtiming}
 
\begin{mashsteps}
\mashstep{\ftoc{152}}{75~min}
\mashstep{\ftoc{168}}{Mash out}
\end{mashsteps}

\begin{fermentationsteps}
\fermentationstep{\ftoc{67}}{}
\end{fermentationsteps}

\end{methodandtiming}

\recipebreak

\begin{ingredientsblock}
  
\begin{malts}
\malt{Pale}{\lbtokg{7}}
\malt{Pilsner}{\lbtokg{7}}
\malt{Carapils / Dextrin}{\lbtokg{2}}
\malt{Flaked Maize}{\lbtokg{2}}
\malt{Flaked Rice}{\lbtokg{1.5}}
\malt{Gambrinus Honey}{\oztog{12}}
\end{malts}

\begin{hops}
\hop{\hopliberty}{4.3~\%}{60~min}{\oztog{1}}
\hop{\hopliberty}{4.3~\%}{30~min}{\oztog{1}}
\hop{\hopmthood}{6~\%}{5~min}{\oztog{2}}
\end{hops}

\singleyeast{Wyeast 2565}

\end{ingredientsblock}

\end{recipe}

% -----------------------------------------------------------------------------
\begin{recipe}{BlytheStone Light Cream Ale} % rechecked
% -----------------------------------------------------------------------------

\begin{aboutblock}
Recipe by Kevin Nanzer of Sacramento, CA. Bronze medal in Category 1: Pale American
Lager during the 2018 National Homebrew Competition in Portland, OR. \sourceaha
\end{aboutblock}

\specifications{\stylecreamale}{\galtol{20}}{1.051}{1.009}{5.5~\%}{12.8}{\srmtoebc{3.6}}{90~min}{2.5}

\begin{methodandtiming}

\begin{mashsteps}
\mashstep{\ftoc{152}}{60~min}
\end{mashsteps}

\begin{fermentationsteps}
\fermentationstep{\ftoc{72}}{10~days}
\fermentationstep{\ftoc{33}}{21~days}
\end{fermentationsteps}

\begin{directions}
Water adjustment: use carbon filtered Sacramento City tap water with
\gtog{10} calcium sulfate, \gtog{17} calcium chloride and \mltoml{23}
lactic acid in the mash.
\end{directions}

\end{methodandtiming}

\recipebreak

\begin{ingredientsblock}

\begin{malts}
\malt{Briess Pilsen}{\lbtokg{24}}
\malt{Rahr Pale Ale}{\lbtokg{14}}
\malt{Flaked Rice}{\lbtokg{5}}
\malt{Flaked Oats}{\lbtokg{2}}
\malt{Flaked Maize}{\lbtokg{1}}
\malt{Melanoidin}{\oztog{4}}
\end{malts}

\begin{hops}
\hop{\hopmagnum}{11.9~\%}{60~min}{\oztog{1}}
\hop{\hophallertaumittelfruh}{3.8~\%}{20~min}{\oztog{2}}
\end{hops}

\singleyeast{Fermentis SafLager W-34/70}

\end{ingredientsblock}

\end{recipe}

% -----------------------------------------------------------------------------
\begin{recipe}{Bosmo's Imperial Cream Ale} % rechecked
% -----------------------------------------------------------------------------

\begin{aboutblock}
Recipe by Rick Debar of the Barrel House Brewing Company and Ray Snyder of the
Bloatarian Brewing League.
\sourcezymurgy{May / June 2008}
\end{aboutblock}

\specifications{\stylecreamale}{\galtol{5}}{1.071}{1.014}{7.45~\%}{37}{\srmtoebc{3.8}}{90~min}{}

\begin{methodandtiming}

\begin{mashsteps}
\mashstep{\ftoc{121}}{30~min}
\mashstep{\ftoc{145}}{45~min}
\mashstep{\ftoc{158}}{60~min}
\mashstep{\ftoc{168}}{Mash out}
\end{mashsteps}

\begin{fermentationsteps}
\fermentationstep{\ftoc{61}}{4~days}
\fermentationstep{\ftoc{45}}{Transfer to secondary; 21~days}
\end{fermentationsteps}

\begin{directions}
Bottle conditioning for a minimum of 14 days.
\end{directions}

\end{methodandtiming}

\recipebreak

\begin{ingredientsblock}

\begin{malts}
\malt{Two-row}{\lbtokg{9.75}}
\malt{Flaked Maize}{\lbtokg{1.25}}
\malt{Briess Carapils}{\lbtokg{0.6}}
\end{malts}

\begin{hops}
\hop{Corn Sugar}{}{90~min}{\lbtokg{1.25}}
\hop{\hophallertaumittelfruh}{4.5~\%}{45~min}{\oztog{0.75}}
\hop{\hopwillamette}{5.5~\%}{45~min}{\oztog{0.9}}
\hop{\hophallertaumittelfruh}{4.5~\%}{5~min}{\oztog{1.1}}
\end{hops}

\singleyeast{White Labs WLP001}

\end{ingredientsblock}

\end{recipe}

% -----------------------------------------------------------------------------
\begin{recipe}{"Cream Corn" Cream Ale} % rechecked
% -----------------------------------------------------------------------------

\begin{aboutblock}
Recipe by Richard Romanko of East Pittsburgh, PA. Silver medal in Category
1: Pale American Beer during the 2019 National Homebrew Competition
in Providence, RI. \sourceaha
\end{aboutblock}
 
\specifications{\stylecreamale}{\galtol{11}}{1.045}{1.004}{5.38~\%}{12.7}{\srmtoebc{2.9}}{60~min}{}

\begin{methodandtiming}
 
\begin{mashsteps}
\mashstep{\ftoc{145}}{75~min}
\end{mashsteps}

\begin{directions}
Water adjustment: reverse osmosis water with \gtog{7.5} of calcium sulfate
and \gtog{4.5} of calcium chloride; mash pH of 5.35. Cold crash and age for
a couple of weeks prior to serving.
\end{directions}

\end{methodandtiming}

\recipebreak

\begin{ingredientsblock}

\begin{malts}
\malt{BEST Heidelberg}{\lbtokg{7}}
\malt{Pale}{\lbtokg{6}}
\malt{Flaked Maize}{\lbtokg{4}}
\malt{Flaked Barley}{\lbtokg{1}}
\malt{Acidulated}{\oztog{5}}
\end{malts}

\begin{hops}
\hop{\hopsterling}{11.6~\%}{60~min}{\oztog{0.6}}
\end{hops}

\singleyeast{White Labs WLP080}

\end{ingredientsblock}

\end{recipe}

% -----------------------------------------------------------------------------
\begin{recipe}{Cream of the Crop} % rechecked
% -----------------------------------------------------------------------------

\begin{aboutblock}
Recipe by Thomas Wallace of Stanardsville, VA. Gold medal in Category 6: Light
Hybrid Beer during the 2015 National Homebrew Competition in San Diego, CA.
\sourceaha
\end{aboutblock}

\specifications{\stylecreamale}{\galtol{5}}{1.057}{1.008}{6.43~\%}{}{\srmtoebc{3.5}}{60~min}{2.5}

\begin{methodandtiming}

\begin{mashsteps}
\mashstep{\ftoc{149}}{60~min}
\end{mashsteps}

\begin{fermentationsteps}
\fermentationstep{\ftoc{65}}{14~days}
\end{fermentationsteps}

\end{methodandtiming}

\recipebreak

\begin{ingredientsblock}

\begin{malts}
\malt{Pilsner}{\lbtokg{4.75}}
\malt{Two-row}{\lbtokg{4.75}}
\malt{Flaked Maize}{\lbtokg{1}}
\end{malts}

\begin{hops}
\hop{Sucrose}{}{}{\lbtokg{0.75}}
\hop{\hophallertaumittelfruh}{2.7~\%}{60~min}{\oztog{2}}
\hop{\hophallertaumittelfruh}{2.7~\%}{1~min}{\oztog{0.5}}
\end{hops}

\singleyeast{Fermentis SafAle US-05}

\end{ingredientsblock}

\end{recipe}

% -----------------------------------------------------------------------------
\begin{recipe}{Kari's Cream Ale} % rechecked
% -----------------------------------------------------------------------------

\begin{aboutblock}
Recipe by David Anderson of Northglenn, CO. Gold medal in Category 6: Light
Hybrid Beer during the 2008 National Homebrew Competition in Cincinnati, OH.
\sourceaha
\end{aboutblock}

\specifications{\stylecreamale}{\galtol{11}}{1.051}{1.008}{5.64~\%}{}{}{60~min}{}

\begin{methodandtiming}

\begin{mashsteps}
\mashstep{\ftoc{149}}{60~min}
\end{mashsteps}

\begin{fermentationsteps}
\fermentationstep{\ftoc{64}}{12~days}
\end{fermentationsteps}

\end{methodandtiming}

\recipebreak

\begin{ingredientsblock}

\begin{malts}
\malt{Pilsner}{\lbtokg{8}}
\malt{Two-row}{\lbtokg{8}}
\malt{Flaked Maize}{\lbtokg{1}}
\end{malts}

\begin{hops}
\hop{Cane Sugar}{}{}{\lbtokg{1}}
\hop{\hophallertaumittelfruh}{3.6~\%}{60~min}{\oztog{2}}
\hop{\hophallertaumittelfruh}{3.6~\%}{1~min}{\oztog{1}}
\end{hops}

\singleyeast{White Labs WLP001}

\end{ingredientsblock}

\end{recipe}

% -----------------------------------------------------------------------------
\begin{recipe}{Lao Kang's Cream Ale} % rechecked
% -----------------------------------------------------------------------------

\begin{aboutblock}
Recipe by Chris "Pacman" Ingermann of Muncie, IN. Gold medal in the 2002 National
Homebrew Competition. Slightly more elevated hop flavor than traditional cream ales,
with some fruitiness and sweet corn notes. \sourceaha
\end{aboutblock}

\specifications{\stylecreamale}{\galtol{12.5}}{1.053}{1.015}{4.9~\%}{}{}{70~min}{2.1}

\begin{methodandtiming}

\begin{mashsteps}
\mashstep{\ftoc{122}}{20~min}
\mashstep{\ftoc{134}}{20~min}
\mashstep{\ftoc{154}}{60~min}
\mashstep{\ftoc{168}}{20~min}
\end{mashsteps}

\end{methodandtiming}

\recipebreak

\begin{ingredientsblock}

\begin{malts}
\malt{Six-row}{\lbtokg{16}}
\malt{Flaked Maize}{\lbtokg{4}}
\malt{Gambrinus Honey}{\lbtokg{0.5}}
\end{malts}

\begin{hops}
\hop{\hopliberty}{4.8~\%}{60~min}{\oztog{3}}
\hop{\hopliberty}{4.8~\%}{5~min}{\oztog{1}}
\end{hops}

\singleyeast{White Labs WLP810}

\end{ingredientsblock}

\end{recipe}

% -----------------------------------------------------------------------------
\begin{recipe}{Roadhouse Brewing Co. Family Vacation Cream Ale Clone} % rechecked
% -----------------------------------------------------------------------------

\begin{aboutblock}
\sourceaha
\end{aboutblock}

\specifications{\stylecreamale}{\galtol{15}}{1.044}{1.007}{4.9~\%}{11}{\srmtoebc{3}}{60~min}{2.5}

\begin{methodandtiming}

\begin{mashsteps}
\mashstep{\ftoc{152}}{60~min}
\end{mashsteps}

\begin{fermentationsteps}
\fermentationstep{\ftoc{69}}{Full attenuation}
\fermentationstep{\ftoc{32}}{7~days}
\end{fermentationsteps}

\end{methodandtiming}

\recipebreak

\begin{ingredientsblock}

\begin{malts}
\malt{Pilsner}{\lbtokg{26}}
\malt{Flaked Barley}{\lbtokg{3}}
\end{malts}

\begin{hops}
\hop{\hopsaaz}{3.4~\%}{60~min}{\oztog{1}}
\hop{\hopsaaz}{3.4~\%}{30~min}{\oztog{1}}
\hop{\hopcomet}{9.5~\%}{5~min}{\oztog{1}}
\hop{\hopsaaz}{3.4~\%}{5~min}{\oztog{1}}
\end{hops}

\singleyeast{American Ale}

\end{ingredientsblock}

\end{recipe}

% -----------------------------------------------------------------------------
\begin{recipe}{Ted's Cream Ale} % rechecked
% -----------------------------------------------------------------------------

\begin{aboutblock}
Recipe by Ted Hausotter of Baker City, OR. Gold medal in Category 6: Light
Hybrid Beer during the 2006 National Homebrew Competition in Orlando, FL.
\sourceaha
\end{aboutblock}

\specifications{\stylecreamale}{\galtol{17}}{1.051}{1.010}{5.51~\%}{}{}{60~min}{}

\begin{methodandtiming}

\begin{mashsteps}
\mashstep{\ftoc{153}}{60~min}
\end{mashsteps}

\begin{fermentationsteps}
\fermentationstep{\ftoc{55}}{1~month}
\fermentationstep{\ftoc{32}}{Transfer to secondary; 4~months}
\end{fermentationsteps}

\end{methodandtiming}

\recipebreak

\begin{ingredientsblock}

\begin{malts}
\malt{Pilsner}{\lbtokg{25}}
\malt{Flaked Maize}{\lbtokg{2}}
\end{malts}

\begin{hops}
\hop{\hophallertaumittelfruh}{}{60~min}{\oztog{3.7}}
\hop{\hophallertaumittelfruh}{}{30~min}{\oztog{2.9}}
\end{hops}

\singleyeast{Wyeast 2112}

\end{ingredientsblock}

\end{recipe}

\stylesection{\styleblondeale}

% -----------------------------------------------------------------------------
\begin{recipe}{All-American Blonde Ale}
% -----------------------------------------------------------------------------

\begin{aboutblock}
Recipe Aspen Ridge Brew Crew. \sourceaha
\end{aboutblock}

\specifications{\styleblondeale}{\galtol{11}}{1.048}{1.012}{4.6~\%}{25}{\srmtoebc{3}}{60~min}{2.3}

\begin{methodandtiming}
 
\begin{mashsteps}
\mashstep{\ftoc{150}}{75~min}
\mashstep{\ftoc{168}}{Mashout}
\end{mashsteps}

\begin{fermentationsteps}
\fermentationstep{64}{}
\end{fermentationsteps}

\end{methodandtiming}

\recipebreak

\begin{ingredientsblock}

\begin{malts}
\malt{Pale}{\lbtokg{20}}
\end{malts}

\begin{hops}
\hop{\hopcascade}{5.5~\%}{60~min}{\oztog{2}}
\hop{\hopcascade}{5.5~\%}{15~min}{\oztog{1}}
\hop{Whirlfloc Tablet}{}{15~min}{2}
\hop{\hopcascade}{5.5~\%}{2~min}{\oztog{2}}
\end{hops}

\singleyeast{Wyeast 1099}

\begin{twists}
\twist{Yeast Nutrient}{Primary}{\tsptog{2}}
\end{twists}

\end{ingredientsblock}

\end{recipe}


% -----------------------------------------------------------------------------
\begin{recipe}{Trans-Atlantic Blonde Ale}
% -----------------------------------------------------------------------------

\begin{aboutblock}
This beer recipe is featured in Simple Homebrewing by Denny Conn and Drew Beechum.
\sourceaha
\end{aboutblock}

\specifications{\styleblondeale}{\galtol{5.5}}{1.048}{1.011}{4.8~\%}{24}{\srmtoebc{4}}{60~min}{2.5}

\begin{methodandtiming}
 
\begin{mashsteps}
\mashstep{\ftoc{154}}{60~min}
\mashstep{\ftoc{168}}{Mashout}
\end{mashsteps}

\begin{fermentationsteps}
\fermentationstep{\ftoc{62}}{1~week}
\end{fermentationsteps}

\begin{directions}
Optional: transfer to secondary fermenter and age for 10--14 days at
\ftoc{65}. For 7~\% ABV \lbtokg{1.5} of dextrose could be added during
the boil.
\end{directions}

\end{methodandtiming}

\recipebreak

\begin{ingredientsblock}

\begin{malts}
\malt{Pilsner}{\lbtokg{9}}
\malt{Carapils / Dextrin}{\lbtokg{1}}
\end{malts}

\begin{hops}
\hop{\hopmagnum}{12~\%}{60~min}{\oztog{0.5}}
\hop{\hopwillamette}{5.5~\%}{\whirl{}{20~min}}{\oztog{0.5}}
\end{hops}

\singleyeast{Wyeast 1272 / Wyeast 1450 / Wyeast 1214 / White Labs WLP550}

\end{ingredientsblock}

\end{recipe}

\stylesection{\stylekolsch}

% -----------------------------------------------------------------------------
\begin{recipe}{Helios Kölsch} % rechecked
% -----------------------------------------------------------------------------

\begin{aboutblock}
Recipe by Heath Haynes of Los Angeles, CA. Gold medal in Category 6: Light Hybrid
Beer during the 2010 National Homebrew Competition in Minneapolis.
\sourceaha
\end{aboutblock}

\specifications{\stylekolsch}{\galtol{5.5}}{1.047}{1.010}{4.82~\%}{21}{\srmtoebc{3.5}}{60~min}{2.4}

\begin{methodandtiming}

\begin{mashsteps}
\mashstep{\ftoc{150}}{75~min}
\mashstep{\ftoc{168}}{10~min}
\end{mashsteps}

\begin{fermentationsteps}
\fermentationstep{\ftoc{60}}{15~days}
\fermentationstep{\ftoc{50}}{50~days}
\fermentationstep{\ftoc{37}}{1~month}
\end{fermentationsteps}

\begin{directions}
Water adjustment: 50~\% West Los Angeles and 50~\% distilled water.
\end{directions}

\end{methodandtiming}

\recipebreak

\begin{ingredientsblock}

\begin{malts}
\malt{Pilsner}{\lbtokg{9.2}}
\malt{Wheat}{\oztog{8}}
\end{malts}

\begin{hops}
\hop{\hophersbrucker}{3~\%}{60~min}{\oztog{2}}
\end{hops}

\singleyeast{Wyeast 2565}

\end{ingredientsblock}

\end{recipe}

% -----------------------------------------------------------------------------
\begin{recipe}{Joe's Kolsch} % rechecked
% -----------------------------------------------------------------------------

\begin{aboutblock}
Recipe by Joe Edidin of Maryville, TN. Gold medal in Category 6: Light Hybrid
Beer during the 2016 National Homebrew Competition in Baltimore, MD.
\sourceaha
\end{aboutblock}

\specifications{\stylekolsch}{\galtol{5}}{1.048}{1.009}{5.12~\%}{}{}{90~min}{2.5}

\begin{methodandtiming}

\begin{mashsteps}
\mashstep{\ftoc{148}}{60~min}
\mashstep{\ftoc{164}}{10~min}
\end{mashsteps}

\begin{fermentationsteps}
\fermentationstep{\ftoc{54}}{14~days}
\end{fermentationsteps}

\end{methodandtiming}

\recipebreak

\begin{ingredientsblock}

\begin{malts}
\malt{Briess Pilsen}{\lbtokg{8}}
\malt{Weyermann Vienna}{\lbtokg{1}}
\malt{Weyermann CARAPILS}{\lbtokg{0.5}}
\malt{Weyermann Acidulated}{\lbtokg{0.25}}
\end{malts}

\begin{hops}
\hop{\hoptettnang}{4.6~\%}{60~min}{\oztog{1.4}}
\hop{\hoptettnang}{4.6~\%}{15~min}{\oztog{0.4}}
\hop{\hoptettnang}{4.6~\%}{1~min}{\oztog{0.4}}
\end{hops}

\singleyeast{White Labs WLP830}

\end{ingredientsblock}

\end{recipe}

% -----------------------------------------------------------------------------
\begin{recipe}{Ramon's Kölsch} % rechecked
% -----------------------------------------------------------------------------

\begin{aboutblock}
Recipe Ramon Astamendi of San Diego, CA. Gold medal in Category 6: Light Hybrid
Beer during the 2007 National Homebrew Competition in Denver, CO.
\sourceaha
\end{aboutblock}

\specifications{\stylekolsch}{\galtol{17}}{1.047}{1.008}{5.12~\%}{}{}{60~min}{2.5}

\begin{methodandtiming}

\begin{mashsteps}
\mashstep{\ftoc{148}}{60~min}
\end{mashsteps}

\begin{fermentationsteps}
\fermentationstep{\ftoc{62}}{14~days}
\fermentationstep{\ftoc{28}}{28~days}
\end{fermentationsteps}

\end{methodandtiming}

\recipebreak

\begin{ingredientsblock}

\begin{malts}
\malt{Weyermann Pilsner}{\lbtokg{25}}
\malt{White Wheat}{\lbtokg{8}}
\malt{Vienna}{\lbtokg{2}}
\end{malts}

\begin{hops}
\hop{\hophallertaumittelfruh}{3.8~\%}{60~min}{\oztog{3.5}}
\hop{\hophersbrucker}{4~\%}{30~min}{\oztog{0.5}}
\hop{\hoptettnang}{4~\%}{30~min}{\oztog{0.5}}
\end{hops}

\singleyeast{White Labs WLP029}

\end{ingredientsblock}

\end{recipe}

\stylesection{\styleamericanwheat}

% -----------------------------------------------------------------------------
\begin{recipe}{Klosterman}
% -----------------------------------------------------------------------------

\begin{aboutblock}
This recipe is courtesy of Fretboard Brewing Co., Cincinnati, Ohio. 
Jim Klosterman is "Lead Guitar" at Fretboard Brewing Co. His last name is attached
to this beer thanks to Fretboard's collaboration with Klosterman Baking Co., a
Cincinnati bakery that supplies bread and buns across the United States. Fretboard
collaborates on two styles with Klosterman, a Rye Lager with caraway seed and the
Honey Wheat Lager below. \sourcezymurgy{November / December 2019}
\end{aboutblock}

\specifications{\styleamericanwheat}{\galtol{5.75}}{1.050}{1.009}{5.2~\%}{10}{\srmtoebc{7}}{90~min}{}

\begin{methodandtiming}
 
\begin{mashsteps}
\mashstep{\ftoc{151}}{35~min}
\mashstep{\ftoc{170}}{Mashout}
\end{mashsteps}

\begin{fermentationsteps}
\fermentationstep{\ftoc{55}}{}
\end{fermentationsteps}

\end{methodandtiming}

\recipebreak

\begin{ingredientsblock}

\begin{malts}
\malt{Pilsner}{\lbtokg{5.5}}
\malt{White Wheat}{\lbtokg{1.75}}
\malt{Gambrinus Honey}{\lbtokg{1.25}}
\end{malts}

\begin{hops}
\hop{\hopnugget}{13~\%}{60~min}{\oztog{0.25}}
\hop{\hophallertaumittelfruh}{4.5~\%}{5~min}{\oztog{0.75}}
\hop{Clover Honey}{}{\foh{}}{\lbtokg{2}}
\end{hops}

\singleyeast{White Labs WLP830}

\end{ingredientsblock}

\end{recipe}

% -----------------------------------------------------------------------------
\begin{recipe}{Triton Brewing Co. Fieldhouse Wheat Clone}
% -----------------------------------------------------------------------------

\begin{aboutblock}
This American wheat ale from Triton Brewing Company is a 2017 Great American
Beer Festival bronze medal winner from Indiana. The golden color, white head
and crisp flavor will satisfy any wheat lover's cravings! \sourceaha
\end{aboutblock}

\specifications{\styleamericanwheat}{\galtol{10}}{1.055}{1.028}{5~\%}{25}{\srmtoebc{4.1}}{90~min}{}

\begin{methodandtiming}
 
\begin{mashsteps}
\mashstep{\ftoc{154}}{}
\end{mashsteps}

\begin{fermentationsteps}
\fermentationstep{\ftoc{68}}{}
\end{fermentationsteps}

\end{methodandtiming}

\recipebreak

\begin{ingredientsblock}

\begin{malts}
\malt{Two-row}{\lbtokg{15.5}}
\malt{White Wheat}{\lbtokg{5}}
\malt{Flaked Barley}{\oztokg{8}}
\end{malts}

\begin{hops}
\hop{\hopgolding}{4.8~\%}{50~min}{\oztog{0.5}}
\hop{\hopliberty}{4.5~\%}{50~min}{\oztog{0.6}}
\hop{\hopfalconersflight}{11.3~\%}{20~min}{\oztog{0.5}}
\hop{\hopfalconersflight}{11.3~\%}{5~min}{\oztog{0.5}}
\hop{\hopfalconersflight}{11.3~\%}{\dryh{}{6~days}}{\oztog{0.9}}
\end{hops}

\singleyeast{Fermentis SafAle US-05}

\end{ingredientsblock}

\end{recipe}

% -----------------------------------------------------------------------------
\begin{recipe}[Uberbrew White Noise American Wheat Ale Clone]{Überbrew White Noise American Wheat Ale Clone}
% -----------------------------------------------------------------------------

\begin{aboutblock}
White Noise from Uberbrew in Billings, Mont. took home a gold medal at the 2016
Great American Beer Festival, adding to a medal count that helped them take home
the award for the 2016 Small Brewing Company and Small Brewing Company Brewer of
the Year. \sourceaha
\end{aboutblock}

\specifications{\styleamericanwheat}{\galtol{5}}{1.055}{}{}{13}{}{90~min}{}

\begin{methodandtiming}
 
\begin{mashsteps}
\mashstep{\ftoc{149}}{}
\end{mashsteps}

\end{methodandtiming}

\recipebreak

\begin{ingredientsblock}

\begin{malts}
\malt{Weyermann Pilsner}{\lbtokg{6.1}}
\malt{Weyermann Pale Wheat}{\lbtokg{4}}
\end{malts}

\begin{hops}
\hop{\hopperle}{8.4~\%}{60~min}{8~g}
\hop{\hopliberty}{4.9~\%}{30~min}{5~g}
\hop{\hopliberty}{4.9~\%}{5~min}{16~g}
\end{hops}

\singleyeast{Wyeast 1010}

\end{ingredientsblock}

\end{recipe}

% -----------------------------------------------------------------------------
\stylecategory{Amber Hybrid Beer}
\stylesection{\styleinternationalamberlager}

% -----------------------------------------------------------------------------
\begin{recipe}{Winnebago International Amber Lager} % checked
% -----------------------------------------------------------------------------

\begin{aboutblock}
Recipe by Sean McCambridge of Lake Winnebago, MO. Gold medal in Category #4: Amber
European Beer during the 2019 National Homebrew Competition in Providence, RI.
\sourceaha
\end{aboutblock}

\specifications{\styleinternationalamberlager}{\galtol{6.5}}{1.056}{1.010}{5.6~\%}{31}{\srmtoebc{9.7}}{60~min}{2.75}

\begin{methodandtiming}
 
\begin{mashsteps}
\mashstep{\ftoc{152}}{60~min}
\end{mashsteps}

\begin{fermentationsteps}
\fermentationstep{\ftoc{50}}{7~days}
\fermentationstep{\ftoc{68}}{Raise to over 3~days}
\fermentationstep{\ftoc{40}}{Carbonate; 14~days}
\end{fermentationsteps}

\end{methodandtiming}

\recipebreak

\begin{ingredientsblock}

\begin{malts}
\malt{Munich}{\lbtokg{10}}
\malt{Vienna}{\lbtokg{2}}
\malt{White Wheat}{\oztog{8}}
\malt{Pilsner}{\lbtokg{1.5}}
\end{malts}

\begin{hops}
\hop{\hophallertaumittelfruh}{4.8~\%}{60~min}{\oztog{1.2}}
\hop{\hophallertaumittelfruh}{4.8~\%}{30~min}{\oztog{1.2}}
\hop{\hophallertaumittelfruh}{4.8~\%}{15~min}{\oztog{0.7}}
\end{hops}

\singleyeast{White Labs WLP800}

\end{ingredientsblock}

\end{recipe}

\stylesection{\stylecaliforniacommon}

% -----------------------------------------------------------------------------
\begin{recipe}{"Dream Steam" California Common} % rechecked
% -----------------------------------------------------------------------------

\begin{aboutblock}
Recipe by Keith Eisel of Raleigh, NC. Gold medal in Category 7: Amber Hybrid Beer
during the 2016 National Homebrew Competition in Baltimore, MD. \sourceaha
\end{aboutblock}

\specifications{\stylecaliforniacommon}{\galtol{5.25}}{1.052}{1.010}{5.5~\%}{}{}{60~min}{2.4}

\begin{methodandtiming}

\begin{mashsteps}
\mashstep{\ftoc{152}}{60~min}
\mashstep{\ftoc{168}}{10~min}
\end{mashsteps}

\begin{fermentationsteps}
\fermentationstep{\ftoc{62}}{12~days}
\fermentationstep{\ftoc{55}}{23~days}
\end{fermentationsteps}

\end{methodandtiming}

\recipebreak

\begin{ingredientsblock}

\begin{malts}
\malt{Pale}{\lbtokg{8}}
\malt{Caramel / Crystal 40 L}{\lbtokg{0.75}}
\malt{Caramel / Crystal 60 L}{\lbtokg{0.75}}
\malt{Carapils / Dextrin}{\oztokg{8}}
\malt{Briess Victory}{\lbtokg{0.38}}
\end{malts}

\begin{hops}
\hop{\hopnorthernbrewer}{10.2~\%}{\fwh}{\oztog{0.75}}
\hop{\hopnorthernbrewer}{10.2~\%}{20~min}{\oztog{0.5}}
\hop{\hopnorthernbrewer}{10.2~\%}{\foh{}}{\oztog{0.5}}
\hop{\hopnorthernbrewer}{10.2~\%}{\dryh{}{}}{\oztog{0.25}}
\end{hops}

\singleyeast{White Labs WLP810}

\end{ingredientsblock}

\end{recipe}

% -----------------------------------------------------------------------------
\begin{recipe}{Wood St. Common} % rechecked
% -----------------------------------------------------------------------------

\begin{aboutblock}
Recipe by Jeremy Adams of Jefferson, MA. Gold medal in Category 11: Amber and Brown
American Beer during the 2018 National Homebrew Competition in Portland, OR.
\sourceaha
\end{aboutblock}

\specifications{\stylecaliforniacommon}{\galtol{5}}{1.049}{1.013}{4.7~\%}{37}{\srmtoebc{11}}{60~min}{2.6}

\begin{methodandtiming}
 
\begin{mashsteps}
\mashstep{\ftoc{152}}{60~min}
\mashstep{\ftoc{168}}{Mash out}
\end{mashsteps}

\begin{fermentationsteps}
\fermentationstep{\ftoc{62}}{14~days}
\end{fermentationsteps}

\begin{directions}
Water adjustment: \waterprofile{66}{6}{5}{56}{95}{15}.
\end{directions}

\end{methodandtiming}

\recipebreak

\begin{ingredientsblock}

\begin{malts}
\malt{Pilsner}{\lbtokg{8}}
\malt{Caramel / Crystal 40 L}{\lbtokg{1}}
\malt{Carapils / Dextrin}{\oztokg{8}}
\malt{Munich}{\oztokg{4}}
\malt{Briess Victory}{\oztokg{4}}
\malt{Chocolate}{\oztokg{1.5}}
\end{malts}

\begin{hops}
\hop{\hopnorthernbrewer}{10.2~\%}{60~min}{\oztog{0.5}}
\hop{\hopnorthernbrewer}{10.2~\%}{20~min}{\oztog{0.8}}
\hop{\hopnorthernbrewer}{10.2~\%}{\foh{}{}}{\oztog{1.25}}
\end{hops}

\singleyeast{White Labs WLP810}

\end{ingredientsblock}

\end{recipe}

\stylesection{\stylealtbier}

% -----------------------------------------------------------------------------
\begin{recipe}{Historic Brewing Company's Deer Lord Altbier Clone}
% -----------------------------------------------------------------------------

\begin{aboutblock}
\sourceaha
\end{aboutblock}

\specifications{\stylealtbier}{\galtol{5}}{1.049}{}{5~\%}{23}{\srmtoebc{16}}{90~min}{}

\begin{methodandtiming}

\begin{mashsteps}
\mashstep{\ftoc{151}}{60~min}
\end{mashsteps}

\begin{fermentationsteps}
\fermentationstep{\ftoc{65}}{}
\end{fermentationsteps}

\begin{directions}
Water adjustment: replicate Düsseldorf water.
\end{directions}

\end{methodandtiming}

\recipebreak

\begin{ingredientsblock}

\begin{malts}
\malt{\maltpilsner}{\lbtokg{5}}
\malt{Briess Aromatic Munich 20 L}{\lbtokg{3}}
\malt{\maltchocolate}{\oztokg{4}}
\malt{\maltwheat}{\oztokg{3.6}}
\malt{\maltcarapils}{\oztokg{2.6}}
\malt{Caramel / Crystal 60 L}{\oztokg{2}}
\end{malts}

\begin{hops}
\hop{\hopmagnum}{12.3~\%}{90~min}{\oztog{0.28}}
\hop{\hopmthood}{5.7~\%}{30~min}{\oztog{0.34}}
\hop{\hopmthood}{5.7~\%}{5~min}{\oztog{0.44}}
\end{hops}

\singleyeast{White Labs WLP036}

\end{ingredientsblock}

\end{recipe}

% -----------------------------------------------------------------------------
\begin{recipe}{Homebrew Challenge Altbier Beer}
% -----------------------------------------------------------------------------

\begin{aboutblock}
Recipe by Martin Keen.
\sourcehomebrewchallenge
\end{aboutblock}

\specifications{\stylealtbier}{\galtol{5}}{1.052}{1.009}{5.6~\%}{37}{}{60~min}{}

\begin{methodandtiming}

\begin{mashsteps}
\mashstep{\ftoc{152}}{60~min}
\end{mashsteps}

\begin{fermentationsteps}
\fermentationstep{\ftoc{55}}{}
\end{fermentationsteps}

\begin{directions}
Age for 4.5 weeks.
\end{directions}

\end{methodandtiming}

\recipebreak

\begin{ingredientsblock}

\begin{malts}
\malt{\maltpilsner}{\lbtokg{9}}
\malt{\maltmunich}{\lbtokg{1}}
\malt{Weyermann CARAMUNICH I}{\oztokg{4}}
\malt{\maltchocolate}{\oztokg{4}}
\end{malts}

\begin{hops}
\hop{\hopperle}{}{60~min}{\oztog{1.25}}
\hop{\hoptettnang}{}{15~min}{\oztog{0.75}}
\end{hops}

\singleyeast{White Labs WLP029}

\end{ingredientsblock}

\end{recipe}


% -----------------------------------------------------------------------------
\stylecategory{English Pale Ale}
\stylesection{\styleordinarybitter}

% -----------------------------------------------------------------------------
\begin{recipe}{Craictastic Ordinary Bitter} % rechecked
% -----------------------------------------------------------------------------

\begin{aboutblock}
Recipe by Patrick Leon of San Diego, CA. Gold medal in Category 8: Pale British
Ale during the 2019 National Homebrew Competition Providence, RI. \sourceaha
\end{aboutblock}

\specifications{\styleordinarybitter}{\galtol{6}}{1.039}{1.010}{3.8~\%}{33}{\srmtoebc{8.3}}{60~min}{2}

\begin{methodandtiming}
 
\begin{mashsteps}
\mashstep{\ftoc{155}}{60~min}
\end{mashsteps}

\begin{fermentationsteps}
\fermentationstep{\ftoc{68}}{Pitch}
\fermentationstep{\ftoc{60}}{1~day}
\fermentationstep{\ftoc{62}}{2~days}
\fermentationstep{\ftoc{64}}{4~days}
\fermentationstep{\ftoc{67}}{6~days}
\end{fermentationsteps}

\begin{directions}
Water adjustment: \waterprofile{86}{6}{12}{155}{20}{80}. Cold crash when fully
attenuated.
\end{directions}

\end{methodandtiming}

\recipebreak

\begin{ingredientsblock}

\begin{malts}
\malt{Golden Promise}{\lbtokg{10.5}}
\malt{Carapils / Dextrin}{\oztokg{8}}
\malt{Caramel / Crystal 120 L}{\oztokg{4}}
\malt{Caramel / Crystal 60 L}{\oztokg{0.38}}
\end{malts}

\begin{hops}
\hop{\hopeastkentgolding}{6~\%}{60~min}{\oztog{1.33}}
\hop{\hopeastkentgolding}{6~\%}{30~min}{\oztog{0.33}}
\hop{\hopeastkentgolding}{6~\%}{\foh{}{}}{\oztog{0.75}}
\end{hops}

\singleyeast{White Labs WLP013}

\end{ingredientsblock}

\end{recipe}

\stylesection{\stylebestbitter}

% -----------------------------------------------------------------------------
\begin{recipe}{Homebrew Challenge Best Bitter}
% -----------------------------------------------------------------------------

\begin{aboutblock}
Recipe by Martin Keen.
\sourcehomebrewchallenge
\end{aboutblock}

\specifications{\stylebestbitter}{\galtol{2.5}}{1.046}{1.012}{4.5~\%}{35}{}{60~min}{}

\begin{methodandtiming}

\begin{mashsteps}
\mashstep{\ftoc{152}}{60~min}
\end{mashsteps}

\begin{fermentationsteps}
\fermentationstep{\ftoc{68}}{}
\end{fermentationsteps}

\begin{directions}
Age for 3 weeks.
\end{directions}

\end{methodandtiming}

\recipebreak

\begin{ingredientsblock}

\begin{malts}
\malt{\maltmarisotter}{\lbtokg{4}}
\malt{\maltcaramel{80}}{\oztokg{6}}
\malt{\maltpalechocolate}{\oztokg{2}}
\end{malts}

\begin{hops}
\hop{\hopfuggle}{4.3~\%}{60~min}{\oztog{0.75}}
\hop{\hopeastkentgolding}{5~\%}{10~min}{\oztog{0.5}}
\end{hops}

\singleyeast{Wyeast 1318}

\end{ingredientsblock}

\end{recipe}

\stylesection{\stylestrongbitter}

% -----------------------------------------------------------------------------
\begin{recipe}{Denizens Brewing Co. Lowest Lord ESB Clone}
% -----------------------------------------------------------------------------

\begin{aboutblock}
Notes of toffee and biscuit with floral and herbal hop character. \sourceaha
\end{aboutblock}

\specifications{\stylestrongbitter}{\galtol{5}}{1.054}{1.014}{5.3~\%}{42}{\srmtoebc{14}}{90~min}{}

\begin{methodandtiming}
 
\begin{mashsteps}
\mashstep{\ftoc{152}}{}
\end{mashsteps}

\end{methodandtiming}

\recipebreak

\begin{ingredientsblock}

\begin{malts}
\malt{\malttworow}{\lbtokg{5}}
\malt{\maltmarisotter}{\lbtokg{2.75}}
\malt{Weyermann Munich I}{\oztokg{12}}
\malt{Aromatic}{\oztokg{12}}
\malt{Caramel / Crystal 45 L}{\oztokg{9}}
\malt{\maltacidulated}{\oztokg{4}}
\malt{\maltpalechocolate}{\oztokg{2}}
\end{malts}

\begin{hops}
\hop{\hopnugget}{13~\%}{45~min}{\oztog{0.5}}
\hop{\hoptarget}{10~\%}{10~min}{\oztog{1}}
\hop{\hoptarget}{10~\%}{5~min}{\oztog{0.5}}
\hop{\hopcrystal}{3.3~\%}{5~min}{\oztog{1}}
\end{hops}

\singleyeast{White Labs WLP007}

\end{ingredientsblock}

\end{recipe}

% -----------------------------------------------------------------------------
\begin{recipe}{Homebrew Challenge ESB (Extra Special Bitter)}
% -----------------------------------------------------------------------------

\begin{aboutblock}
Recipe by Martin Keen.
\sourcehomebrewchallenge
\end{aboutblock}

\specifications{\stylestrongbitter}{\galtol{2.5}}{}{}{}{45}{}{60~min}{}

\begin{methodandtiming}

\begin{mashsteps}
\mashstep{\ftoc{152}}{60~min}
\end{mashsteps}

\begin{fermentationsteps}
\fermentationstep{\ftoc{68}}{}
\end{fermentationsteps}

\begin{directions}
Age for 3 weeks.
\end{directions}

\end{methodandtiming}

\recipebreak

\begin{ingredientsblock}

\begin{malts}
\malt{\maltmarisotter}{\lbtokg{5}}
\malt{Caramel / Crystal 80 L}{\oztokg{8}}
\malt{\maltbrown}{\oztokg{4}}
\end{malts}

\begin{hops}
\hop{\hopfuggle}{4.3~\%}{60~min}{\oztog{1}}
\hop{\hopeastkentgolding}{5~\%}{20~min}{\oztog{0.5}}
\hop{\hopeastkentgolding}{5~\%}{\foh{}}{\oztog{0.5}}
\end{hops}

\singleyeast{Wyeast 1318}

\end{ingredientsblock}

\end{recipe}

% -----------------------------------------------------------------------------
\begin{recipe}{Up the Junction Bitter}
% -----------------------------------------------------------------------------

\begin{aboutblock}
Recipe by Brian Phad of Lockport, IL. Bronze medal in Category 8: Pale British Ale
during the 2019 National Homebrew Competition in Providence, RI. \sourceaha
\end{aboutblock}

\specifications{\stylestrongbitter}{\galtol{6}}{1.060}{1.010}{5.9~\%}{34.9}{\srmtoebc{14}}{}{2}

\begin{methodandtiming}
 
\begin{mashsteps}
\mashstep{\ftoc{152}}{60~min}
\end{mashsteps}

\begin{fermentationsteps}
\fermentationstep{\ftoc{65}}{Pitch}
\fermentationstep{\ftoc{69}}{Free raise; full attenuation}
\end{fermentationsteps}

\end{methodandtiming}

\recipebreak

\begin{ingredientsblock}

\begin{malts}
\malt{\maltmarisotter}{\lbtokg{11}}
\malt{Caramel / Crystal 60 L}{\oztokg{16}}
\malt{Aromatic}{\lbtokg{8}}
\malt{Briess Victory}{\oztokg{8}}
\malt{Simpsons Crystal Dark}{\oztokg{4.5}}
\end{malts}

\begin{hops}
\hop{\hopeastkentgolding}{6~\%}{\fwh}{\oztog{1}}
\hop{\hopeastkentgolding}{6~\%}{30~min}{\oztog{1}}
\hop{\hopeastkentgolding}{6~\%}{15~min}{\oztog{0.5}}
\hop{Light Brown Sugar}{}{10~min}{\oztog{4}}
\hop{\hopeastkentgolding}{6~\%}{\whirl{}{20~min}}{\oztog{0.5}}
\end{hops}

\singleyeast{Wyeast Labs 1469}

\end{ingredientsblock}

\end{recipe}

\stylesection{\stylebritishgoldenale}

% -----------------------------------------------------------------------------
\begin{recipe}{Homebrew Challenge British Golden Ale}
% -----------------------------------------------------------------------------

\begin{aboutblock}
Recipe by Martin Keen.
\sourcehomebrewchallenge
\end{aboutblock}

\specifications{\stylebritishgoldenale}{\galtol{5}}{1.051}{1.014}{4.9~\%}{42}{\srmtoebc{4}}{60~min}{}

\begin{methodandtiming}

\begin{mashsteps}
\mashstep{\ftoc{152}}{60~min}
\end{mashsteps}

\begin{fermentationsteps}
\fermentationstep{\ftoc{68}}{}
\end{fermentationsteps}

\begin{directions}
Age for 4 weeks.
\end{directions}

\end{methodandtiming}

\recipebreak

\begin{ingredientsblock}

\begin{malts}
\malt{\maltmarisotter}{\lbtokg{5}}
\malt{\maltpale}{\lbtokg{4}}
\malt{\maltwheat}{\lbtokg{1}}
\end{malts}

\begin{hops}
\hop{\hopfuggle}{4.5~\%}{60~min}{\oztog{1.5}}
\hop{\hoptarget}{10.7~\%}{10~min}{\oztog{1}}
\hop{\hoptarget}{10.7~\%}{\dryh{}{}}{\oztog{1}}
\end{hops}

\singleyeast{Wyeast 1318}

\end{ingredientsblock}

\end{recipe}

\stylesection{\styleaustraliansparklingale}

% -----------------------------------------------------------------------------
\begin{recipe}{Homebrew Challenge Australian Sparkling Ale} % rechecked
% -----------------------------------------------------------------------------

\begin{aboutblock}
Recipe by Martin Keen.
\sourcehomebrewchallenge
\end{aboutblock}

\specifications{\styleaustraliansparklingale}{\galtol{5}}{1.051}{1.012}{5.1~\%}{34}{\srmtoebc{4}}{60~min}{}

\begin{methodandtiming}

\begin{mashsteps}
\mashstep{\ftoc{152}}{60~min}
\end{mashsteps}

\begin{fermentationsteps}
\fermentationstep{\ftoc{68}}{}
\end{fermentationsteps}

\end{methodandtiming}

\recipebreak

\begin{ingredientsblock}

\begin{malts}
\malt{Maris Otter}{\lbtokg{5}}
\malt{Two-row}{\lbtokg{4}}
\malt{Caramel / Crystal 45 L}{\oztokg{8}}
\malt{Wheat}{\oztokg{8}}
\end{malts}

\begin{hops}
\hop{\hoppacificjade}{}{45~min}{\oztog{0.5}}
\hop{\hopgalaxy}{}{10~min}{\oztog{0.5}}
\hop{\hoppacificjade}{}{\foh{}{}}{\oztog{0.5}}
\hop{\hopgalaxy}{}{\foh{}{}}{\oztog{0.5}}
\end{hops}

\singleyeast{Wyeast 1318}

\end{ingredientsblock}

\end{recipe}


% -----------------------------------------------------------------------------
\stylecategory{Scottish \& Irish Ale}
\stylesection{\stylescottishlight}

% -----------------------------------------------------------------------------
\begin{recipe}{Iowa Brewers Bevvy Scottish Light}
% -----------------------------------------------------------------------------

\begin{aboutblock}
Recipe by Zack Rice of West Des Moines, IA. Gold medal in Category \#9: Scottish \& Irish
Ale during the 2019 National Homebrew Competition in Providence, RI. \sourceaha
\end{aboutblock}

\specifications{\stylescottishlight}{\galtol{5.2}}{1.040}{1.011}{3.8~\%}{23}{\srmtoebc{18}}{60~min}{}

\begin{methodandtiming}
 
\begin{mashsteps}
\mashstep{\ftoc{158}}{60~min}
\mashstep{\ftoc{168}}{Raise to over 10~min}
\end{mashsteps}

\begin{fermentationsteps}
\fermentationstep{\ftoc{67}}{10~days}
\end{fermentationsteps}

\begin{directions}
Water adjustment: \waterprofile{48}{16}{22}{75}{46}{36}.
\end{directions}

\end{methodandtiming}

\recipebreak

\begin{ingredientsblock}

\begin{malts}
\malt{Muntons Maris Otter}{\lbtokg{4.5}}
\malt{Simpsons Finest Pale Ale Golden Promise}{\lbtokg{2}}
\malt{Caramel / Crystal 40 L}{\lbtokg{1}}
\malt{Munich}{\oztog{8}}
\malt{Vienna}{\oztog{8}}
\malt{Caramel / Crystal 120 L}{\oztog{5}}
\malt{Chocolate}{\oztog{5}}
\malt{Acidulated}{\oztog{2}}
\end{malts}

\begin{hops}
\hop{\hopeastkentgolding}{6~\%}{60~min}{\oztog{1}}
\end{hops}

\singleyeast{Wyeast 1098}

\end{ingredientsblock}

\end{recipe}

% -----------------------------------------------------------------------------
\begin{recipe}{Stewpid Propino Scottish Ale}
% -----------------------------------------------------------------------------

\begin{aboutblock}
Recipe by Todd Stewart of Chicago, IL. Gold medal in Category 9: Scottish and Irish
Ale during the 2018 National Homebrew Competition in Portland, OR.
\sourceaha
\end{aboutblock}

\specifications{\stylescottishlight}{\galtol{6}}{1.034}{1.010}{3.3~\%}{17}{\srmtoebc{18}}{60~min}{}

\begin{methodandtiming}

\begin{mashsteps}
\mashstep{\ftoc{152}}{60~min}
\mashstep{\ftoc{168}}{Mash out}
\end{mashsteps}

\begin{fermentationsteps}
\fermentationstep{\ftoc{55}}{3~weeks}
\fermentationstep{\ftoc{33}}{Transfer to secondary; 1~month}
\end{fermentationsteps}

\begin{directions}
Boil \galtol{1} of first runnings until reduced by half, then add to brew
kettle.
\end{directions}

\end{methodandtiming}

\recipebreak

\begin{ingredientsblock}

\begin{malts}
\malt{Muntons Propino Pale Ale}{\lbtokg{6.4}}
\malt{Roasted Barley}{\oztog{4.8}}
\malt{Chocolate}{\oztog{3}}
\malt{Simpsons DRC}{\oztog{2.5}}
\end{malts}

\begin{hops}
\hop{\hopfuggle}{4.4~\%}{60~min}{\oztog{1}}
\hop{Whirlfloc Tablet}{}{10~min}{1}
\hop{Yest Nutrient}{}{10~min}{--}
\end{hops}

\singleyeast{Wyeast 1728}

\end{ingredientsblock}

\end{recipe}

\stylesection{\stylescottishheavy}

% -----------------------------------------------------------------------------
\begin{recipe}{Argentine Strong Scotch Ale} % checked
% -----------------------------------------------------------------------------

\begin{aboutblock}
Recipe by Ron Yabut. Took part in the 2007 GABF Pro-Am competition. Toasty, sweet malt
nose and a medium-full body with some pleasant fruity esters.
\sourcezymurgy{January / February 2008}
\end{aboutblock}

\specifications{\stylescottishheavy}{\galtol{5.5}}{1.080}{}{}{17}{\srmtoebc{16}}{120~min}{}

\begin{methodandtiming}

\begin{mashsteps}
\mashstep{\ftoc{150}}{60~min}
\mashstep{\ftoc{170}}{Mash out}
\end{mashsteps}

\begin{fermentationsteps}
\fermentationstep{\ftoc{50}}{4~days}
\fermentationstep{\ftoc{50}}{Transfer to secondary; 7~days}
\end{fermentationsteps}

\end{methodandtiming}

\recipebreak

\begin{ingredientsblock}

\begin{malts}
\malt{Maris Otter}{\lbtokg{10.5}}
\malt{Munich}{\lbtokg{3.5}}
\malt{Caramel / Crystal 60 L}{\lbtokg{1.25}}
\malt{Weyermann CARAMUNICH I}{\lbtokg{0.5}}
\end{malts}

\begin{hops}
\hop{\hopeastkentgolding}{3.5~\%}{120~min}{\oztog{1}}
\end{hops}

\singleyeast{Wyeast 1728}

\end{ingredientsblock}

\end{recipe}

% -----------------------------------------------------------------------------
\begin{recipe}{"Don't Look Up\ldots" Scottish Ale} % checked
% -----------------------------------------------------------------------------

\begin{aboutblock}
Recipe by Mark Peterson of Queen Creek, AZ. Slver medal in Category 9:
Scottish \& Irish Ale during the 2019 National Homebrew Competition
in Providence, RI. \sourceaha
\end{aboutblock}

\specifications{\stylescottishheavy}{\galtol{5}}{1.080}{1.023}{8.9~\%}{22}{\srmtoebc{18}}{80~min}{2.4}

\begin{methodandtiming}
 
\begin{mashsteps}
\mashstep{\ftoc{150}}{60~min}
\end{mashsteps}

\begin{fermentationsteps}
\fermentationstep{\ftoc{59}}{8~days}
\fermentationstep{\ftoc{59}}{Transfer to secondary; 2~weeks}
\fermentationstep{\ftoc{65}}{3~days}
\end{fermentationsteps}

\begin{directions}
Caramelized wort: boil \galtol{1.5} of first runnings until reduced to \galtol{0.5}.
\end{directions}

\end{methodandtiming}

\recipebreak

\begin{ingredientsblock}

\begin{malts}
\malt{Pale}{\lbtokg{11}}
\malt{Munich}{\lbtokg{1.5}}
\malt{Caramel / Crystal 60 L}{\lbtokg{1}}
\malt{Chocolate}{\oztog{3.2}}
\malt{Crisp Roast Barley}{\oztog{3.2}}
\end{malts}

\begin{hops}
\hop{Caramelized Wort}{}{80~min}{\galtol{0.5}}
\hop{\hopfuggle}{4.5~\%}{60~min}{\oztog{1}}
\hop{\hopfuggle}{4.5~\%}{15~min}{\oztog{0.5}}
\end{hops}

\singleyeast{Wyeast 1728}

\end{ingredientsblock}

\end{recipe}

\stylesection{\stylescottishexport}

% -----------------------------------------------------------------------------
\begin{recipe}{80 Shilling Scottish Ale}
% -----------------------------------------------------------------------------

\begin{aboutblock}
Recipe by Amahl Turczyn. \sourcezymurgy{July / August 2017}
\end{aboutblock}

\specifications{\stylescottishexport}{\galtol{5.5}}{1.055}{1.016}{5.1~\%}{30}{\srmtoebc{15}}{120~min}{}

\begin{methodandtiming}

\begin{mashsteps}
\mashstep{\ftoc{154}}{60~min}
\mashstep{\ftoc{168}}{Mash out}
\end{mashsteps}

\begin{fermentationsteps}
\fermentationstep{\ftoc{63}}{Pitch}
\fermentationstep{\ftoc{67}}{Free raise over 4~days; 10~days}
\fermentationstep{\ftoc{36}}{Transfer to secondary; 14~days}
\end{fermentationsteps}

\end{methodandtiming}

\recipebreak

\begin{ingredientsblock}

\begin{malts}
\malt{Golden Promise}{\lbtokg{11}}
\malt{Roasted Barley}{\oztokg{5}}
\end{malts}

\begin{hops}
\hop{\hopeastkentgolding}{5~\%}{60~min}{\oztog{2}}
\end{hops}

\singleyeast{White Labs WLP028 / Wyeast 1728}

\end{ingredientsblock}

\end{recipe}

% -----------------------------------------------------------------------------
\begin{recipe}{Odell Brewing Company 90 Shilling Ale Clone} % rehecked
% -----------------------------------------------------------------------------

\begin{aboutblock}
Smooth and complex. A medium-bodied amber ale with a distinct burnished copper
color. \sourceaha
\end{aboutblock}

\specifications{\stylescottishexport}{\galtol{5}}{1.048}{1.008}{5.3~\%}{27}{\srmtoebc{8.5}}{90~min}{}

\begin{methodandtiming}

\begin{mashsteps}
\mashstep{\ftoc{153}}{}
\end{mashsteps}

\begin{fermentationsteps}
\fermentationstep{\ftoc{67}}{}
\end{fermentationsteps}

\begin{directions}
Chill for 14 days before packaging.
\end{directions}

\end{methodandtiming}

\recipebreak

\begin{ingredientsblock}

\begin{malts}
\malt{\maltpale}{\lbtokg{7.2}}
\malt{\maltmunich}{\lbtokg{1}}
\malt{\maltcaramel{10}}{\lbtokg{0.75}}
\malt{Flaked Wheat}{\lbtokg{0.5}}
\malt{\maltcaramel{15}}{\oztokg{4.8}}
\malt{\maltchocolate}{\oztokg{1.6}}
\end{malts}

\begin{hops}
\hop{\hopcascade}{5.5~\%}{90~min}{\oztog{0.2}}
\hop{\hopperle}{8~\%}{90~min}{\oztog{0.2}}
\hop{\hopnugget}{13~\%}{90~min}{\oztog{0.1}}
\hop{\hopperle}{8~\%}{45~min}{\oztog{0.2}}
\hop{\hopcascade}{5.5~\%}{45~min}{\oztog{0.1}}
\hop{\hopnugget}{13~\%}{45~min}{\oztog{0.1}}
\hop{\hopperle}{8~\%}{\foh{}{}}{\oztog{0.2}}
\hop{\hopcascade}{5.5~\%}{\foh{}{}}{\oztog{0.1}}
\end{hops}

\singleyeast{Scottish Ale}

\end{ingredientsblock}

\end{recipe}

% -----------------------------------------------------------------------------
\begin{recipe}{Bonnie Prince Charlie's 80/- Shilling}
% -----------------------------------------------------------------------------

\begin{aboutblock}
Recipe by Michael Kelly. Gold medal at the 2014 GABF Pro-Am competition.
\sourceaha
\end{aboutblock}

\specifications{\stylescottishexport}{\galtol{12}}{1.054}{1.016}{5~\%}{23}{\srmtoebc{13}}{90~min}{}

\begin{methodandtiming}

\begin{mashsteps}
\mashstep{\ftoc{156}}{60~min}
\mashstep{\ftoc{168}}{Raise over 10~min}
\end{mashsteps}

\begin{fermentationsteps}
\fermentationstep{\ftoc{17}}{14~days}
\fermentationstep{\ftoc{36}}{Transfer to secondary; 14~days}
\end{fermentationsteps}

\begin{directions}
Caramelized wort: boil \galtol{4} of first runnings for 30 minutes.
\end{directions}

\end{methodandtiming}

\recipebreak

\begin{ingredientsblock}

\begin{malts}
\malt{\maltmarisotter}{\lbtokg{24}}
\malt{Roasted Barley}{\lbtokg{0.75}}
\end{malts}

\begin{hops}
\hop{Caramelized Wort}{}{90~min}{--}
\hop{\hopeastkentgolding}{5.7~\%}{60~min}{\oztog{4}}
\end{hops}

\singleyeast{White Labs WLP028}

\end{ingredientsblock}

\end{recipe}

% -----------------------------------------------------------------------------
\begin{recipe}{"D3 Scottish Ale" Scottish Export}
% -----------------------------------------------------------------------------

\begin{aboutblock}
Recipe by Matthew and Brennan Weissenbuehler of Parker, CO. Gold medal in
Category 9: Scottish and Irish Ale during the 2016 National Homebrew Competition
in Baltimore, MD. \sourceaha
\end{aboutblock}

\specifications{\stylescottishexport}{\galtol{5}}{1.052}{1.010}{5.5~\%}{}{}{60~min}{2.3}

\begin{methodandtiming}

\begin{mashsteps}
\mashstep{\ftoc{156}}{60~min}
\mashstep{\ftoc{168}}{10~min}
\end{mashsteps}

\begin{fermentationsteps}
\fermentationstep{\ftoc{61}}{14~days}
\fermentationstep{\ftoc{40}}{28~days}
\end{fermentationsteps}

\end{methodandtiming}

\recipebreak

\begin{ingredientsblock}

\begin{malts}
\malt{Simpsons Finest Pale Ale Golden Promise}{\lbtokg{8.5}}
\malt{\maltweyermannmunichtwo}{\lbtokg{1}}
\malt{Briess Caramel 30 L}{\lbtokg{1}}
\malt{\maltweyermannpalewheat}{\lbtokg{0.25}}
\malt{Roasted Barley}{\lbtokg{0.25}}
\end{malts}

\begin{hops}
\hop{\hopfuggle}{5.2~\%}{\fwh}{\oztog{1}}
\end{hops}

\singleyeast{White Labs WLP028}

\end{ingredientsblock}

\end{recipe}

\stylesection{\styleirishredale}

% -----------------------------------------------------------------------------
\begin{recipe}{Barley Phillip Irish Red}
% -----------------------------------------------------------------------------

\begin{aboutblock}
Recipe by Amahl Turczyn. \sourcezymurgy{January / February 2018}
\end{aboutblock}

\specifications{\styleirishredale}{\galtol{5.5}}{1.047}{1.009}{5~\%}{20}{\srmtoebc{13}}{90~min}{}

\begin{methodandtiming}
 
\begin{mashsteps}
\mashstep{\ftoc{148}}{60~min}
\mashstep{\ftoc{168}}{Raise to over 20~min; 10~min}
\end{mashsteps}

\begin{fermentationsteps}
\fermentationstep{\ftoc{60}}{Pitch}
\fermentationstep{\ftoc{65}}{Until fully attenuated}
\end{fermentationsteps}

\begin{directions}
Water adjustment: use reverse osmosis water with \gpgaltogpl{1} calcium chloride.
\end{directions}

\end{methodandtiming}

\recipebreak

\begin{ingredientsblock}

\begin{malts}
\malt{Maris Otter}{\lbtokg{8}}
\malt{Flaked Maize}{\lbtokg{1}}
\malt{Roast Barley}{\oztog{4}}
\end{malts}

\begin{hops}
\hop{\hopeastkentgolding}{5~\%}{60~min}{\oztog{1.25}}
\end{hops}

\singleyeast{White Labs WLP039}

\end{ingredientsblock}

\end{recipe}

% -----------------------------------------------------------------------------
\begin{recipe}{Brighid's Irish Red Ale}
% -----------------------------------------------------------------------------

\begin{aboutblock}
Recipe by Grote. Won best of show at the Liquid Poetry Slam competition in
Fort Collins. \sourcezymurgy{January / February} 
\end{aboutblock}

\specifications{\styleirishredale}{\galtol{5}}{1.058}{1.012}{6~\%}{25}{\srmtoebc{13}}{60~min}{}

\begin{methodandtiming}

\begin{directions}
Ferment for 1 week, transfer to secondary, and condition for an additional 2 weeks.
\end{directions}

\end{methodandtiming}

\recipebreak

\begin{ingredientsblock}

\begin{malts}
\malt{Pale}{\lbtokg{6}}
\malt{Munich}{\lbtokg{3.25}}
\malt{Rye}{\lbtokg{2.5}}
\malt{Roasted Barley}{\oztog{4}}
\end{malts}

\begin{hops}
\hop{\hopchallenger}{7.5~\%}{60~min}{\oztog{1}}
\hop{\hopfuggle}{4.5~\%}{15~min}{\oztog{1}}
\hop{Whirlfloc Tablet}{}{}{1}
\end{hops}

\singleyeast{White Labs WLP013}

\end{ingredientsblock}

\end{recipe}

% -----------------------------------------------------------------------------
\begin{recipe}{Finnegan's Irish Ale}
% -----------------------------------------------------------------------------

\begin{aboutblock}
Recipe by Dennis Mitchell of Chandler, AZ. Bronze medal in Category \#9: Scottish
\& Irish Ale during the 2019 National Homebrew Competition in Providence, RI.
\sourceaha
\end{aboutblock}

\specifications{\styleirishredale}{\galtol{5}}{1.050}{1.010}{9.3~\%}{4.8}{}{90~min}{}

\begin{methodandtiming}
 
\begin{mashsteps}
\mashstep{\ftoc{152}}{60~min}
\end{mashsteps}

\begin{fermentationsteps}
\fermentationstep{\ftoc{64}}{}
\end{fermentationsteps}

\begin{directions}
Water adjustment: reverse osmosis water with \tsptog{1} calcium chloride added to
mash.
\end{directions}

\end{methodandtiming}

\recipebreak

\begin{ingredientsblock}

\begin{malts}
\malt{Mild}{\lbtokg{6.5}}
\malt{Vienna}{\lbtokg{3}}
\malt{Caramel / Crystal 40 L}{\lbtokg{3}}
\malt{Flaked Maize}{\lbtokg{1}}
\malt{Roasted Barley}{\lbtokg{0.25}}
\end{malts}

\begin{hops}
\hop{\hopeastkentgolding}{6~\%}{60~min}{\oztog{0.7}}
\hop{\hopeastkentgolding}{6~\%}{30~min}{\oztog{0.5}}
\hop{\hopeastkentgolding}{6~\%}{10~min}{\oztog{0.25}}
\end{hops}

\singleyeast{Imperial Yeast A10}

\end{ingredientsblock}

\end{recipe}

% -----------------------------------------------------------------------------
\begin{recipe}{Joe Gillian's Red}
% -----------------------------------------------------------------------------

\begin{aboutblock}
Irish red ale from Amal Turczyn's book, A Year of Beer. Buttery character, with
a well-balanced maltiness and a round, fruity finish. \sourceaha
\end{aboutblock}

\specifications{\styleirishredale}{\galtol{5}}{1.050}{1.010}{5.25~\%}{}{}{60~min}{}

\begin{methodandtiming}
 
\begin{mashsteps}
\mashstep{\ftoc{150}}{90~min}
\end{mashsteps}

\begin{fermentationsteps}
\fermentationstep{\ftoc{68}}{7~days}
\fermentationstep{\ftoc{65}}{7~days}
\end{fermentationsteps}

\end{methodandtiming}

\recipebreak

\begin{ingredientsblock}

\begin{malts}
\malt{Two-row}{\lbtokg{7}}
\malt{Caramel / Crystal 55 L}{\lbtokg{0.25}}
\malt{Caramel / Crystal 80 L}{\lbtokg{0.25}}
\malt{Roasted Barley}{\oztog{2}}
\end{malts}

\begin{hops}
\hop{\hopfuggle}{4.4~\%}{60~min}{\oztog{1}}
\hop{\hopfuggle}{4.4~\%}{30~min}{\oztog{0.75}}
\hop{\hopeastkentgolding}{5.1~\%}{5~min}{\oztog{0.75}}
\end{hops}

\singleyeast{Wyeast 1968}

\end{ingredientsblock}

\end{recipe}

% -----------------------------------------------------------------------------
\begin{recipe}{Scarlett O'Hara Irish Red Ale}
% -----------------------------------------------------------------------------

\begin{aboutblock}
Recipe by Robert Smith of Newton, NJ. Silver medal in Category 9: Scottish \&
Irish Ale during the 2018 National Homebrew Competition in Portland, OR.
\sourceaha
\end{aboutblock}

\specifications{\styleirishredale}{\galtol{6}}{1.051}{1.012}{5.1~\%}{19}{\srmtoebc{14.6}}{90~min}{}

\begin{methodandtiming}
 
\begin{mashsteps}
\mashstep{\ftoc{152}}{75~min}
\mashstep{\ftoc{168}}{15~min}
\end{mashsteps}

\begin{fermentationsteps}
\fermentationstep{\ftoc{68}}{12~days}
\end{fermentationsteps}

\begin{directions}
Water adjustment: use \galtol{4.5} reverse osmosis water with 4.5~g calcium chloride
to mash and \galtol{5} reverse osmosis water with pH adjusted to 5.5 for sparge.
\end{directions}

\end{methodandtiming}

\recipebreak

\begin{ingredientsblock}

\begin{malts}
\malt{Muntons Mild}{\lboztokg{6}{7}}
\malt{Weyermann Vienna}{\lboztokg{2}{12}}
\malt{Briess Caramel}{\oztog{15}}
\malt{Briess Carapils}{\oztog{15}}
\malt{Thomas Fawcett Flaked Maize}{\oztog{15}}
\malt{Muntons Roasted Barley}{\oztog{3.2}}
\malt{Acidulated}{\oztog{3}}
\end{malts}

\begin{hops}
\hop{\hopeastkentgolding}{5.1~\%}{60~min}{22~g}
\hop{\hopeastkentgolding}{5.1~\%}{30~min}{9~g}
\hop{Whirlfloc Tablet}{}{30~min}{1}
\end{hops}

\singleyeast{Wyeast 1084}

\end{ingredientsblock}

\end{recipe}

% -----------------------------------------------------------------------------
\begin{recipe}{Tennessee Valley Brewing Co. DIVARTY Redlegs Clone}
% -----------------------------------------------------------------------------

\begin{aboutblock}
Slight malty flavor, a soft touch of caramel and lightly roasty finish. \sourceaha
\end{aboutblock}

\specifications{\styleirishredale}{\galtol{5}}{1.046}{1.011}{4.5~\%}{24.7}{\srmtoebc{14.1 }}{60~min}{2.3}

\begin{methodandtiming}
 
\begin{mashsteps}
\mashstep{\ftoc{153}}{}
\end{mashsteps}

\begin{fermentationsteps}
\fermentationstep{\ftoc{68}}{1~weeks}
\end{fermentationsteps}

\begin{directions}
Transfer into secondary for 1 week.
\end{directions}

\end{methodandtiming}

\recipebreak

\begin{ingredientsblock}

\begin{malts}
\malt{Two-row}{\lbtokg{8}}
\malt{Caramel / Crystal 60 L}{\oztog{5}}
\malt{Caramel / Crystal 120 L}{\oztog{4}}
\malt{Chocolate}{\oztog{4}}
\end{malts}

\begin{hops}
\hop{\hopwillamette}{5.5~\%}{60~min}{\oztog{1.25}}
\hop{Irish Moss}{}{10~min}{--}
\end{hops}

\singleyeast{Fermentis SafAle US-05}

\end{ingredientsblock}

\end{recipe}

\stylesection{\styleweeheavy}

% -----------------------------------------------------------------------------
\begin{recipe}{3 Angels Kilt Lifter} % rechecked
% -----------------------------------------------------------------------------

\begin{aboutblock}
Recipe by Richard Kearns of Marlborough, MA. Gold medal in Category 9: Scottish
and Irish Ale during the 2015 National Homebrew Competition in San Diego, CA.
\sourceaha
\end{aboutblock}

\specifications{\styleweeheavy}{\galtol{6.5}}{1.100}{1.027}{9.58~\%}{}{\srmtoebc{14}}{90~min}{}

\begin{methodandtiming}

\begin{mashsteps}
\mashstep{\ftoc{156}}{60~min}
\mashstep{\ftoc{168}}{10~min}
\end{mashsteps}

\begin{fermentationsteps}
\fermentationstep{\ftoc{68}}{3~months}
\end{fermentationsteps}

\end{methodandtiming}

\recipebreak

\begin{ingredientsblock}

\begin{malts}
\malt{Golden Promise}{\lbtokg{15}}
\malt{Aromatic}{\lbtokg{1}}
\malt{Carapils / Dextrin}{\lbtokg{1}}
\malt{Weyermann CARAMUNICH I}{\lbtokg{1}}
\malt{Weyermann CARAAMBER}{\lbtokg{1}}
\malt{Melanoidin}{\lbtokg{1}}
\malt{Weyermann Munich I}{\lbtokg{1}}
\malt{Dingemans Biscuit}{\lbtokg{0.5}}
\malt{Brown}{\lbtokg{0.5}}
\malt{Roasted Barley}{\oztokg{3}}
\malt{Weyermann Munich II}{\oztokg{0.2}}
\end{malts}

\begin{hops}
\hop{\hopnorthernbrewer}{8.5~\%}{90~min}{\oztog{0.5}}
\hop{\hopnorthernbrewer}{8.5~\%}{45~min}{\oztog{0.5}}
\hop{\hopeastkentgolding}{5~\%}{20~min}{\oztog{0.5}}
\hop{\hopeastkentgolding}{5~\%}{15~min}{\oztog{0.5}}
\end{hops}

\singleyeast{Wyeast 1728}

\end{ingredientsblock}

\end{recipe}

% -----------------------------------------------------------------------------
\begin{recipe}{Albanach Láidir} % rechecked
% -----------------------------------------------------------------------------

\begin{aboutblock}
Recipe by Mark Schoppe of Austin, TX. Gold medal in Category 9: Scottish and Irish
Ale during the 2012 National Homebrew Competition in Seattle, WA.
\sourceaha
\end{aboutblock}

\specifications{\styleweeheavy}{\galtol{6}}{1.085}{1.023}{8.14~\%}{22}{\srmtoebc{18}}{75~min}{}

\begin{methodandtiming}

\begin{mashsteps}
\mashstep{\ftoc{155}}{}
\end{mashsteps}

\begin{fermentationsteps}
\fermentationstep{\ftoc{65}}{21~days}
\end{fermentationsteps}

\begin{directions}
Caramelized wort: boil \galtol{1} of first runnings until condensed to a syrup.
Condition for 14 days at room temperature.
\end{directions}

\end{methodandtiming}

\recipebreak

\begin{ingredientsblock}

\begin{malts}
\malt{Rahr Pale Ale}{\lbtokg{13.75}}
\malt{Carapils / Dextrin}{\lbtokg{2.4}}
\malt{Weyermann Beech Smoked Barley Malt}{\oztokg{9.5}}
\malt{Dingemans Biscuit}{\oztokg{9.5}}
\malt{Briess Victory}{\oztokg{9.5}}
\malt{Caramel / Crystal 60 L}{\oztokg{9.5}}
\malt{Roasted Barley}{\oztokg{5}}
\end{malts}

\begin{hops}
\hop{Caramelized Wort}{}{75~min}{--}
\hop{\hopsummit}{18.5~\%}{75~min}{\oztog{0.6}}
\hop{\hopeastkentgolding}{5~\%}{15~min}{\oztog{0.25}}
\hop{\hopeastkentgolding}{5~\%}{5~min}{\oztog{0.25}}
\end{hops}

\singleyeast{White Labs WLP028}

\end{ingredientsblock}

\end{recipe}

% -----------------------------------------------------------------------------
\begin{recipe}{Gunn Clan Scotch Ale} % rechecked
% -----------------------------------------------------------------------------

\begin{aboutblock}
This recipe was originally featured in "Master Malt: Selecting the Best Base
for Your Beer" by Gordon Strong. \sourcezymurgy{May / June 2011}
\end{aboutblock}

\specifications{\styleweeheavy}{\galtol{5.5}}{1.130}{}{}{32}{\srmtoebc{23}}{60~min}{}

\begin{methodandtiming}

\begin{mashsteps}
\mashstep{\ftoc{158}}{120~min}
\end{mashsteps}

\begin{fermentationsteps}
\fermentationstep{\ftoc{60}}{}
\end{fermentationsteps}

\begin{directions}
Caramelized wort: hard boil \galtol{1} of first runnings.
\end{directions}

\end{methodandtiming}

\recipebreak

\begin{ingredientsblock}

\begin{malts}
\malt{Cargill Pauls Mild Ale (Dextrin)}{\lbtokg{15}}
\malt{Crist Finest Maris Otter Ale}{\lbtokg{15}}
\malt{Roasted Barley}{\oztokg{6}}
\end{malts}

\begin{hops}
\hop{Caramelized Wort}{}{}{--}
\hop{\hopnorthernbrewer}{8.5~\%}{60~min}{\oztog{1.5}}
\hop{\hopnorthernbrewer}{8.5~\%}{30~min}{\oztog{0.5}}
\end{hops}

\singleyeast{White Labs WLP028}

\end{ingredientsblock}

\end{recipe}

% -----------------------------------------------------------------------------
\begin{recipe}{Revelry Brewing Co Oh My Darlyn Scotch Ale Clone} % rechecked
% -----------------------------------------------------------------------------

\begin{aboutblock}
Dominated by malt character, with a rich and dominant sweet malt flavor and aroma.
\sourceaha
\end{aboutblock}

\specifications{\styleweeheavy}{\galtol{5}}{1.093}{1.019}{9.6~\%}{30}{}{60~min}{}

\begin{methodandtiming}

\begin{mashsteps}
\mashstep{\ftoc{152}}{60~min}
\end{mashsteps}

\begin{fermentationsteps}
\fermentationstep{\ftoc{64}}{21~days}
\end{fermentationsteps}

\end{methodandtiming}

\recipebreak

\begin{ingredientsblock}

\begin{malts}
\malt{Maris Otter}{\lbtokg{14.5}}
\malt{Roasted Barley}{\oztokg{7}}
\malt{Flaked Oats}{\oztokg{7}}
\malt{Melanoidin}{\oztokg{7}}
\malt{Caramel / Crystal 15 L}{\oztokg{7}}
\malt{Caramel / Crystal 60 L}{\oztokg{7}}
\end{malts}

\begin{hops}
\hop{\hopeastkentgolding}{5~\%}{60~min}{\oztog{2.75}}
\end{hops}

\singleyeast{White Labs WLP028}

\end{ingredientsblock}

\end{recipe}

% -----------------------------------------------------------------------------
\begin{recipe}{Russell's Scottish 80} % rechecked
% -----------------------------------------------------------------------------

\begin{aboutblock}
Recipe by Russell Brunner of Tamarac, FL. Gold medal in Category 9: Scottish
and Irish Ale during the 2013 National Homebrew Competition in Philadelphia, PA.
\sourceaha
\end{aboutblock}

\specifications{\styleweeheavy}{\galtol{7}}{1.069}{1.019}{6.7}{14.4}{}{60~min}{}

\begin{methodandtiming}

\begin{mashsteps}
\mashstep{\ftoc{158}}{60~min}
\mashstep{\ftoc{168}}{Mash out}
\end{mashsteps}

\begin{fermentationsteps}
\fermentationstep{\ftoc{67}}{14~days}
\end{fermentationsteps}

\begin{directions}
Water adjustment: reverse osmosis water with \gtog{4.2} calcium chloride.
\end{directions}

\end{methodandtiming}

\recipebreak

\begin{ingredientsblock}

\begin{malts}
\malt{Maris Otter}{\lbtokg{13.5}}
\malt{Caramel / Crystal 40 L}{\lbtokg{1.38}}
\malt{Gambrinus Honey}{\lbtokg{0.69}}
\malt{Munich}{\lbtokg{0.69}}
\malt{Caramel / Crystal 120 L}{\lbtokg{0.38}}
\malt{Pale Chocolate}{\oztokg{4}}
\malt{Rice Hulls}{\lbtokg{1}}
\end{malts}

\begin{hops}
\hop{\hopeastkentgolding}{6.4~\%}{120~min}{\oztog{1}}
\end{hops}

\singleyeast{White Labs WLP001}

\end{ingredientsblock}

\end{recipe}

% -----------------------------------------------------------------------------
\begin{recipe}{Southern Highlander Wee Heavy} % rechecked
% -----------------------------------------------------------------------------

\begin{aboutblock}
Recipe by Drew Beechum. Taste of a dark, nutty, caramel flavor with a hint of smoke.
Mimics a pecan pie's brown sugar syrup with the dose of maple syrup during secondary
fermentation. \sourcezymurgy{March / April 2010}.
\end{aboutblock}

\specifications{\styleweeheavy}{\galtol{5.5}}{1.084}{}{}{19}{\srmtoebc{21}}{180~min}{}

\begin{methodandtiming}

\begin{mashsteps}
\mashstep{\ftoc{155}}{60~min}
\end{mashsteps}

\begin{directions}
Caramelized wort: boil \qttol{2.5} of first runnings until reduced to \cuptoml{3}.
\end{directions}

\end{methodandtiming}

\recipebreak

\begin{ingredientsblock}

\begin{malts}
\malt{Maris Otter}{\lbtokg{15.5}}
\malt{Weyermann CARAAROMA}{\lbtokg{0.5}}
\malt{Double Pecan-Smoked}{\lbtokg{0.5}}
\malt{Roasted Barley}{\lbtokg{0.25}}
\end{malts}

\begin{hops}
\hop{Caramelized Wort}{}{90~min}{\cuptoml{3}}
\hop{\hopmagnum}{12.9~\%}{90~min}{\oztog{0.45}}
\end{hops}

\singleyeast{Wyeast 1728}

\begin{twists}
\twist{Grade B Maple Syrup}{Secondary}{\oztoml{8}}
\end{twists}

\end{ingredientsblock}

\end{recipe}

% -----------------------------------------------------------------------------
\begin{recipe}{Wee Heavy / Strong Scotch Ale} % rechecked
% -----------------------------------------------------------------------------

\begin{aboutblock}
Recipe by Amahl Turczyn. \sourcezymurgy{July / August 2017}
\end{aboutblock}

\specifications{\styleweeheavy}{\galtol{5.5}}{1.091}{1.025}{8.8~\%}{36}{\srmtoebc{20}}{120~min}{}

\begin{methodandtiming}

\begin{mashsteps}
\mashstep{\ftoc{152}}{60~min}
\mashstep{\ftoc{168}}{Mash out}
\end{mashsteps}

\begin{fermentationsteps}
\fermentationstep{\ftoc{63}}{Pitch}
\fermentationstep{\ftoc{67}}{Free raise over 4~days; 14~days}
\fermentationstep{\ftoc{36}}{Transfer to secondary; 1~month}
\end{fermentationsteps}

\begin{directions}
Age up to 10 months at cellar temperatures before serving.
\end{directions}

\end{methodandtiming}

\recipebreak

\begin{ingredientsblock}

\begin{malts}
\malt{Golden Promise}{\lbtokg{18.25}}
\malt{Roasted Barley}{\oztokg{8}}
\end{malts}

\begin{hops}
\hop{\hopeastkentgolding}{5~\%}{60~min}{\oztog{3}}
\end{hops}

\singleyeast{White Labs WLP028 / Wyeast 1728}

\end{ingredientsblock}

\end{recipe}


% -----------------------------------------------------------------------------
\stylecategory{American Pale Ale}
\part{\styleamericanpaleale}

\begin{recipie}{3 Floyds Alpha King Clone}

\begin{aboutblock}
Alpha King, a bold, citrus-forward American pale ale has been consistently ranked as
one of the best pale ales made in America. If you're a fan of the versatile Centennial
hop, this beer from Munster, Ind.'s 3 Floyds Brewing Co. is for you!
\end{aboutblock}

\specifications{\styleamericanpaleale}{\galtol{5}}{1.059}{1.014}{6.4~\%}{69}{}{60~min}

\begin{methodandtiming}
 
\begin{mashsteps}
\mashstep{\ftoc{154}}{}
\end{mashsteps}

\begin{fermentationsteps}
\fermentationstep{\ftoc{68}}{7~days}
\end{fermentationsteps}

\end{methodandtiming}

\pagebreak

\begin{ingredientsblock}

\begin{malts}
\malt{Pale Two-row}{\lbtokg{10}}
\malt{Simpsons Crystal Medium}{\lbtokg{1}}
\malt{Dingemans Cara 45 L}{\lbtokg{0.5}}
\end{malts}

\begin{hops}
\hop{\hopcolumbus}{15.5~\%}{60~min}{\oztog{1}}
\hop{\hopwarrior}{17~\%}{30~min}{\oztog{0.5}}
\hop{\hopcentennial}{10.5~\%}{10~min}{\oztog{1}}
\hop{\hopwarrior}{17~\%}{\dryh{}{7~days}}{\oztog{0.5}}
\hop{\hopcentennial}{10.5~\%}{\dryh{}{7~days}}{\oztog{0.5}}
\end{hops}

\begin{yeasts}
\yeast{Wyeast 1056}
\end{yeasts}

\end{ingredientsblock}

\end{recipie}

\begin{recipie}{Biloxi Brewing Company Pale Ale}

\begin{aboutblock}
This is Biloxi Brewing Company's sessionable pale ale that bursts with grapefruit
flavors and aromas from the generous amount of Citra hops used, especially in the
dry hop addition.
\end{aboutblock}

\specifications{\styleamericanpaleale}{\galtol{5}}{1.049}{1.009}{5.2~\%}{37}{}{60~min}

\begin{methodandtiming}
 
\begin{mashsteps}
\mashstep{\ftoc{150}}{75~min}
\mashstep{\ftoc{168}}{10~min}
\end{mashsteps}

\begin{fermentationsteps}
\fermentationstep{\ftoc{67}}{}
\end{fermentationsteps}

\begin{directions}
Age at \ftoc{65} for 30 days.
\end{directions}

\end{methodandtiming}

\pagebreak

\begin{ingredientsblock}

\begin{malts}
\malt{Pale Two-row}{\lbtokg{6.47}}
\malt{Munich}{\lbtokg{1}}
\malt{Carapils}{\oztokg{12.4}}
\malt{Crystal 40 L}{\oztokg{12.4}}
\end{malts}

\begin{hops}
\hop{\hopcitra}{14~\%}{60~min}{\oztog{0.42}}
\hop{\hopcascade}{6.3~\%}{15~min}{\oztog{0.42}}
\hop{\hopcitra}{12.2~\%}{15~min}{\oztog{0.42}}
\hop{\hopcascade}{}{\whirl{}{}}{\oztog{0.42}}
\hop{\hopcitra}{}{\whirl{}{}}{\oztog{0.42}}
\hop{\hopcitra}{}{\dryh{}{}}{\oztog{0.83}}
\hop{\hopcascade}{}{\dryh{}{}}{\oztog{0.42}}

\end{hops}

\begin{yeasts}
\yeast{Fermentis SafAle US-05}
\end{yeasts}

\end{ingredientsblock}

\end{recipie}

\begin{recipie}{Indeed Brewing Co. Day Tripper Pale Ale}

\begin{aboutblock}
Day Tripper is a West Coast-style pale ale out of Minneapolis. Indeed Brewing describes Day
Tripper as having a heady, dank, citrus-laced aroma supported by a complex and subtly sweet
malt backbone.
\end{aboutblock}

\specifications{\styleamericanpaleale}{\galtol{5}}{1.052}{}{5.4~\%}{}{}{90~min}

\begin{methodandtiming}
 
\begin{mashsteps}
\mashstep{\ftoc{150}}{60~min}
\end{mashsteps}

\begin{fermentationsteps}
\fermentationstep{\ftoc{67}}{}
\end{fermentationsteps}

\begin{directions}
Age one week in secondary. Keg, or bottle with \oztog{5} priming sugar and condition
for 2 weeks.
\end{directions}

\end{methodandtiming}

\pagebreak

\begin{ingredientsblock}

\begin{malts}
\malt{Briess Pale Ale}{\lbtokg{5.5}}
\malt{Maris Otter}{\lbtokg{3.5}}
\malt{White Wheat}{\lbtokg{0.75}}
\malt{Briess Caramel 20 L}{\lbtokg{0.5}}
\malt{Briess Carapils}{\lbtokg{0.5}}
\malt{Briess Bonlander Munich 10 L}{\lbtokg{0.25}}

\end{malts}

\begin{hops}
\hop{\hopwillamette}{}{\fwh}{\oztog{0.25}}
\hop{\hopcascade}{}{20~min}{\oztog{1.5}}
\hop{\hopcascade}{}{10~min}{\oztog{1.5}}
\hop{\hopcolumbus}{}{10~min}{\oztog{0.5}}
\hop{\hopsummit}{}{10~min}{\oztog{0.5}}
\hop{\hopcascade}{}{\foh{}}{\oztog{1.5}}
\hop{\hopcolumbus}{}{\foh{}}{\oztog{1.5}}
\hop{\hopsummit}{}{\foh{}}{\oztog{1}}
\hop{\hopcascade}{}{\dryh{}{3~days}}{\oztog{1.5}}
\hop{\hopcolumbus}{}{\dryh{}{3~days}}{\oztog{1.5}}
\hop{\hopsummit}{}{\dryh{}{3~days}}{\oztog{1}}

\end{hops}

\begin{yeasts}
\yeast{Fermentis SafAle US-05 / Wyeast 1272}
\end{yeasts}

\end{ingredientsblock}

\end{recipie}

\begin{recipie}{Lynnwood Brewing Drop Bear APA}

\begin{aboutblock}
Drop Bear APA from Lynnwood Grill \& Brewing (Raleigh, N.C.) is brewed with Mosaic,
El Dorado, and Galaxy hops that are nicely balanced by a mix of two-row, Munich,
and caramel malts.
\end{aboutblock}

\specifications{\styleamericanpaleale}{\galtol{5.5}}{}{}{5.5~\%}{}{\srmtoebc{5.3}}{60~min}

\begin{methodandtiming}
 
\begin{mashsteps}
\mashstep{\ftoc{153}}{60~min}
\end{mashsteps}

\begin{fermentationsteps}
\fermentationstep{\ftoc{67}}{}
\end{fermentationsteps}

\begin{directions}
Sulfate to Chloride at 1:1.
\end{directions}

\end{methodandtiming}

\pagebreak

\begin{ingredientsblock}

\begin{malts}
\malt{Great Western Premium Tow-row}{\lbtokg{11}}
\malt{Briess Bonlander Munich 10 L}{\oztokg{6}}
\malt{Briess Caramel 20 L}{\oztokg{4}}
\malt{Briess Caramel 40 L}{\oztokg{4}}
\end{malts}

\begin{hops}
\hop{\hopmosaic}{12.25~\%}{15~min}{\oztog{0.5}}
\hop{\hopeldorado}{15~\%}{10~min}{\oztog{0.7}}
\hop{\hopgalaxy}{14~\%}{5~min}{\oztog{0.7}}
\hop{\hopeldorado}{15~\%}{\whirl{}{}}{\oztog{0.8}}
\hop{\hopgalaxy}{14~\%}{\whirl{}{}}{\oztog{0.8}}
\hop{\hopmosaic}{12.25~\%}{\whirl{}{}}{\oztog{0.8}}
\hop{\hopgalaxy}{14~\%}{\dryh{}{}}{\oztog{3}}
\hop{\hopeldorado}{15~\%}{\dryh{}{}}{\oztog{1.5}}
\hop{\hopmosaic}{12.25~\%}{\dryh{}{}}{\oztog{1.5}}
\end{hops}

\begin{yeasts}
\yeast{Fermentis SafAle S04 / Wyeast 1056 / White Labs WLP001}
\end{yeasts}

\end{ingredientsblock}

\end{recipie}

\begin{recipie}{Maine Beer Co. Peeper Ale}

\begin{aboutblock}
Peeper Pale Ale was the first recipe the Kleban brothers perfected when they decided
to open their brewery in 2009. Maine Beer Co. (Freeport, Maine) describes Peeper as dry,
clean, and well balanced, with a generous dose of American hops.
\end{aboutblock}

\specifications{\styleamericanpaleale}{\galtol{5}}{1.047}{1.007}{}{}{}{60~min}

\begin{methodandtiming}
 
\begin{mashsteps}
\mashstep{\ftoc{150}}{}
\end{mashsteps}

\begin{fermentationsteps}
\fermentationstep{\ftoc{68}}{}
\end{fermentationsteps}

\end{methodandtiming}

\pagebreak

\begin{ingredientsblock}

\begin{malts}
\malt{Pale Two-row}{\lbtokg{8}}
\malt{Vienna}{\lbtokg{0.5}}
\malt{Red Wheat}{\oztokg{6}}
\malt{Carapils}{\oztokg{6}}
\end{malts}

\begin{hops}
\hop{\hopmagnum}{10~\%}{60~min}{\oztog{0.25}}
\hop{\hopamarillo}{9.2~\%}{10~min}{\oztog{0.15}}
\hop{\hopcascade}{5.5~\%}{10~min}{\oztog{0.2}}
\hop{\hopcentennial}{10~\%}{10~min}{\oztog{0.2}}
\hop{\hopamarillo}{9.2~\%}{\whirl{}{}}{\oztog{1.5}}
\hop{\hopcascade}{5.5~\%}{\whirl{}{}}{\oztog{2}}
\hop{\hopcentennial}{10~\%}{\whirl{}{}}{\oztog{2}}
\hop{\hopamarillo}{9.2~\%}{\dryh{}{}}{\oztog{2.5}}
\hop{\hopcentennial}{10~\%}{\dryh{}{}}{\oztog{2.5}}
\end{hops}

\begin{yeasts}
\yeast{Wyeast 1056}
\end{yeasts}

\end{ingredientsblock}

\end{recipie}

\begin{recipie}{Maplewood Brewing Co. Charlatan American Pale Ale}

\begin{aboutblock}
Charlatan is hopped with Simcoe, Citra, and Centennial, and this pale ale is packed with
tropical flavors like mango, passionfruit, and grapefruit according to Chicago's Maplewood
Brewery.
\end{aboutblock}

\specifications{\styleamericanpaleale}{\galtol{5}}{1.059}{1.012}{6.1~\%}{32}{\srmtoebc{5.1}}{90~min}

\begin{methodandtiming}
 
\begin{mashsteps}
\mashstep{\ftoc{152}}{45~min}
\mashstep{\ftoc{170}}{Sparge}
\end{mashsteps}

\begin{directions}
Target mash pH 5.3. At flameout, use a large spoon or stirrer
and create a whirlpool to cool the wort for a few minutes. The closer you can get to \ftoc{190}
the better, but anything under \ftoc{200} will work. Add the whirlpool hops and let rest 10
minutes, then whirlpool again for a few minutes to create a nice compact trub / hop pile.
When the fermentation is about 1~°P from FG, add the dry hops. After 3--7 days after FG, crash
and rack off of the hops. Carbonate to 2.55 volumes \ce{CO2}.

Water profile: use Gypsum and Calcium Chloride to adjust to: 115--125 ppm Calcium,
65--75 ppm Chloride, 145--155 ppm Sulfate. Finished \ce{SO4}/\ce{Cl} ratio: 2.3--2.5.
Use Lactic Acid (88~\% strength) to adjust water pH, we use a 0.55~ml/gal rate, and we
adjust both the mash water and sparge water separately (based on the volume of each).
If you do not have the capability to adjust mash and sparge water separately, you may
make all of the mineral / acid additions based on your mash water volume. Always use
de-chlorinated water.
\end{directions}

\end{methodandtiming}

\begin{ingredientsblock}

\begin{malts}
\malt{Rahr Standard Two-row}{\lbtokg{8}}
\malt{Weyermann Munich I}{\lbtokg{0.75}}
\malt{Crisp Dextrin}{\lbtokg{0.33}}
\malt{Crisp Cara}{\lbtokg{0.33}}
\malt{Rahr White Wheat}{\lbtokg{0.33}}
\end{malts}

\begin{hops}
\hop{\hopcolumbus}{16.5~\%}{90~min}{\oztog{0.1}}
\hop{\hopsimcoe}{13~\%}{15~min}{\oztog{0.2}}
\hop{\hopcitra}{12.5~\%}{10~min}{\oztog{0.1}}
\hop{\hopcentennial}{10~\%}{10~min}{\oztog{0.1}}
\hop{\hopsimcoe}{13~\%}{10~min}{\oztog{0.2}}
\hop{Whirlfloc Tablet}{}{10~min}{1}
\hop{Yeast Nutrient}{}{10~min}{1}
\hop{\hopcitra}{12.5~\%}{\whirl{}{10~min}}{\oztog{0.33}}
\hop{\hopcentennial}{10~\%}{\whirl{}{10~min}}{\oztog{0.33}}
\hop{\hopsimcoe}{13~\%}{\whirl{}{10~min}}{\oztog{0.165}}
\hop{\hopcitra}{12.5~\%}{\dryh{}{3~days}}{\oztog{1.25}}
\hop{\hopcentennial}{10~\%}{\dryh{}{3~days}}{\oztog{0.75}}
\hop{\hopsimcoe}{13~\%}{\dryh{}{3~days}}{\oztog{0.75}}
\end{hops}

\begin{yeasts}
\yeast{Wyeast 1318}
\end{yeasts}

\end{ingredientsblock}

\end{recipie}

% -----------------------------------------------------------------------------
\stylecategory{Other American Ale}
\stylesection{\styleamericanamberale}

% -----------------------------------------------------------------------------
\begin{recipie}{Hoppy New Year Amber Ale}
% -----------------------------------------------------------------------------

\begin{aboutblock}
Jason Bryant of Reston, VA, member of the Wort Hogs, won a silver medal in
Category \#11: Amber \& Brown American Ale with an American Amber Ale during the
2019 National Homebrew Competition Final Round in Providence, RI. Bryant's American
Amber Ale was chosen as the runner-up entry among 281 entries in the category.
\end{aboutblock}

\specifications{\styleamericanamberale}{\galtol{6.5}}{1.058}{1.011}{5.7~\%}{65.3}{\srmtoebc{12.3}}{90~min}{2.5}

\begin{methodandtiming}
 
\begin{mashsteps}
\mashstep{\ftoc{156}}{}
\end{mashsteps}

\begin{fermentationsteps}
\fermentationstep{\ftoc{65}}{}
\end{fermentationsteps}

\begin{directions}
Acidify sparge water to pH 5.6 with lactic acid. Target pH of 5.2 to 5.5.
Chill to \ftoc{195} before adding the whirlpool hops.
\end{directions}

\end{methodandtiming}

\pagebreak

\begin{ingredientsblock}

\begin{malts}
\malt{Pilsner}{\lbtokg{8.5}}
\malt{Pale}{\lbtokg{1.8}}
\malt{Weyermann Munich II}{\oztokg{11.8}}
\malt{Aromatic}{\oztokg{7.2}}
\malt{Weyermann CARAAMBER}{\oztokg{7.2}}
\malt{Caramel / Crystal 20 L}{\oztokg{7.2}}
\malt{Caramel / Crystal 120 L}{\oztokg{7.2}}
\malt{Weyermann CARARED}{\oztokg{7.2}}
\malt{Briess Caravienne}{\oztokg{7.2}}
\malt{Simpsons Dark Crystal}{\oztokg{7.2}}
\end{malts}

\begin{hops}
\hop{\hopsummit}{19~\%}{90~min}{\oztog{0.2}}
\hop{\hopcitra}{11.1~\%}{5~min}{\oztog{0.4}}
\hop{\hopidahoseven}{11.6~\%}{5~min}{\oztog{0.4}}
\hop{\hopcitra}{11.1~\%}{\whirl{}{15~min}}{\oztog{1.2}}
\hop{\hopidahoseven}{11.6~\%}{\whirl{}{15~min}}{\oztog{1.2}}
\hop{\hopsummit}{19~\%}{\whirl{}{15~min}}{\oztog{0.7}}
\hop{\hoppalisade ~Cryo}{14~\%}{\whirl{}{15~min}}{\oztog{0.4}}
\end{hops}

\begin{yeasts}
\yeast{American Ale}
\end{yeasts}

\end{ingredientsblock}

\end{recipie}

% -----------------------------------------------------------------------------
\begin{recipie}{Iron Monk Brewing Co. Velvet Antler Amber}
% -----------------------------------------------------------------------------

\begin{aboutblock}
Iron Monk Brewing's American amber ale is very malt-forward and smooth. They hear from
fans all the time that this is their all-time favorite beer, and we think you’ll agree
once you brew some yourself.
\end{aboutblock}

\specifications{\styleamericanamberale}{\galtol{5}}{1.045}{}{4.6~\%}{14}{}{60~min}{}

\begin{methodandtiming}
 
\begin{mashsteps}
\mashstep{\ftoc{154}}{60~min}
\mashstep{\ftoc{170}}{10~min}
\end{mashsteps}

\begin{fermentationsteps}
\fermentationstep{\ftoc{68}}{}
\end{fermentationsteps}

\end{methodandtiming}

\pagebreak

\begin{ingredientsblock}

\begin{malts}
\malt{Two-row}{\lbtokg{7}}
\malt{Caramel / Crystal 60 L}{\lbtokg{2}}
\malt{Munich Light}{\lbtokg{0.5}}
\malt{Red}{\lbtokg{0.5}}
\malt{Flaked Wheat}{\lbtokg{0.5}}
\malt{Flaked Barley}{\lbtokg{0.5}}
\end{malts}

\begin{hops}
\hop{\hopsimcoe}{}{60~min}{\oztog{0.4}}
\hop{\hopcascade}{}{\foh{}}{\oztog{0.3}}
\end{hops}

\begin{yeasts}
\yeast{Fermentis SafAle S-05}
\end{yeasts}

\end{ingredientsblock}

% -----------------------------------------------------------------------------
\begin{recipie}{Kilauea Punch American Amber Ale}
% -----------------------------------------------------------------------------

\begin{aboutblock}
Derek Springer of San Marcos, CA, member of the Society of Barley Engineers,
won a bronze medal in Category \#11: Amber \& Brown American Ale with an
American Amber Ale during the 2019 National Homebrew Competition Final Round
in Providence, RI. Springer's American Amber Ale was chosen as a top three
entry among 281 entries in the category.
\end{aboutblock}

\specifications{\styleamericanamberale}{\galtol{5.5}}{1.057}{1.015}{5.6~\%}{180.9}{\srmtoebc{14.7}}{}{}

\begin{methodandtiming}
 
\begin{mashsteps}
\mashstep{\ftoc{153}}{60~min}
\mashstep{\ftoc{168}}{10~min}
\end{mashsteps}

\begin{fermentationsteps}
\fermentationstep{\ftoc{65}}{1~day}
\fermentationstep{\ftoc{68}}{6~days}
\fermentationstep{\ftoc{32}}{1~day}
\end{fermentationsteps}

\begin{directions}
After 4 days add dry hops. After fermentation add gelatin and a day after that
keg and carbonate.
\end{directions}

\end{methodandtiming}

\pagebreak

\begin{ingredientsblock}

\begin{malts}
\malt{Great Western Premium Two-row}{\lbtokg{4}}
\malt{Crisp Maris Otter}{\lbtokg{4}}
\malt{Weyermann Munich II}{\oztog{16}}
\malt{Dingemans}{\oztog{8}}
\malt{Great Western Crystal 120}{\oztog{8}}
\malt{Great Western Crystal 40}{\oztog{8}}
\malt{Great Western Crystal 75}{\oztog{8}}
\end{malts}

\begin{hops}
\hop{\hopgalaxy}{18.1~\%}{15~min}{\oztog{1}}
\hop{\hopcentennial}{10.4~\%}{5~min}{\oztog{1}}
\hop{\hopmotueka}{7.3~\%}{5~min}{\oztog{1}}
\hop{\hopcentennial}{10.4~\%}{\dryh{}{3~days}}{\oztog{1}}
\hop{\hopgalaxy}{18.1~\%}{\dryh{}{3~days}}{\oztog{1}}
\hop{\hopmotueka}{7.3~\%}{\dryh{}{3~days}}{\oztog{1}}
\end{hops}

\begin{yeasts}
\yeast{White Labs WLP090}
\end{yeasts}

\end{ingredientsblock}

\end{recipie}

% -----------------------------------------------------------------------------
\begin{recipie}{Possibility and Promise Pacific Amber}
% -----------------------------------------------------------------------------

\begin{aboutblock}
Justin Westcott of Federal Way, WA, member of the Browns Point Homebrew Club, won
a gold medal in Category \#11: Amber \& Brown American Ale with an American Amber
Ale during the 2019 National Homebrew Competition Final Round in Providence, RI.
Westcott's Amber \& Brown American Ale was chosen as the best among 281 entries in
the category.
\end{aboutblock}

\specifications{\styleamericanamberale}{\galtol{5.2}}{1.057}{1.011}{6.1~\%}{35}{\srmtoebc{11}}{60~min}{2.6}

\begin{methodandtiming}
 
\begin{mashsteps}
\mashstep{\ftoc{152}}{60~min}
\mashstep{\ftoc{168}}{Mashout}
\end{mashsteps}

\begin{fermentationsteps}
\fermentationstep{\ftoc{64}}{14~days}
\end{fermentationsteps}

\begin{directions}
Condition at \ftoc{65} for 30 days.
\end{directions}

\end{methodandtiming}

\pagebreak

\begin{ingredientsblock}

\begin{malts}
\malt{Pale}{\lbtokg{10}}
\malt{Weyermann CARARED}{\lbtokg{2.2}}
\malt{Caramel / Crystal 80 L}{\oztog{8}}
\malt{Weyermann Melanoidin}{\oztog{8}}
\end{malts}

\begin{hops}
\hop{\hopcascade}{7.8~\%}{60~min}{\oztog{0.5}}
\hop{\hopcentennial}{8~\%}{20~min}{\oztog{0.5}}
\hop{\hopcascade}{7.8~\%}{15~min}{\oztog{0.5}}
\hop{Whirlfloc Tablet}{}{15~min}{1}
\hop{Yeast Nnutrient}{}{15~min}{2.2~g}
\hop{\hopamarillo}{7.7~\%}{5~min}{\oztog{1}}
\hop{\hopcentennial}{8~\%}{5~min}{\oztog{0.5}}
\end{hops}

\begin{yeasts}
\yeast{Imperial Yeast A07}
\end{yeasts}

\end{ingredientsblock}

\end{recipie}

\end{recipie}
\stylesection{\styleamericanbrownale}

% -----------------------------------------------------------------------------
\begin{recipe}{Betty's Brown Ale}
% -----------------------------------------------------------------------------

\begin{aboutblock}
Recipe by Jim Elmhirst.
\sourcezymurgy{January / February 2019}
\end{aboutblock}

\specifications{\styleamericanbrownale}{\galtol{5}}{1.056}{1.014}{5.5~\%}{28}{\srmtoebc{20}}{60~min}{}

\begin{methodandtiming}

\begin{mashsteps}
\mashstep{\ftoc{152}}{60~min}
\mashstep{\ftoc{168}}{Mash out}
\end{mashsteps}

\begin{fermentationsteps}
\fermentationstep{\ftoc{66}}{}
\end{fermentationsteps}

\end{methodandtiming}

\recipebreak

\begin{ingredientsblock}

\begin{malts}
\malt{\maltmarisotter}{\lbtokg{8.75}}
\malt{Caramel / Crystal 60 L}{\oztokg{12}}
\malt{Briess Victory}{\oztokg{12}}
\malt{Weyermann CARAMUNICH I}{\oztokg{8}}
\malt{Weyermann Munich II}{\oztokg{8}}
\malt{\maltchocolate}{\oztokg{4}}
\end{malts}

\begin{hops}
\hop{\hopwarrior}{15~\%}{60~min}{\oztog{0.5}}
\end{hops}

\singleyeast{White Labs WLP002}

\end{ingredientsblock}

\end{recipe}

% -----------------------------------------------------------------------------
\begin{recipe}{Big Brown Ale}
% -----------------------------------------------------------------------------

\begin{aboutblock}
Recipe by Brian Linder of Tewksbury, MA. Gold medal in Category 10: American
Ale during the 2014 National Homebrew Competition in Grand Rapids, MI.
\sourceaha
\end{aboutblock}

\specifications{\styleamericanbrownale}{\galtol{6}}{1.062}{1.014}{6.3~\%}{}{}{60~min}{}

\begin{methodandtiming}

\begin{mashsteps}
\mashstep{\ftoc{145}}{30~min}
\mashstep{\ftoc{158}}{40~min}
\mashstep{\ftoc{170}}{Mash out}
\end{mashsteps}

\begin{fermentationsteps}
\fermentationstep{\ftoc{68}}{7~days}
\fermentationstep{\ftoc{65}}{14~days}
\end{fermentationsteps}

\end{methodandtiming}

\recipebreak

\begin{ingredientsblock}

\begin{malts}
\malt{\maltpale}{\lbtokg{9.6}}
\malt{\maltmunich}{\lbtokg{1.8}}
\malt{Caramel / Crystal 60 L}{\lbtokg{0.6}}
\malt{\maltchocolate}{\lbtokg{0.6}}
\end{malts}

\begin{hops}
\hop{\hopcascade}{7.3~\%}{60~min}{\oztog{0.3}}
\hop{\hopcascade}{7.3~\%}{30~min}{\oztog{0.3}}
\hop{\hopwillamette}{5.3~\%}{15~min}{\oztog{1}}
\hop{\hopcascade}{7.3~\%}{15~min}{\oztog{1}}
\hop{\hopwillamette}{5.3~\%}{5~min}{\oztog{0.5}}
\hop{\hopcascade}{7.3~\%}{5~min}{\oztog{0.5}}
\hop{\hopwillamette}{5.3~\%}{1~min}{\oztog{0.3}}
\hop{\hopcascade}{7.3~\%}{1~min}{\oztog{0.3}}
\end{hops}

\singleyeast{Wyeast 1056}

\end{ingredientsblock}

\end{recipe}

% -----------------------------------------------------------------------------
\begin{recipe}{Cookie Dough Brown Ale}
% -----------------------------------------------------------------------------

\begin{aboutblock}
Recipe by Mark Ranes of Turlock, CA. Silver medal in Category 11: Amber and Brown
American Ale during the 2018 National Homebrew Competition in Portland, OR.
\sourceaha
\end{aboutblock}

\specifications{\styleamericanbrownale}{\galtol{10}}{1.074}{1.015}{7.7~\%}{38}{\srmtoebc{25}}{60~min}{}

\begin{methodandtiming}

\begin{mashsteps}
\mashstep{\ftoc{149}}{60~min}
\end{mashsteps}

\begin{fermentationsteps}
\fermentationstep{\ftoc{66}}{}
\end{fermentationsteps}

\begin{directions}
Add dry hops on day 6.
\end{directions}

\end{methodandtiming}

\recipebreak

\begin{ingredientsblock}

\begin{malts}
\malt{Fawcett Golden Promise Pale Ale}{\lbtokg{13}}
\malt{Briess Pale Ale}{\lbtokg{8}}
\malt{Crisp Brown}{\lbtokg{3}}
\malt{Briess Carapils}{\lbtokg{1}}
\malt{Briess Chocolate}{\lbtokg{0.5}}
\malt{Weyermann CARAFA II}{\oztokg{3}}
\malt{Great Western Crystal 60}{\lbtokg{1}}
\end{malts}

\begin{hops}
\hop{\hopnorthernbrewer}{}{60~min}{\oztog{1.5}}
\hop{\hopcentennial}{}{20~min}{\oztog{1.5}}
\hop{Dark Brown Sugar}{}{10~min}{\lbtokg{2}}
\hop{Dark Maple Syrup}{}{10~min}{\lbtokg{1}}
\hop{Raisins}{}{10~min}{\oztog{22}}
\hop{\hopnorthernbrewer}{}{\foh{}{}}{\oztog{2}}
\hop{\hopcentennial}{}{\foh{}{}}{\oztog{2}}
\hop{\hopnorthernbrewer}{}{\dryh{}{}}{\oztog{2}}
\hop{\hopcascade}{}{\dryh{}{}}{\oztog{2}}
\end{hops}

\singleyeast{White Labs WLP001}

\end{ingredientsblock}

\end{recipe}

% -----------------------------------------------------------------------------
\begin{recipe}{Dogfish Head Indian Brown Ale Clone}
% -----------------------------------------------------------------------------

\begin{aboutblock}
Recipe by Amahl Turczyn. \sourcezymurgy{July / August 2011}
\end{aboutblock}

\specifications{\styleamericanbrownale}{\galtol{5.5}}{1.072}{}{}{21}{\srmtoebc{22}}{90~min}{}

\begin{methodandtiming}

\begin{mashsteps}
\mashstep{\ftoc{152}}{60~min}
\mashstep{\ftoc{168}}{10~min}
\end{mashsteps}

\begin{fermentationsteps}
\fermentationstep{\ftoc{70}}{Full attenuation}
\fermentationstep{\ftoc{60}}{Transfer to secondary; 7~days}
\end{fermentationsteps}

\end{methodandtiming}

\recipebreak

\begin{ingredientsblock}

\begin{malts}
\malt{\maltpale}{\lbtokg{12}}
\malt{Amber}{\oztokg{10}}
\malt{Caramel / Crystal 60 L}{\oztokg{10}}
\malt{Roasted Barley}{\oztokg{2}}
\end{malts}

\begin{hops}
\hop{Caramelized / Brown Sugar}{}{}{\oztog{8}}
\hop{\hopwarrior}{16~\%}{60~min}{\oztog{0.5}}
\hop{\hopvanguard}{4.5~\%}{\foh{}{}}{\oztog{1}}
\end{hops}

\singleyeast{Wyeast 1187}

\end{ingredientsblock}

\end{recipe}

% -----------------------------------------------------------------------------
\begin{recipe}{Janet's Brown Ale}
% -----------------------------------------------------------------------------

\begin{aboutblock}
Recipe by Mike McDole. Gold medal in Category 10: Brown Ale at the American
Homebrewers Assocation National Homebrew Competition in 2004.
\sourceaha
\end{aboutblock}

\specifications{\styleamericanbrownale}{\galtol{12}}{1.074}{1.018}{7.35~\%}{}{}{60~min}{2.4}

\begin{methodandtiming}

\begin{mashsteps}
\mashstep{\ftoc{154}}{30~min}
\mashstep{\ftoc{170}}{15~min}
\end{mashsteps}

\begin{fermentationsteps}
\fermentationstep{\ftoc{68}}{7~days}
\fermentationstep{\ftoc{70}}{9~days}
\end{fermentationsteps}

\begin{directions}
Water adjustment: \waterprofile{110}{18}{17}{350}{50}{}. Cold condition for 1 month
and 14 days.
\end{directions}

\end{methodandtiming}

\recipebreak

\begin{ingredientsblock}

\begin{malts}
\malt{\maltpale}{\lbtokg{27.5}}
\malt{\maltcarapils}{\lbtokg{3}}
\malt{Caramel / Crystal 40 L}{\lbtokg{2.5}}
\malt{\maltwheat}{\lbtokg{2}}
\malt{\maltchocolate}{\lbtokg{1}}
\end{malts}

\begin{hops}
\hop{Corn Sugar}{}{}{\lbtokg{1}}
\hop{\hopnorthernbrewer}{5.1~\%}{Mash}{\oztog{3}}
\hop{\hopnorthernbrewer}{5.1~\%}{60~min}{\oztog{3}}
\hop{\hopnorthernbrewer}{5.1~\%}{15~min}{\oztog{2}}
\hop{\hopcascade}{5.6~\%}{10~min}{\oztog{3}}
\hop{\hopcascade}{5.6~\%}{\foh{}{}}{\oztog{4}}
\hop{\hopcentennial}{10.5~\%}{\dryh{}{7~days}}{\oztog{4}}
\end{hops}

\singleyeast{White Labs WLP001}

\end{ingredientsblock}

\end{recipe}

% -----------------------------------------------------------------------------
\begin{recipe}{VCR Brown Ale}
% -----------------------------------------------------------------------------

\begin{aboutblock}
Recipe by Rob Hardisty of Colleyville, TX. Gold medal in Category 11: Amber and
Brown American Ale during the 2017 National Homebrew Competition in Minneapolis, MN.
\sourceaha
\end{aboutblock}

\specifications{\styleamericanbrownale}{\galtol{5}}{1.053}{1.012}{}{}{}{60~min}{2.4}

\begin{methodandtiming}

\begin{mashsteps}
\mashstep{\ftoc{145}}{30~min}
\mashstep{\ftoc{162}}{30~min}
\mashstep{\ftoc{170}}{10~min}
\end{mashsteps}

\begin{fermentationsteps}
\fermentationstep{\ftoc{64}}{7~days}
\end{fermentationsteps}

\begin{directions}
Water adjustment: \waterprofile{78}{1}{13}{80}{93}{}.
\end{directions}

\end{methodandtiming}

\recipebreak

\begin{ingredientsblock}

\begin{malts}
\malt{\maltmarisotter}{\lbtokg{6.5}}
\malt{\maltmunich}{\lbtokg{2.5}}
\malt{Caramel / Crystal 60 L}{\lbtokg{0.6}}
\malt{\maltchocolate}{\lbtokg{0.37}}
\malt{Weyermann CARAFA SPECIAL II}{\oztokg{1.4}}
\end{malts}

\begin{hops}
\hop{\hopmagnum}{12.4~\%}{50~min}{\oztog{0.25}}
\hop{\hopwillamette}{4.6~\%}{40~min}{\oztog{0.6}}
\hop{\hopcolumbus}{15~\%}{40~min}{\oztog{0.25}}
\hop{\hopwillamette}{4.6~\%}{30~min}{\oztog{0.6}}
\hop{\hopwillamette}{4.6~\%}{15~min}{\oztog{0.7}}
\hop{\hopchinook}{11.8~\%}{\foh{}{}}{\oztog{1}}
\end{hops}

\singleyeast{Wyeast 1450}

\end{ingredientsblock}

\end{recipe}


% -----------------------------------------------------------------------------
\stylecategory{English Brown Ale}
\stylesection{\styledarkmild}

% -----------------------------------------------------------------------------
\begin{recipe}{Mike Dawson's Wild Rice Mild}
% -----------------------------------------------------------------------------

\begin{aboutblock}
John DiSpirito of East Meadow, NY, member of the Handgrenades Homebrew and Craft Beer
Club, won a silver medal in Category \#12: Brown British Beer with a Dark Mild during
the 2019 National Homebrew Competition Final Round in Providence, RI. DiSpirito's
Dark Mild was chosen as the runner-up entry among 259 entries in the category.
\sourceaha
\end{aboutblock}

\specifications{\styledarkmild}{\galtol{7}}{1.046}{1.010}{4.1~\%}{14.8}{\srmtoebc{23.4}}{75~min}{1.5}

\begin{methodandtiming}
 
\begin{mashsteps}
\mashstep{\ftoc{152}}{60~min}
\mashstep{\ftoc{170}}{Raise to over 15~min; mashout}
\end{mashsteps}

\begin{fermentationsteps}
\fermentationstep{\ftoc{66}}{2~days}
\fermentationstep{\ftoc{72}}{Free raise to; until fully attenuated}
\end{fermentationsteps}

\begin{directions}
Target a mash pH of 5.2.
\end{directions}

\end{methodandtiming}

\recipebreak

\begin{ingredientsblock}

\begin{malts}
\malt{Maris Otter}{\lbtokg{8.5}}
\malt{Flaked Rice}{\lbtokg{1.4}}
\malt{Simpsons Crystal Dark}{\oztog{13.8}}
\malt{Chateau Black Nature}{\oztog{6.1}}
\end{malts}

\begin{hops}
\hop{\hopeastkentgolding}{5~\%}{60~min}{\oztog{1}}
\end{hops}

\singleyeast{Wyeast 1968}

\end{ingredientsblock}

\end{recipe}

% -----------------------------------------------------------------------------
\begin{recipe}{Mike's Mild}
% -----------------------------------------------------------------------------

\begin{aboutblock}
Mike's Mild, courtesy of Mike Volpe, is a standard ale with a malty palate. The
East Kent Golding additions add only the bare minimum bitterness needed to balance
this malty ale out; roasty malt lovers are sure to be pleased if you stick to
the recipe on this one. \sourcezymurgy{November / December 2019}
\end{aboutblock}

\specifications{\styledarkmild}{\galtol{5}}{1.045}{}{}{}{}{60~min}{}

\begin{methodandtiming}
 
\begin{mashsteps}
\mashstep{\ftoc{153}}{60~min}
\end{mashsteps}

\begin{fermentationsteps}
\fermentationstep{\ftoc{67}}{}
\end{fermentationsteps}

\end{methodandtiming}

\recipebreak

\begin{ingredientsblock}

\begin{malts}
\malt{Maris Otter}{\lbtokg{4}}
\malt{Pale}{\lbtokg{3.5}}
\malt{Weyermann CARAFOAM}{\lbtokg{0.5}}
\malt{Simpsons Crystal Dark}{\lbtokg{0.25}}
\end{malts}

\begin{hops}
\hop{\hopeastkentgolding}{5~\%}{60~min}{\oztog{0.5}}
\hop{\hopeastkentgolding}{5~\%}{25~min}{\oztog{1.25}}
\hop{\hopeastkentgolding}{5~\%}{5~min}{\oztog{1.25}}
\end{hops}

\singleyeast{Wyeast 1469}

\end{ingredientsblock}

\end{recipe}

% -----------------------------------------------------------------------------
\begin{recipe}{Missouri Beer Company English Dark Mild}
% -----------------------------------------------------------------------------

\begin{aboutblock}
This malt-focused session ale from Missouri Beer Company is dark, low gravity
and really refreshing. The low ABV makes the English dark mild suited for session
drinking. \sourceaha
\end{aboutblock}

\specifications{\styledarkmild}{\galtol{5}}{1.044}{1.012}{4.2~\%}{15}{\srmtoebc{19}}{60~min}{}

\begin{methodandtiming}
 
\begin{mashsteps}
\mashstep{\ftoc{150}}{60~min}
\mashstep{\ftoc{170}}{Mashout}
\end{mashsteps}

\begin{fermentationsteps}
\fermentationstep{\ftoc{65}}{}
\end{fermentationsteps}

\end{methodandtiming}

\recipebreak

\begin{ingredientsblock}

\begin{malts}
\malt{Maris Otter}{\lbtokg{6.6}}
\malt{Brown}{\lbtokg{0.57}}
\malt{Amber}{\lbtokg{0.57}}
\malt{Chocolate Rye}{\lbtokg{0.57}}
\end{malts}

\begin{hops}
\hop{\hopcolumbus}{16.4~\%}{60~min}{\oztog{0.2}}
\end{hops}

\singleyeast{Wyeast 1968}

\begin{twists}
\twist{Coconut (Medium Toast)}{Secondary}{\lbtokg{1.5}}
\twist{Whole Arabica Coffee Beans}{Secondary}{\oztog{0.5}}
\end{twists}

\end{ingredientsblock}

\end{recipe}


\stylesection{\stylebritishbrownale}

% -----------------------------------------------------------------------------
\begin{recipe}{Back Forty Beer Co. Truck Stop Honey Brown Ale Clone}
% -----------------------------------------------------------------------------

\begin{aboutblock}
Traditional English brown ale with a slightly sweet finish. \sourceaha
\end{aboutblock}

\specifications{\stylebritishbrownale}{\galtol{5}}{1.059}{1.012}{6.23~\%}{15.4}{\srmtoebc{15.85}}{60~min}{}

\begin{methodandtiming}
 
\begin{mashsteps}
\mashstep{\ftoc{160}}{}
\end{mashsteps}

\begin{fermentationsteps}
\fermentationstep{\ftoc{68}}{}
\end{fermentationsteps}

\begin{directions}
After fermentation, cold crash at \ftoc{32}.
\end{directions}

\end{methodandtiming}

\recipebreak

\begin{ingredientsblock}

\begin{malts}
\malt{Two-row}{\lbtokg{7}}
\malt{Gambrinus Hone}{\lbtokg{1}}
\malt{Caramel / Crystal 20 L}{\lbtokg{0.5}}
\malt{Dark Chocolate}{\lbtokg{0.2}}
\end{malts}

\begin{hops}
\hop{\hopapollo}{}{60~min}{\oztog{0.12}}
\hop{\hopwillamette}{}{10~min}{\oztog{0.3}}
\hop{Wildflower Honey}{}{5~min}{\lbtokg{0.5}}
\end{hops}

\singleyeast{Fermentis SafAle US-05}

\end{ingredientsblock}

\end{recipe}


% -----------------------------------------------------------------------------
\stylecategory{Porter}
\stylesection{English Porter}

% -----------------------------------------------------------------------------
\begin{recipe}{Holy City Brewing Pluff Mud Porter Clone} % rechecked
% -----------------------------------------------------------------------------

\begin{aboutblock}
Like a classic porter, with subtle chocolate notes and a silky finish.
\sourceaha
\end{aboutblock}

\specifications{\styleenglishporter}{\galtol{5}}{1.061}{1.021}{5.5~\%}{21}{\srmtoebc{24}}{60~min}{}

\begin{methodandtiming}
 
\begin{mashsteps}
\mashstep{\ftoc{149}}{45~min}
\end{mashsteps}

\end{methodandtiming}

\recipebreak

\begin{ingredientsblock}

\begin{malts}
\malt{Pale}{\lbtokg{6.6}}
\malt{Caramel / Crystal 20 L}{\lbtokg{2.4}}
\malt{Caramel / Crystal 60 L}{\lbtokg{1.2}}
\malt{Weyermann Munich II}{\lbtokg{1.2}}
\malt{Weyermann CARAFA I}{\oztokg{9.5}}
\end{malts}

\begin{hops}
\hop{\hopnorthernbrewer}{8.5~\%}{60~min}{\oztog{0.78}}
\end{hops}

\singleyeast{British Ale}

\end{ingredientsblock}

\end{recipe}

% -----------------------------------------------------------------------------
\begin{recipe}{Homebrew Challenge English Porter} % rechecked
% -----------------------------------------------------------------------------

\begin{aboutblock}
Recipe by Martin Keen.
\sourcehomebrewchallenge
\end{aboutblock}

\specifications{\styleenglishporter}{\galtol{5}}{1.050}{1.010}{5~\%}{32}{\srmtoebc{25}}{60~min}{}

\begin{methodandtiming}

\begin{mashsteps}
\mashstep{\ftoc{152}}{60~min}
\end{mashsteps}

\begin{fermentationsteps}
\fermentationstep{\ftoc{65}}{}
\end{fermentationsteps}

\end{methodandtiming}

\recipebreak

\begin{ingredientsblock}

\begin{malts}
\malt{Maris Otter}{\lbtokg{7}}
\malt{Brown}{\lbtokg{1}}
\malt{Caramel / Crystal 45 L}{\lbtokg{1}}
\malt{Chocolate}{\lbtokg{0.5}}
\malt{Caramel / Crystal 80 L}{\lbtokg{0.25}}
\end{malts}

\begin{hops}
\hop{\hopeastkentgolding}{}{60~min}{\oztog{1.5}}
\hop{\hopeastkentgolding}{}{10~min}{\oztog{0.5}}
\end{hops}

\singleyeast{Wyeast 1028}

\end{ingredientsblock}

\end{recipe}

% -----------------------------------------------------------------------------
\begin{recipe}{Peppermint Porter} % rechecked
% -----------------------------------------------------------------------------

\begin{aboutblock}
Recipe by Mike Volpe. Balanced with roasty chocolate malts, coffee beans and
flaked oats. \sourcezymurgy{November / December 2019}
\end{aboutblock}

\specifications{\styleenglishporter}{\galtol{5}}{1.068}{1.014}{7.2~\%}{29}{\srmtoebc{22}}{75~min}{}

\begin{methodandtiming}
 
\begin{mashsteps}
\mashstep{\ftoc{156}}{60~min}
\end{mashsteps}

\begin{fermentationsteps}
\fermentationstep{\ftoc{64}}{Full attenuation}
\fermentationstep{\ftoc{67}}{Transfer to secondary; 1~day}
\end{fermentationsteps}

\end{methodandtiming}

\recipebreak

\begin{ingredientsblock}

\begin{malts}
\malt{Pale}{\lbtokg{5.5}}
\malt{Maris Otter}{\lbtokg{4}}
\malt{Flaked Oats}{\lbtokg{2}}
\malt{Chocolate}{\lbtokg{0.5}}
\malt{Dark Crystal}{\oztokg{6}}
\end{malts}

\begin{hops}
\hop{\hopwillamette}{5~\%}{60~min}{\oztog{1}}
\hop{\hopeastkentgolding}{5~\%}{25~min}{\oztog{1.25}}
\hop{\hopeastkentgolding}{5~\%}{5~min}{\oztog{1.25}}
\hop{\hopmthood}{5.1~\%}{\foh{}}{\oztog{1.5}}
\hop{Peppermint Tea Bags}{}{\foh{5~min}}{15}
\end{hops}

\singleyeast{Lallemand Nottingham High Performance Ale}

\begin{twists}
\twist{Whole Coffee Beans}{Secondary}{\oztog{2}}
\end{twists}

\end{ingredientsblock}

\end{recipe}

% -----------------------------------------------------------------------------
\begin{recipe}[Sket English Porter]{'Sket English Porter} % rechecked
% -----------------------------------------------------------------------------

\begin{aboutblock}
Recipe by Ryan Celia of Easton, PA. Gold medal in Category 12: Brown British
Beer during the 2019 National Homebrew Competition Final Round in Providence, RI.
\sourceaha
\end{aboutblock}

\specifications{\styleenglishporter}{\galtol{6}}{1.049}{1.013}{4.7~\%}{30}{\srmtoebc{29}}{60~min}{2.1}

\begin{methodandtiming}
 
\begin{mashsteps}
\mashstep{\ftoc{155}}{60~min}
\end{mashsteps}

\begin{fermentationsteps}
\fermentationstep{\ftoc{63}}{Pitch}
\fermentationstep{\ftoc{65}}{3~days}
\fermentationstep{\ftoc{67}}{3~days}
\fermentationstep{\ftoc{70}}{Full attenuation}
\end{fermentationsteps}

\end{methodandtiming}

\recipebreak

\begin{ingredientsblock}

\begin{malts}
\malt{Maris Otter}{\lbtokg{6.5}}
\malt{Brown}{\lbtokg{2}}
\malt{Munich}{\lbtokg{1}}
\malt{Caramel / Crystal 90 L}{\lbtokg{0.8}}
\malt{Amber}{\lbtokg{0.7}}
\malt{Chocolate}{\oztokg{8}}
\end{malts}

\begin{hops}
\hop{\hopchallenger}{5.1~\%}{60~min}{\oztog{1.05}}
\hop{\hopwillamette}{5.8~\%}{20~min}{\oztog{1}}
\end{hops}

\singleyeast{Wyeast 1335}

\end{ingredientsblock}

\end{recipe}

\stylesection{\styleamericanporter}

% -----------------------------------------------------------------------------
\begin{recipe}{Crow Peak Brewing Co. Pile 'O Dirt Porter Clone}
% -----------------------------------------------------------------------------

\begin{aboutblock}
Very dark in color with a nice tan head and malt complexity. \sourceaha
\end{aboutblock}

\specifications{\styleamericanporter}{\galtol{5.5}}{1.062}{1.015}{5.5~\%}{20}{\srmtoebc{35}}{60~min}{}

\begin{methodandtiming}
 
\begin{mashsteps}
\mashstep{\ftoc{156}}{60~min}
\end{mashsteps}

\begin{fermentationsteps}
\fermentationstep{\ftoc{63}}{}
\end{fermentationsteps}

\end{methodandtiming}

\recipebreak

\begin{ingredientsblock}

\begin{malts}
\malt{Briess Pale Ale}{\lbtokg{9}}
\malt{Briess Bonlander Munich 10 L}{\lbtokg{1.5}}
\malt{Briess Chocolate}{\lbtokg{0.5}}
\malt{Briess Carapils}{\lbtokg{0.5}}
\malt{Briess Extra Special}{\lbtokg{0.5}}
\malt{Briess Blackprinz}{\lbtokg{0.5}}
\end{malts}

\begin{hops}
\hop{\hopperle}{8.2~\%}{\fwh}{\oztog{0.5}}
\hop{\hopwillamette}{4.5~\%}{10~min}{\oztog{0.3}}
\hop{\hopwillamette}{4.5~\%}{\whirl{}{}}{\oztog{0.3}}
\end{hops}

\singleyeast{White Labs WLP041}

\end{ingredientsblock}

\end{recipe}

% -----------------------------------------------------------------------------
\begin{recipe}{Great Lakes Brewing Edmund Fitzgerald Porter Clone}
% -----------------------------------------------------------------------------

\begin{aboutblock}
\sourceaha
\end{aboutblock}

\specifications{\styleamericanporter}{\galtol{5}}{1.062}{1.014}{6.3~\%}{38.6}{\srmtoebc{41}}{60~min}{}

\begin{methodandtiming}

\begin{mashsteps}
\mashstep{\ftoc{152}}{60~min}
\end{mashsteps}

\begin{fermentationsteps}
\fermentationstep{\ftoc{68}}{}
\end{fermentationsteps}

\end{methodandtiming}

\recipebreak

\begin{ingredientsblock}

\begin{malts}
\malt{\maltpale}{\lbtokg{9.5}}
\malt{\maltcaramel{55}}{\lbtokg{0.75}}
\malt{\maltblack}{\lbtokg{0.5}}
\malt{Roasted Barley}{\lbtokg{0.5}}
\end{malts}

\begin{hops}
\hop{\hopnorthernbrewer}{8~\%}{60~min}{\oztog{0.75}}
\hop{\hopeastkentgolding}{4.75~\%}{60~min}{\oztog{0.25}}
\hop{\hopeastkentgolding}{4.75~\%}{35~min}{\oztog{0.5}}
\hop{\hopeastkentgolding}{4.75~\%}{10~min}{\oztog{0.5}}
\end{hops}

\singleyeast{Irish Ale / American Ale}

\end{ingredientsblock}

\end{recipe}

% -----------------------------------------------------------------------------
\begin{recipe}{Mid-Sember Knight's Dream American Porter}
% -----------------------------------------------------------------------------

\begin{aboutblock}
Recipe by Matt Knight of Milford, CT. Gold medal in Category 14: American
Porter and Stout during the 2019 National Homebrew Competition in
Providence, RI. \sourceaha
\end{aboutblock}

\specifications{\styleamericanporter}{\galtol{10}}{1.067}{1.021}{6~\%}{27}{\srmtoebc{38}}{60~min}{1.6}

\begin{methodandtiming}
 
\begin{mashsteps}
\mashstep{\ftoc{152}}{60~min}
\mashstep{\ftoc{165}}{Mash out}
\end{mashsteps}

\begin{fermentationsteps}
\fermentationstep{\ftoc{66}}{}
\end{fermentationsteps}

\begin{directions}
Water adjustment: \waterprofile{120}{10}{27}{92}{165}{208}.
\end{directions}

\end{methodandtiming}

\recipebreak

\begin{ingredientsblock}

\begin{malts}
\malt{\maltmarisotter}{\lbtokg{17.5}}
\malt{Flaked Oats}{\lbtokg{2}}
\malt{\maltchocolate}{\lbtokg{1.5}}
\malt{\maltcaramel{40}}{\lbtokg{1.25}}
\malt{\maltweyermanncarafaone}{\lbtokg{1.13}}
\malt{\maltblack}{\oztokg{8}}
\malt{\maltacidulated}{\oztokg{8}}
\end{malts}

\begin{hops}
\hop{\hopmagnum}{12.4~\%}{60~min}{\oztog{1.2}}
\end{hops}

\singleyeast{Fermentis SafAle US-05}

\end{ingredientsblock}

\end{recipe}

\stylesection{\stylebalticporter}

% -----------------------------------------------------------------------------
\begin{recipie}{Continental-Style Baltic Porter}
% -----------------------------------------------------------------------------

\begin{aboutblock}
In the mid-1800s, Baltic porter as a style relieved a formative transformation:
lager yeast. Cold shipping of the first strong porter exports had mellowed the
English-brewed porters, but beers brewed in the Baltic countries needed yeast
adapted to ferment cool, not just condition cold. Lager yeast therefore became
the Baltic brewery standard; ale yeasts were unsuitable, and strong porters
lost much of their ale yeast-derived ester and phenols, gaining signature
lager smoothness. \sourcezymurgy{September / October 2017}
\end{aboutblock}

\specifications{\stylebalticporter}{\galtol{5.5}}{1.077}{1.012}{8.6~\%}{29}{\srmtoebc{29}}{120~min}{}

\begin{methodandtiming}
 
\begin{mashsteps}
\mashstep{\ftoc{148}}{60~min}
\mashstep{\ftoc{168}}{Mashout}
\end{mashsteps}

\begin{fermentationsteps}
\fermentationstep{\ftoc{48}}{36~hours / fermentation start}
\fermentationstep{\ftoc{50}}{2~weeks}
\fermentationstep{\ftoc{55}}{Free raise to; until fermentation slowdown}
\fermentationstep{\ftoc{60}}{1--2~weeks / until fully attenuated}
\end{fermentationsteps}

\begin{directions}
Cold condition for at least 1 month at \ftoc{35} before packaging, although
3 months is better.
\end{directions}

\end{methodandtiming}

\pagebreak

\begin{ingredientsblock}

\begin{malts}
\malt{Pilsner}{\lbtokg{5}}
\malt{Munich}{\lbtokg{4.25}}
\malt{Vienna}{\lbtokg{4}}
\malt{Weyermann CARAMUNICH I}{\lbtokg{1}}
\malt{Dingemans Special B}{\oztokg{8}}
\malt{Briess Extra Special}{\oztokg{8}}
\malt{Weyermann CARAFA II}{\oztokg{0.5}}
\end{malts}

\begin{hops}
\hop{\hopmarynka}{10.5~\%}{60~min}{\oztog{0.5}}
\hop{\hoplubelska}{5~\%}{60~min}{\oztog{2}}
\hop{Whirlfloc Tablet}{}{15~min}{1}
\end{hops}

\singleyeast{White Labs WLP802}

\end{ingredientsblock}

\end{recipie}

% -----------------------------------------------------------------------------
\begin{recipie}{Odin's Beard Baltic Porter}
% -----------------------------------------------------------------------------

\begin{aboutblock}
Justin McClenahan of Silver Spring, MD, member of the The Brewing Network, won a
gold medal in Category \#6: Strong European Lager with a Baltic Porter during the
2019 National Homebrew Competition Final Round in Providence, RI. McClenahan's
Strong European Lager was chosen as the best among 269 entries in the category.
\sourceaha
\end{aboutblock}

\specifications{\stylebalticporter}{\galtol{12}}{1.086}{1.016}{8.7~\%}{43}{\srmtoebc{30}}{90~min}{2.6}

\begin{methodandtiming}
 
\begin{mashsteps}
\mashstep{\ftoc{156}}{60~min}
\mashstep{\ftoc{168}}{Mashout}
\end{mashsteps}

\begin{fermentationsteps}
\fermentationstep{\ftoc{54}}{Primary fermentation}
\fermentationstep{\ftoc{64}}{3~days}
\fermentationstep{\ftoc{38}}{Transfer to secondary; slowly reduce to \ftoc{38}}
\end{fermentationsteps}

\begin{directions}
Water adjustment: \waterprofile{50}{10}{33}{57}{44}{142}. Lager for 30 days.
\end{directions}

\end{methodandtiming}

\pagebreak

\begin{ingredientsblock}

\begin{malts}
\malt{Swaen Pilsner}{\lbtokg{25}}
\malt{BEST Munich Dark}{\lbtokg{6}}
\malt{Simpsons Golden Naked Oats}{\lbtokg{2.5}}
\malt{Pale Chocolate}{\lbtokg{1.5}}
\malt{Weyermann CARAMUNICH I}{\lbtokg{1}}
\malt{Thomas Fawcett Roasted Barley}{\oztog{12}}
\end{malts}

\begin{hops}
\hop{\hopmagnum}{14~\%}{\fwh}{\oztog{1.5}}
\hop{\hopmagnum}{12~\%}{40~min}{\oztog{0.5}}
\hop{Light Brown Sugar}{}{15~min}{\lbtokg{2}}
\hop{\hopsterling}{7.5~\%}{\whirl{}{20~min}}{\oztog{2}}
\end{hops}

\begin{yeastsx}
\yeastx{White Labs WLP833}{Primary}
\yeastx{White Labs WLP838}{Primary}
\end{yeastsx}

\end{ingredientsblock}

\end{recipie}

% -----------------------------------------------------------------------------
\stylecategory{Stout}
\stylesection{\styleirishextrastout}

% -----------------------------------------------------------------------------
\begin{recipe}{Homebrew Challenge Irish Extra Stout}
% -----------------------------------------------------------------------------

\begin{aboutblock}
Recipe by Martin Keen.
\sourcehomebrewchallenge
\end{aboutblock}

\specifications{\styleirishextrastout}{\galtol{5}}{1.060}{1.015}{6~\%}{43}{\srmtoebc{37}}{60~min}{}

\begin{methodandtiming}

\begin{mashsteps}
\mashstep{\ftoc{152}}{60~min}
\mashstep{\ftoc{168}}{Mash out}
\end{mashsteps}

\begin{fermentationsteps}
\fermentationstep{\ftoc{65}}{Full attenuation}
\end{fermentationsteps}
 
\end{methodandtiming}

\recipebreak

\begin{ingredientsblock}

\begin{malts}
\malt{\maltmarisotter}{\lbtokg{9}}
\malt{Flaked Barley}{\lbtokg{2}}
\malt{Roasted Barley}{\lbtokg{1}}
\end{malts}

\begin{hops}
\hop{\hoptarget}{}{60~min}{\oztog{1}}
\hop{\hopfuggle}{}{10~min}{\oztog{1}}
\end{hops}

\singleyeast{Fermentis SafAle S-04}

\end{ingredientsblock}

\end{recipe}

\stylesection{\stylesweetstout}

% -----------------------------------------------------------------------------
\begin{recipe}{Belgian Dip Chocolate Stout} % rechecked
% -----------------------------------------------------------------------------

\begin{aboutblock}
Recipe by Amahl Turczyn based off of Mountain Sun Pub \& Brewery's
(Boulder, CO) chocolate stout. Has sweet chocolaty notes.
\sourcezymurgy{November / December 2007}
\end{aboutblock}

\specifications{\stylesweetstout}{\galtol{5}}{1.072}{}{}{}{\srmtoebc{45}}{90~min}{}

\begin{methodandtiming}
 
\begin{mashsteps}
\mashstep{\ftoc{154}}{60~min}
\end{mashsteps}

\begin{directions}
Add lactose to kettle before running clear wort into the kettle. Coarsely chop
chocolate and add to wort after boil is complete.
\end{directions}

\end{methodandtiming}

\recipebreak

\begin{ingredientsblock}

\begin{malts}
\malt{Pale}{\lbtokg{8.87}}
\malt{Black}{\lbtokg{0.81}}
\malt{Briess Bonlander Munich 10 L}{\lbtokg{0.81}}
\malt{Briess Caramel Vienne 20 L}{\lbtokg{0.56}}
\malt{Brown}{\lbtokg{0.48}}
\malt{Briess Victory}{\lbtokg{0.48}}
\malt{Wheat}{\lbtokg{0.32}}
\malt{Chocolate}{\lbtokg{0.24}}
\malt{Chocolate Rye}{\lbtokg{0.24}}
\end{malts}

\begin{hops}
\hop{Lactose}{}{\fwh}{\lbtokg{0.89}}
\hop{\hophallertaumittelfruh}{}{60~min}{\oztog{0.65}}
\hop{\hopliberty}{}{\foh{}}{\oztog{0.52}}
\hop{Callebaut Milk Chocolate}{}{\foh{}}{\lbtokg{0.53}}
\hop{Callebaut Dark Chocolate}{}{\foh{}}{\lbtokg{0.18}}
\end{hops}

\singleyeast{Mountain Sun Ale / American Ale}

\end{ingredientsblock}

\end{recipe}

% -----------------------------------------------------------------------------
\begin{recipe}{Boise Brewing Dark Daisy Chocolate Milk Stout Clone} % rechecked
% -----------------------------------------------------------------------------

\begin{aboutblock}
This overall roast-y and dark chocolate-y brew has a creamy, sweet character.
\sourceaha
\end{aboutblock}

\specifications{\stylesweetstout}{\galtol{5}}{1.070}{1.020}{7.1~\%}{40}{\srmtoebc{63}}{60~min}{}

\begin{methodandtiming}

\begin{mashsteps}
\mashstep{\ftoc{154}}{}
\end{mashsteps}

\end{methodandtiming}

\recipebreak

\begin{ingredientsblock}

\begin{malts}
\malt{Pale}{\lbtokg{10.1}}
\malt{Chocolate}{\lbtokg{1.1}}
\malt{Weyermann CARAFA SPECIAL II}{\oztog{11.2}}
\malt{Wheat}{\oztog{11.2}}
\malt{Brown}{\oztog{8}}
\end{malts}

\begin{hops}
\hop{Lactose}{}{}{\oztog{6.4}}
\hop{\hopcentennial}{10~\%}{60~min}{\oztog{0.6}}
\hop{\hopcentennial}{10~\%}{30~min}{\oztog{0.75}}
\end{hops}

\singleyeast{American Ale}

\end{ingredientsblock}

\end{recipe}

% -----------------------------------------------------------------------------
\begin{recipe}{Crooked Hammock Brewery Haulin' Oats Milk Stout Clone} % rechecked
% -----------------------------------------------------------------------------

\begin{aboutblock}
Heavy in body and color but light in its mouthfeel thanks to rolled oats and lactose,
providing an underlying sweetness. \sourceaha
\end{aboutblock}

\specifications{\stylesweetstout}{\galtol{15}}{1.068}{}{5.8~\%}{18}{}{60~min}{}

\begin{methodandtiming}

\begin{mashsteps}
\mashstep{\ftoc{151}}{}
\end{mashsteps}

\begin{fermentationsteps}
\fermentationstep{\ftoc{65}}{}
\end{fermentationsteps}

\begin{directions}
Original gravity includes lactose addition.
\end{directions}

\end{methodandtiming}

\recipebreak

\begin{ingredientsblock}

\begin{malts}
\malt{Two-row}{\lbtokg{16.5}}
\malt{Flaked Oats}{\lbtokg{3.5}}
\malt{Munich}{\lbtokg{3}}
\malt{Caramel / Crystal 60 L}{\lbtokg{1}}
\malt{Chocolate Wheat}{\lbtokg{1}}
\malt{Roasted Barley}{\lbtokg{1}}
\malt{Patagonia Black Pearl}{\lbtokg{1}}
\malt{Weyermann CARAFA III}{\lbtokg{0.5}}
\malt{Black}{\lbtokg{0.5}}
\end{malts}

\begin{hops}
\hop{\hopcolumbus}{}{60~min}{\oztog{0.77}}
\hop{\hopfuggle}{}{30~min}{\oztog{0.42}}
\hop{Lactose}{}{15~min}{\lbtokg{3}}
\end{hops}

\singleyeast{White Labs WLP001}

\end{ingredientsblock}

\end{recipe}

% -----------------------------------------------------------------------------
\begin{recipe}{Exile Brewing Company Sir Moch-A-Lot Mocha Stout Clone} % rechecked
% -----------------------------------------------------------------------------

\begin{aboutblock}
Brewed with Brazilian coffee beans. Roasted coffee and chocolate bitterness balance
the smooth malty sweetness. \sourceaha
\end{aboutblock}

\specifications{\stylesweetstout}{\galtol{5}}{1.069}{}{7.1~\%}{20}{}{60~min}{2}

\begin{methodandtiming}

\begin{mashsteps}
\mashstep{\ftoc{154}}{}
\end{mashsteps}

\begin{fermentationsteps}
\fermentationstep{\ftoc{68}}{Full attenuation}
\fermentationstep{\ftoc{38}}{Cold crash; transfer to secondary}
\end{fermentationsteps}

\end{methodandtiming}

\recipebreak

\begin{ingredientsblock}

\begin{malts}
\malt{Pilsner}{\lbtokg{5.8}}
\malt{Golden Promise}{\lbtokg{2.8}}
\malt{Flaked Oats}{\lbtokg{2}}
\malt{Simpson Brown}{\lbtokg{1.2}}
\malt{Chocolate}{\lbtokg{0.9}}
\malt{Caramel / Crystal 45 L}{\lbtokg{0.8}}
\malt{Black}{\lbtokg{0.6}}
\end{malts}

\begin{hops}
\hop{\hopcascade}{5.5~\%}{60~min}{\oztog{1.2}}
\end{hops}

\singleyeast{American Ale}

\begin{twists}
\twist{Cacao Nibs}{Secondary (5~days)}{\lbtokg{0.2}}
\twist{Cold Brew Coffee}{Bottling}{--}
\end{twists}

\end{ingredientsblock}

\end{recipe}

% -----------------------------------------------------------------------------
\begin{recipe}{Left Hand Milk Stout Clone} % rechecked
% -----------------------------------------------------------------------------

\begin{aboutblock}
Recipe by Amahl Turczyn. \sourcezymurgy{July / August 2009}
\end{aboutblock}

\specifications{\stylesweetstout}{\galtol{5}}{1.068}{1.016}{7~\%}{19}{\srmtoebc{41}}{90~min}{}

\begin{methodandtiming}

\begin{mashsteps}
\mashstep{\ftoc{151}}{90~min}
\end{mashsteps}

\begin{fermentationsteps}
\fermentationstep{\ftoc{70}}{Full attenuation}
\fermentationstep{\ftoc{60}}{Transfer to secondary; 7~days}
\end{fermentationsteps}

\end{methodandtiming}

\recipebreak

\begin{ingredientsblock}

\begin{malts}
\malt{Pale}{\lbtokg{7}}
\malt{Roasted Barley}{\lbtokg{1}}
\malt{Caramel / Crystal 60 L}{\lbtokg{0.75}}
\malt{Munich}{\lbtokg{0.75}}
\malt{Chocolate}{\lbtokg{0.75}}
\malt{Flaked Barley}{\lbtokg{0.5}}
\malt{Flaked Oats}{\lbtokg{0.5}}
\end{malts}

\begin{hops}
\hop{\hopmagnum}{13~\%}{60~min}{\oztog{0.3}}
\hop{Lactose}{}{15~min}{\lbtokg{1}}
\hop{\hopeastkentgolding}{5~\%}{10~min}{\oztog{1}}
\end{hops}

\singleyeast{White Labs WLP001 / Wyeast 1056}

\end{ingredientsblock}

\end{recipe}

% -----------------------------------------------------------------------------
\begin{recipe}{Lov-a-ly Bunch of Coconuts} % rechecked
% -----------------------------------------------------------------------------

\begin{aboutblock}
Recipe by Nate Levengood with Millersburg Brewing Company, OH.
\sourceaha
\end{aboutblock}

\specifications{\stylesweetstout}{\galtol{10}}{1.058}{1.036}{}{28}{}{60~min}{2}

\begin{methodandtiming}

\begin{mashsteps}
\mashstep{\ftoc{156}}{60~min}
\end{mashsteps}

\begin{fermentationsteps}
\fermentationstep{\ftoc{68}}{}
\end{fermentationsteps}

\end{methodandtiming}

\recipebreak

\begin{ingredientsblock}

\begin{malts}
\malt{Golden Promise}{\lbtokg{9}}
\malt{Flaked Barley}{\lbtokg{3}}
\malt{Pale Chocolate}{\lbtokg{1.75}}
\malt{Roasted Barley}{\lbtokg{1.75}}
\malt{Caramel / Crystal 60 L}{\lbtokg{1}}
\malt{Caramel / Crystal 120 L}{\lbtokg{1}}
\malt{Briess Victory}{\oztog{12}}
\end{malts}

\begin{hops}
\hop{\hopmagnum}{14.7~\%}{60~min}{\oztog{1}}
\hop{Lactose}{}{10~min}{\lbtokg{3}}
\end{hops}

\singleyeast{White Labs WLP028 / Wyeast 1728}

\begin{twists}
\twist{Toasted Coconut}{Secondary (3~days)}{\oztog{12}}
\twist{Toasted Almonds}{Secondary (3~days)}{\oztog{2}}
\end{twists}

\end{ingredientsblock}

\end{recipe}

% -----------------------------------------------------------------------------
\begin{recipe}{Nortons Brewing Co. Don't Poke the Bear Milk Stout Clone} % rechecked
% -----------------------------------------------------------------------------

\begin{aboutblock}
A chocolate imperial milk stout with vanilla. \sourceaha
\end{aboutblock}

\specifications{\stylesweetstout}{\galtol{5}}{1.096}{1.035}{8.2~\%}{17.1}{\srmtoebc{38.3}}{60~min}{}

\begin{methodandtiming}

\begin{mashsteps}
\mashstep{\ftoc{152}}{60~min}
\mashstep{\ftoc{168}}{10~min}
\end{mashsteps}

\begin{fermentationsteps}
\fermentationstep{\ftoc{68}}{}
\end{fermentationsteps}

\begin{directions}
Add coca nibs after 75~\% of fermentation is complete. Create a tincture out of
vanilla beans at brew day.
\end{directions}

\end{methodandtiming}

\recipebreak

\begin{ingredientsblock}

\begin{malts}
\malt{Maris Otter}{\lbtokg{12}}
\malt{Rahr White Wheat}{\lbtokg{1.3}}
\malt{Weyermann Munich I}{\oztog{11.5}}
\malt{Crisp Chocolate}{\oztog{9.7}}
\malt{Flaked Oats}{\oztog{9.6}}
\malt{Flaked Barley}{\oztog{9.6}}
\malt{Crisp Roasted Barley}{\oztog{7.8}}
\end{malts}

\begin{hops}
\hop{\hopgalena}{12.5~\%}{60~min}{\oztog{0.4}}
\hop{Dutch Cocoa}{}{30~min}{\oztog{1}}
\hop{Lactose}{}{20~min}{\lbtokg{1.5}}
\hop{\hopmthood}{6~\%}{15~min}{\oztog{0.9}}
\end{hops}

\singleyeast{Wyeast 1028}

\begin{twists}
\twist{Cocoa Nibs}{Primary}{\oztog{6}}
\twist{Cold Brew Coffee}{Secondary}{\oztoml{2}}
\twist{Vanilla Beans}{Secondary (1~week)}{2}
\end{twists}

\end{ingredientsblock}

\end{recipe}

% -----------------------------------------------------------------------------
\begin{recipe}{Outlaw Brewing Udder Madness Chocolate Milk Stout Clone} % rechecked
% -----------------------------------------------------------------------------

\begin{aboutblock}
Full body and rich mouthfeel. \sourceaha
\end{aboutblock}

\specifications{\stylesweetstout}{\galtol{5}}{1.058}{1.014}{5.7~\%}{21}{}{60~min}{}

\begin{methodandtiming}
 
\begin{mashsteps}
\mashstep{\ftoc{152}}{60~min}
\end{mashsteps}

\begin{fermentationsteps}
\fermentationstep{\ftoc{68}}{Attenuation to \sgtop{1.024}}
\fermentationstep{\ftoc{70}}{Free raise; full attenuation}
\end{fermentationsteps}

\begin{directions}
After diacetyl rest, remove from yeast and cold crash. Add chocolate of
choice (Outlaw Brewing prefers an organic cocao at a rate of \ozpgaltogpl{6}).
Swirl fermenter periodically for first day, then allow to settle for 4 days.
Transfer off chocolate and carbonate.
\end{directions}

\end{methodandtiming}

\recipebreak

\begin{ingredientsblock}
 
\begin{malts}
\malt{Two-row}{\lbtokg{5}}
\malt{Flaked Wheat}{\oztog{10}}
\malt{Caramel / Crystal 45 L}{\oztog{6}}
\malt{Chocolate}{\oztog{6}}
\malt{Flaked Oats}{\oztog{5}}
\malt{Roasted Barley}{\oztog{3.7}}
\end{malts}

\begin{hops}
\hop{\hopnorthernbrewer}{}{60~min}{\oztog{0.4}}
\hop{Lactose}{}{10~min}{\oztog{5}}
\hop{\hopfuggle}{}{5~min}{\oztog{0.4}}
\end{hops}

\singleyeast{American Ale}

\begin{twists}
\twist{Chocolate}{Secondary (5~days)}{\oztog{30}}
\end{twists}

\end{ingredientsblock}

\end{recipe}

% -----------------------------------------------------------------------------
\begin{recipe}{The Olga Milk Stout} % rechecked
% -----------------------------------------------------------------------------

\begin{aboutblock}
Recipe by Ryan DeLaRosa of Houston, TX. Gold medal in Category 14: Stout during
the 2017 National Homebrew Competition in Minneapolis, MN. \sourceaha
\end{aboutblock}

\specifications{\stylesweetstout}{\galtol{6.2}}{1.080}{1.030}{}{}{}{60~min}{2.2}

\begin{methodandtiming}

\begin{mashsteps}
\mashstep{\ftoc{156}}{60~min}
\mashstep{\ftoc{169}}{Mash out}
\end{mashsteps}

\begin{fermentationsteps}
\fermentationstep{\ftoc{68}}{5~days}
\fermentationstep{\ftoc{72}}{4~days}
\fermentationstep{\ftoc{36}}{2~days}
\end{fermentationsteps}

\end{methodandtiming}

\recipebreak

\begin{ingredientsblock}

\begin{malts}
\malt{Two-row}{\lbtokg{7.45}}
\malt{Caramel / Crystal 60 L}{\lbtokg{3}}
\malt{Weyermann CARAMUNICH II}{\lbtokg{3}}
\malt{Chocolate}{\oztog{12}}
\malt{Flaked Oats}{\oztog{11}}
\malt{Roasted Barley}{\oztog{6}}
\malt{Flaked Barley}{\oztog{5}}
\end{malts}

\begin{hops}
\hop{\hopcolumbus}{16~\%}{60~min}{\oztog{0.6}}
\hop{\hopeastkentgolding}{5.6~\%}{20~min}{\oztog{0.5}}
\hop{\hopeastkentgolding}{5.6~\%}{\foh{}{}}{\oztog{0.5}}
\hop{Lactose}{}{}{\lbtokg{1}}
\end{hops}

\singleyeast{White Labs WLP001}

\end{ingredientsblock}

\end{recipe}

\stylesection{\styleoatmealstout}

% -----------------------------------------------------------------------------
\begin{recipe}{Beerwreckdus Oatmeal Stout}
% -----------------------------------------------------------------------------

\begin{aboutblock}
Recipe by James Slattery of Vista, CA. Gold medal in Category 13: British and
Irish Stout during the 2018 National Homebrew Competition in Portland, OR.
\sourceaha
\end{aboutblock}

\specifications{\styleoatmealstout}{\galtol{12.5}}{1.076}{1.021}{7.2~\%}{30.5}{\srmtoebc{49}}{75~min}{2}

\begin{methodandtiming}

\begin{mashsteps}
\mashstep{\ftoc{155}}{}
\end{mashsteps}

\begin{fermentationsteps}
\fermentationstep{\ftoc{67}}{1~week}
\fermentationstep{\ftoc{35}}{Transfer to secondary; 1~week}
\end{fermentationsteps}

\end{methodandtiming}

\recipebreak

\begin{ingredientsblock}

\begin{malts}
\malt{Malteurop Pilsen}{\lbtokg{12}}
\malt{Crisp Maris Otter}{\lbtokg{12}}
\malt{Simpsons Chocolate}{\lbtokg{4}}
\malt{Flaked Oats}{\lbtokg{2.5}}
\malt{Flaked Barley}{\lbtokg{2.5}}
\malt{Simpsons DRC}{\lbtokg{2.25}}
\malt{Gambrinus Honey}{\lbtokg{1.5}}
\end{malts}

\begin{hops}
\hop{\hopeastkentgolding}{5~\%}{60~min}{\oztog{2.5}}
\hop{\hopnorthernbrewer}{8~\%}{60~min}{\oztog{1.25}}
\hop{\hopeastkentgolding}{5~\%}{10~min}{\oztog{2.5}}
\end{hops}

\singleyeast{White Labs WLP028}

\end{ingredientsblock}

\end{recipe}

% -----------------------------------------------------------------------------
\begin{recipe}{Danado de Bom Oatmeal Stout}
% -----------------------------------------------------------------------------

\begin{aboutblock}
Recipe by Charlie Papazian. Based on his experience at Brewerkz Riverside
Point in Singapore. Smooth, velvety texture and a creamy brown head. Light cocoa
and coffee flavor with subtle enhancement of aroma hops.
\sourcezymurgy{January / February 2009}
\end{aboutblock}

\specifications{\styleoatmealstout}{\galtol{6}}{1.048}{1.016}{4.2~\%}{34}{\srmtoebc{45}}{60~min}{}

\begin{methodandtiming}

\begin{mashsteps}
\mashstep{\ftoc{132}}{30~min}
\mashstep{\ftoc{155}}{30~min}
\mashstep{\ftoc{167}}{Mash out}
\end{mashsteps}

\begin{fermentationsteps}
\fermentationstep{\ftoc{70}}{1~week}
\fermentationstep{\ftoc{55}}{Transfer to secondary; 1~week}
\end{fermentationsteps}

\begin{directions}
Add dry hops after transfer to secondary.
\end{directions}

\end{methodandtiming}

\recipebreak

\begin{ingredientsblock}

\begin{malts}
\malt{Maris Otter}{\lbtokg{7}}
\malt{Caramel / Crystal 15 L}{\lbtokg{1}}
\malt{Aromatic}{\oztog{8}}
\malt{Roasted Barley}{\oztog{12}}
\malt{Black}{\oztog{8}}
\malt{Chocolate}{\oztog{8}}
\malt{Rolled Oats}{\oztog{12}}
\end{malts}

\begin{hops}
\hop{\hopchallenger}{5.8~\%}{60~min}{\oztog{1.25}}
\hop{\hopeastkentgolding}{6.3~\%}{10~min}{\oztog{1}}
\hop{Irish Moss}{}{10~min}{\tsptog{0.25}}
\hop{\hopcrystal}{5.5~\%}{\dryh{}{}}{\oztog{0.5}}
\end{hops}

\singleyeast{White Labs WLP1983 / English Ale}

\end{ingredientsblock}

\end{recipe}

% -----------------------------------------------------------------------------
\begin{recipe}{Glenn's Oatmeal Stout}
% -----------------------------------------------------------------------------

\begin{aboutblock}
Glenn Quinting of Timonium, MD. Gold medal in Category \#13: Stout during the
2010 National Homebrew Competition in Minneapolis, MN. \sourceaha
\end{aboutblock}

\specifications{\styleoatmealstout}{\galtol{10}}{1.060}{1.018}{5.51~\%}{35}{\srmtoebc{33.5}}{60~min}{2.5}

\begin{methodandtiming}

\begin{mashsteps}
\mashstep{\ftoc{154}}{90~min}
\end{mashsteps}

\begin{fermentationsteps}
\fermentationstep{\ftoc{65}}{2~weeks}
\end{fermentationsteps}

\begin{directions}
Flaked oat preparation: spread on cookie sheets, sprinkle with water and bake for 15--20
min at \ftoc{300}, turning and mixing midway through. Allow them to sit in a brown paper bag
or empty cereal box for 1--2 weeks so flavors smooth out. 
\end{directions}

\end{methodandtiming}

\recipebreak

\begin{ingredientsblock}

\begin{malts}
\malt{Pale}{\lbtokg{18}}
\malt{Flaked Oats}{\lbtokg{2}}
\malt{Briess Victory}{\lbtokg{1.5}}
\malt{Black}{\lbtokg{1}}
\malt{Chocolate}{\oztog{12}}
\malt{Pale Chocolate}{\oztog{12}}
\malt{Caramel / Crystal 80 L}{\lbtokg{1.0}}
\end{malts}

\begin{hops}
\hop{\hopeastkentgolding}{4.95~\%}{60~min}{\oztog{3.6}}
\hop{Brewers Licorice}{}{60~min}{\oztog{0.2}}
\hop{Yeast Energizer}{}{}{\tsptog{1}}
\end{hops}

\singleyeast{White Labs WLP002}

\end{ingredientsblock}

\end{recipe}

% -----------------------------------------------------------------------------
\begin{recipe}{Rob's Oatmeal Stout}
% -----------------------------------------------------------------------------

\begin{aboutblock}
Recipe by Rob Vrabel of Farmington Hills, MI. Gold medal in Category \#13: Stout
during the 2014 National Homebrew Competition in Grand Rapids, MI. \sourceaha
\end{aboutblock}

\specifications{\styleoatmealstout}{\galtol{10}}{1.062}{1.016}{6~\%}{}{}{60~min}{}

\begin{methodandtiming}

\begin{mashsteps}
\mashstep{\ftoc{154}}{60~min}
\end{mashsteps}

\begin{fermentationsteps}
\fermentationstep{\ftoc{68}}{2~weeks}
\end{fermentationsteps}

\end{methodandtiming}

\recipebreak

\begin{ingredientsblock}

\begin{malts}
\malt{Pale}{\lbtokg{17.375}}
\malt{Flaked Oats}{\lbtokg{2}}
\malt{Briess Victory}{\lbtokg{1.5}}
\malt{Caramel / Crystal 80 L}{\lbtokg{1.0}}
\malt{Black}{\lbtokg{1.0}}
\malt{Chocolate}{\lbtokg{0.75}}
\malt{Pale Chocolate}{\lbtokg{0.75}}
\end{malts}

\begin{hops}
\hop{\hopgolding}{}{60~min}{\oztog{4}}
\end{hops}

\singleyeast{White Labs WLP002}

\end{ingredientsblock}

\end{recipe}

% -----------------------------------------------------------------------------
\begin{recipe}{Sticky Faucet Oatmeal Milk Stout}
% -----------------------------------------------------------------------------

\begin{aboutblock}
Recipe by Josh Baas of Elk Grove, CA. Gold medal in Category \#13:
British \& Irish Stout during the 2019 National Homebrew Competition
in Providence, RI. \sourceaha
\end{aboutblock}

\specifications{\styleoatmealstout}{\galtol{10}}{1.077}{1.032}{5.5~\%}{40}{\srmtoebc{40}}{60~min}{2.5}

\begin{methodandtiming}
 
\begin{mashsteps}
\mashstep{\ftoc{156}}{60~min}
\mashstep{\ftoc{168}}{10~min}
\end{mashsteps}

\begin{fermentationsteps}
\fermentationstep{\ftoc{67}}{1~week}
\fermentationstep{\ftoc{70}}{3~days}
\fermentationstep{\ftoc{33}}{5~days}
\end{fermentationsteps}

\begin{directions}
Target 5.2 pH.
\end{directions}

\end{methodandtiming}

\recipebreak

\begin{ingredientsblock}

\begin{malts}
\malt{Pale}{\lbtokg{13}}
\malt{Flaked Oats}{\lbtokg{5}}
\malt{Flaked Barley}{\lbtokg{2}}
\malt{Chocolate}{\lbtokg{1.7}}
\malt{Caramel / Crystal 60 L}{\lbtokg{1.5}}
\malt{Munich}{\lbtokg{1.5}}
\malt{Roasted Barley}{\lbtokg{1.5}}
\malt{Briess Black Patent}{\lbtokg{1}}
\end{malts}

\begin{hops}
\hop{\hopmagnum}{14~\%}{60~min}{\oztog{1.75}}
\hop{Medium Crushed Oreo Cookies}{}{15~min}{\lbtokg{5}}
\hop{Lactose}{}{15~min}{\lbtokg{4}}
\hop{\hopstyriangolding}{5.4~\%}{10~min}{\oztog{1.75}}
\end{hops}

\singleyeast{White Labs WLP002}

\end{ingredientsblock}

\end{recipe}


% -----------------------------------------------------------------------------
\stylecategory{Strong Stout}
\stylesection{\styleamericanstout}

% -----------------------------------------------------------------------------
\begin{recipie}{Rogue Chocolate Stout Clone}
% -----------------------------------------------------------------------------

\begin{aboutblock}
Many chocolate stouts seem to have acidic coffee flavors and end up falling short
of real chocolate flavor. However, Rogue's Chocolate Stout is smooth, sweet and
velvety, and tastes like a chocolate malted milkshake with slight hoppiness and
bittersweet chocolate notes. The allure of this beer comes down to the chocolate
flavors, which makes it enjoyable for both the chocolate and beer lover. The sweet,
creamy head gives way to a rich chocolate truffle finish. This beer pairs perfectly
with desserts as it's balanced by plenty of roasty flavor and bitterness.
\end{aboutblock}

\specifications{\styleamericanstout}{\galtol{5}}{1.069}{1.017}{6.8~\%}{30}{\srmtoebc{25}}{90~min}{}

\begin{methodandtiming}
 
\begin{mashsteps}
\mashstep{\ftoc{150}}{60~min}
\end{mashsteps}

\begin{fermentationsteps}
\fermentationstep{\ftoc{60}}{7~days}
\fermentationstep{\ftoc{50}}{--}
\end{fermentationsteps}

\begin{directions}
Siphon into secondary at \ftoc{50} onto chocolate extract and hold until
fermentation is complete, then package and condition.
\end{directions}

\end{methodandtiming}

\pagebreak

\begin{ingredientsblock}

\begin{malts}
\malt{Great Western Premium Tow-row}{\lbtokg{11}}
\malt{Caramel / Crystal 120 L}{\lbtokg{0.5}}
\malt{Chocolate}{\lbtokg{0.5}}
\malt{Rolled Oats}{\lbtokg{0.5}}
\malt{Roast Barley}{\oztokg{3}}
\end{malts}

\begin{hops}
\hop{\hopcascade}{5~\%}{90~min}{\oztog{1}}
\hop{\hopcascade}{5~\%}{30~min}{\oztog{1}}
\hop{Irish Moss}{}{20~min}{\tsptoml{1}}
\hop{\hopcascade}{5~\%}{\foh{}}{\oztog{1}}
\end{hops}

\begin{yeasts}
\yeast{Wyeast 1764-PC / White Labs WLP001}
\end{yeasts}

\begin{twists}
\twist{Chocolate Extract}{Secondary}{\oztog{1.51}}
\end{twists}

\end{ingredientsblock}

\end{recipie}

% -----------------------------------------------------------------------------
\begin{recipie}{Stout Trousers}
% -----------------------------------------------------------------------------

\begin{aboutblock}
Chris O'Brien is a homebrewer with a mission: How to Drink
Beer and Save the World, O'Brien takes homebrewing to the next level
by promoting the use of responsibly cultivated, organic ingredients.
"Making organic homebrew is no different than making regular homebrew.
Just start with fresh, organic ingredients and the rest is the art and
science of homebrewing -- same as usual. The one big challenege is finding those ingredients." Of course, if you can't get your hands on strictly
organic ingredients, Stout Trousers -- a deliciously hoppy American
stout -- can still be brewed with non-organic substitutes.
\end{aboutblock}

\specifications{\styleamericanstout}{\galtol{5}}{1.072}{1.018}{7.1~\%}{}{}{100~min}{}

\begin{methodandtiming}
 
\begin{mashsteps}
\mashstep{\ftoc{154}}{60~min}
\end{mashsteps}

\begin{fermentationsteps}
\fermentationstep{\ftoc{64}}{1~day}
\fermentationstep{\ftoc{66}}{Until 75~\% fermented}
\fermentationstep{\ftoc{70}}{1~week}
\fermentationstep{\ftoc{40}}{2~days}
\end{fermentationsteps}

\begin{directions}
Do not use soft water for this brew. Acidify the strike water to a pH of 4.8.
This should result in a mash pH of 5.2. Also acidify the sparge water to 4.8.
\end{directions}

\end{methodandtiming}

\pagebreak

\begin{ingredientsblock}

\begin{malts}
\malt{Great Western Organic Premium Two-row}{\lbtokg{11.5}}
\malt{Briess Organic Roasted Barley}{\lbtokg{0.75}}
\malt{Briess Organic Chocolate}{\oztokg{8.8}}
\malt{Briess Organic Caramel 60 L}{\oztokg{8.8}}
\malt{Weyermann CARAFA II}{\oztokg{3.2}}
\end{malts}

\begin{hops}
\hop{\hoppacificgem}{}{60~min}{\oztog{0.75}}
\hop{\hopfuggle}{}{15~min}{\oztog{1.5}}
\hop{\hopcascade}{}{10~min}{\oztog{1.5}}
\hop{Irish Moss}{}{10~min}{\tsptoml{1}}
\hop{\hopcascade}{}{5~min}{\oztog{1}}
\end{hops}

\begin{yeasts}
\yeast{Wyeast 1272}
\end{yeasts}

\end{ingredientsblock}

\end{recipie}
\stylesection{\styleimperialstout}

% -----------------------------------------------------------------------------
\begin{recipe}{Bourbon Barrel Aged Full Eclipse II}
% -----------------------------------------------------------------------------

\begin{aboutblock}
Recipe by Ryan Golden and Matt Warren of Signal Mountain, TN. Silver medal
during the 2017 National Homebrew Competition. \sourceaha
\end{aboutblock}

\specifications{\styleimperialstout}{\galtol{12}}{1.124}{1.030}{}{}{}{120~min}{}

\begin{methodandtiming}
 
\begin{mashsteps}
\mashstep{\ftoc{153}}{60~min}
\end{mashsteps}

\begin{fermentationsteps}
\fermentationstep{\ftoc{65}}{}
\end{fermentationsteps}

\begin{directions}
When fully attenuated, transfer to a bourbon barrel and age until the desired amount
of oak and bourbon character is reached.
\end{directions}

\end{methodandtiming}

\recipebreak

\begin{ingredientsblock}

\begin{malts}
\malt{Rice Hulls}{\lbtokg{1}}
\malt{Maris Otter}{\lbtokg{40}}
\malt{Flaked Wheat}{\lbtokg{5}}
\malt{Pale Chocolate}{\lbtokg{4}}
\malt{Chocolate}{\lbtokg{3}}
\malt{Roasted Barley}{\lbtokg{3}}
\malt{Dingemans Special B}{\lbtokg{3}}
\malt{Caramel / Crystal 120 L}{\lbtokg{1}}
\end{malts}

\begin{hops}
\hop{\hopmagnum}{12.3}{90~min}{\oztog{5.5}}
\hop{\hopfuggle}{4.2}{20~min}{\oztog{2}}
\hop{\hopstyriangolding}{3.4}{20~min}{\oztog{2}}
\hop{Whirlfloc Tablet}{}{5~min}{1}
\hop{\hopfuggle}{4.2}{\whirl{}{15~min}}{\oztog{2}}
\hop{\hopstyriangolding}{3.4}{\whirl{}{15~min}}{\oztog{2}}

\end{hops}

\singleyeast{Wyeast 1028}

\end{ingredientsblock}

\end{recipe}

% -----------------------------------------------------------------------------
\begin{recipe}{Hammer of the Gord's Imperial Stout}
% -----------------------------------------------------------------------------

\begin{aboutblock}
Recipe by Tyler Mangin of Fargo, ND. Bronze medal in Category \#15: Imperial
Stout during the 2019 National Homebrew Competition in Providence, RI.
\sourceaha
\end{aboutblock}

\specifications{\styleimperialstout}{\galtol{5.5}}{1.100}{1.025}{10~\%}{86}{\srmtoebc{86}}{90~min}{2.4}

\begin{methodandtiming}
 
\begin{mashsteps}
\mashstep{\ftoc{131}}{Only base malts; 15~min}
\mashstep{\ftoc{144}}{50~min}
\mashstep{\ftoc{158}}{50~min}
\mashstep{\ftoc{168}}{Add caramel and roasted malts; Mash out}
\end{mashsteps}

\begin{fermentationsteps}
\fermentationstep{\ftoc{67}}{3~days}
\fermentationstep{\ftoc{72}}{Raise to over 5~days}
\end{fermentationsteps}

\begin{directions}
Target a mash pH of 5.3. Dry hop on day 7.
\end{directions}

\end{methodandtiming}

\recipebreak

\begin{ingredientsblock}

\begin{malts}
\malt{Crisp Finest Maris Otter Ale}{\lbtokg{12}}
\malt{Briess Blackprinz}{\lbtokg{2}}
\malt{Crisp Brown}{\lbtokg{2}}
\malt{BEST Munich}{\lbtokg{2}}
\malt{Roasted Barley}{\lbtokg{2}}
\malt{Dingemans Aromatic / Amber}{\oztog{16}}
\malt{Weyermann Munich II}{\oztog{16}}
\malt{Crisp Chocolate}{\oztog{16}}
\malt{Weyermann Vienna}{\oztog{16}}
\malt{Weyermann CARAMUNICH III}{\oztog{8}}
\malt{Flaked Oats}{\oztog{8}}
\malt{Dingemans Special B}{\oztog{8}}
\end{malts}

\begin{hops}
\hop{\hopmagnum}{13.2~\%}{90~min}{\oztog{1.6}}
\hop{\hopsterling}{7.4~\%}{30~min}{\oztog{1}}
\hop{\hopsterling}{8.1~\%}{\foh{}{}}{\oztog{2}}
\hop{\hopfuggle}{4.5~\%}{\dryh{}{1~week}}{\oztog{2}}
\end{hops}

\singleyeast{Wyeast 1335}

\end{ingredientsblock}

\end{recipe}

% -----------------------------------------------------------------------------
\begin{recipe}{Mill 'em All Imperial Stout}
% -----------------------------------------------------------------------------

\begin{aboutblock}
Recipe by Tim Hill of St. Louis, MO. Silver medal in Category \#15: Imperial Stout
during the 2019 National Homebrew Competition in Providence, RI. \sourceaha
\end{aboutblock}

\specifications{\styleimperialstout}{\galtol{5.25}}{1.104}{1.031}{9.8~\%}{78}{\srmtoebc{65}}{120~min}{2.2}

\begin{methodandtiming}
 
\begin{mashsteps}
\mashstep{\ftoc{156}}{60~min}
\end{mashsteps}

\begin{fermentationsteps}
\fermentationstep{\ftoc{66}}{2~weeks}
\end{fermentationsteps}

\begin{directions}
Allow another 6 weeks in primary before packaging.
\end{directions}

\end{methodandtiming}

\recipebreak

\begin{ingredientsblock}

\begin{malts}
\malt{Two-row}{\lbtokg{14.25}}
\malt{Flaked Oats}{\lbtokg{2.75}}
\malt{Caramel / Crystal 60 L}{\lbtokg{1.25}}
\malt{Caramel / Crystal 120 L}{\lbtokg{1.25}}
\malt{Roasted Barley}{\lbtokg{1.13}}
\malt{Black}{\oztokg{10}}
\malt{Chocolate}{\oztokg{10}}
\end{malts}

\begin{hops}
\hop{\hoppolaris}{19~\%}{60~min}{\oztog{1.5}}
\hop{\hoppolaris}{19~\%}{20~min}{\oztog{0.35}}
\hop{Light Brown Sugar}{}{}{\oztog{8}}
\end{hops}

\singleyeast{Fermentis SafAle US-05}

\end{ingredientsblock}

\end{recipe}

% -----------------------------------------------------------------------------
\begin{recipe}{Richard's Brownie Batter Imperial Stout}
% -----------------------------------------------------------------------------

\begin{aboutblock}
Recipe by Tom Lawrence of San Diego, CA. Gold medal in Category \#25: Spiced
Beer during the 2019 National Homebrew Competition in Providence, RI. \sourceaha
\end{aboutblock}

\specifications{\styleimperialstout}{\galtol{3}}{1.093}{1.030}{8.3~\%}{52}{\srmtoebc{40}}{240~min}{2.2}

\begin{methodandtiming}
 
\begin{mashsteps}
\mashstep{\ftoc{150}}{60~min}
\mashstep{\ftoc{170}}{Mash out}
\end{mashsteps}

\begin{fermentationsteps}
\fermentationstep{\ftoc{66}}{3~days}
\fermentationstep{\ftoc{72}}{Raise to by 0.5~°C/day; 28~days}
\fermentationstep{\ftoc{72}}{Transfer to secondary; 2~weeks}
\end{fermentationsteps}

\begin{directions}
Water adjustment: \waterprofile{53}{5}{61}{37}{45}{217}. Lightly toast shredded
coconut on the stove until light brown in color and let cool. Put coconut and
coffee beans in a muslin bag. Let beer condition on coconut and coffee at least
2 weeks.
\end{directions}

\end{methodandtiming}

\recipebreak

\begin{ingredientsblock}

\begin{malts}
\malt{Maris Otter}{\lbtokg{6}}
\malt{Weyermann CARAFA II}{\lbtokg{0.69}}
\malt{Briess Midnight Wheat}{\lbtokg{0.69}}
\malt{Flaked Oats}{\lbtokg{0.69}}
\malt{BEST Chit}{\lbtokg{0.69}}
\malt{Caramel / Crystal 10 L}{\lbtokg{0.69}}
\malt{Caramel / Crystal 20 L}{\lbtokg{0.5}}
\malt{Caramel / Crystal 40 L}{\lbtokg{0.5}}
\malt{Caramel / Crystal 60 L}{\oztog{3.04}}
\malt{Caramel / Crystal 80 L}{\oztog{3.04}}
\malt{Caramel / Crystal 120 L}{\oztog{3.04}}
\end{malts}

\begin{hops}
\hop{\hopmosaic}{}{240~min}{\oztog{1}}
\end{hops}

\singleyeast{RVA Yeast Labs RVA 132}

\begin{twists}
\twist{Coconut (Medium Toast)}{Secondary}{\lbtokg{1.5}}
\twist{Whole Arabica Coffee Beans}{Secondary}{\oztog{0.5}}
\end{twists}

\end{ingredientsblock}

\end{recipe}


% -----------------------------------------------------------------------------
\stylecategory{American IPA}
\stylesection{\styleamericanipa}

% -----------------------------------------------------------------------------
\begin{recipie}{Bell's Two Hearted Ale Clone}
% -----------------------------------------------------------------------------

\begin{aboutblock}
Bell's Brewery of Kalamazoo, Mich. brews a little beer called Two Hearted Ale.
Maybecyou've heard of it? This India pale ale is bursting with hop aromas ranging
from pinecto grapefruit thanks to the use of 100 percent Centennial hops. This
recipe was createdcby David Curtis and Ryan Kramer of Bell's General Store.
\sourceaha
\end{aboutblock}

\specifications{\styleamericanipa}{\galtol{5}}{1.063}{1.012}{6.7~\%}{55}{\srmtoebc{10}}{75~min}{}

\begin{methodandtiming}
 
\begin{mashsteps}
\mashstep{\ftoc{150}}{45~min}
\mashstep{\ftoc{170}}{Raise to over 15~min; 10~min}
\end{mashsteps}

\begin{directions}
Water adjustment: use carbon filtered water with 4~g calcium sulfate.
Dry hop 1 week into fermentation. Allow two hearted clone to stay warm with hops
for 1 week. Transfer beer, cold crash, and cold age for 1 week.
\end{directions}

\end{methodandtiming}

\pagebreak

\begin{ingredientsblock}

\begin{malts}
\malt{Briess Brewers}{\lbtokg{10}}
\malt{Briess Pale Ale}{\lbtokg{2.83}}
\malt{Briess Caramel 40 L}{\oztokg{8}}
\end{malts}

\begin{hops}
\hop{\hopcentennial}{9.1~\%}{45~min}{\oztog{1.2}}
\hop{\hopcentennial}{9.1~\%}{30~min}{\oztog{1.2}}
\hop{\hopcentennial}{9.1~\%}{\dryh{}{1~week}}{\oztog{3.5}}
\end{hops}

\singleyeast{White Labs WLP001 / White Labs WLP051}

\end{ingredientsblock}

\end{recipie}

% -----------------------------------------------------------------------------
\begin{recipie}{Bissel Brothers Brewing The Substance New England IPA Clone}
% -----------------------------------------------------------------------------

\begin{aboutblock}
The Substance from Bissel Brothers Brewing Co. flirts with the new world IPA style
in a way that intrigues and compels, adding complexity and not detracting from the
beer. It does have notes of tropical citrus, but it is still first and foremost
"dank" with a perceived bitterness that contributes to an overall balanced experience.
\sourceaha
\end{aboutblock}

\specifications{\styleamericanipa}{\galtol{5.16}}{1.061}{1.011}{}{}{}{}{}

\begin{methodandtiming}
 
\begin{mashsteps}
\mashstep{\ftoc{150}}{}
\end{mashsteps}

\begin{fermentationsteps}
\fermentationstep{\ftoc{68}}{1~day}
\fermentationstep{\ftoc{71}}{Until fully attenuated}
\end{fermentationsteps}

\end{methodandtiming}

\pagebreak

\begin{ingredientsblock}

\begin{malts}
\malt{Pale}{\lbtokg{10}}
\malt{Flaked Wheat}{\lbtokg{1.18}}
\malt{Caramel / Crystal 20 L}{\oztokg{5.6}}
\malt{Flaked Oats}{\oztokg{3.8}}
\end{malts}

\begin{hops}
\hop{\hopapollo}{}{Start}{\oztog{1}}
\hop{\hopfalconersflight}{}{\foh{}}{\oztog{1}}
\hop{\hopcentennial}{}{\foh{}}{\oztog{1}}
\hop{\hopfalconersflight}{}{\dryh{}{}}{\oztog{3}}
\hop{\hopcentennial}{}{\dryh{}{}}{\oztog{1.5}}
\hop{\hopeureka}{}{\dryh{}{}}{\oztog{1}}
\hop{\hopapollo}{}{\dryh{}{}}{\oztog{1}}
\hop{\hopchinook}{}{\dryh{}{}}{\oztog{1}}

\end{hops}

\singleyeast{Wyeast 2112 / White Labs WLP810}

\end{ingredientsblock}

\end{recipie}

% -----------------------------------------------------------------------------
\begin{recipie}{Cigar City Jai Alai IPA Classic Clone}
% -----------------------------------------------------------------------------

\begin{aboutblock}
This recipe courtesy of Amahl Turczyn. Cigar City brewmaster Wayne Wambles deemed
this recipe "pretty close to classic Jai Alai circa 2009--2010." He made a few
suggestions to bring this recipe closer to the original, including the use of
the Thames Valley ale yeast strain, Munich malt in the 6--10~L range, and an
updated water profile that "pushes the sulfate up to 150~ppm and the chloride
up to 125~ppm. This will jack up the calcium, but we find that quite a bit of
it precipitates out." \sourcezymurgy{July/August 2019}
\end{aboutblock}

\specifications{\styleamericanipa}{\galtol{5.5}}{1.074}{1.020}{7.5~\%}{70}{\srmtoebc{10}}{60~min}{}

\begin{methodandtiming}
 
\begin{mashsteps}
\mashstep{\ftoc{150}}{60~min}
\end{mashsteps}

\begin{fermentationsteps}
\fermentationstep{\ftoc{65}}{1~week}
\fermentationstep{\ftoc{34}}{Reduce to over 2~weeks}
\end{fermentationsteps}

\begin{directions}
Water adjustment: \waterprofile{100}{<10}{150}{<10}{125}{23}. Cold condition
10 days at \ftoc{34}, then carbonate and package.
\end{directions}

\end{methodandtiming}

\pagebreak

\begin{ingredientsblock}

\begin{malts}
\malt{Pale}{\lbtokg{13.25}}
\malt{Munich}{\oztog{12}}
\malt{Caramel / Crystal 60 L}{\oztog{12}}
\malt{Briess Victory}{\oztog{4}}
\end{malts}

\begin{hops}
\hop{\hopahtanum}{6~\%}{\fwh}{\oztog{0.5}}
\hop{\hopcolumbus}{14~\%}{\fwh}{\oztog{0.5}}
\hop{\hopahtanum}{6~\%}{60~min}{\oztog{0.5}}
\hop{\hopcolumbus}{14~\%}{60~min}{\oztog{0.25}}
\hop{\hopamarillo}{8.5~\%}{60~min}{\oztog{0.5}}
\hop{\hopcentennial}{10~\%}{15~min}{\oztog{0.5}}
\hop{\hopcascade}{5.5~\%}{15~min}{\oztog{0.5}}
\hop{\hopcolumbus}{14~\%}{15~min}{\oztog{0.5}}
\hop{Whirlfloc Tablet}{}{10~min}{1}
\hop{\hopahtanum}{6~\%}{5~min}{\oztog{0.5}}
\hop{\hopcascade}{5.5~\%}{5~min}{\oztog{0.5}}
\hop{\hopamarillo}{8.5~\%}{5~min}{\oztog{0.5}}
\hop{\hopsimcoe}{13~\%}{\dryh{}{2~weeks}}{\oztog{0.5}}
\end{hops}

\singleyeast{Wyeast 1275}

\end{ingredientsblock}

\end{recipie}

% -----------------------------------------------------------------------------
\begin{recipie}{Corridor Brewery Wizard Fight American IPA Clone}
% -----------------------------------------------------------------------------

\begin{aboutblock}
This flagship beer from Corridor Brewery features a plethora of cool kid hops
including Mosaic, Citra, and El Dorado create a citrus and tropical paradise.
\sourceaha
\end{aboutblock}

\specifications{\styleamericanipa}{\galtol{5}}{1.059}{1.010}{6.5~\%}{60}{}{90~min}{}

\begin{methodandtiming}
 
\begin{mashsteps}
\mashstep{\ftoc{152}}{60~min}
\end{mashsteps}

\begin{fermentationsteps}
\fermentationstep{\ftoc{68}}{}
\end{fermentationsteps}

\begin{directions}
Cold crash 1 day after dry hopping.
\end{directions}

\end{methodandtiming}

\pagebreak

\begin{ingredientsblock}

\begin{malts}
\malt{Two-row}{\lbtokg{8}}
\malt{Briess Bonlander Munich 10 L}{\oztokg{12}}
\malt{Carapils / Dextrin}{\oztokg{12}}
\malt{Flaked Oats}{\oztokg{4}}
\end{malts}

\begin{hops}
\hop{\hopwarrior}{16~\%}{90~min}{\oztog{0.75}}
\hop{\hopchinook}{13~\%}{15~min}{\oztog{0.5}}
\hop{\hopcitra}{13~\%}{\foh{}}{\oztog{0.64}}
\hop{\hopeldorado}{15~\%}{\foh{}}{\oztog{0.64}}
\hop{\hopmosaic}{11~\%}{\foh{}}{\oztog{0.64}}
\hop{\hopmosaic}{}{\dryh{}{5~days}}{\oztog{0.5}}
\hop{\hopcitra}{}{\dryh{}{5~days}}{\oztog{0.5}}
\hop{\hopeldorado}{}{\dryh{}{5~days}}{\oztog{0.5}}
\end{hops}

\singleyeast{Brewing Science Institute A-18}

\end{ingredientsblock}

\end{recipie}

% -----------------------------------------------------------------------------
\begin{recipie}{Deschutes' Fresh Squeezed IPA Clone}
% -----------------------------------------------------------------------------

\begin{aboutblock}
The name Fresh Squeezed IPA gives you an idea of what's in store when you brew
this clone. First and foremost, you should drink this beer fresh. This hop-centric
IPA has big, piney hop aroma that's full of fruit and peppery notes. It drips with
juicy citrus and grapefruit flavor thanks to the Citra hops, while the Mosaic hops
present soft, fruit flavors like honeydew. A mild malt profile of pale, Munich and
crystal take a back seat to the hops, making this easy to drink IPA. \sourceaha
\end{aboutblock}

\specifications{\styleamericanipa}{\galtol{5.5}}{1.066}{1.014}{7~\%}{60}{\srmtoebc{10}}{90~min}{}

\begin{methodandtiming}
 
\begin{mashsteps}
\mashstep{\ftoc{150}}{60~min}
\end{mashsteps}

\begin{fermentationsteps}
\fermentationstep{\ftoc{70}}{}
\end{fermentationsteps}

\begin{directions}
Water adjustment: use \gpgaltogpl{1} calcium sulphate to treat distilled or
reverse osmosis water. Drink as fresh as possible (2--3 weeks after packaging)
for maximum late hop character.
\end{directions}

\end{methodandtiming}

\pagebreak

\begin{ingredientsblock}

\begin{malts}
\malt{Pale}{\lbtokg{11}}
\malt{Munich}{\lbtokg{1.75}}
\malt{Caramel / Crystal 75 L}{\lbtokg{0.75}}
\end{malts}

\begin{hops}
\hop{\hopnugget}{13~\%}{60~min}{\oztog{0.5}}
\hop{\hopcitra}{12~\%}{15~min}{\oztog{1}}
\hop{\hopmosaic}{12~\%}{15~min}{\oztog{1}}
\hop{Whirlfloc Tablet}{}{10~min}{1}
\hop{\hopcitra}{12~\%}{\whirl{}{10~min}}{\oztog{1}}
\hop{\hopcitra}{12~\%}{\dryh{}{5~days}}{\oztog{1}}
\hop{\hopmosaic}{12~\%}{\dryh{}{5~days}}{\oztog{1}}

\end{hops}

\singleyeast{White Labs WLP001}

\end{ingredientsblock}

\end{recipie}

% -----------------------------------------------------------------------------
\begin{recipie}{Fargo Brewing Company Wood Chipper IPA Clone}
% -----------------------------------------------------------------------------

\begin{aboutblock}
This classic American IPA from Fargo Brewing Company showcases aromatic and bold
hop flavors. Horizon hops and oats provide a sleek, velvety body and balanced
bitterness while pounds per barrel of Cascade, Centennial, Chinook and Simcoe hops
give this IPA waves of citrus and pine flavors. That's one delicious beer eh? Oh
yeah, you betcha!
\end{aboutblock}

\specifications{\styleamericanipa}{\galtol{5}}{1.062}{}{6.5~\%}{59}{\srmtoebc{4.5}}{60~min}{}

\begin{methodandtiming}
 
\begin{mashsteps}
\mashstep{\ftoc{155}}{}
\end{mashsteps}

\end{methodandtiming}

\pagebreak

\begin{ingredientsblock}

\begin{malts}
\malt{Pale}{\lbtokg{10.5}}
\malt{Weyermann Munich I}{\lbtokg{1}}
\malt{Rice Hulls}{--}
\end{malts}

\begin{hops}
\hop{\hopcentennial}{5.8~\%}{FWH}{\oztog{0.5}}
\hop{\hophorizon}{11.3~\%}{60~min}{\oztog{0.75}}
\hop{Whirlfloc Tablet}{}{15~min}{1}
\hop{\hopcascade}{7.55~\%}{10~min}{\oztog{0.75}}
\hop{\hopchinook}{11.1~\%}{10~min}{\oztog{0.5}}
\hop{Servomyces}{}{10~min}{1}
\hop{\hopsimcoe}{}{\dryh{}{}}{\oztog{1}}
\hop{\hopchinook}{}{\dryh{}{}}{\oztog{1}}
\hop{\hopcentennial}{}{\dryh{}{}}{\oztog{1}}
\hop{\hopcascade}{}{\dryh{}{}}{\oztog{1}}
\end{hops}

\singleyeast{American Ale}

\end{ingredientsblock}

\end{recipie}

% -----------------------------------------------------------------------------
\begin{recipie}{Focal Point (Inspired by The Alchemist's Focal Banger)}
% -----------------------------------------------------------------------------

\begin{aboutblock}
The Alchemist's Focal Banger recipe is a closely guarded secret, but this American
IPA is inspired by famous hoppy ale. This recipe was formulated by the AHA with a
few tips from John Kimmich, head brewer and owner of the Stow, Vt.-based brewery.
\sourceaha
\end{aboutblock}

\specifications{\styleamericanipa}{\galtol{5.5}}{1.064}{1.012}{7~\%}{80}{\srmtoebc{5}}{60~min}{}

\begin{methodandtiming}

\begin{mashsteps}
\mashstep{\ftoc{150}}{75~min}
\end{mashsteps}

\begin{directions}
If you don't have the equipment to conduct a whirlpool at the end of the boil, 
simply conduct a hop stand by steeping the final addition of Mosaic in the hot
wort for 10 minutes before chilling. Add dry hops for 3 days, prior to packaging. 
\end{directions}

\end{methodandtiming}

\pagebreak

\begin{ingredientsblock}

\begin{malts}
\malt{Pale}{\lbtokg{9}}
\malt{Pilsner}{\lbtokg{4.8}}
\end{malts}

\begin{hops}
\hop{Hop Extract}{}{60~min}{6~ml}
\hop{\hopmosaic}{12.25~\%}{10~min}{\oztog{1}}
\hop{\hopmosaic}{12.25~\%}{5~min}{\oztog{1}}
\hop{\hopmosaic}{12.25~\%}{\whirl{}{}}{\oztog{1}}
\hop{\hopcitra}{12.25~\%}{\dryh{}{3~days}}{\oztog{4}}
\end{hops}

\singleyeast{White Labs WLP095 / Omega Yeast OLY-052 / GigaYeast GY054 / The Yeast Bay WLP4000 / Imperial Yeast A04}

\end{ingredientsblock}

\end{recipie}

% -----------------------------------------------------------------------------
\begin{recipie}{Good Word Brewing Never Sleep New England IPA Clone}
% -----------------------------------------------------------------------------

\begin{aboutblock}
This is one of Good Word Brewing's most popular beers, described as "juicy" without
being overly sweet thanks to Vic Secret and Citra hops. Pilsner malt dominates
alongside English pale malt and oats for a little more body and mouthfeel.
\sourceaha
\end{aboutblock}

\specifications{\styleamericanipa}{\galtol{5}}{1.065}{1.013}{7~\%}{45}{\srmtoebc{3.8}}{90~min}{}

\begin{methodandtiming}
 
\begin{mashsteps}
\mashstep{\ftoc{146}}{15~min}
\mashstep{\ftoc{156}}{30~min}
\mashstep{\ftoc{168}}{10~min}
\end{mashsteps}

\begin{fermentationsteps}
\fermentationstep{\ftoc{68}}{2~days}
\fermentationstep{\ftoc{70}}{2~days}
\fermentationstep{\ftoc{72}}{Until fully attenuated}
\end{fermentationsteps}

\begin{directions}
After boil, add whirlpool hops once wort is below \ftoc{180} to prevent 
isomerization of hops. Dry hop for 3 days when attenuation is within
0.5--1.0~°P. Complete a diacetyl rest before cold crashing. Cold crash at
\ftoc{32} for 4--6 days.
\end{directions}

\end{methodandtiming}

\pagebreak

\begin{ingredientsblock}

\begin{malts}
\malt{Pilsner}{\lbtokg{6.4}}
\malt{Pale}{\lbtokg{3.25}}
\malt{Oat}{\lbtokg{1.5}}
\end{malts}

\begin{hops}
\hop{Dextrose}{}{}{\oztog{14}}
\hop{\hopvicsecret}{}{10~min}{\oztog{1}}
\hop{\hopvicsecret}{}{5~min}{\oztog{1.5}}
\hop{\hopcitra}{}{5~min}{\oztog{1.5}}
\hop{\hopcitra}{}{\whirl{}{}}{\oztog{5}}
\hop{\hopvicsecret}{}{\whirl{}{}}{\oztog{5}}
\hop{\hopvicsecret}{}{\dryh{}{3~days}}{\oztog{8}}
\hop{\hopcitra}{}{\dryh{}{3~days}}{\oztog{8}}
\end{hops}

\singleyeast{Unspecified}

\end{ingredientsblock}

\end{recipie}

% -----------------------------------------------------------------------------
\begin{recipie}{Great South Bay Brewery Massive IPA}
% -----------------------------------------------------------------------------

\begin{aboutblock}
Let's just say this Massive IPA from New York's Great South Bay Brewery doesn't
skimp on the hops! With a hefty dose of flameout hops, Massive IPA is packed full
of American hop flavor and aroma. \sourceaha
\end{aboutblock}

\specifications{\styleamericanipa}{\galtol{5}}{1.065}{1.011}{6.7~\%}{55}{\srmtoebc{10}}{60~min}{2.5}

\begin{methodandtiming}
 
\begin{mashsteps}
\mashstep{\ftoc{151}}{}
\end{mashsteps}

\begin{directions}
Dry hop at high krausen.
\end{directions}

\end{methodandtiming}

\pagebreak

\begin{ingredientsblock}

\begin{malts}
\malt{Two-row}{\lbtokg{10.5}}
\malt{Weyermann CARAAMBER}{\lbtokg{0.75}}
\malt{Caramel / Crystal 30 L}{\lbtokg{0.5}}
\malt{Carapils / Dextrin}{\lbtokg{0.6}}
\malt{Flaked Oats}{\lbtokg{0.5}}
\end{malts}

\begin{hops}
\hop{\hopchinook}{}{60~min}{\oztog{0.75}}
\hop{\hopcentennial}{}{1~min}{\oztog{0.5}}
\hop{\hopcascade}{}{1~min}{\oztog{1}}
\hop{\hopsimcoe}{}{1~min}{\oztog{0.5}}
\hop{\hopcascade}{}{\foh{}}{\oztog{0.25}}
\hop{\hopcentennial}{}{\foh{}}{\oztog{0.25}}
\hop{\hopsimcoe}{}{\foh{}}{\oztog{0.5}}
\hop{\hopcascade}{}{\dryhbt{}{}}{\oztog{0.5}}
\hop{\hopsimcoe}{}{\dryhbt{}{}}{\oztog{0.5}}
\end{hops}

\singleyeast{American Ale}

\end{ingredientsblock}

\end{recipie}

% -----------------------------------------------------------------------------
\begin{recipie}{Green Flash West Coast IPA Clone}
% -----------------------------------------------------------------------------

\begin{aboutblock}
The plethora of hops creates a layered drinking experience of bitterness and
hop-forward flavor and aroma. The additions of Simcoe and Centennial give off
tropical grapefruit and piney notes respectively, while the Cascade hops add some
floral quality to compliment the spicy citrus notes of the Columbus and Amarillo
hops. With an assertive flavor and grapefruit bitterness held up by crystal malt,
it's a beer of extremes -- overdoing things just because they wanted to.
\sourcezymurgy{July/August 2013}
\end{aboutblock}

\specifications{\styleamericanipa}{\galtol{5.5}}{1.075}{1.018}{7.48~\%}{90}{\srmtoebc{7}}{90~min}{}

\begin{methodandtiming}
 
\begin{mashsteps}
\mashstep{\ftoc{152}}{60~min}
\mashstep{\ftoc{165}}{10~min}
\end{mashsteps}

\begin{fermentationsteps}
\fermentationstep{\ftoc{70}}{}
\end{fermentationsteps}

\begin{directions}
Water adjustment: to brew Green Flash West Coast IPA, the water profile should
be similar to Burton-Upon-Trent's, but with roughly half the mineral content.
When fermentation is complete, dry hop in primary fermenter for 1 week.
Drink it fresh for maximum late hop character.
\end{directions}

\end{methodandtiming}

\pagebreak

\begin{ingredientsblock}

\begin{malts}
\malt{Pale}{\lbtokg{12.5}}
\malt{Bairds Carastan}{\lbtokg{1.25}}
\malt{Briess Carapils}{\lbtokg{1.25}}
\end{malts}

\begin{hops}
\hop{\hopsimcoe}{13~\%}{90~min}{\oztog{0.5}}
\hop{\hopcolumbus}{14~\%}{90~min}{\oztog{0.25}}
\hop{\hopsimcoe}{13~\%}{60~min}{\oztog{0.25}}
\hop{\hopcolumbus}{14~\%}{60~min}{\oztog{0.25}}
\hop{\hopsimcoe}{13~\%}{30~min}{\oztog{0.25}}
\hop{\hopcolumbus}{14~\%}{30~min}{\oztog{0.25}}
\hop{\hopsimcoe}{13~\%}{15~min}{\oztog{0.75}}
\hop{\hopcolumbus}{14~\%}{15~min}{\oztog{0.75}}
\hop{Whirlfloc Tablet}{}{10~min}{1}
\hop{\hopsimcoe}{13~\%}{\whirl{}{10~min}}{\oztog{0.5}}
\hop{\hopcolumbus}{14~\%}{\whirl{}{10~min}}{\oztog{0.5}}
\hop{\hopcascade}{5.75~\%}{\whirl{}{10~min}}{\oztog{1}}
\hop{\hopsimcoe}{13~\%}{\dryh{}{1~week}}{\oztog{0.5}}
\hop{\hopcascade}{5.75~\%}{\dryh{}{1~week}}{\oztog{0.5}}
\hop{\hopamarillo}{10~\%}{\dryh{}{1~week}}{\oztog{0.5}}
\hop{\hopcentennial}{10.5~\%}{\dryh{}{1~week}}{\oztog{0.5}}
\end{hops}

\singleyeast{White Labs WLP001}

\end{ingredientsblock}

\end{recipie}

% -----------------------------------------------------------------------------
\begin{recipie}{Lazy Magnolia Brewery Southern Hops'pitality IPA Clone}
% -----------------------------------------------------------------------------

\begin{aboutblock}
This session IPA from Lazy Magnolia Brewing in Kiln, Miss. is meant for sharing
with family and friends. It's described as having a bold citrus burst on the front
end, with hints of tropical fruits such as grapefruit, orange, and mango in the
finish. \sourceaha
\end{aboutblock}

\specifications{\styleamericanipa}{\galtol{5.5}}{1.054}{1.010}{5.8~\%}{}{}{90~min}{}

\begin{methodandtiming}
 
\begin{mashsteps}
\mashstep{\ftoc{152}}{60~min}
\end{mashsteps}

\begin{fermentationsteps}
\fermentationstep{\ftoc{65}}{}
\end{fermentationsteps}

\begin{directions}
Water adjustment: use water with calcium chloride and food-grade acid added to
increase hardness and reduce alkalinity if necessary. Our water is extremely
alkaline with zero hardness and calcium. Unless your water is extremely soft,
mineral additions may not be required.
\end{directions}

\end{methodandtiming}

\pagebreak

\begin{ingredientsblock}

\begin{malts}
\malt{Two-row}{\lbtokg{10}}
\malt{Carapils / Dextrin}{\lbtokg{0.5}}
\malt{Caramel / Crystal 40 L}{\lbtokg{0.5}}
\end{malts}

\begin{hops}
\hop{\hopnugget}{}{60~min}{\oztog{0.25}}
\hop{\hopcentennial}{}{10~min}{\oztog{0.5}}
\hop{\hopcitra}{}{10~min}{\oztog{0.5}}
\hop{\hopsimcoe}{}{10~min}{\oztog{1}}
\hop{Whirlfloc Tablet}{}{10~min}{1}
\hop{\hopcentennial}{}{5~min}{\oztog{0.6}}
\hop{\hopcitra}{}{5~min}{\oztog{0.6}}
\hop{\hopsimcoe}{}{5~min}{\oztog{0.6}}
\hop{\hopcitra}{}{\whirl{}{}}{\oztog{0.75}}
\hop{\hopsummit}{}{\whirl{}{}}{\oztog{0.6}}
\hop{\hopsimcoe}{}{\whirl{}{}}{\oztog{0.6}}
\hop{\hopfalconersflight}{}{\whirl{}{}}{\oztog{0.6}}
\hop{\hopcitra}{}{\dryh{}{}}{\oztog{1.5}}
\end{hops}

\singleyeast{Fermentis SafAle US-05 / White Labs WLP001}

\end{ingredientsblock}

\end{recipie}

% -----------------------------------------------------------------------------
\begin{recipie}{Lupulin Brewing Straight Hash Homie IPA Clone}
% -----------------------------------------------------------------------------

\begin{aboutblock}
This appropriately named IPA from Lupulin Brewing is made with pure lupulin
powder, paying homage to the brewery's name. The bursting tropical flavors and
soft bitterness may fool you, but no pellet or hop touched this beer. Brew your
own and see what you think! \sourceaha
\end{aboutblock}

\specifications{\styleamericanipa}{\galtol{6}}{1.076}{1.018}{7.75~\%}{60}{\srmtoebc{5.5}}{60~min}{}

\begin{methodandtiming}
 
\begin{mashsteps}
\mashstep{\ftoc{152}}{}
\end{mashsteps}

\begin{fermentationsteps}
\fermentationstep{\ftoc{66}}{Fermentation start}
\fermentationstep{\ftoc{72}}{Raise to over 1~week}
\end{fermentationsteps}

\end{methodandtiming}

\pagebreak

\begin{ingredientsblock}

\begin{malts}
\malt{Two-row}{\lbtokg{10.5}}
\malt{Caramalt}{\lbtokg{1}}
\malt{Flaked Rye}{\lbtokg{1}}
\malt{Vienna}{\lbtokg{1}}
\end{malts}

\begin{hops}
\hop{Dextrose}{}{60~min}{\oztog{12}}
\hop{\hopcitra ~Hash}{}{\whirl{}{}}{\oztog{2}}
\hop{\hopsimcoe ~Hash}{}{\whirl{}{}}{\oztog{2}}
\hop{\hopcitra ~Hash}{}{\dryh{}{3~days}}{\oztog{3}}
\hop{\hopmosaic ~Hash}{}{\dryh{}{3~days}}{\oztog{2}}
\end{hops}

\singleyeast{Omega Yeast OYL-052}

\end{ingredientsblock}

\end{recipie}

% -----------------------------------------------------------------------------
\begin{recipie}{Odell IPA Clone}
% -----------------------------------------------------------------------------

\begin{aboutblock}
Odell Brewing Co. of Fort Collins, Colorado makes one tasty IPA, and the 2007
Great American Beer Festival judges agreed when they awarded Odell IPA with a
gold medal in the "American-style IPA" category. \sourcezymurgy{July/August 2011}
\end{aboutblock}

\specifications{\styleamericanipa}{\galtol{5.5}}{1.067}{}{}{47}{\srmtoebc{7}}{90~min}{}

\begin{methodandtiming}
 
\begin{mashsteps}
\mashstep{\ftoc{154}}{60~min}
\mashstep{\ftoc{168}}{10~min}
\end{mashsteps}

\begin{fermentationsteps}
\fermentationstep{\ftoc{68}}{Until fully attenuated}
\fermentationstep{\ftoc{60}}{Transfer to secondary; 1~week}
\end{fermentationsteps}

\begin{directions}
Use a hopback at runoff for \oztog{1} Simcoe and Chinook additions, or steep
whole flowers at flameout for 10 minutes. Condition in secondary on dry hops.
\end{directions}

\end{methodandtiming}

\pagebreak

\begin{ingredientsblock}

\begin{malts}
\malt{Gambrinus ESB Pale}{\lbtokg{6}}
\malt{Pale}{\lbtokg{5}}
\malt{Vienna}{\lbtokg{2}}
\malt{Thomas Fawcett Caramalt}{\oztokg{10}}
\malt{Weyermann CARAFOAM}{\oztokg{8}}
\end{malts}

\begin{hops}
\hop{\hophorizon}{13~\%}{90~min}{\oztog{0.75}}
\hop{\hopsimcoe}{13~\%}{90~min}{\oztog{0.5}}
\hop{\hopcolumbus}{15~\%}{Hopback}{\oztog{1}}
\hop{\hopchinook}{13~\%}{Hopback}{\oztog{1}}
\hop{\hopsimcoe}{13~\%}{\dryh{}{1~week}}{\oztog{0.5}}
\hop{\hophorizon}{13~\%}{\dryh{}{1~week}}{\oztog{0.5}}
\hop{\hopamarillo}{13~\%}{\dryh{}{1~week}}{\oztog{0.5}}
\hop{\hopcentennial}{13~\%}{\dryh{}{1~week}}{\oztog{0.5}}
\end{hops}

\singleyeast{Nottingham Ale}

\end{ingredientsblock}

\end{recipie}

% -----------------------------------------------------------------------------
\begin{recipie}{Perfect Plain Brewing Co. Holy Spin American IPA Clone}
% -----------------------------------------------------------------------------

\begin{aboutblock}
Dubbed the Holy Spin, the third spin of a vinyl record is known to be when the
tunes are at their best. This IPA recipe from Perfect Plain Brewing Co. was the
third turn of their brewhouse and is dry-hopped, abundantly so, with citra hops.
\sourceaha
\end{aboutblock}

\specifications{\styleamericanipa}{\galtol{5}}{1.061}{1.010}{6.5~\%}{43}{\srmtoebc{5}}{90~min}{2.5}

\begin{methodandtiming}
 
\begin{mashsteps}
\mashstep{\ftoc{150}}{}
\end{mashsteps}

\begin{directions}
Targeting a mash pH between 5.2 and 5.4. Chill wort below \ftoc{170} or below
before adding whirlpool hops for 15 minutes. Dry hop with \oztog{2} of Citra
hops 2 days after the start of fermentation. Once full attenuation
is reached, add the remaining \oztog{5} of Citra dry hops.
\end{directions}

\end{methodandtiming}

\pagebreak

\begin{ingredientsblock}

\begin{malts}
\malt{Pale}{\lbtokg{8.5}}
\malt{Unmalted White Wheat}{\lbtokg{2.1}}
\malt{Flaked Oats}{\oztokg{10}}
\end{malts}

\begin{hops}
\hop{\hopnugget}{12.5~\%}{60~min}{\oztog{1}}
\hop{\hopcitra}{}{\whirl{}{}}{\oztog{2}}
\hop{\hopcitra}{}{\dryhbt{}{}}{\oztog{2}}
\hop{\hopcitra}{}{\dryh{}{5~days}}{\oztog{5}}
\end{hops}

\singleyeast{Fermentis SafAle US-05}

\end{ingredientsblock}

\end{recipie}

% -----------------------------------------------------------------------------
\begin{recipie}{Providence Brewing Company Battlecow Galacticose New England IPA Clone}
% -----------------------------------------------------------------------------

\begin{aboutblock}
This deceptively strong beer from Providence Brewing Company is brewed with
Two-row pale malt, CARA\-FOAM, rolled oats, milk sugar, and intensely double
dry-hopped with Citra and Mosaic giving it a decidedly dank aroma. Bursting
with mango, orange and pineapple flavors, this is a juicy milkshake New England
IPA that'll have you begging for another sip. \sourceaha
\end{aboutblock}

\specifications{\styleamericanipa}{\galtol{5.5}}{1.071}{1.019}{8.1~\%}{70}{\srmtoebc{3.6}}{60~min}{}

\begin{methodandtiming}
 
\begin{mashsteps}
\mashstep{\ftoc{155}}{60~min}
\end{mashsteps}

\begin{fermentationsteps}
\fermentationstep{\ftoc{70}}{}
\end{fermentationsteps}

\begin{directions}
Turn off heat and let temperature drop to \ftoc{198}, then add the whirlpool
hops and whirlpool for 30 minutes. Two days after pitching the yeast, add the
first dry hopadditions. After active fermentation has stopped, typically 5~days
after pitching the yeast, add the second round of dry hops. A day later, fill your
fermenter with a 30 second burst of \ce{CO2} to rouse the hops. Add \oztog{2}
Citra and \oztog{2} Mosaic pellet hops 8--10 days later. Just before kegging
your beer, bag and add last dry hop additions to the keg or bottling bucket.
\end{directions}

\end{methodandtiming}

\pagebreak

\begin{ingredientsblock}

\begin{malts}
\malt{Two-row}{\lbtokg{10}}
\malt{Flaked Oats}{\lbtokg{2}}
\malt{Weyermann CARAFOAM}{\lbtokg{1.5}}
\malt{White Wheat}{\lbtokg{1.5}}
\end{malts}

\begin{hops}
\hop{\hopcolumbus ~Cones}{14~\%}{FWH}{\oztog{2}}
\hop{\hopcolumbus}{14~\%}{60~min}{\oztog{2}}
\hop{Lactose}{}{10~min}{\lbtokg{2}}
\hop{\hopcitra}{12~\%}{\whirl{}{30~min}}{\oztog{2}}
\hop{\hopmosaic}{12.25~\%}{\whirl{}{30~min}}{\oztog{2}}
\hop{\hopcitra}{12~\%}{\dryhbt{}{}}{\oztog{1}}
\hop{\hopmosaic}{12.25~\%}{\dryhbt{}{}}{\oztog{1}}
\hop{\hopcitra}{12~\%}{\dryh{1}{4~days}}{\oztog{1}}
\hop{\hopmosaic}{12.25~\%}{\dryh{1}{4~days}}{\oztog{1}}
\hop{\hopcitra}{12~\%}{\dryh{2}{10~days}}{\oztog{2}}
\hop{\hopmosaic}{12.25~\%}{\dryh{2}{10~days}}{\oztog{2}}
\hop{\hopcitra ~Cryo}{26~\%}{\dryh{3}{kegging}}{\oztog{1}}
\hop{\hopmosaic ~Cryo}{26~\%}{\dryh{3}{kegging}}{\oztog{1}}
\end{hops}

\singleyeast{The Yeast Bay WLP4042}

\end{ingredientsblock}

\end{recipie}

% -----------------------------------------------------------------------------
\begin{recipie}{Red Door Brewing Company New England IPA Clone}
% -----------------------------------------------------------------------------

\begin{aboutblock}
This juicy and hazy India pale ale from Red Door Brewing Co. features an intense
tropical fruit and floral nose. This is a perfect choice for warm weather.
\sourceaha
\end{aboutblock}

\specifications{\styleamericanipa}{\galtol{5}}{1.066}{1.014}{7.1~\%}{69}{\srmtoebc{4.3}}{60~min}{}

\begin{methodandtiming}
 
\begin{mashsteps}
\mashstep{\ftoc{154}}{60~min}
\end{mashsteps}

\begin{fermentationsteps}
\fermentationstep{\ftoc{68}}{}
\end{fermentationsteps}

\begin{directions}
Add the first dry hops and recirculate. After 2 days, add the second dry hops
and recirculate. Cold crash the next day to \ftoc{33}. Leave cold crashed
for 5 days.
\end{directions}

\end{methodandtiming}

\pagebreak

\begin{ingredientsblock}

\begin{malts}
\malt{Two-row}{\lbtokg{7.45}}
\malt{Malted White Wheat}{\lbtokg{3.77}}
\malt{Flaked Oats}{\lbtokg{2.18}}
\end{malts}

\begin{hops}
\hop{\hopcitra}{13.7~\%}{\fwh}{\oztog{0.2}}
\hop{\hopmosaic}{10.7~\%}{\fwh}{\oztog{0.2}}
\hop{\hopeldorado}{15~\%}{\fwh}{\oztog{0.2}}
\hop{Zinc}{}{15~min}{16~ml}
\hop{\hopcitra}{13.7~\%}{\whirl{}{15~min}}{\oztog{1.3}}
\hop{\hopmosaic}{10.7~\%}{\whirl{}{15~min}}{\oztog{1.3}}
\hop{\hopeldorado}{15~\%}{\whirl{}{15~min}}{\oztog{1.3}}
\hop{\hopcitra}{13.7~\%}{\dryh{1}{2~days}}{\oztog{1.3}}
\hop{\hopmosaic}{10.7~\%}{\dryh{1}{2~days}}{\oztog{1.3}}
\hop{\hopeldorado}{15~\%}{\dryh{1}{2~days}}{\oztog{1.3}}
\hop{\hopcitra}{13.7~\%}{\dryh{2}{5~days}}{\oztog{1.3}}
\hop{\hopmosaic}{10.7~\%}{\dryh{2}{5~days}}{\oztog{1.3}}
\hop{\hopeldorado}{15~\%}{\dryh{2}{5~days}}{\oztog{1.3}}
\end{hops}

\singleyeast{Wyeast 1318}

\end{ingredientsblock}

\end{recipie}

% -----------------------------------------------------------------------------
\begin{recipie}{Russian River Blind Pig IPA Clone}
% -----------------------------------------------------------------------------

\begin{aboutblock}
Vinnie Cilurzo says about the groundbreaking Blind Pig IPA: "Blind Pig IPA was
first brewed in Temecula, Calif. at my first brewery, Blind Pig Brewing Company,
in 1994. This version was 92 bittering units and had very little malt with a very
forward hop character. In December 1996, I left the brewery and my former business
partner continued the brewery for a few years. After the original brewery closed
and I was at Russian River Brewing Company, we were able to trademark the name and
start making Blind Pig IPA again. The recipe has changed, in that we have
added a couple of new hop varieties that were not in existence when the brewery
in Temecula was open." \sourcezymurgy{July/August 2012}
\end{aboutblock}

\specifications{\styleamericanipa}{\galtol{5}}{1.057}{1.013}{6.1~\%}{62}{}{90~min}{}

\begin{methodandtiming}
 
\begin{mashsteps}
\mashstep{\ftoc{153}}{60~min}
\end{mashsteps}

\begin{fermentationsteps}
\fermentationstep{\ftoc{68}}{}
\end{fermentationsteps}

\end{methodandtiming}

\pagebreak

\begin{ingredientsblock}

\begin{malts}
\malt{Pale}{\lbtokg{9.8}}
\malt{Caramel / Crystal 40 L}{\oztokg{6.5}}
\malt{Carapils / Dextrin}{\oztokg{5}}
\end{malts}

\begin{hops}
\hop{\hopcolumbus}{16~\%}{90~min}{\oztog{0.25}}
\hop{\hopchinook}{13~\%}{90~min}{\oztog{0.5}}
\hop{\hopamarillo}{7.5~\%}{30~min}{\oztog{0.5}}
\hop{\hopcascade}{5.75~\%}{\foh{}}{\oztog{0.5}}
\hop{\hopamarillo}{7.5~\%}{\foh{}}{\oztog{0.5}}
\hop{\hopcentennial}{10.5~\%}{\foh{}}{\oztog{0.5}}
\hop{\hopsimcoe}{10.5~\%}{\foh{}}{\oztog{0.5}}
\hop{\hopcascade}{5.75~\%}{\dryh{}{10~days}}{\oztog{0.5}}
\hop{\hopamarillo}{7.5~\%}{\dryh{}{10~days}}{\oztog{0.5}}
\hop{\hopcolumbus}{16~\%}{\dryh{}{10~days}}{\oztog{0.5}}
\end{hops}

\singleyeast{White Labs WLP001 / Wyeast 1056}

\end{ingredientsblock}

\end{recipie}

% -----------------------------------------------------------------------------
\begin{recipie}{Spice Trade Brewing Sun Temple IPA}
% -----------------------------------------------------------------------------

\begin{aboutblock}
Recipe courtesy Jeff Tyler, Spice Trade Brewing Co., Arvada, Colo. Tyler says:
Don't use any cold-side clarifying agents! Haze means there are hop polyphenols
in solution, which promote magical, juicy, fruit-forward flavor!
\sourcezymurgy{July/August 2017}
\end{aboutblock}

\specifications{\styleamericanipa}{\galtol{5.5}}{1.064}{1.010}{7.1~\%}{75}{\srmtoebc{7}}{60~min}{2.6}

\begin{methodandtiming}
 
\begin{mashsteps}
\mashstep{\ftoc{148}}{60~min}
\end{mashsteps}

\begin{fermentationsteps}
\fermentationstep{\ftoc{68}}{}
\end{fermentationsteps}

\begin{directions}
Water adjustment: 100~ppm chloride and 200~ppm sulfate. After flameout,
add whirlpool hops and stir wort for 30 minutes to create a whirlpool and
precipitate out the trub.
\end{directions}

\end{methodandtiming}

\pagebreak

\begin{ingredientsblock}

\begin{malts}
\malt{Pale}{\lbtokg{10.63}}
\malt{Weyermann CARAAMER}{\oztokg{6}}
\malt{Dingemans Special B}{\oztokg{3}}
\end{malts}

\begin{hops}
\hop{\hopmagnum}{12.3~\%}{60~min}{\oztog{0.6}}
\hop{\hopeldorado}{9~\%}{20~min}{\oztog{1.9}}
\hop{Yeast Nutrient}{}{15~min}{\tsptog{0.25}}
\hop{\hopcitra}{14.1~\%}{10~min}{\oztog{0.8}}
\hop{Whirlfloc Tablet}{}{10~min}{1}
\hop{Dextrose}{}{10~min}{\lbtokg{1}}
\hop{\hopeldorado}{9~\%}{FO}{\oztog{0.8}}
\hop{\hopsimcoe}{12.3~\%}{FO}{\oztog{0.3}}
\hop{\hopeldorado}{9~\%}{\whirl{}{30~min}}{\oztog{0.75}}
\hop{\hopsimcoe}{12.3~\%}{\whirl{}{30~min}}{\oztog{1.25}}
\hop{\hopeldorado}{9~\%}{\dryh{1}{4~days}}{\oztog{1.2}}
\hop{\hopsimcoe}{12.3~\%}{\dryh{1}{4~days}}{\oztog{0.7}}
\hop{\hopcitra}{14.1~\%}{\dryh{1}{4~days}}{\oztog{0.7}}
\hop{\hopeldorado}{9~\%}{\dryh{2}{3~days}}{\oztog{1.2}}
\hop{\hopsimcoe}{12.3~\%}{\dryh{2}{3~days}}{\oztog{0.7}}
\hop{\hopcitra}{14.1~\%}{\dryh{2}{3~days}}{\oztog{0.7}}
\hop{\hopeldorado}{9~\%}{\dryh{3}{}}{\oztog{1.2}}
\hop{\hopsimcoe}{12.3~\%}{\dryh{3}{}}{\oztog{0.7}}
\hop{\hopcitra}{14.1~\%}{\dryh{3}{}}{\oztog{0.8}}
\end{hops}

\singleyeast{GigaYeast GY054 / Inland Island Yeast Lab. INIS-003 / The Yeast Bay WLP 4000}

\end{ingredientsblock}

\end{recipie}

% -----------------------------------------------------------------------------
\begin{recipie}{Uinta Brewing Co. Hop Nosh Tangerine Clone}
% -----------------------------------------------------------------------------

\begin{aboutblock}
Hop Nosh Tangerine is a play on Uinta Brewing's (Salt Lake City, Utah) flagship
IPA. This homebrew recipe features an aromatic menagerie of tropical hops and
tangerine zest and wraps up with a crisp, bitter finish. \sourceaha
\end{aboutblock}

\specifications{\styleamericanipa}{\galtol{5}}{1.065}{}{6.7~\%}{}{}{60~min}{}

\begin{methodandtiming}
 
\begin{mashsteps}
\mashstep{\ftoc{154}}{}
\end{mashsteps}

\begin{fermentationsteps}
\fermentationstep{\ftoc{66}}{}
\end{fermentationsteps}

\end{methodandtiming}

\pagebreak

\begin{ingredientsblock}

\begin{malts}
\malt{Pale}{\lbtokg{11}}
\malt{Munich}{\lbtokg{1}}
\malt{Caramel / Crystal 40 L}{\lbtokg{0.5}}
\end{malts}

\begin{hops}
\hop{\hopchinook}{}{60~min}{\oztog{1}}
\hop{\hopcascade}{}{30~min}{\oztog{1.5}}
\hop{\hopbravo}{}{5~min}{\oztog{1}}
\hop{\hopcascade}{}{5~min}{\oztog{1}}
\hop{Tangerine Concentrate}{}{1~min}{1~l}
\hop{\hopbravo}{}{\whirl{}{}}{\oztog{0.5}}
\hop{\hopcitra}{}{\whirl{}{}}{\oztog{0.5}}
\hop{\hopgalaxy}{}{\whirl{}{}}{\oztog{1}}
\hop{\hopgalaxy}{}{\dryh{}{}}{\oztog{1}}
\hop{\hopcitra}{}{\dryh{}{}}{\oztog{0.75}}
\hop{\hopchinook}{}{\dryh{}{}}{\oztog{0.75}}
\end{hops}

\singleyeast{Wyeast 1007}

\end{ingredientsblock}

\end{recipie}

% -----------------------------------------------------------------------------
\begin{recipie}{Von Ebert Brewing Sabrage Brut IPA Clone}
% -----------------------------------------------------------------------------

\begin{aboutblock}
Sabrage is the technique for opening a champagne bottle with a saber, a fitting name
for this beer recipe from Von Ebert Brewing. This recipe is dry and effervescent,
with minimal bitterness and hop character that's dominated by grapefruit-heavy Citra
and a touch of resinous Chinook. You don't need a ceremonial occasion to drink
this beer! \sourceaha
\end{aboutblock}

\specifications{\styleamericanipa}{\galtol{5.5}}{1.050}{}{6.6~\%}{10}{}{90~min}{3}

\begin{methodandtiming}
 
\begin{mashsteps}
\mashstep{\ftoc{142}}{60~min}
\mashstep{\ftoc{168}}{Mashout}
\end{mashsteps}

\begin{fermentationsteps}
\fermentationstep{\ftoc{65}}{}
\end{fermentationsteps}

\begin{directions}
For a special version of this recipe, try incorporating a grape varietal into the
fermentation.
\end{directions}

\end{methodandtiming}

\pagebreak

\begin{ingredientsblock}

\begin{malts}
\malt{Pilsner}{\lbtokg{8}}
\malt{Flaked Rice}{\lbtokg{2}}
\end{malts}

\begin{hops}
\hop{\hopcitra}{}{Mash}{\oztog{2}}
\hop{\hopcitra}{}{\fwh}{\oztog{2}}
\hop{\hopcitra}{}{\whirl{}{}}{\oztog{2}}
\hop{\hopcitra}{}{\dryh{}{3~days}}{\oztog{6}}
\hop{\hopchinook}{}{\dryh{}{3~days}}{\oztog{2}}
\end{hops}

\singleyeast{Neutral Ale}

\end{ingredientsblock}

\end{recipie}

% -----------------------------------------------------------------------------
\begin{recipie}{WeldWerks Brewing Juicy Bits NEIPA Clone}
% -----------------------------------------------------------------------------

\begin{aboutblock}
From WeldWerks: "Our version of a New England-style IPA featuring a huge citrus
and tropical fruit character from the Mosaic, Citra, and El Dorado hops, a softer,
fluffier mouthfeel from the lower attenuation, and the characteristic New England
hop haze. The end result is a beer reminiscent of citrus juice with extra pulp,
thus the name Juicy Bits." \sourcezymurgy{July/August 2018}
\end{aboutblock}

\specifications{\styleamericanipa}{\galtol{5}}{1.062}{1.012}{}{45}{\srmtoebc{4.5}}{90~min}{}

\begin{methodandtiming}
 
\begin{mashsteps}
\mashstep{\ftoc{149}}{45~min}
\end{mashsteps}

\begin{fermentationsteps}
\fermentationstep{\ftoc{67}}{}
\end{fermentationsteps}

\begin{directions}
Water adjustment: about 250~ppm chloride and 80~ppm sulfate. Add the first dry hop
addition when the beer has fermented to about 2--3~°P from final gravity,
and then add the last two dry hop additions after final gravity has been reached.
\end{directions}

\end{methodandtiming}

\pagebreak

\begin{ingredientsblock}

\begin{malts}
\malt{Pale}{\lbtokg{4}}
\malt{Pilsner}{\lbtokg{4}}
\malt{Carapils}{\lbtokg{1}}
\malt{Pale Wheat}{\lbtokg{1}}
\malt{Flaked Oats}{\oztokg{12}}
\malt{Flaked Wheat}{\oztokg{12}}
\malt{Wheat}{\oztokg{12}}
\end{malts}

\begin{hops}
\hop{\hopmagnum}{14~\%}{FWH}{\oztog{0.33}}
\hop{Dextrose}{}{90~min}{\oztog{6}}
\hop{\hopcitra}{12.5~\%}{\whirl{1}{10~min}}{\oztog{0.33}}
\hop{\hopeldorado}{15.7~\%}{\whirl{1}{10~min}}{\oztog{0.33}}
\hop{\hopmosaic}{13.1~\%}{\whirl{1}{10~min}}{\oztog{0.33}}
\hop{\hopcitra}{12.5~\%}{\whirl{2}{10~min}}{\oztog{0.66}}
\hop{\hopeldorado}{15.7~\%}{\whirl{2}{10~min}}{\oztog{0.66}}
\hop{\hopmosaic}{13.1~\%}{\whirl{2}{10~min}}{\oztog{0.66}}
\hop{\hopcitra}{12.5~\%}{\whirl{3}{20~min}}{\oztog{1}}
\hop{\hopeldorado}{15.7~\%}{\whirl{3}{20~min}}{\oztog{1}}
\hop{\hopmosaic}{13.1~\%}{\whirl{3}{20~min}}{\oztog{1}}
\hop{\hopcitra}{12.5~\%}{\dryhbt{}{}}{\oztog{0.5}}
\hop{\hopeldorado}{15.7~\%}{\dryhbt{}{}}{\oztog{0.5}}
\hop{\hopmosaic}{13.1~\%}{\dryhbt{}{}}{\oztog{0.5}}
\hop{\hopcitra}{12.5~\%}{\dryh{1}{3~days}}{\oztog{0.5}}
\hop{\hopeldorado}{15.7~\%}{\dryh{1}{3~days}}{\oztog{0.5}}
\hop{\hopmosaic}{13.1~\%}{\dryh{1}{3~days}}{\oztog{0.5}}
\hop{\hopcitra}{12.5~\%}{\dryh{2}{3~days}}{\oztog{1}}
\hop{\hopeldorado}{15.7~\%}{\dryh{2}{3~days}}{\oztog{1}}
\hop{\hopmosaic}{13.1~\%}{\dryh{2}{3~days}}{\oztog{1}}
\end{hops}

\end{ingredientsblock}

\pagebreak

\begin{ingredientsblock}

\singleyeast{Wyeast 1318}

\end{ingredientsblock}

\end{recipie}

% -----------------------------------------------------------------------------
\begin{recipie}{Whetstone Station Brewery Whetstoner Session IPA Clone}
% -----------------------------------------------------------------------------

\begin{aboutblock}
This bright and delicious session IPA from Whetstone Craft Beers features Simcoe,
Amarillo and Citra hops. While it's hazy, aromatic and full of flavor, at just
4.5~\% ABV this crisp beer is perfect for when you've got thirst that needs quenching.
\sourceaha
\end{aboutblock}

\specifications{\styleamericanipa}{\galtol{5}}{1.045}{1.010}{4.6~\%}{36}{\srmtoebc{4.8}}{90~min}{2.3}

\begin{methodandtiming}
 
\begin{mashsteps}
\mashstep{\ftoc{150}}{60~min}
\mashstep{\ftoc{168}}{Mashout}
\end{mashsteps}

\begin{fermentationsteps}
\fermentationstep{\ftoc{65}}{}
\end{fermentationsteps}

\end{methodandtiming}

\pagebreak

\begin{ingredientsblock}

\begin{malts}
\malt{Pale}{\lbtokg{7}}
\malt{White Wheat}{\lbtokg{2}}
\malt{Caramel / Crystal 30 L}{\lbtokg{0.5}}
\end{malts}

\begin{hops}
\hop{\hopamarillo}{9.2~\%}{\whirl{}{}}{\oztog{2.25}}
\hop{\hopsimcoe}{13~\%}{\whirl{}{}}{\oztog{1.5}}
\hop{\hopamarillo}{9.2~\%}{\dryh{}{2~days}}{\oztog{0.75}}
\hop{\hopcitra}{13.4~\%}{\dryh{}{2~days}}{\oztog{0.75}}
\hop{\hopsimcoe}{13~\%}{\dryh{}{2~days}}{\oztog{0.75}}
\end{hops}

\singleyeast{American Ale}

\end{ingredientsblock}

\end{recipie}

% -----------------------------------------------------------------------------
\begin{recipie}{Zipline Brewing Co. NZ IPA Clone}
% -----------------------------------------------------------------------------

\begin{aboutblock}
Zipline's India pale ale is brewed in Lincoln, Neb. and is packed with New Zealand
hop varieties like Pacific Jade, Rakau, and Wakatu, known for their exotic fruity
flavors and aromas. \sourceaha
\end{aboutblock}

\specifications{\styleamericanipa}{\galtol{5.25}}{1.062}{}{}{}{}{60~min}{}

\begin{methodandtiming}
 
\begin{mashsteps}
\mashstep{\ftoc{152}}{}
\end{mashsteps}

\begin{directions}
Target mash pH is 5.3. Add dry hops after 9~days in primary fermentation.
\end{directions}

\end{methodandtiming}

\pagebreak

\begin{ingredientsblock}

\begin{malts}
\malt{Briess Pilsner}{\lbtokg{9.15}}
\malt{Weyermann Vienna}{\lbtokg{0.9}}
\malt{Weyermann CARAMUNICH II}{\lbtokg{0.65}}
\malt{Briess Victory}{\lbtokg{0.48}}
\malt{Flaked White Wheat}{\lbtokg{0.8}}
\end{malts}

\begin{hops}
\hop{\hoppacificjade}{}{60~min}{11~g}
\hop{\hoppacificgem}{}{30~min}{11~g}
\hop{\hoprakau}{}{30~min}{11~g}
\hop{\hopwakatu}{}{10~min}{10~g}
\hop{\hoprakau}{}{10~min}{11~g}
\hop{\hopmotueka}{}{10~min}{10~g}
\hop{\hoppacifica}{}{10~min}{11~g}
\hop{\hoppacificgem}{}{\whirl{}{}}{14~g}
\hop{\hopwakatu}{}{\whirl{}{}}{14~g}
\hop{\hoprakau}{}{\whirl{}{}}{14~g}
\hop{\hopmotueka}{}{\whirl{}{}}{14~g}
\hop{\hoppacifica}{}{\whirl{}{}}{14~g}
\hop{\hoppacificgem}{}{\dryh{}{}}{14~g}
\hop{\hopwakatu}{}{\dryh{}{}}{14~g}
\hop{\hoprakau}{}{\dryh{}{}}{14~g}
\hop{\hopmotueka}{}{\dryh{}{}}{14~g}
\hop{\hoppacifica}{}{\dryh{}{}}{14~g}
\end{hops}

\singleyeast{IPA}

\end{ingredientsblock}

\end{recipie}

% -----------------------------------------------------------------------------
\stylecategory{India Pale Ale}
\stylesection{\styleenglishipa}

% -----------------------------------------------------------------------------
\begin{recipe}{Briess RoastOat Red} % rechecked
% -----------------------------------------------------------------------------

\begin{aboutblock}
Recipe by Briess Malt \& Ingredients Co. Developed and served at the 2019
Craft Brewers Conference in Denver, Colo., to mark the official launch of
Briess Blonde RoastOat Malt. \sourcezymurgy{September / October 2019}
\end{aboutblock}

\specifications{\styleenglishipa}{\galtol{5}}{1.048}{1.014}{4.5~\%}{20}{\srmtoebc{9}}{60~min}{2}

\begin{methodandtiming}
 
\begin{mashsteps}
\mashstep{\ftoc{152}}{60~min}
\end{mashsteps}

\begin{fermentationsteps}
\fermentationstep{\ftoc{66}}{}
\end{fermentationsteps}

\end{methodandtiming}

\recipebreak

\begin{ingredientsblock}

\begin{malts}
\malt{Briess Synergy Select Pilsen}{\lbtokg{6.9}}
\malt{Briess Blonde RoastOat}{\lbtokg{1.9}}
\malt{Briess Caramel 80 L}{\lbtokg{0.6}}
\end{malts}

\begin{hops}
\hop{\hopeastkentgolding}{4.5~\%}{60~min}{\oztog{1.1}}
\end{hops}

\singleyeast{White Labs WLP004 / Wyeast 1084 / Omega Yeast OYL-005}

\end{ingredientsblock}

\end{recipe}

% -----------------------------------------------------------------------------
\begin{recipe}{Homebrew Challenge English IPA} % rechecked
% -----------------------------------------------------------------------------

\begin{aboutblock}
Recipe by Martin Keen.
\sourcehomebrewchallenge
\end{aboutblock}

\specifications{\styleenglishipa}{\galtol{5}}{1.064}{1.018}{6.1~\%}{53}{\srmtoebc{11}}{60~min}{}

\begin{methodandtiming}

\begin{mashsteps}
\mashstep{\ftoc{152}}{60~min}
\end{mashsteps}

\begin{fermentationsteps}
\fermentationstep{\ftoc{65}}{}
\end{fermentationsteps}

\end{methodandtiming}

\recipebreak

\begin{ingredientsblock}

\begin{malts}
\malt{Golden Promise}{\lbtokg{11}}
\malt{Caramel / Crystal 45 L}{\lbtokg{1}}
\malt{Dingemans Biscuit}{\lbtokg{0.5}}
\end{malts}

\begin{hops}
\hop{\hoptarget}{}{60~min}{\oztog{1}}
\hop{\hopfuggle}{}{10~min}{\oztog{1}}
\hop{\hopeastkentgolding}{}{10~min}{\oztog{1}}
\hop{\hopfuggle}{}{\foh{}{}}{\oztog{1}}
\hop{\hopeastkentgolding}{}{\dryh{}{4~days}}{\oztog{1}}
\end{hops}

\singleyeast{Wyeast 1099}

\end{ingredientsblock}

\end{recipe}

% -----------------------------------------------------------------------------
\begin{recipe}{"The Duke" English IPA} % rechecked
% -----------------------------------------------------------------------------

\begin{aboutblock}
Recipe by Mike Treadway and Sean Vreeland of Keller, TX. Gold medal in Category
17: India Pale Ale (IPA)during the 2016 National Homebrew Competition in Baltimore, MD. \sourceaha
\end{aboutblock}

\specifications{\styleenglishipa}{\galtol{11}}{1.063}{1.014}{6.3~\%}{50}{\srmtoebc{12}}{90~min}{2.5}

\begin{methodandtiming}

\begin{mashsteps}
\mashstep{\ftoc{152}}{60~min}
\mashstep{\ftoc{170}}{Mash out}
\end{mashsteps}

\begin{fermentationsteps}
\fermentationstep{\ftoc{66}}{2~days}
\fermentationstep{\ftoc{72}}{Slow raise; full attenuation}
\end{fermentationsteps}

\end{methodandtiming}

\recipebreak

\begin{ingredientsblock}

\begin{malts}
\malt{Maris Otter}{\lbtokg{12.75}}
\malt{Dingemans Biscuit}{\lbtokg{1.125}}
\malt{Caramel / Crystal 40 L}{\lbtokg{0.875}}
\malt{Caramel / Crystal 120 L}{\lbtokg{0.875}}
\malt{Wheat}{\lbtokg{0.875}}
\end{malts}

\begin{hops}
\hop{\hopmagnum}{14~\%}{60~min}{\oztog{2}}
\hop{\hopfuggle}{4.5~\%}{10~min}{\oztog{2.25}}
\hop{\hopeastkentgolding}{7.2~\%}{\foh{}}{\oztog{3}}
\hop{\hopeastkentgolding}{7.2~\%}{\dryh{}{5~days}}{\oztog{1}}
\end{hops}

\singleyeast{White Labs WLP005}

\end{ingredientsblock}

\end{recipe}

\stylesection{\styledoubleipa}

% -----------------------------------------------------------------------------
\begin{recipe}{Roughtail Brewing Co. Hoptometrist Double IPA Clone}
% -----------------------------------------------------------------------------

\begin{aboutblock}
\sourceaha
\end{aboutblock}

\specifications{\styledoubleipa}{\galtol{6}}{1.079}{1.009}{9.3~\%}{100}{\srmtoebc{9.5}}{60~min}{}

\begin{methodandtiming}

\begin{mashsteps}
\mashstep{\ftoc{147}}{60~min}
\mashstep{\ftoc{170}}{Mash out}
\end{mashsteps}

\begin{fermentationsteps}
\fermentationstep{\ftoc{65}}{2~days}
\fermentationstep{\ftoc{70}}{Full attenuation; 6~days}
\fermentationstep{\ftoc{32}}{3~days}
\end{fermentationsteps}

\end{methodandtiming}

\recipebreak

\begin{ingredientsblock}

\begin{malts}
\malt{\malttworow}{\lbtokg{17}}
\malt{Caramel / Crystal 40}{\lbtokg{0.75}}
\malt{\maltcarapils}{\lbtokg{0.75}}
\end{malts}

\begin{hops}
\hop{\hopcolumbus}{}{60~min}{\oztog{2}}
\hop{\hopdenali}{}{\foh{15~min}}{\oztog{2}}
\hop{\hopsummit}{}{\foh{15~min}}{\oztog{2}}
\hop{\hopeureka}{}{\foh{15~min}}{\oztog{2}}
\hop{\hopmosaic}{}{\dryh{1}{3~days}}{\oztog{1}}
\hop{\hopidahoseven}{}{\dryh{1}{3~days}}{\oztog{1}}
\hop{\hopgalaxy}{}{\dryh{1}{3~days}}{\oztog{1}}
\hop{\hopmosaic}{}{\dryh{2}{3~days}}{\oztog{1}}
\hop{\hopidahoseven}{}{\dryh{2}{3~days}}{\oztog{1}}
\hop{\hopgalaxy}{}{\dryh{2}{3~days}}{\oztog{1}}
\end{hops}

\singleyeast{White Labs WLP001 / Fermentis SafAle US-05}

\end{ingredientsblock}

\end{recipe}

\stylesection{\styleryeipa}

% -----------------------------------------------------------------------------
\begin{recipe}{Atlas Brew Works Rowdy Rye Ale Clone}
% -----------------------------------------------------------------------------

\begin{aboutblock}
\sourceaha
\end{aboutblock}

\specifications{\styleryeipa}{\galtol{5}}{1.057}{}{6.2~\%}{50}{\srmtoebc{19}}{90~min}{}

\begin{methodandtiming}
 
\begin{mashsteps}
\mashstep{\ftoc{154}}{}
\end{mashsteps}

\begin{directions}
Water adjustment: carbon filtered DC tap water with \gtog{1.5} of calcium sulfate
and \gtog{1.5} calcium chloride; \mltoml{2} of food grade 85~\% phosphoric acid
for the sparge.
\end{directions}

\end{methodandtiming}

\recipebreak

\begin{ingredientsblock}

\begin{malts}
\malt{Briess Pilsen}{\lbtokg{6}}
\malt{Briess Rye}{\lbtokg{1.5}}
\malt{Weyermann CARARED}{\lbtokg{0.8}}
\malt{Briess Aromatic Munich 20 L}{\lbtokg{0.8}}
\malt{Briess Victory}{\lbtokg{0.4}}
\malt{Briess Carapils}{\lbtokg{0.4}}
\malt{Briess Midnight Wheat}{\lbtokg{0.2}}
\end{malts}

\begin{hops}
\hop{\hopbravo}{}{90~min}{\oztog{0.32}}
\hop{\hopzythos}{}{20~min}{\oztog{0.48}}
\hop{\hopcentennial}{}{5~min}{\oztog{1.23}}
\hop{\hopcentennial}{}{\dryh{}{}}{\oztog{0.63}}
\hop{\hopzythos}{}{\dryh{}{}}{\oztog{0.63}}
\end{hops}

\singleyeast{American Ale}

\end{ingredientsblock}

\end{recipe}

% -----------------------------------------------------------------------------
\begin{recipe}{Wry Smile Rye IPA}
% -----------------------------------------------------------------------------

\begin{aboutblock}
Recipe by Denny Conn. \sourcezymurgy{November / December 2019}
\end{aboutblock}

\specifications{\styleryeipa}{\galtol{5}}{1.074}{1.020}{7.2~\%}{77}{\srmtoebc{12}}{75~min}{}

\begin{methodandtiming}

\begin{mashsteps}
\mashstep{\ftoc{153}}{60~min}
\end{mashsteps}

\begin{fermentationsteps}
\fermentationstep{\ftoc{65}}{}
\end{fermentationsteps}

\end{methodandtiming}

\recipebreak

\begin{ingredientsblock}

\begin{malts}
\malt{\maltpale}{\lbtokg{11}}
\malt{Rye}{\lbtokg{3}}
\malt{\maltcaramel{60}}{\lbtokg{1.25}}
\malt{\maltcarapils}{\lbtokg{0.5}}
\malt{\maltwheat}{\lbtokg{0.5}}
\end{malts}

\begin{hops}
\hop{\hopmthood}{5.1~\%}{\fwh}{\oztog{1}}
\hop{\hopcolumbus}{16~\%}{60~min}{\oztog{1.25}}
\hop{\hopmthood}{5.1~\%}{30~min}{\oztog{0.5}}
\hop{\hopmthood}{5.1~\%}{\foh{}}{\oztog{1.5}}
\hop{\hopcolumbus}{16~\%}{\dryh{}{7~days}}{\oztog{1}}
\end{hops}

\singleyeast{Wyeast 1450}

\end{ingredientsblock}

\end{recipe}


% -----------------------------------------------------------------------------
\stylecategory{German Wheat \& Rye Beer}
\stylesection{\styleweissbier}

% -----------------------------------------------------------------------------
\begin{recipe}{"Barb's Hef" Weissbier}
% -----------------------------------------------------------------------------

\begin{aboutblock}
Recipe by Nick Corona of San Marcos, CA. Gold medal in Category 18: German
Wheat and Rye Beer during the 2016 National Homebrew Competition in
Baltimore, MD. \sourceaha
\end{aboutblock}

\specifications{\styleweissbier}{\galtol{5}}{1.049}{1.010}{}{}{}{90~min}{3.5}

\begin{methodandtiming}
 
\begin{mashsteps}
\mashstep{\ftoc{115}}{10~min}
\mashstep{\ftoc{127}}{10~min}
\mashstep{\ftoc{149}}{60~min}
\mashstep{\ftoc{168}}{Mash out}
\end{mashsteps}

\begin{fermentationsteps}
\fermentationstep{\ftoc{62}}{3~days}
\fermentationstep{\ftoc{65}}{3~days}
\fermentationstep{\ftoc{68}}{5~days}
\end{fermentationsteps}

\end{methodandtiming}

\recipebreak

\begin{ingredientsblock}

\begin{malts}
\malt{\maltpilsner}{\lbtokg{4.25}}
\malt{\maltwheat}{\lbtokg{4.25}}
\malt{Rice Hulls}{\lbtokg{0.5}}
\end{malts}

\begin{hops}
\hop{\hophallertaumittelfruh}{3.75~\%}{90~min}{\oztog{0.25}}
\hop{\hophallertaumittelfruh}{3.75~\%}{60~min}{\oztog{0.75}}
\end{hops}

\singleyeast{White Labs WLP380}

\end{ingredientsblock}

\end{recipe}

% -----------------------------------------------------------------------------
\begin{recipe}{Bigfoot's (D)elight Weissbier}
% -----------------------------------------------------------------------------

\begin{aboutblock}
Recipe by Sean Manrique of San Lorenzo, CA. Gold medal in Category 7: German
Wheat Beer during the 2019 National Homebrew Competition in Providence, RI.
\sourceaha
\end{aboutblock}

\specifications{\styleweissbier}{\galtol{11}}{1.046}{1.010}{5.7~\%}{15}{\srmtoebc{4}}{60~min}{}

\begin{methodandtiming}
 
\begin{mashsteps}
\mashstep{\ftoc{154}}{60~min}
\mashstep{\ftoc{168}}{Mash out}
\end{mashsteps}

\begin{fermentationsteps}
\fermentationstep{\ftoc{60}}{14~days}
\fermentationstep{\ftoc{35}}{7~days}
\end{fermentationsteps}

\end{methodandtiming}

\recipebreak

\begin{ingredientsblock}

\begin{malts}
\malt{White Wheat}{\lbtokg{13}}
\malt{\maltpilsner}{\lbtokg{10}}
\malt{Rice Hulls}{\lbtokg{8}}
\end{malts}

\begin{hops}
\hop{\hophallertaumittelfruh}{4.8~\%}{60~min}{\oztog{2}}
\end{hops}

\singleyeast{White Labs WLP300}

\end{ingredientsblock}

\end{recipe}

% -----------------------------------------------------------------------------
\begin{recipe}{Cane Toad Weisse}
% -----------------------------------------------------------------------------

\begin{aboutblock}
An attempt at Redback from Matilda Bay Brewing in Australia. Somewhat drier,
but with high carbonation, low hopping, and subtle hints of clove and banana.
\sourcezymurgy{March / April 2019}
\end{aboutblock}

\specifications{\styleweissbier}{\galtol{5.5}}{1.050}{1.007}{5.7~\%}{13}{\srmtoebc{3}}{60~min}{3.25}

\begin{methodandtiming}
 
\begin{mashsteps}
\mashstep{\ftoc{125}}{20~min}
\mashstep{\ftoc{140}}{30~min}
\mashstep{\ftoc{152}}{40~min}
\mashstep{\ftoc{168}}{Mash out}
\end{mashsteps}

\begin{fermentationsteps}
\fermentationstep{\ftoc{64}}{3~days}
\fermentationstep{\ftoc{68}}{Free raise; full attenuation}
\end{fermentationsteps}

\begin{directions}
Water adjustment: reverse osmosis water with \gpgaltogpl{1} calcium cloride.
Condition bottles at \ftoc{70} for 7 days or until they begin to clear. Store
at cellar temperature for 14 days.
\end{directions}

\end{methodandtiming}

\recipebreak

\begin{ingredientsblock}
    
\begin{malts}
\malt{\maltwheat}{\lbtokg{4.5}}
\malt{Briess Brewers}{\lbtokg{4.5}}
\end{malts}

\begin{hops}
\hop{\hopsterling}{2.3~\%}{60~min}{\oztog{2}}
\hop{Raw Organic Cane Sugar}{}{--}{\oztog{12}}
\end{hops}

\singleyeast{White Labs WLP380}

\end{ingredientsblock}

\end{recipe}

% -----------------------------------------------------------------------------
\begin{recipe}{Homebrew Challenge Hefeweizen (Weissbier)}
% -----------------------------------------------------------------------------

\begin{aboutblock}
Recipe by Martin Keen.
\sourcehomebrewchallenge
\end{aboutblock}

\specifications{\styleweissbier}{\galtol{5}}{1.048}{1.008}{5.2~\%}{13}{}{60~min}{}

\begin{methodandtiming}

\begin{mashsteps}
\mashstep{\ftoc{152}}{60~min}
\end{mashsteps}

\begin{fermentationsteps}
\fermentationstep{\ftoc{68}}{}
\end{fermentationsteps}

\begin{directions}
Age for 3 weeks.
\end{directions}

\end{methodandtiming}

\recipebreak

\begin{ingredientsblock}

\begin{malts}
\malt{\maltwheat}{\lbtokg{5}}
\malt{Weyermann Bohemian Pilsner}{\lbtokg{2}}
\malt{\maltpilsner}{\lbtokg{2}}
\malt{Melanoidin}{\oztokg{4}}
\malt{Rice Hulls}{\oztokg{8}}
\end{malts}

\begin{hops}
\hop{\hopperle}{7~\%}{60~min}{\oztog{0.5}}
\end{hops}

\singleyeast{White Labs WLP300}

\end{ingredientsblock}

\end{recipe}

% -----------------------------------------------------------------------------
\begin{recipe}{Kevin's Mom}
% -----------------------------------------------------------------------------

\begin{aboutblock}
Recipe by Chris Colby. \sourcezymurgy{January / February 2018}
\end{aboutblock}

\specifications{\styleweissbier}{\galtol{5}}{1.052}{1.012}{5.1~\%}{19}{\srmtoebc{3.9}}{90~min}{4}

\begin{methodandtiming}
 
\begin{mashsteps}
\mashstep{\ftoc{104}}{Mash in}
\mashstep{\ftoc{113}}{5~min}
\mashstep{\ftoc{122}}{15~min}
\mashdecoctthick{with 40~\% of mash}
\mashstep{\ftoc{158}}{Full conversion}
\mashdecoctboil{20~min}
\mashdecoctreturn{\ftoc{149}}{15~min}
\mashstep{\ftoc{158}}{Full conversion}
\mashstep{\ftoc{168}}{Mash out}
\end{mashsteps}

\begin{fermentationsteps}
\fermentationstep{\ftoc{54}}{Pitch}
\fermentationstep{\ftoc{64}}{Free raise; full attenuation}
\end{fermentationsteps}

\begin{directions}
Open fermentation for the 1 to 2 days when the fermentation is at
it's most vigorous. Condition bottles at room temperature for 14 days.
\end{directions}

\end{methodandtiming}

\recipebreak

\begin{ingredientsblock}

\begin{malts}
\malt{Red Wheat}{\lbtokg{6.75}}
\malt{Undermodified Pilsner}{\lbtokg{3}}
\end{malts}

\begin{hops}
\hop{\hophallertaumittelfruh}{4~\%}{60~min}{\oztog{1.3}}
\end{hops}

\begin{yeastsx}
\yeastx{Wyeast 3068 / White Labs WLP300}{Primary}
\yeastx{Lager}{Bottling}
\end{yeastsx}

\begin{twists}
\twist{Wort / Speise}{Bottling}{\qttol{3.6}}
\end{twists}

\end{ingredientsblock}

\end{recipe}

% -----------------------------------------------------------------------------
\begin{recipe}{Hackysack Superstar}
% -----------------------------------------------------------------------------

\begin{aboutblock}
Recipe by Rob Knighton of Columbia, PA. Gold medal in Category 19: German
Wheat and Rye Beer during the 2017 National Homebrew Competition in Minneapolis,
MN. \sourceaha
\end{aboutblock}

\specifications{\styleweissbier}{\galtol{6}}{1.048}{1.010}{}{}{}{60~min}{3}

\begin{methodandtiming}
 
\begin{mashsteps}
\mashstep{\ftoc{96}}{Mash in}
\mashstep{\ftoc{115}}{Raise over 10~min; 10~min}
\mashstep{\ftoc{127}}{Raise over 10~min; 10~min}
\mashstep{\ftoc{149}}{Raise over 15~min; 45~min}
\mashstep{\ftoc{168}}{Mash out}
\end{mashsteps}

\begin{fermentationsteps}
\fermentationstep{\ftoc{62}}{3~days}
\fermentationstep{\ftoc{68}}{4~days}
\fermentationstep{Transfer to secondary; \ftoc{68}}{3~days}
\end{fermentationsteps}

\begin{directions}
Ferment only \galtol{4.25} of wort. Ferment open, or with only tin foil as
cover. Apply airlock on the fourth day. Use closed fermenter in secondary.
\end{directions}

\end{methodandtiming}

\recipebreak

\begin{ingredientsblock}

\begin{malts}
\malt{Briess Red Wheat}{\lbtokg{5}}
\malt{\maltweyermannpilsner}{\lbtokg{4.5}}
\malt{\maltacidulated}{\oztokg{6}}
\malt{Weyermann Melanoidin}{\oztokg{2}}
\end{malts}

\begin{hops}
\hop{\hopmagnum}{12.4~\%}{60~min}{\oztog{0.25}}
\end{hops}

\singleyeast{White Labs WLP380}

\begin{twists}
\twist{Wort / Speise}{Secondary}{\qttol{1}}
\end{twists}

\end{ingredientsblock}

\end{recipe}

% -----------------------------------------------------------------------------
\begin{recipe}{Honolulu BeerWorks CocoWeizen Clone}
% -----------------------------------------------------------------------------

\begin{aboutblock}
\sourceaha
\end{aboutblock}

\specifications{\styleweissbier}{\galtol{5}}{1.048}{1.006}{5.5~\%}{14}{\srmtoebc{5.5}}{60~min}{}

\begin{methodandtiming}
 
\begin{mashsteps}
\mashstep{\ftoc{150}}{60~min}
\end{mashsteps}

\begin{fermentationsteps}
\fermentationstep{\ftoc{72}}{10~days}
\end{fermentationsteps}

\begin{directions}
Toast coconut until very dark, but not burnt.
\end{directions}

\end{methodandtiming}

\recipebreak

\begin{ingredientsblock}

\begin{malts}
\malt{White Wheat}{\lbtokg{8}}
\malt{\malttworow}{\lbtokg{2.5}}
\end{malts}

\begin{hops}
\hop{\hopcrystal}{3.9~\%}{60~min}{\oztog{1.2}}
\end{hops}

\singleyeast{Wyeast 3068}

\begin{twists}
\twist{Toasted and Shredded Coconut}{Secondary (2~days)}{\lbtokg{1}}
\end{twists}

\end{ingredientsblock}

\end{recipe}

% -----------------------------------------------------------------------------
\begin{recipe}{Siebte Flasche}
% -----------------------------------------------------------------------------

\begin{aboutblock}
Recipe by Dave Carpenter. Inspired by Schneider Weisse Tap 7 "Mein Original."
The key to this beer is open fermentation. \sourcezymurgy{November / December 2017}
\end{aboutblock}

\specifications{\styleweissbier}{\galtol{5}}{1.052}{1.011}{5.4~\%}{14}{\srmtoebc{9}}{90~min}{3.5}

\begin{methodandtiming}
 
\begin{mashsteps}
\mashstep{\ftoc{113}}{10~min}
\mashstep{\ftoc{122}}{10~min}
\mashstep{\ftoc{147}}{5~min}
\mashdecoctthick{with 1/3 of mash}
\mashstep{\ftoc{152}}{10~min}
\mashstep{\ftoc{158}}{20~min}
\mashdecoctboil{}
\mashdecoctreturn{\ftoc{168}}{mash out}
\end{mashsteps}

\begin{fermentationsteps}
\fermentationstep{\ftoc{62}}{Pitch}
\fermentationstep{\ftoc{72}}{Free raise over 5~days}
\end{fermentationsteps}

\begin{directions}
Leave the fermenter open until a gravity reading indicates that fermentation is
at or very near completion, then seal with an airlock. The amount of carbonation
requires sturdy glass bottles.
\end{directions}

\end{methodandtiming}

\recipebreak

\begin{ingredientsblock}

\begin{malts}
\malt{\maltpilsner}{\lbtokg{6}}
\malt{Pale Wheat}{\lboztokg{3}{11}}
\malt{Chocolate Wheat}{\oztokg{2}}
\end{malts}

\begin{hops}
\hop{\hopherkules}{14~\%}{45~min}{\oztog{0.25}}
\hop{\hophallertautradition}{5~\%}{15~min}{\oztog{0.25}}
\end{hops}

\singleyeast{Wyeast 3068 / White Labs WLP300}

\end{ingredientsblock}

\end{recipe}

% -----------------------------------------------------------------------------
\begin{recipe}{Spicy Nana Weissbier}
% -----------------------------------------------------------------------------

\begin{aboutblock}
Recipe by Dennis Pike of Chapel Hill, NC. Silver medal in Category 7: German Wheat
Beer during the 2018 National Homebrew Competition in Portland, OR. \sourceaha
\end{aboutblock}

\specifications{\styleweissbier}{\galtol{8}}{1.048}{1.010}{5~\%}{13}{\srmtoebc{3.4}}{90~min}{}

\begin{methodandtiming}
 
\begin{mashsteps}
\mashstep{\ftoc{105}}{Mash in}
\mashstep{\ftoc{112}}{10~min}
\mashstep{\ftoc{126}}{10~min}
\mashstep{\ftoc{149}}{5~min}
\mashdecoctthick{with 1/3 of mash}
\mashstep{\ftoc{158}}{20~min}
\mashdecoctboil{10~min}
\mashdecoctreturn{\ftoc{158}}{10~min}
\mashstep{\ftoc{169}}{10~min}
\end{mashsteps}

\begin{fermentationsteps}
\fermentationstep{\ftoc{58}}{Pitch}
\fermentationstep{\ftoc{63}}{Raise over 1~day; 7~days}
\fermentationstep{\ftoc{68}}{Raise over 1~day; 14~days}
\end{fermentationsteps}

\begin{directions}
Water adjustment: \ppmtopptm{100} calcium, mash pH of 5.4. Condition bottles at
room temperature for 14 days.
\end{directions}

\end{methodandtiming}

\recipebreak

\begin{ingredientsblock}

\begin{malts}
\malt{Durst Wheat}{\lbtokg{8.75}}
\malt{Durst Pilsner}{\lbtokg{5.75}}
\end{malts}

\begin{hops}
\hop{\hophallertaumittelfruh}{4.1~\%}{90~min}{\oztog{0.2}}
\hop{\hophallertaumittelfruh}{4.1~\%}{60~min}{\oztog{1}}
\end{hops}

\singleyeast{White Labs WLP300}

\begin{twists}
\twist{Wort / Speise}{Bottling}{\qttol{2}}
\end{twists}

\end{ingredientsblock}

\end{recipe}

% -----------------------------------------------------------------------------
\begin{recipe}{Stew's Brew Hefeweizen}
% -----------------------------------------------------------------------------

\begin{aboutblock}
Recipe by Zach Gelfand. Won first prize, production competition.
\sourcezymurgy{May / June 2018}
\end{aboutblock}

\specifications{\styleweissbier}{\galtol{5.5}}{1.046}{1.010}{4.7~\%}{9}{\srmtoebc{4}}{90~min}{}

\begin{methodandtiming}
 
\begin{mashsteps}
\mashstep{\ftoc{122}}{20~min}
\mashstep{\ftoc{150}}{60~min}
\mashstep{\ftoc{168}}{Mash out}
\end{mashsteps}

\begin{fermentationsteps}
\fermentationstep{\ftoc{65}}{7~days}
\end{fermentationsteps}

\begin{directions}
Priming: \sugarcuptog{0.75} dextrose for bottles, \sugarcuptog{0.33} for kegs.
\end{directions}

\end{methodandtiming}

\recipebreak

\begin{ingredientsblock}

\begin{malts}
\malt{Weyermann Pale Wheat}{\lbtokg{4.75}}
\malt{\maltpilsner}{\lbtokg{3.75}}
\malt{Weyermann Melanoidin}{\oztokg{8}}
\end{malts}

\begin{hops}
\hop{\hopeastkentgolding}{3.9~\%}{60~min}{\oztog{0.67}}
\end{hops}

\singleyeast{White Labs WLP300}

\end{ingredientsblock}

\end{recipe}

% -----------------------------------------------------------------------------
\begin{recipe}{The Bizarro Jerry Weissbier}
% -----------------------------------------------------------------------------

\begin{aboutblock}
Recipe by Tyler Cipriani and Blazey Heier of Longmont, CO. Bronze medal in 
Category 7: German Wheat Beer during the 2018 National Homebrew Competition
in Portland, OR. \sourceaha
\end{aboutblock}

\specifications{\styleweissbier}{\galtol{6}}{1.050}{}{}{12.3}{\srmtoebc{4.8}}{60~min}{}

\begin{methodandtiming}
 
\begin{mashsteps}
\mashstep{\ftoc{148}}{During decoction}
\mashdecoctthick{with 1/3 of mash}
\mashstep{\ftoc{160}}{15~min}
\mashdecoctboil{20~min}
\mashdecoctreturn{\ftoc{158}}{25~min}
\mashstep{\ftoc{170}}{10~min}
\end{mashsteps}

\begin{fermentationsteps}
\fermentationstep{\ftoc{56}}{Pitch}
\fermentationstep{\ftoc{64}}{Free raise start}
\fermentationstep{\ftoc{68}}{Free raise over 3~days}
\end{fermentationsteps}

\begin{directions}
Water adjustment: carbon filtered Longmont, Colorado water with \tsptog{2} of
calcium sulfate.
\end{directions}

\end{methodandtiming}

\recipebreak

\begin{ingredientsblock}

\begin{malts}
\malt{Weyermann Pale Wheat}{\lbtokg{8}}
\malt{Root Shoot Odyssey Pilsner}{\lbtokg{4.75}}
\malt{Weyermann CARAHELL}{\oztokg{12}}
\end{malts}

\begin{hops}
\hop{\hophallertaumittelfruh}{4.5~\%}{60~min}{\oztog{1}}
\end{hops}

\singleyeast{White Labs WLP300}

\end{ingredientsblock}

\end{recipe}

% -----------------------------------------------------------------------------
\begin{recipe}{Where's Fluffy Weissbier}
% -----------------------------------------------------------------------------

\begin{aboutblock}
Recipe by Paul Brown of Pinole, CA. Bronze medal in Category 7: German Wheat
Beer during the 2019 National Homebrew Competition Final Round in Providence, RI.
\sourceaha
\end{aboutblock}

\specifications{\styleweissbier}{\galtol{5.5}}{1.044}{1.006}{6.5~\%}{9.8}{\srmtoebc{5}}{90~min}{3}

\begin{methodandtiming}
 
\begin{mashsteps}
\mashstep{\ftoc{122}}{30~min}
\mashstep{\ftoc{148}}{30~min}
\mashstep{\ftoc{168}}{Mash out}
\end{mashsteps}

\begin{fermentationsteps}
\fermentationstep{\ftoc{64}}{5~days}
\fermentationstep{\ftoc{66}}{Full attenuation}
\end{fermentationsteps}

\end{methodandtiming}

\recipebreak

\begin{ingredientsblock}

\begin{malts}
\malt{Rahr Red Wheat}{\lbtokg{6}}
\malt{\maltweyermannpilsner}{\lbtokg{5}}
\malt{Weyermann Melanoidin}{\oztokg{12}}
\malt{\maltweyermannacidulated}{\oztokg{8}}
\end{malts}

\begin{hops}
\hop{\hopmagnum}{12~\%}{\fwh}{\oztog{0.3}}
\end{hops}

\singleyeast{White Labs WLP300}

\end{ingredientsblock}

\end{recipe}

\stylesection{\styledunklesweissbier}

% -----------------------------------------------------------------------------
\begin{recipe}{Homebrew Challenge Dunkles Weissbier} % rechecked
% -----------------------------------------------------------------------------

\begin{aboutblock}
Recipe by Martin Keen.
\sourcehomebrewchallenge
\end{aboutblock}

\specifications{\styledunklesweissbier}{\galtol{5}}{1.054}{1.008}{6.3~\%}{}{}{60~min}{}

\begin{methodandtiming}

\begin{mashsteps}
\mashstep{\ftoc{152}}{60~min}
\end{mashsteps}

\begin{fermentationsteps}
\fermentationstep{\ftoc{68}}{}
\end{fermentationsteps}

\begin{directions}
Age for 3 weeks.
\end{directions}

\end{methodandtiming}

\recipebreak

\begin{ingredientsblock}

\begin{malts}
\malt{Wheat}{\lbtokg{6}}
\malt{Weyermann Bohemian Pilsner}{\lboztokg{1}{8}}
\malt{Weyermann Vienna}{\lboztokg{1}{8}}
\malt{Weyermann CARAMUNICH III}{\lbtokg{1}}
\malt{Pilsner}{\lbtokg{1}}
\malt{Weyermann CARAFA II}{\oztokg{4}}
\malt{Rice Hulls}{\oztokg{8}}
\end{malts}

\begin{hops}
\hop{\hopperle}{7~\%}{60~min}{\oztog{0.5}}
\end{hops}

\singleyeast{White Labs WLP300}

\end{ingredientsblock}

\end{recipe}


% -----------------------------------------------------------------------------
\begin{recipe}{Tall, Dark and Bready} % rechecked
% -----------------------------------------------------------------------------

\begin{aboutblock}
\sourcezymurgy{September / October 2018}
\end{aboutblock}

\specifications{\styledunklesweissbier}{\galtol{5.5}}{1.055}{1.013}{5.5~\%}{15}{\srmtoebc{20}}{90~min}{}

\begin{methodandtiming}

\begin{mashsteps}
\mashstep{\ftoc{122}}{20~min}
\mashdecoctthick{with \qttol{9} of mash}
\mashdecoctboil{15~min}
\mashdecoctreturn{\ftoc{150}}{30~min}
\mashstep{\ftoc{168}}{Mash out}
\end{mashsteps}

\begin{fermentationsteps}
\fermentationstep{\ftoc{64}}{}
\end{fermentationsteps}

\begin{directions}
Priming: \sugarcuptog{1} dextrose at \ftoc{70} for 7 days. Age for at least 1 month.
\end{directions}

\end{methodandtiming}

\recipebreak

\begin{ingredientsblock}

\begin{malts}
\malt{White Wheat}{\lbtokg{5}}
\malt{Munich}{\lbtokg{3}}
\malt{Vienna}{\lbtokg{2}}
\malt{Dingemans Special B}{\oztokg{8}}
\malt{Chocolate Wheat}{\oztokg{4}}
\malt{Rice Hulls}{\oztokg{4}}
\end{malts}

\begin{hops}
\hop{\hoptettnang}{5~\%}{60~min}{\oztog{1}}
\end{hops}

\singleyeast{White Labs WLP380 / Wyeast 3333-PC}

\end{ingredientsblock}

\end{recipe}

% -----------------------------------------------------------------------------
\begin{recipe}{Trigo Oscuro} % rechecked
% -----------------------------------------------------------------------------

\begin{aboutblock}
No too big or rich but fruity and malty. Featured in the book Brewing Classic
Styles by Jamil Zainasheff and John Palmer. \sourceaha
\end{aboutblock}

\specifications{\styledunklesweissbier}{\galtol{5}}{1.056}{1.014}{5.5~\%}{16}{\srmtoebc{16}}{90~min}{3}

\begin{methodandtiming}

\begin{mashsteps}
\mashstep{\ftoc{152}}{}
\end{mashsteps}

\begin{fermentationsteps}
\fermentationstep{\ftoc{62}}{}
\end{fermentationsteps}

\end{methodandtiming}

\recipebreak

\begin{ingredientsblock}

\begin{malts}
\malt{Pilsner}{\lbtokg{2}}
\malt{Wheat}{\lbtokg{6.9}}
\malt{Munich}{\lbtokg{3}}
\malt{Dingemans Special B}{\oztokg{6}}
\malt{Caramel / Crystal 40 L}{\oztokg{6}}
\malt{Weyermann CARAFA SPECIAL II}{\oztokg{2}}
\end{malts}

\begin{hops}
\hop{\hophallertaumittelfruh}{4~\%}{60~min}{\oztog{1}}
\end{hops}

\singleyeast{White Labs WLP300 / Wyeast 3068}

\end{ingredientsblock}

\end{recipe}

\stylesection{\styleweizenbock}

% -----------------------------------------------------------------------------
\begin{recipe}{German WeizenBock}
% -----------------------------------------------------------------------------

\begin{aboutblock}
Recipe by Rodney Kibzey. \sourcezymurgy{September / October 2017}
\end{aboutblock}

\specifications{\styleweizenbock}{\galtol{5}}{1.076}{}{}{18}{\srmtoebc{16}}{90~min}{}

\begin{methodandtiming}

\begin{mashsteps}
\mashstep{\ftoc{111}}{15~min}
\mashstep{\ftoc{122}}{15~min}
\mashstep{\ftoc{154}}{60~min}
\mashstep{\ftoc{168}}{10~min}
\end{mashsteps}

\end{methodandtiming}

\recipebreak

\begin{ingredientsblock}

\begin{malts}
\malt{Wheat}{\lbtokg{7.25}}
\malt{Pale}{\lbtokg{5.25}}
\malt{Munich}{\oztog{8}}
\malt{Dingemans Cara 120}{\oztog{4}}
\malt{Chocolate}{\oztog{5.5}}
\end{malts}

\begin{hops}
\hop{\hoptettnang}{4.7~\%}{90~min}{\oztog{0.8}}
\hop{\hoptettnang}{4.7~\%}{\foh{}{}}{\oztog{0.25}}
\end{hops}

\singleyeast{White Labs WLP380}

\end{ingredientsblock}

\end{recipe}

% -----------------------------------------------------------------------------
\begin{recipe}{Winter Wave Weizenbock}
% -----------------------------------------------------------------------------

\begin{aboutblock}
Recipe by Jim Rupert of Germantown, OH. Silver medal in Category \#7: German
Wheat Beer during the 2019 National Homebrew Competition in Providence, RI.
\sourceaha
\end{aboutblock}

\specifications{\styleweizenbock}{\galtol{5.75}}{1.075}{1.015}{8.9~\%}{24.4}{\srmtoebc{23}}{60~min}{}

\begin{methodandtiming}
 
\begin{mashsteps}
\mashstep{\ftoc{95}}{10~min}
\mashstep{\ftoc{113}}{Raise to over 15~min; 10~min}
\mashstep{\ftoc{135}}{Raise to over 15~min; 15~min}
\mashstep{\ftoc{145}}{Raise to over 15~min; 20~min}
\mashstep{\ftoc{154}}{Raise to over 15~min; 20~min}
\mashstep{\ftoc{168}}{Raise to over 15~min; 10~min}
\end{mashsteps}

\begin{directions}
Water adjustment: use reverse osmosis water with 3.03~g calcium sulfate, 2.33~g
magnesium sulfate, 1.86~g calcium chloride and 1~g sodium chloride in mash.
\end{directions}

\end{methodandtiming}

\recipebreak

\begin{ingredientsblock}

\begin{malts}
\malt{Dark Wheat}{\lbtokg{9}}
\malt{Pilsner}{\lbtokg{4.5}}
\malt{Munich}{\lbtokg{2}}
\malt{Rice Hulls}{\lbtokg{1.2}}
\malt{Caramel / Crystal 60 L}{\oztog{8}}
\malt{Dingemans Special B}{\oztog{8}}
\malt{Pale Chocolate}{\oztog{4}}
\end{malts}

\begin{hops}
\hop{\hophallertaumittelfruh}{4.1~\%}{60~min}{\oztog{2}}
\hop{Irish Moss}{}{15~min}{--}
\hop{Yeast Nutrient}{}{15~min}{--}
\end{hops}

\singleyeast{White Labs WLP300}

\end{ingredientsblock}

\end{recipe}

% -----------------------------------------------------------------------------
\begin{recipe}{Weize-Ass Bock}
% -----------------------------------------------------------------------------

\begin{aboutblock}
Recipe by Harry Pilgrim of Fredericksburg, VA. Gold medal in Category \#15: German
Wheat and Rye Beer during the 2015 National Homebrew Competition in San Diego.
\sourceaha
\end{aboutblock}

\specifications{\styleweizenbock}{\galtol{6}}{1.070}{1.018}{7.4~\%}{20}{\srmtoebc{19}}{90~min}{3}

\begin{methodandtiming}

\begin{mashsteps}
\mashstep{\ftoc{156}}{90~min}
\end{mashsteps}

\begin{fermentationsteps}
\fermentationstep{\ftoc{66}}{5~days}
\fermentationstep{\ftoc{70}}{1~week}
\end{fermentationsteps}

\begin{directions}
Bottle and prime with dry malt extract. Age at cellar temperature for several months
to reach optimum flavor.
\end{directions}

\end{methodandtiming}

\recipebreak

\begin{ingredientsblock}

\begin{malts}
\malt{Rice Hulls}{\lbtokg{1}}
\malt{Dark Wheat}{\lbtokg{9}}
\malt{Pilsner}{\lbtokg{5}}
\malt{Munich}{\lbtokg{2}}
\malt{Melanoidin}{\lbtokg{0.75}}
\malt{Dingemans Special B}{\lbtokg{0.5}}
\malt{Caramel / Crystal 40 L}{\lbtokg{0.5}}
\malt{Pale Chocolate}{\oztog{4}}
\end{malts}

\begin{hops}
\hop{\hophersbrucker}{4.4~\%}{60~min}{\oztog{2}}
\end{hops}

\singleyeast{White Labs WLP380}

\end{ingredientsblock}

\end{recipe}

% -----------------------------------------------------------------------------
\begin{recipe}{Weizenbock}
% -----------------------------------------------------------------------------

\begin{aboutblock}
Recipe by Jamil Zainasheff and John Palmer. From the book Brewing Classic Styles.
\sourceaha
\end{aboutblock}

\specifications{\styleweizenbock}{\galtol{6}}{1.081}{1.021}{8~\%}{23}{\srmtoebc{16}}{90~min}{}

\begin{methodandtiming}

\begin{mashsteps}
\mashstep{\ftoc{152}}{}
\end{mashsteps}

\begin{fermentationsteps}
\fermentationstep{\ftoc{62}}{1~week}
\fermentationstep{\ftoc{62}}{Transfer to secondary; 2~weeks}
\end{fermentationsteps}

\end{methodandtiming}

\recipebreak

\begin{ingredientsblock}

\begin{malts}
\malt{Pilsner}{\lbtokg{5}}
\malt{Dark Wheat}{\lbtokg{10}}
\malt{Munich}{\lbtokg{2}}
\malt{Dingemans Special B}{\lbtokg{0.5}}
\malt{Caramel / Crystal 40 L}{\lbtokg{0.5}}
\malt{Pale Chocolate}{\lbtokg{0.25}}
\end{malts}

\begin{hops}
\hop{\hophallertaumittelfruh}{4~\%}{60~min}{\oztog{1.6}}
\end{hops}

\singleyeast{Wyeast 3068 / White Labs WLP380}

\end{ingredientsblock}

\end{recipe}


% -----------------------------------------------------------------------------
\begin{recipe}{Yyy's Man Says}
% -----------------------------------------------------------------------------

\begin{aboutblock}
Recipe by Scott Hixson of Gulfport, MS. Gold medal in Category \#15: German Wheat
and Rye Beer during the 2014 National Homebrew Competition in Grand Rapids, MI.
\sourceaha
\end{aboutblock}

\specifications{\styleweizenbock}{\galtol{5}}{1.072}{1.018}{7.09~\%}{}{}{90~min}{4.2}

\begin{methodandtiming}

\begin{mashsteps}
\mashstep{\ftoc{111}}{15~min}
\mashstep{\ftoc{125}}{15~min}
\mashstep{\ftoc{153}}{60~min}
\end{mashsteps}

\begin{fermentationsteps}
\fermentationstep{\ftoc{68}}{8~days}
\fermentationstep{\ftoc{65}}{8~days}
\fermentationstep{\ftoc{36}}{6~months}
\end{fermentationsteps}

\end{methodandtiming}

\recipebreak

\begin{ingredientsblock}

\begin{malts}
\malt{White Wheat}{\lbtokg{6.9}}
\malt{Pale}{\lbtokg{4.4}}
\malt{Munich}{\lbtokg{0.5}}
\malt{Pale Chocolate}{\lbtokg{0.4}}
\malt{Weyermann CARAMUNICH I}{\lbtokg{0.3}}
\malt{Weyermann CARAMUNICH III}{\lbtokg{0.3}}
\malt{Melanoidin}{\lbtokg{0.3}}
\end{malts}

\begin{hops}
\hop{\hophallertaumittelfruh}{4~\%}{\fwh}{\oztog{1}}
\hop{\hophallertaumittelfruh}{4~\%}{15~min}{\oztog{0.3}}
\end{hops}

\singleyeast{White Labs WLP380}

\end{ingredientsblock}

\end{recipe}


% -----------------------------------------------------------------------------
\stylecategory{Belgian \& French Ale}
\stylesection{\stylewitbier}

% -----------------------------------------------------------------------------
\begin{recipe}{Brunchmaster 2000 Witbier}
% -----------------------------------------------------------------------------

\begin{aboutblock}
Recipe by Thomas Kinzer of Milwaukie, OR. Gold medal in Category 21: Belgian Ale
during the 2018 National Homebrew Competition in Portland, OR. \sourceaha
\end{aboutblock}

\specifications{\stylewitbier}{\galtol{5.5}}{1.050}{1.012}{5~\%}{18}{\srmtoebc{3}}{60~min}{}

\begin{methodandtiming}
 
\begin{mashsteps}
\mashstep{\ftoc{152}}{90~min}
\mashstep{\ftoc{168}}{Mash out}
\end{mashsteps}

\begin{fermentationsteps}
\fermentationstep{\ftoc{64}}{Pitch}
\fermentationstep{\ftoc{78}}{Raise to over 2~weeks}
\end{fermentationsteps}

\end{methodandtiming}

\recipebreak

\begin{ingredientsblock}

\begin{malts}
\malt{Flaked Wheat}{\lbtokg{6}}
\malt{Briess Brewers}{\lbtokg{4.5}}
\malt{Flaked Oats}{\oztog{4}}
\end{malts}

\begin{hops}
\hop{\hopamarillo}{9~\%}{20~min}{\oztog{0.65}}
\hop{\hopamarillo}{9~\%}{15~min}{\oztog{0.5}}
\hop{Cracked Coriander Seeds}{}{5~min}{\oztog{0.1}}
\hop{Ground Bitter Orange Peel}{}{5~min}{\oztog{0.1}}
\hop{Wyeast Beer Nutrient Blend}{}{1~min}{\tsptog{0.25}}
\hop{\hopamarillo}{9~\%}{\foh{}{}}{\oztog{1}}
\end{hops}

\singleyeast{Wyeast 3944}

\end{ingredientsblock}

\end{recipe}

% -----------------------------------------------------------------------------
\begin{recipe}{Pedal Haus Brewery Biere Blanche Belgian Witbier Clone}
% -----------------------------------------------------------------------------

\begin{aboutblock}
\sourceaha
\end{aboutblock}

\specifications{\stylewitbier}{\galtol{10}}{1.048}{1.009}{5.1~\%}{14}{\srmtoebc{3.4}}{90~min}{}

\begin{methodandtiming}
 
\begin{mashsteps}
\mashstep{\ftoc{122}}{20~min}
\mashstep{\ftoc{149}}{20~min}
\mashstep{\ftoc{158}}{20~min}
\end{mashsteps}

\begin{fermentationsteps}
\fermentationstep{\ftoc{65}}{}
\end{fermentationsteps}

\begin{directions}
Conduct a diacetyl rest for at least 5 days. Do not use kettle coagulant and do
not cold crash too fast or the yeast will drop the protein haze with it.
\end{directions}

\end{methodandtiming}

\recipebreak

\begin{ingredientsblock}

\begin{malts}
\malt{Pale Wheat}{\lbtokg{8.25}}
\malt{Pilsner}{\lbtokg{5.4}}
\malt{Two-row}{\lbtokg{2.3}}
\malt{Flaked Wheat}{\lbtokg{1.4}}
\malt{Acidulated}{\oztog{3.5}}
\malt{Acidulated}{\oztog{3.5}}
\end{malts}

\begin{hops}
\hop{\hophallertautradition}{4~\%}{90~min}{\oztog{1.75}}
\hop{Dried Moroccan Orange Peel}{}{1~min}{\oztog{5}}
\hop{Cracked Coriander Seeds}{}{\whirl{}{}}{\oztog{5}}
\end{hops}

\singleyeast{Wyeast 3787 / White Labs WLP530}

\end{ingredientsblock}

\end{recipe}

% -----------------------------------------------------------------------------
\begin{recipe}{Witty Dutchman Witbier}
% -----------------------------------------------------------------------------

\begin{aboutblock}
Recipe by Mark Peterson of Queen Creek, AZ. Gold medal in Category \#21: Belgian
Ale during the 2019 National Homebrew Competition in Providence, RI. \sourceaha
\end{aboutblock}

\specifications{\stylewitbier}{\galtol{5}}{1.045}{1.011}{4.4~\%}{17}{\srmtoebc{3}}{60~min}{2.25}

\begin{methodandtiming}
 
\begin{mashsteps}
\mashstep{\ftoc{155}}{60~min}
\end{mashsteps}

\begin{fermentationsteps}
\fermentationstep{\ftoc{65}}{14~days}
\fermentationstep{\ftoc{65}}{Transfer to secondary; 5~days}
\fermentationstep{\ftoc{68}}{2~days}
\end{fermentationsteps}

\begin{directions}
Target a mash pH of 5.2.
\end{directions}

\end{methodandtiming}

\recipebreak

\begin{ingredientsblock}

\begin{malts}
\malt{Pilsner}{\lbtokg{4}}
\malt{White Wheat}{\lbtokg{2}}
\malt{Red Wheat}{\lbtokg{1.5}}
\malt{Yellow Flaked Maize}{\oztog{12}}
\malt{Flaked Oats}{\oztog{4}}
\end{malts}

\begin{hops}
\hop{\hopnorthernbrewer}{7~\%}{60~min}{\oztog{0.7}}
\hop{Orange Peel}{}{10~min}{\oztog{1}}
\hop{Cracked Coriander Seeds}{}{5~min}{\oztog{5}}
\end{hops}

\singleyeast{Wyeast 3944}

\end{ingredientsblock}

\end{recipe}

\stylesection{\stylebelgianpaleale}

% -----------------------------------------------------------------------------
\begin{recipe}{Ben's Belgian Pale Ale}
% -----------------------------------------------------------------------------

\begin{aboutblock}
Andrew Johnson of Salem, OR, member of the Capitol Brewers, won a silver medal in
Category \#21: Belgian Ale with a Belgian Pale Ale during the 2019 National Homebrew
Competition Final Round in Providence, RI. Johnson's Belgian Pale Ale was chosen as
the runner-up entry among 231 entries in the category. \sourceaha
\end{aboutblock}

\specifications{\stylebelgianpaleale}{\galtol{5}}{1.054}{1.012}{5~\%}{33.2}{\srmtoebc{8.8}}{70~min}{2.5}

\begin{methodandtiming}
 
\begin{mashsteps}
\mashstep{\ftoc{152}}{60~min}
\end{mashsteps}

\begin{fermentationsteps}
\fermentationstep{\ftoc{68}}{Pitch}
\fermentationstep{\ftoc{74}}{Free raise to; 10~days}
\end{fermentationsteps}

\begin{directions}
After fermentation, transfer to secondary and let condition for another
10 days.
\end{directions}

\end{methodandtiming}

\recipebreak

\begin{ingredientsblock}

\begin{malts}
\malt{Pilsner}{\lbtokg{7}}
\malt{Caravienne}{\oztog{12}}
\malt{Aromatic}{\oztog{9}}
\malt{Flaked Oats}{\oztog{7}}
\malt{Weyermann CARAMUNICH II}{\oztog{6}}
\malt{Carapils / Dextrin}{\oztog{2.3}}
\malt{Red Wheat}{\oztog{2.3}}
\end{malts}

\begin{hops}
\hop{\hopmagnum}{16.4~\%}{60~min}{\oztog{0.1}}
\hop{\hopsaaz}{5.1~\%}{60~min}{\oztog{0.8}}
\hop{\hopsaaz}{5.1~\%}{20~min}{\oztog{0.7}}
\hop{Whirlfloc Tablet}{}{10~min}{1}
\hop{\hopsaaz}{5.1~\%}{\foh{}}{\oztog{0.4}}
\end{hops}

\singleyeast{Wyeast 3522}

\end{ingredientsblock}

\end{recipe}

\stylesection{\stylesaison}

\begin{recipie}{Zon Saison}

\begin{aboutblock}
Mike Vandervoort of Toronto, CAN, member of the GTA Brews, won a silver medal
in Category \#20: Saison during the 2019 National Homebrew Competition Final Round
in Providence, RI. Vandervoort’s Saison was chosen as the runner-up entry among 253
entries in the category.
\end{aboutblock}

\specifications{\stylesaison}{\galtol{5}}{1.049}{1.007}{5.5~\%}{27}{\srmtoebc{3.2}}{60~min}

\begin{methodandtiming}
 
\begin{mashsteps}
\mashstep{\ftoc{135}--\ftoc{140}}{Only unmalted wheat, 30--45~min}
\mashstep{\ftoc{142}}{30~min}
\mashstep{\ftoc{156}}{30~min}
\end{mashsteps}

\begin{fermentationsteps}
\fermentationstep{\ftoc{70}}{Pitch}
\fermentationstep{\ftoc{71}--\ftoc{80}}{--}
\end{fermentationsteps}

\begin{directions}
Target 5.4 for pH.
\end{directions}

\end{methodandtiming}

\pagebreak

\begin{ingredientsblock}

\begin{malts}
\malt{Pilsner}{\lbtokg{5.75}}
\malt{Weyermann CARAFOAM}{\lbtokg{0.625}}
\malt{Unmalted Wheat}{\lbtokg{1.875}}
\end{malts}

\begin{hops}
\hop{\hopsaaz}{2.8~\%}{60~min}{\oztog{0.75}}
\hop{\hopcitra}{13.8~\%}{20~min}{\oztog{0.4}}
\hop{\hopmandarinabavaria}{9.6~\%}{5~min}{\oztog{0.75}}
\hop{\hopmosaic}{12.5~\%}{\whirl{}{20~min}}{\oztog{0.4}}
\end{hops}

\begin{yeasts}
\yeast{Escarpment Labs Ontario Farmhouse Ale Blend}
\end{yeasts}

\end{ingredientsblock}

\end{recipie}



% -----------------------------------------------------------------------------
%\stylecategory{Sour Ale}

% -----------------------------------------------------------------------------
\stylecategory{Belgian Strong Ale}
\stylesection{\stylebelgianblondale}

% -----------------------------------------------------------------------------
\begin{recipe}{Brilliant Blond}
% -----------------------------------------------------------------------------

\begin{aboutblock}
Recipe by John Groeger of Mechanicsville, MD. Bronze medal in Category 21:
Belgian Ale during the 2019 National Homebrew Competition Final Round in
Providence, RI. \sourceaha
\end{aboutblock}

\specifications{\stylebelgianblondale}{\galtol{5.5}}{1.065}{1.010}{7.2~\%}{19.9}{\srmtoebc{5.1}}{90~min}{2.8}

\begin{methodandtiming}
 
\begin{mashsteps}
\mashstep{\ftoc{150}}{60~min}
\end{mashsteps}

\begin{fermentationsteps}
\fermentationstep{\ftoc{65}}{2~days}
\fermentationstep{\ftoc{72}}{5~days}
\end{fermentationsteps}

\begin{directions}
Add cane sugar solution on day 3. Transfer to secondary for 7 days.
\end{directions}

\end{methodandtiming}

\recipebreak

\begin{ingredientsblock}

\begin{malts}
\malt{\maltpilsner}{\lbtokg{13.25}}
\malt{\maltwheat}{\lbtokg{1}}
\malt{Aromatic}{\lbtokg{0.25}}
\end{malts}

\begin{hops}
\hop{\hophallertaumittelfruh}{3.4~\%}{60~min}{\oztog{1.8}}
\hop{\hoptettnang}{2.2~\%}{10~min}{\oztog{0.75}}
\end{hops}

\singleyeast{White Labs WLP500}

\begin{twists}
\twist{Cane Sugar Solution}{Primary}{\lbtokg{0.75}}
\end{twists}

\end{ingredientsblock}

\end{recipe}

% -----------------------------------------------------------------------------
\begin{recipe}{More Fun Blonde}
% -----------------------------------------------------------------------------

\begin{aboutblock}
Recipe by Jay Shambo. Gold medal at 2011 GABF Pro-Am.
\sourcezymurgy{January / February 2013}
\end{aboutblock}

\specifications{\stylebelgianblondale}{\galtol{5}}{1.067}{1.007}{7.8~\%}{34}{\srmtoebc{4}}{60~min}{}

\begin{methodandtiming}

\begin{mashsteps}
\mashstep{\ftoc{152}}{60~min}
\end{mashsteps}

\begin{fermentationsteps}
\fermentationstep{\ftoc{65}}{Pitch}
\fermentationstep{\ftoc{75}}{Raise over 1.5 days; full attenuation}
\end{fermentationsteps}

\begin{directions}
Water adjustment: mash pH of 5.4.
\end{directions}

\end{methodandtiming}

\recipebreak

\begin{ingredientsblock}

\begin{malts}
\malt{\maltpilsner}{\lbtokg{8.5}}
\malt{\maltpale}{\lbtokg{2}}
\end{malts}

\begin{hops}
\hop{Dextrose}{}{}{\oztog{13}}
\hop{\hopnorthernbrewer}{9~\%}{60~min}{\oztog{0.5}}
\hop{\hopnorthernbrewer}{9~\%}{20~min}{\oztog{0.6}}
\hop{\hopwillamette}{5~\%}{20~min}{\oztog{0.7}}
\hop{\hopwillamette}{5~\%}{\foh{}}{\oztog{0.5}}
\end{hops}

\singleyeast{Wyeast 3787 / White Labs WLP530}

\end{ingredientsblock}

\end{recipe}

\stylesection{\stylebelgiantripel}

% -----------------------------------------------------------------------------
\begin{recipe}{Coppertail Brewing Co. Unholy Trippel Clone} % rechecked
% -----------------------------------------------------------------------------

\begin{aboutblock}
\sourceaha
\end{aboutblock}

\specifications{\stylebelgiantripel}{\galtol{10}}{1.080}{}{9.5~\%}{}{}{90~min}{}

\begin{methodandtiming}

\begin{mashsteps}
\mashstep{\ftoc{149}}{60~min}
\end{mashsteps}

\begin{fermentationsteps}
\fermentationstep{\ftoc{76}}{}
\end{fermentationsteps}

\end{methodandtiming}

\recipebreak

\begin{ingredientsblock}

\begin{malts}
\malt{Pilsner}{\lbtokg{20}}
\malt{Wheat}{\lbtokg{2}}
\malt{Acidulated}{\lbtokg{0.15}}
\end{malts}

\begin{hops}
\hop{\hopazacca}{14.6~\%}{\fwh}{\oztog{0.1}}
\hop{\hopcolumbus}{11.1~\%}{60~min}{\oztog{0.4}}
\hop{\hopmatueka}{7.5~\%}{15~min}{\oztog{0.4}}
\hop{Demerara Sugar}{}{10~min}{\lbtokg{4}}
\hop{\hopazacca}{14.6~\%}{5~min}{\oztog{0.2}}
\hop{\hopmatueka}{7.5~\%}{\whirl{}{}}{\oztog{0.5}}
\hop{\hopsimcoe}{12.2~\%}{\whirl{}{}}{\oztog{0.5}}
\hop{\hopmatueka}{7.5~\%}{\dryh{}{}}{\oztog{1.6}}
\hop{\hopsimcoe}{12.2~\%}{\dryh{}{}}{\oztog{5.3}}
\end{hops}

\singleyeast{Belgian Ale}

\end{ingredientsblock}

\end{recipe}

\stylesection{\stylebelgiangoldenstrongale}

% -----------------------------------------------------------------------------
\begin{recipie}{Nuetnigenough Belgian Golden Strong}
% -----------------------------------------------------------------------------

\begin{aboutblock}
Wesley Carmichael \& Mandy Naglich of New York, NY, won a silver medal in
Category \#22: Strong Belgian Ale with a Belgian Golden Strong Ale during the
2019 National Homebrew Competition Final Round in Providence, RI. Carmichael's
Belgian Golden Strong Ale was chosen as the runner-up entry among 389 entries
in the category.
\end{aboutblock}

\specifications{\stylebelgiangoldenstrongale}{\galtol{6}}{1.071}{1.010}{8.9~\%}{27}{\srmtoebc{3.3}}{90~min}{3}

\begin{methodandtiming}
 
\begin{mashsteps}
\mashstep{\ftoc{154}}{30~min}
\mashstep{\ftoc{170}}{Raise to over 4~min}
\mashstep{\ftoc{170}}{10~min}
\end{mashsteps}

\begin{fermentationsteps}
\fermentationstep{\ftoc{60}}{Until activity slows}
\fermentationstep{\ftoc{75}}{Serveral days}
\end{fermentationsteps}

\begin{directions}
Cold crash with gelatin when fully attenuated. Bottle condition with priming
sugar.
\end{directions}

\end{methodandtiming}

\pagebreak

\begin{ingredientsblock}

\begin{malts}
\malt{Dingemans Pilsen MD}{\lbtokg{9.7}}
\malt{Flaked Wheat}{\oztog{16}}
\malt{Weyermann Acidulated}{\oztog{7}}
\end{malts}

\begin{hops}
\hop{Dextrose}{}{}{\lbtokg{2.7}}
\hop{\hopmagnum}{14~\%}{90~min}{\oztog{0.6}}
\hop{\hopsaaz}{3.8~\%}{5~min}{\oztog{1}}
\hop{\hopstyriangolding}{5.4~\%}{5~min}{\oztog{1}}
\hop{\hopsaaz}{3.8~\%}{\whirl{}{}}{\oztog{1}}
\hop{\hopstyriangolding}{5.4~\%}{\whirl{}{}}{\oztog{1}}
\end{hops}

\begin{yeasts}
\yeast{Wyeast 1388}
\end{yeasts}

\end{ingredientsblock}

\end{recipie}

\stylesection{\stylebelgiandarkstrongale}

% -----------------------------------------------------------------------------
\begin{recipie}{Easter Quad}
% -----------------------------------------------------------------------------

\begin{aboutblock}
Michael Tonsmeire has made a name for himself in the beer world -- especially
with sour beers. However, in his Brewer of the Week post, he shared a not-so-sour
Easter Quad recipe he brewed with his neighbor, a priest, for a church's Easter Vigil.
Tonsmeire was expecting a light wheat beer, but instead was excited when his neighbor
wanted to brew a Belgian Quad with a couple ingredients mentioned in the Bible
(pomegranate and cardamom). Needless to say, it turned out delicious.
\end{aboutblock}

\specifications{\stylebelgiandarkstrongale}{\galtol{10}}{1.082}{}{}{24}{\srmtoebc{23}}{70~min}

\begin{methodandtiming}
 
\begin{mashsteps}
\mashstep{\ftoc{152}}{60~min}
\end{mashsteps}

\begin{fermentationsteps}
\fermentationstep{\ftoc{63}}{3~days}
\fermentationstep{\ftoc{70}}{--}
\end{fermentationsteps}

\begin{directions}
Carbonate to 2.4 volumes \ce{CO2}.
\end{directions}

\end{methodandtiming}

\pagebreak

\begin{ingredientsblock}

\begin{malts}
\malt{Pale}{\lbtokg{33}}
\malt{Weyermann CARMUNICH I}{\lbtokg{2}}
\malt{Weyermann CARAFA Special II}{\lbtokg{0.38}}
\end{malts}

\begin{hops}
\hop{Scurose}{}{--}{\lbtokg{2}}
\hop{\hophallertauertradition}{6~\%}{60~min}{\oztog{2.5}}
\hop{Yeast Nutrient}{}{10~min}{\tsptog{0.5}}
\hop{Whirlfloc Tablet}{}{10~min}{1}
\hop{Cardamon Seeds}{}{\foh{}}{0.5~g}
\end{hops}

\begin{yeasts}
\yeast{White Labs WLP530}
\end{yeasts}

\begin{twists}
\twist{Pomegranate Molasses}{Secondary}{\lbtokg{2}}
\end{twists}

\end{ingredientsblock}

\end{recipie}

% -----------------------------------------------------------------------------
\stylecategory{Strong Ale}
\stylesection{\styleoldale}

% -----------------------------------------------------------------------------
\begin{recipe}{Clan MacIntosh Old Ale} % rechecked
% -----------------------------------------------------------------------------

\begin{aboutblock}
Recipe by Alex MacIntosh of Providence, RI. Bronze medal in Category 19: Strong
UK Ale during the 2019 National Homebrew Competition in Providence, RI. \sourceaha
\end{aboutblock}

\specifications{\styleoldale}{\galtol{6.1}}{1.105}{1.030}{10~\%}{32.7}{\srmtoebc{18.7}}{200~min}{}

\begin{methodandtiming}
 
\begin{mashsteps}
\mashstep{\ftoc{152}}{60~min}
\end{mashsteps}

\begin{fermentationsteps}
\fermentationstep{\ftoc{63}}{Pitch}
 \fermentationstep{\ftoc{65}}{Fermentation start; 14~days}
\end{fermentationsteps}

\begin{directions}
Water adjustment: mash pH of 5.4. Transfer to an oak barrel and age for
3 months.
\end{directions}

\end{methodandtiming}

\recipebreak

\begin{ingredientsblock}

\begin{malts}
\malt{Crisp Finest Maris Otter}{\lbtokg{18}}
\malt{Briess Caramel 40 L}{\oztokg{12}}
\malt{Gambrinus Honey}{\oztokg{8}}
\malt{Munich}{\oztokg{8}}
\malt{Weyermann Chocolate Wheat}{\oztokg{4}}
\malt{Crisp Medium Crystal}{\oztokg{4}}
\end{malts}

\begin{hops}
\hop{\hopwillamette}{4.4~\%}{60~min}{\oztog{2}}
\hop{\hopmagnum}{14~\%}{60~min}{\oztog{0.3}}
\end{hops}

\singleyeast{GigaYeast GY044}

\end{ingredientsblock}

\end{recipe}

\stylesection{\styleamericanbarleywine}

% -----------------------------------------------------------------------------
\begin{recipe}{Big Blimp Barleywine} % rechecked
% -----------------------------------------------------------------------------

\begin{aboutblock}
Recipe by Donna Reuter of Akron, OH. Gold medal in Category 18: Strong American
Ale during the 2019 National Homebrew Competition in Providence, RI. \sourceaha
\end{aboutblock}

\specifications{\styleamericanbarleywine}{\galtol{5.5}}{1.096}{1.018}{10.31~\%}{100}{\srmtoebc{17}}{90~min}{}

\begin{methodandtiming}
 
\begin{mashsteps}
\mashstep{\ftoc{150}}{90~min}
\end{mashsteps}

\begin{fermentationsteps}
\fermentationstep{\ftoc{60}}{21~days}

\end{fermentationsteps}

\begin{directions}
Water adjustment: Akron, Ohio, city water with \tsptog{1} calcium chloride and
\tsptog{0.5} calcium sulfate; mash pH of 5.3. Priming: \oztog{3.5} dextrose
over 14 days weeks. Cool crash to \ftoc{30} and hold 14 days, then age until
ready to serve.
\end{directions}

\end{methodandtiming}

\recipebreak

\begin{ingredientsblock}

\begin{malts}
\malt{Maris Otter}{\lbtokg{19}}
\malt{Caramel / Crystal 40 L}{\lbtokg{1}}
\malt{Caramel / Crystal 80 L}{\oztog{8}}
\malt{Dingemans Special B}{\oztog{4}}
\malt{Carapils / Dextrin}{\oztog{4}}
\end{malts}

\begin{hops}
\hop{\hopcascade}{5.75~\%}{\fwh}{\oztog{0.75}}
\hop{\hopchinook}{12.8~\%}{\fwh}{\oztog{0.75}}
\hop{\hopcentennial}{10~\%}{\fwh}{\oztog{0.75}}
\hop{Dextrose}{}{}{\lbtokg{1}}
\hop{\hopcascade}{5.75~\%}{1~min}{\oztog{1.25}}
\hop{\hopchinook}{12.8~\%}{1~min}{\oztog{1.25}}
\hop{\hopcentennial}{10~\%}{1~min}{\oztog{1.25}}
\hop{\hopcascade}{5.75~\%}{\dryh{}{5~days}}{\oztog{1}}
\hop{\hopchinook}{12.8~\%}{\dryh{}{5~days}}{\oztog{2}}
\hop{\hopcentennial}{10~\%}{\dryh{}{5~days}}{\oztog{1}}
\end{hops}

\singleyeast{Wyeast 1056}

\end{ingredientsblock}

\end{recipe}

\stylesection{\styleenglishbarleywine}

% -----------------------------------------------------------------------------
\begin{recipe}{"I See London" English Barleywine}
% -----------------------------------------------------------------------------

\begin{aboutblock}
Recipe by Metts Potter of Browns Summit, NC. Silver medal in Category 19:
Strong UK Ale during the 2019 National Homebrew Competition in Providence,
RI. \sourceaha
\end{aboutblock}

\specifications{\styleenglishbarleywine}{\galtol{5.5}}{1.104}{1.032}{9.3~\%}{43}{\srmtoebc{18}}{90~min}{2.5}

\begin{methodandtiming}
 
\begin{mashsteps}
\mashstep{\ftoc{152}}{60~min}
\mashstep{\ftoc{170}}{Mash out}
\end{mashsteps}

\begin{fermentationsteps}
\fermentationstep{\ftoc{66}}{1~day}
\fermentationstep{\ftoc{68}}{5~days / fermentation slowdown}
\fermentationstep{\ftoc{72}}{10~days}
\end{fermentationsteps}

\begin{directions}
Water adjustment: mash pH of 5.4.
\end{directions}

\end{methodandtiming}

\recipebreak

\begin{ingredientsblock}

\begin{malts}
\malt{Bairds Maris Otter Pale Ale}{\lbtokg{12}}
\malt{BEST Vienna}{\lbtokg{5}}
\malt{Dingemans Organic Biscuit}{\lbtokg{2}}
\malt{Caramel / Crystal 15 L}{\lbtokg{1}}
\malt{Bairds Caramalt Medium}{\lbtokg{1}}
\malt{Fawcett Dark Crystal I}{\lbtokg{1}}
\end{malts}

\begin{hops}
\hop{\hopeastkentgolding}{6~\%}{60~min}{\oztog{1}}
\hop{\hopfuggle}{4.2~\%}{60~min}{\oztog{1}}
\hop{\hopeastkentgolding}{6~\%}{30~min}{\oztog{0.5}}
\hop{\hopfuggle}{4.2~\%}{30~min}{\oztog{0.5}}
\hop{\hopeastkentgolding}{6~\%}{15~min}{\oztog{0.5}}
\hop{\hopfuggle}{4.2~\%}{15~min}{\oztog{0.5}}
\end{hops}

\singleyeast{Wyeast 1968}

\end{ingredientsblock}

\end{recipe}

% -----------------------------------------------------------------------------
\begin{recipe}{Just Another Pretty Face}
% -----------------------------------------------------------------------------

\begin{aboutblock}
Recipe by Doug Thiel. Gold medal at 2017 GABF Pro-Am.
\sourcezymurgy{March/April 2018}
\end{aboutblock}

\specifications{\styleenglishbarleywine}{\galtol{5.5}}{1.093}{1.018}{10~\%}{40}{\srmtoebc{18}}{60~min}{}

\begin{methodandtiming}

\begin{mashsteps}
\mashstep{\ftoc{154}}{60~min}
\end{mashsteps}

\begin{fermentationsteps}
\fermentationstep{\ftoc{68}}{}
\end{fermentationsteps}

\begin{directions}
Water adjustment: \waterprofile{32}{9}{26}{48}{27}{\cacotohco{89}}. Age in bourbon barrel for 2 months
or add bourbon-soaked oak spiral until desired flavour is achieved.
\end{directions}

\end{methodandtiming}

\recipebreak

\begin{ingredientsblock}

\begin{malts}
\malt{\maltpale}{\lbtokg{16.38}}
\malt{Briess Aromatic Munich 20 L}{\lbtokg{1}}
\malt{Caramel / Crystal 120 L}{\lbtokg{1}}
\end{malts}

\begin{hops}
\hop{\hopchallenger}{9.6~\%}{60~min}{\oztog{2}}
\hop{Turbinado Sugar}{}{15~min}{\lbtokg{1}}
\hop{\hopfuggle}{4~\%}{10~min}{\oztog{1}}
\end{hops}

\singleyeast{White Labs WLP028}

\end{ingredientsblock}

\end{recipe}


% -----------------------------------------------------------------------------
\stylecategory{Fruit Beer}
\stylesection{\stylefruitbeer}

% -----------------------------------------------------------------------------
\begin{recipe}{Sour Blonde Ale with Blackberries and Raspberries}
% -----------------------------------------------------------------------------

\begin{aboutblock}
Damien Jones of Petaluma, CA, won a silver medal in Category \#28: American Wild
Ale with a Soured Blonde Ale with raspberry, blackberry during the 2019 National
Homebrew Competition Final Round in Providence, RI. Jones's Soured Blonde Ale
with raspberry, blackberry was chosen as the runner-up entry among 334 entries in
the category. \sourceaha
\end{aboutblock}

\specifications{\stylefruitbeer}{\galtol{5.5}}{1.046}{1.010}{5.1~\%}{}{\srmtoebc{14.6}}{60~min}{2.5}

\begin{methodandtiming}
 
\begin{mashsteps}
\mashstep{\ftoc{156}}{45~min}
\end{mashsteps}

\begin{fermentationsteps}
\fermentationstep{\ftoc{60}}{Pitch}
\fermentationstep{\ftoc{75}}{Free raise to; until fully attenuated}
\end{fermentationsteps}

\begin{directions}
Adjust hopping rate based on your mixed culture hop tolerance. After airlock
activity has subsided, transfer to a purged vessel and minimize head space.
Keep airlock topped up. Condition for 4--6 months at cellar temperatures.
Transfer to purged vessel on top of fruit puree. After secondary fermentation
is complete, condition for several months on the fruit. Cold crash and transfer
to a purged keg leaving the fruit behind.
\end{directions}

\end{methodandtiming}

\recipebreak

\begin{ingredientsblock}

\begin{malts}
\malt{Pilsner}{\lbtokg{7.5}}
\malt{Wheat}{\lbtokg{1.5}}
\malt{Aromatic}{\oztog{8}}
\malt{Chateau Spelt}{\oztog{8}}
\malt{Thomas Fawcett Crystal Malt II}{\oztog{8}}
\malt{Flaked Oats}{\oztog{8}}
\malt{Weyermann CARAFA Special III}{\oztog{4}}
\end{malts}

\begin{hops}
\hop{\hopstrisselspalt}{4.0~\%}{60~min}{\oztog{0.1}}
\end{hops}

\singleyeast{Mixed Culture}

\begin{twists}
\twist{Blackberry Puree}{Secondary}{\oztokg{48}}
\twist{Raspberry Puree}{Secondary}{\oztokg{48}}
\end{twists}

\end{ingredientsblock}

\end{recipe}


% -----------------------------------------------------------------------------
\stylecategory{Spice / Herb / Vegetable Beer}
\stylesection{\stylewinterseasonalbeer}

% -----------------------------------------------------------------------------
\begin{recipe}{Holiday Pumpkin Ale}
% -----------------------------------------------------------------------------

\begin{aboutblock}
Recipe by Mark Pasquinelli. \sourceaha
\end{aboutblock}

\specifications{\stylewinterseasonalbeer}{\galtol{6}}{1.071}{1.015}{7.4~\%}{19}{\srmtoebc{12}}{90~min}{}

\begin{methodandtiming}
 
\begin{mashsteps}
\mashstep{\ftoc{155}}{60~min}
\end{mashsteps}

\begin{directions}
Ferment for 7 days, then transfer to secondary for 14 days.
\end{directions}

\end{methodandtiming}

\recipebreak

\begin{ingredientsblock}

\begin{malts}
\malt{\maltmarisotter}{\lbtokg{8}}
\malt{\maltmunich}{\lbtokg{4}}
\malt{Aromatic}{\lbtokg{2}}
\malt{\maltweyermanncaramunichone}{\oztokg{10}}
\end{malts}

\begin{hops}
\hop{Pumkin}{}{90~min}{\lbtokg{5}}
\hop{Brown Sugar}{}{60~min}{\oztog{8}}
\hop{\hopfuggle}{4.6~\%}{45~min}{\oztog{1.25}}
\hop{Cinnamon}{}{5~min}{\tsptog{3}}
\hop{Nutmeg}{}{5~min}{\tsptog{1.9}}
\hop{Ginger Root}{}{5~min}{\tbsptog{1}}
\end{hops}

\begin{twists}
\twist{Vanilla Extract}{Secondary}{\tsptog{4}}
\end{twists}

\singleyeast{White Labs WLP002}

\end{ingredientsblock}

\end{recipe}


% -----------------------------------------------------------------------------
\stylecategory{Smoke-Flavored \& Wood-Aged Beer}
\stylesection{\stylerauchbier}

% -----------------------------------------------------------------------------
\begin{recipe}{Homebrew Challenge Smoke Beer (Rauchbier)}
% -----------------------------------------------------------------------------

\begin{aboutblock}
Recipe by Martin Keen.
\sourcehomebrewchallenge
\end{aboutblock}

\specifications{\stylerauchbier}{\galtol{5}}{}{}{5~\%}{27}{}{60~min}{}

\begin{methodandtiming}

\begin{mashsteps}
\mashstep{\ftoc{152}}{60~min}
\end{mashsteps}

\begin{fermentationsteps}
\fermentationstep{\ftoc{50}}{}
\end{fermentationsteps}

\end{methodandtiming}

\recipebreak

\begin{ingredientsblock}

\begin{malts}
\malt{Cherry Wood Smoked Malt}{\lbtokg{10}}
\malt{\maltweyermanncarafaspecialtwo}{\oztokg{4}}
\malt{\maltweyermanncaramunichone}{\oztokg{4}}
\end{malts}

\begin{hops}
\hop{\hopperle}{}{60~min}{\oztog{1}}
\hop{\hophallertaumittelfruh}{}{10~min}{\oztog{0.5}}
\end{hops}

\singleyeast{White Labs WLP830}

\end{ingredientsblock}

\end{recipe}


% -----------------------------------------------------------------------------
\stylecategory{Specialty Beer}
\stylesection{\stylealternativesugarbeer}

% -----------------------------------------------------------------------------
\begin{recipe}{Chomolungma Brown Ale} % rechecked
% -----------------------------------------------------------------------------

\begin{aboutblock}
Recipe by Jackie O's Pub \& Brewery, Athens, OH. Brad Clark created the recipe
as an American interpretation of Newcastle Brown Ale.
\sourcezymurgy{November / December 2019}
\end{aboutblock}

\specifications{\stylealternativesugarbeer}{\galtol{5}}{1.060}{1.013}{6.5~\%}{25}{\srmtoebc{25}}{75~min}{}

\begin{methodandtiming}
 
\begin{mashsteps}
\mashstep{\ftoc{152}}{30~min}
\mashstep{\ftoc{165}}{Mash out}
\end{mashsteps}

\begin{fermentationsteps}
\fermentationstep{\ftoc{65}}{}
\end{fermentationsteps}

\begin{directions}
Water adjustment: \waterprofile{48}{12}{80}{30}{46}{}; mash pH of 5.2.
\end{directions}

\end{methodandtiming}

\recipebreak

\begin{ingredientsblock}

\begin{malts}
\malt{Briess Brewers}{\lbtokg{4.6}}
\malt{Briess Bonlander Munich 10 L}{\lbtokg{1.4}}
\malt{Briess Caramel 60 L}{\lbtokg{0.75}}
\malt{Briess White Wheat}{\lbtokg{0.5}}
\malt{Dingemans Mroost 900 (Chocolate)}{\lbtokg{0.4}}
\malt{Dingemans Aromatic / Amber}{\lbtokg{0.4}}
\malt{Briess Special Roast}{\lbtokg{0.4}}
\end{malts}

\begin{hops}
\hop{\hopnorthernbrewer}{8.5~\%}{75~min}{\oztog{0.2}}
\hop{\hopnorthernbrewer}{8.5~\%}{60~min}{\oztog{0.2}}
\hop{\hopnorthernbrewer}{8.5~\%}{45~min}{\oztog{0.2}}
\hop{Minimally Processed Honey}{}{30~min}{\lbtokg{2.5}}
\hop{\hopwillamette}{5.5~\%}{15~min}{\oztog{0.7}}
\end{hops}

\singleyeast{English Ale}

\end{ingredientsblock}

\end{recipe}


\appendix

\twocolumn

\part{Appendix}

\chapter{Malt}

\begin{maltinfos}{Vendor Neutral}
\maltinfo{Black}{\ltoebc{470}}{\ltoebc{620}}
\maltinfo{Brown}{\ltoebc{50}}{\ltoebc{75}}
\maltinfo{Caramel / Crystal 10 L}{\ltoebc{10}}{\ltoebc{10}}
\maltinfo{Caramel / Crystal 15 L}{\ltoebc{15}}{\ltoebc{15}}
\maltinfo{Caramel / Crystal 20 L}{\ltoebc{20}}{\ltoebc{20}}
\maltinfo{Caramel / Crystal 30 L}{\ltoebc{30}}{\ltoebc{30}}
\maltinfo{Caramel / Crystal 40 L}{\ltoebc{40}}{\ltoebc{40}}
\maltinfo{Caramel / Crystal 45 L}{\ltoebc{45}}{\ltoebc{45}}
\maltinfo{Caramel / Crystal 55 L}{\ltoebc{55}}{\ltoebc{55}}
\maltinfo{Caramel / Crystal 60 L}{\ltoebc{60}}{\ltoebc{60}}
\maltinfo{Caramel / Crystal 65 L}{\ltoebc{65}}{\ltoebc{65}}
\maltinfo{Caramel / Crystal 75 L}{\ltoebc{75}}{\ltoebc{75}}
\maltinfo{Caramel / Crystal 80 L}{\ltoebc{80}}{\ltoebc{80}}
\maltinfo{Caramel / Crystal 90 L}{\ltoebc{90}}{\ltoebc{90}}
\maltinfo{Caramel / Crystal 120 L}{\ltoebc{120}}{\ltoebc{120}}
\maltinfo{Chocolate}{\ltoebc{400}}{\ltoebc{500}}
\maltinfo{Pale Chocolate}{\ltoebc{200}}{\ltoebc{300}}
\end{maltinfos}

\begin{maltinfos}{Admiral}
\maltinfo{Feldblume}{\srmtoebc{2}}{\srmtoebc{3}}
\end{maltinfos}

\begin{maltinfos}{Avangard}
\maltinfo{Pilsner}{\ebctoebc{3}}{\ebctoebc{3.5}}
\maltinfo{Vienna}{\ebctoebc{8}}{\ebctoebc{15}}
\end{maltinfos}

\begin{maltinfos}{Bairds}
\maltinfo{Brown}{\ebctoebc{110}}{\ebctoebc{130}}
\maltinfo{Caramalt Medium}{\ebctoebc{55}}{\ebctoebc{75}}
\maltinfo{Crystal Medium}{\ebctoebc{140}}{\ebctoebc{160}}
\maltinfo{Maris Otter Pale Ale}{\ebctoebc{5}}{\ebctoebc{7}}
\end{maltinfos}

\pagebreak

\begin{maltinfos}{BEST}
\maltinfo{Caramel Aromatic}{\ebctoebc{41}}{\ebctoebc{60}}
\maltinfo{Chit}{\ebctoebc{2}}{\ebctoebc{3}}
\maltinfo{Heidelberg}{\ebctoebc{2.9}}{\ebctoebc{2.9}}
\maltinfo{Munich}{\ebctoebc{11}}{\ebctoebc{20}}
\maltinfo{Munich Dark}{\ebctoebc{21}}{\ebctoebc{35}}
\maltinfo{Pilsen}{\ebctoebc{3}}{\ebctoebc{4.9}}
\maltinfo{Vienna}{\ebctoebc{8}}{\ebctoebc{10}}
\end{maltinfos}

\begin{maltinfos}{Briess}
\maltinfo{Aromatic Munich 20 L}{\ltoebc{20}}{\ltoebc{20}}
\maltinfo{Blackprinz}{\ltoebc{500}}{\ltoebc{500}}
\maltinfo{Blonde RoastOat}{\ltoebc{4}}{\ltoebc{4}}
\maltinfo{Bonlander Munich 10 L}{\ltoebc{10}}{\ltoebc{10}}
\maltinfo{Brewers}{\ltoebc{1.8}}{\ltoebc{1.8}}
\maltinfo{Caramel 20 L}{\ltoebc{20}}{\ltoebc{20}}
\maltinfo{Caramel 30 L}{\ltoebc{30}}{\ltoebc{30}}
\maltinfo{Caramel 40 L}{\ltoebc{40}}{\ltoebc{40}}
\maltinfo{Caramel 60 L}{\ltoebc{60}}{\ltoebc{60}}
\maltinfo{Caramel 80 L}{\ltoebc{80}}{\ltoebc{80}}
\maltinfo{Caramel Vienne 20 L}{\ltoebc{20}}{\ltoebc{20}}
\maltinfo{Carapils}{\ltoebc{1.5}}{\ltoebc{1.5}}
\maltinfo{Chocolate}{\ltoebc{350}}{\ltoebc{350}}
\maltinfo{Extra Special}{\ltoebc{130}}{\ltoebc{130}}
\maltinfo{Midnight Wheat}{\ltoebc{550}}{\ltoebc{550}}
\maltinfo{Organic Caramel 60 L}{\ltoebc{60}}{\ltoebc{60}}
\maltinfo{Organic Chocolate}{\ltoebc{350}}{\ltoebc{350}}
\maltinfo{Roasted Barley}{\ltoebc{300}}{\ltoebc{300}}
\maltinfo{Pale Ale}{\ltoebc{3.5}}{\ltoebc{3.5}}
\maltinfo{Pilsen}{\ltoebc{1.2}}{\ltoebc{1.2}}
\maltinfo{Red Wheat}{\ltoebc{3}}{\ltoebc{3}}
\maltinfo{Rye}{\ltoebc{3.7}}{\ltoebc{3.7}}
\maltinfo{Special Roast}{\ltoebc{40}}{\ltoebc{40}}
\maltinfo{Synergy Select Pilsen}{\ltoebc{1.8}}{\ltoebc{1.8}}
\maltinfo{Victory}{\ltoebc{28}}{\ltoebc{28}}
\maltinfo{White Wheat}{\ltoebc{2.8}}{\ltoebc{2.8}}
\end{maltinfos}

\pagebreak

\begin{maltinfos}{Cargill}
\maltinfo{Pauls Mild Ale (Dextrin)}{\ebctoebc{8}}{\ebctoebc{10}}
\end{maltinfos}

\begin{maltinfos}{Castle}
\maltinfo{Chateau Abbey}{\ebctoebc{45}}{\ebctoebc{45}}
\maltinfo{Chateau Black Nature}{\ebctoebc{1150}}{\ebctoebc{1400}}
\maltinfo{Chateau Pale Ale}{\ebctoebc{7}}{\ebctoebc{10}}
\maltinfo{Chateau Special Belgium}{\ebctoebc{260}}{\ebctoebc{320}}
\maltinfo{Chateau Spelt}{\ebctoebc{3}}{\ebctoebc{7}}
\end{maltinfos}

\begin{maltinfos}{Crisp}
\maltinfo{Amber}{\ebctoebc{60}}{\ebctoebc{80}}
\maltinfo{Brown}{\ebctoebc{120}}{\ebctoebc{150}}
\maltinfo{Cara}{\ebctoebc{25}}{\ebctoebc{35}}
\maltinfo{Chocolate}{\ebctoebc{900}}{\ebctoebc{1100}}
\maltinfo{Dextrin}{\ebctoebc{2.5}}{\ebctoebc{3.5}}
\maltinfo{Finest Maris Otter Ale}{\ebctoebc{5.5}}{\ebctoebc{7.5}}
\maltinfo{Light Crystal}{\ebctoebc{160}}{\ebctoebc{180}}
\maltinfo{Medium Crystal}{\ebctoebc{250}}{\ebctoebc{290}}
\maltinfo{Roast Barley}{\ebctoebc{1250}}{\ebctoebc{1450}}
\end{maltinfos}

\begin{maltinfos}{Dingemans}
\maltinfo{Aromatic / Amber}{\ebctoebc{40}}{\ebctoebc{60}}
\maltinfo{Biscuit}{\ebctoebc{50}}{\ebctoebc{70}}
\maltinfo{Cara 120}{\ebctoebc{100}}{\ebctoebc{140}}
\maltinfo{Mroost 900 (Chocolate)}{\ebctoebc{800}}{\ebctoebc{1000}}
\maltinfo{Organic Biscuit}{\ebctoebc{50}}{\ebctoebc{70}}
\maltinfo{Pilsen}{\ebctoebc{2.5}}{\ebctoebc{3.5}}
\maltinfo{Special B}{\ebctoebc{300}}{\ebctoebc{350}}
\end{maltinfos}

\begin{maltinfos}{Durst}
\maltinfo{Pilsner}{\ebctoebc{3}}{\ebctoebc{4}}
\maltinfo{Wheat}{\ebctoebc{3.2}}{\ebctoebc{4}}
\end{maltinfos}

\pagebreak

\begin{maltinfos}{Fawcett}
\maltinfo{Brown}{\ltoebc{66}}{\ltoebc{76}}
\maltinfo{Caramalt}{\ltoebc{10}}{\ltoebc{12}}
\maltinfo{Chocolate}{\ltoebc{390}}{\ltoebc{450}}
\maltinfo{Crystal Malt II}{\ltoebc{60}}{\ltoebc{70}}
\maltinfo{Dark Crystal I}{\ltoebc{80}}{\ltoebc{90}}
\maltinfo{Dark Crystal II}{\ltoebc{110}}{\ltoebc{130}}
\maltinfo{Golden Promise Pale Ale}{\ltoebc{2.3}}{\ltoebc{3}}
\maltinfo{Pale Chocolate}{\ltoebc{200}}{\ltoebc{260}}
\maltinfo{Pearl Pale Ale}{\ltoebc{2.3}}{\ltoebc{3}}
\maltinfo{Roasted Barley}{\ltoebc{490}}{\ltoebc{600}}
\maltinfo{Torrified Wheat}{\ltoebc{1.5}}{\ltoebc{3}}
\end{maltinfos}

\begin{maltinfos}{Gambrinus}
\maltinfo{ESB Pale}{\ltoebc{2.5}}{\ltoebc{3.5}}
\maltinfo{Honey}{\ebctoebc{17}}{\ebctoebc{25}}
\end{maltinfos}

\begin{maltinfos}{Great Western}
\maltinfo{Crystal 40}{\ltoebc{40}}{\ltoebc{40}}
\maltinfo{Crystal 60}{\ltoebc{60}}{\ltoebc{60}}
\maltinfo{Crystal 75}{\ltoebc{75}}{\ltoebc{75}}
\maltinfo{Crystal 120}{\ltoebc{120}}{\ltoebc{120}}
\maltinfo{Organic Premium Two-row}{\ltoebc{2}}{\ltoebc{2}}
\maltinfo{Premium Two-row}{\ltoebc{1.6}}{\ltoebc{1.6}}
\maltinfo{Pure California}{\ltoebc{2}}{\ltoebc{2}}
\maltinfo{Superior Pilsen}{\ltoebc{1.6}}{\ltoebc{1.6}}
\end{maltinfos}

\begin{maltinfos}{Malteurop}
\maltinfo{Pilsen}{\ebctoebc{3}}{\ebctoebc{4.5}}
\end{maltinfos}

\begin{maltinfos}{Mecca Grade}
\maltinfo{Pelton}{\srmtoebc{1.6}}{\srmtoebc{1.8}}
\end{maltinfos}

\begin{maltinfos}{Patagonia}
\maltinfo{Black Pearl}{\ebctoebc{800}}{\ebctoebc{1000}}
\end{maltinfos}

\pagebreak

\begin{maltinfos}{Rahr}
\maltinfo{Pale Ale}{\ltoebc{3}}{\ltoebc{4}}
\maltinfo{Premium Pilsner}{\ltoebc{1.5}}{\ltoebc{2.0}}
\maltinfo{Red Wheat}{\ltoebc{3}}{\ltoebc{3.5}}
\maltinfo{Standard Six-row}{\ltoebc{2.1}}{\ltoebc{2.5}}
\maltinfo{Standard Two-row}{\ltoebc{1.7}}{\ltoebc{2}}
\maltinfo{White Wheat}{\ltoebc{3}}{\ltoebc{3.5}}
\end{maltinfos}

\begin{maltinfos}{Root Shoot}
\maltinfo{Odyssey Pilsner}{\ltoebc{2}}{\ltoebc{2}}
\end{maltinfos}

\begin{maltinfos}{Simpsons}
\maltinfo{Amber}{\ebctoebc{54}}{\ebctoebc{71}}
\maltinfo{Brown}{\ebctoebc{162}}{\ebctoebc{226}}
\maltinfo{Chocolate}{\ebctoebc{1067}}{\ebctoebc{1300}}
\maltinfo{Crystal Dark}{\ebctoebc{250}}{\ebctoebc{285}}
\maltinfo{Crystal Medium}{\ebctoebc{167}}{\ebctoebc{190}}
\maltinfo{Crystal T50}{\ebctoebc{130}}{\ebctoebc{145}}
\maltinfo{DRC}{\ebctoebc{280}}{\ebctoebc{320}}
\maltinfo{Finest Pale Ale Golden Promise}{\ebctoebc{4.5}}{\ebctoebc{6.5}}
\maltinfo{Finest Pale Ale Maris Otter}{\ebctoebc{4.5}}{\ebctoebc{6.5}}
\maltinfo{Golden Naked Oats}{\ebctoebc{12}}{\ebctoebc{25}}
\maltinfo{Roasted Barley}{\ebctoebc{1300}}{\ebctoebc{1900}}
\end{maltinfos}

\begin{maltinfos}{The Swaen}
\maltinfo{Pilsner}{\ebctoebc{3}}{\ebctoebc{4.5}}
\end{maltinfos}

\pagebreak

\begin{maltinfos}{Weyermann}
\maltinfo{Barke Pilsner}{\ebctoebc{2.5}}{\ebctoebc{4.5}}
\maltinfo{Beech Smoked Barley Malt}{\ebctoebc{4}}{\ebctoebc{8}}
\maltinfo{Bohemian Pilsner}{\ebctoebc{3}}{\ebctoebc{5}}
\maltinfo{CARAAMBER}{\ebctoebc{60}}{\ebctoebc{80}}
\maltinfo{CARAAROMA}{\ebctoebc{350}}{\ebctoebc{450}}
\maltinfo{CARABOHEMIAN}{\ebctoebc{170}}{\ebctoebc{220}}
\maltinfo{CARAFA I}{\ebctoebc{800}}{\ebctoebc{1000}}
\maltinfo{CARAFA II}{\ebctoebc{1100}}{\ebctoebc{1200}}
\maltinfo{CARAFA SPECIAL II}{\ebctoebc{1100}}{\ebctoebc{1200}}
\maltinfo{CARAFA SPECIAL III}{\ebctoebc{1300}}{\ebctoebc{1500}}
\maltinfo{CARAHELL}{\ebctoebc{20}}{\ebctoebc{30}}
\maltinfo{CARAMUNICH I}{\ebctoebc{80}}{\ebctoebc{100}}
\maltinfo{CARAMUNICH II}{\ebctoebc{110}}{\ebctoebc{130}}
\maltinfo{CARAMUNICH III}{\ebctoebc{140}}{\ebctoebc{160}}
\maltinfo{CARAPILS}{\ebctoebc{2.5}}{\ebctoebc{6.5}}
\maltinfo{CARARED}{\ebctoebc{40}}{\ebctoebc{60}}
\maltinfo{Chocolate Wheat}{\ebctoebc{900}}{\ebctoebc{1200}}
\maltinfo{Melanoidin}{\ebctoebc{60}}{\ebctoebc{80}}
\maltinfo{Munich I}{\ebctoebc{12}}{\ebctoebc{18}}
\maltinfo{Munich II}{\ebctoebc{20}}{\ebctoebc{25}}
\maltinfo{Pale Wheat}{\ebctoebc{3}}{\ebctoebc{5}}
\maltinfo{Pilsner}{\ebctoebc{2.5}}{\ebctoebc{4.5}}
\maltinfo{Vienna}{\ebctoebc{6}}{\ebctoebc{9}}
\end{maltinfos}

\chapter{Yeast}

\begin{yeastinfos}{Brewing Science Institute}
\yeastinfo{A-18}{London Ale III}
\yeastinfo{B-22}{LaChouffe}
\yeastinfo{S-26}{Farmhouse Ale}
\end{yeastinfos}

\begin{yeastinfos}{GigaYeast}
\yeastinfo{GY001}{Norcal Ale \#1}
\yeastinfo{GY002}{Czech Pilsner}
\yeastinfo{GY005}{Golden Gate Lager}
\yeastinfo{GY014}{Belgian Abbey Ale}
\yeastinfo{GY017}{Bavarian Hefe}
\yeastinfo{GY020}{Portland Hefe}
\yeastinfo{GY021}{Kölsch Bier}
\yeastinfo{GY027}{Saison \#2}
\yeastinfo{GY031}{British Ale \#2}
\yeastinfo{GY044}{Scotch Ale \#1}
\yeastinfo{GY045}{German Lager}
\yeastinfo{GY054}{Vermont IPA}
\yeastinfo{GY080}{Irish Stout}
\end{yeastinfos}

\begin{yeastinfos}{Imperial Yeast}
\yeastinfo{A01}{House}
\yeastinfo{A04}{Barbarian}
\yeastinfo{A07}{Flagship}
\yeastinfo{A09}{Pub}
\yeastinfo{A10}{Darkness}
\yeastinfo{A15}{Independence}
\yeastinfo{A24}{Dry Hop}
\yeastinfo{A31}{Tartan}
\yeastinfo{B44}{Whiteout}
\yeastinfo{B45}{Gnome}
\yeastinfo{B48}{Triple Double}
\yeastinfo{B63}{Monastic}
\yeastinfo{G01}{Stefon}
\yeastinfo{G02}{Kaiser}
\yeastinfo{G03}{Dieter}
\yeastinfo{L05}{Cablecar}
\yeastinfo{L13}{Global}
\yeastinfo{L17}{Harvest}
\end{yeastinfos}

\pagebreak

\begin{yeastinfos}{Inland Island Yeast Labs}
\yeastinfo{INIS-003}{Colorado IPA}
\end{yeastinfos}

\begin{yeastinfos}{Omega Yeast}
\yeastinfo{OYL-001}{Alt}
\yeastinfo{OYL-002}{American Wheat}
\yeastinfo{OYL-003}{London Ale}
\yeastinfo{OYL-004}{West Coast Ale I}
\yeastinfo{OYL-005}{Irish Ale}
\yeastinfo{OYL-006}{British Ale I}
\yeastinfo{OYL-007}{British Ale II}
\yeastinfo{OYL-009}{West Coast Ale II}
\yeastinfo{OYL-010}{British Ale IV}
\yeastinfo{OYL-012}{British Ale V}
\yeastinfo{OYL-014}{British Ale VII}
\yeastinfo{OYL-015}{Scottish Ale}
\yeastinfo{OYL-016}{British Ale VIII}
\yeastinfo{OYL-017}{Kolsch}
\yeastinfo{OYL-018}{Abbey Ale C}
\yeastinfo{OYL-019}{Belgian Ale D}
\yeastinfo{OYL-021}{Hefeweizen Ale I}
\yeastinfo{OYL-022}{Hefeweizen Ale II}
\yeastinfo{OYL-027}{Belgian Saison I}
\yeastinfo{OYL-030}{Wit}
\yeastinfo{OLY-041}{CL-50 Ale}
\yeastinfo{OLY-052}{DIPA Ale}
\yeastinfo{OLY-101}{Pilsner I}
\yeastinfo{OLY-105}{West Coast Lager}
\yeastinfo{OLY-106}{German Lager I}
\yeastinfo{OLY-107}{Oktoberfest}
\yeastinfo{OLY-108}{Pilsner II}
\yeastinfo{OLY-109}{German Lager II}
\yeastinfo{OYL-500}{Saisonstein's Monster}
\end{yeastinfos}

\begin{yeastinfos}{RVA Yeast Labs}
\yeastinfo{RVA 132}{Manchester Ale}
\end{yeastinfos}

\pagebreak

\begin{yeastinfos}{The Yeast Bay}
\yeastinfo{WLP4000}{Vermont Ale}
\yeastinfo{WLP4021}{Saison Blend II}
\yeastinfo{WLP4042}{Hazy Daze}
\yeastinfo{WLP4627}{Funktown Pale Ale}
\end{yeastinfos}

\pagebreak

\begin{yeastinfos}{White Labs}
\yeastinfo{WLP001}{California Ale}
\yeastinfo{WLP002}{English Ale}
\yeastinfo{WLP004}{Irish Ale}
\yeastinfo{WLP007}{Dry English Ale}
\yeastinfo{WLP013}{London Ale}
\yeastinfo{WLP023}{Burton Ale}
\yeastinfo{WLP025}{Southwold Ale}
\yeastinfo{WLP028}{Edinburgh Scottish Ale}
\yeastinfo{WLP029}{German Ale / Kölsch}
\yeastinfo{WLP036}{Düsseldorf Alt Ale}
\yeastinfo{WLP039}{East Midlands Ale}
\yeastinfo{WLP041}{Pacific Ale}
\yeastinfo{WLP051}{California V Ale}
\yeastinfo{WLP080}{Cream Ale Yeast Blend}
\yeastinfo{WLP090}{San Diego Super}
\yeastinfo{WLP095}{Burlington Ale}
\yeastinfo{WLP300}{Hefeweizen Ale}
\yeastinfo{WLP320}{American Hefeweizen Ale}
\yeastinfo{WLP380}{Hefeweizen IV Ale}
\yeastinfo{WLP400}{Belgian Wit Ale}
\yeastinfo{WLP500}{Monastery Ale}
\yeastinfo{WLP510}{Bastogne Belgian Ale}
\yeastinfo{WLP530}{Abbey Ale}
\yeastinfo{WLP550}{Belgian Ale}
\yeastinfo{WLP565}{Belgian Saison I Ale}
\yeastinfo{WLP570}{Belgian Golden Ale}
\yeastinfo{WLP800}{Pilsner Lager}
\yeastinfo{WLP802}{Czech Budejovice Lager}
\yeastinfo{WLP810}{San Francisco Lager}
\yeastinfo{WLP820}{Oktoberfest/Märzen Lager}
\yeastinfo{WLP830}{German Lager}
\yeastinfo{WLP833}{German Bock Lager}
\yeastinfo{WLP838}{Southern German Lager}
\yeastinfo{WLP840}{American Lager Yeast}
\yeastinfo{WLP920}{Old Bavarian Lager}
\yeastinfo{WLP940}{Mexican Lager}
\yeastinfo{WLP1983}{Charlie's Fist Bump}
\end{yeastinfos}

\pagebreak

\begin{yeastinfos}{Wyeast}
\yeastinfo{1007}{German Ale}
\yeastinfo{1010}{American Wheat}
\yeastinfo{1028}{London Ale}
\yeastinfo{1056}{American Ale}
\yeastinfo{1084}{Irish Ale}
\yeastinfo{1098}{British Ale}
\yeastinfo{1099}{Whitbread Ale}
\yeastinfo{1187}{Ringwood Ale}
\yeastinfo{1214}{Belgian Abbey Style Ale}
\yeastinfo{1272}{American Ale II}
\yeastinfo{1275}{Thames Valley Ale}
\yeastinfo{1318}{London Ale III}
\yeastinfo{1335}{British Ale II}
\yeastinfo{1388}{Belgian Strong Ale}
\yeastinfo{1450}{Denny's Favorite 50 Ale}
\yeastinfo{1469}{West Yorkshire Ale}
\yeastinfo{1728}{Scottish Ale}
\yeastinfo{1764-PC}{ROGUE Pacman}
\yeastinfo{1968}{London ESB Ale}
\yeastinfo{2001}{Pilsner Urquell H-Strain}
\yeastinfo{2035-PC}{American Lager}
\yeastinfo{2042-PC}{Danish Lager}
\yeastinfo{2112}{California Lager}
\yeastinfo{2124}{Bohemian Lager}
\yeastinfo{2206}{Bavarian Lager}
\yeastinfo{2247-PC}{European Lager}
\yeastinfo{2278}{Czech Pils}
\yeastinfo{2308}{Munich Lager}
\yeastinfo{2352}{Munich Lager II}
\yeastinfo{2565}{Kölsch}
\yeastinfo{2633}{Octoberfest Lager Blend}
\yeastinfo{3068}{Weihenstephan Weizen}
\yeastinfo{3333-PC}{German Wheat}
\yeastinfo{3522}{Belgian Ardennes}
\yeastinfo{3711}{French Saison}
\yeastinfo{3724}{Belgian Saison}
\yeastinfo{3726}{Farmhouse Ale}
\yeastinfo{3787}{Trappist Style High Gravity}
\yeastinfo{3944}{Belgian Witbier}
\end{yeastinfos}

\onecolumn

\chapter{Yeast Substitution}

\begin{recipeblock}{Substitution Table} 

\begin{tabu} to \textwidth {cccccX}

\textbf{Wyeast} & \textbf{White Labs} & \textbf{Omega} & \textbf{Imperial} & \textbf{GigaYeast} & \textbf{Dry} \\ \midrule

1007 & WLP036 & OYL-001 & G02 & n/a & Fermentis SafeAle K-97 \\ \midrule
1010 & WLP320 & OYL-002 & n/a & GY020 & n/a \\ \midrule
1028 & WLP013 & OYL-003 & n/a & n/a & Lallemand Windstor British-Style Beer \\ \midrule
1056 & WLP001 & OYL-004	& A07 & GY001 & Fermentis SafAle US-05 / Lallemand BRY-97 American West Coast Ale \\ \midrule
1084 & WLP004 & OYL-005 & A10 & GY080 & n/a \\ \midrule
1098 & WLP007 & OYL-006 & A01 & GY031 & Fermentis SafAle S-04 \\ \midrule
1099 & n/a & OYL-007 & n/a & n/a & n/a \\ \midrule
1214 & WLP500 & OYL-018 & B63 & GY014 & Fermentis SafAle T-58 / Mangrove Jack's M31 Belgian Tripel \\ \midrule
1272 & WLP051 & OYL-009 & A15 & n/a & n/a \\ \midrule
1275 & WLP023 & OYL-010 & n/a & n/a & Lallemand Windstor British-Style Beer \\ \midrule
1318 & n/a & OYL-011 & n/a & n/a & n/a \\ \midrule
1335 & WLP025 & n/a & n/a & n/a & n/a \\ \midrule
1388 & WLP570 & OYL-019 & n/a & GY048 & Mangrove Jack's M31 Belgian Tripel \\ \midrule
1450 & n/a & OYL-041 & n/a & n/a & n/a \\ \midrule
1469 & n/a & OYL-014 & n/a & n/a & n/a \\ \midrule
1728 & WLP028 & OYL-015 & A31 & GY044 & n/a \\ \midrule
1764-PC & n/a & n/a & n/a & n/a & Mangrove Jack's M44 US West Coast \\ \midrule
1968 & WLP002 & OYL-016 & A09 & n/a & Lallemand London ESB English-Style Ale / Mangrove Jack's M15 Empire Ale \\ \midrule
2001 & WLP800 & OYL-101 & n/a & GY002 & n/a \\ \midrule
2112 & WLP810 & OYL-105 & L05 & GY005 & Mangrove Jack's M54 Californian Lager \\ \midrule
2124 & WLP830 & OYL-106 & L13 & GY045 & Fermentis SafLager W-34/70 / Mangrove Jack's M84 Bohemian Lager \\ \midrule
2206 & WLP820 & OYL-107 & n/a & n/a & Fermentis SafLager S-23 \\ \midrule
2247-PC	& WLP920 & n/a & n/a & n/a & n/a \\ \midrule
2278 & n/a & OYL-108 & n/a & n/a & n/a \\ \midrule
2308 & WLP838 & OYL-109 & n/a & n/a & n/a \\ \midrule
2487-PC & WLP833 & n/a & n/a & n/a & n/a \\ \midrule
2565 & WLP029 & OYL-017 & G03 & GY021 & Fermentis SafAle K-97 \\ \midrule
3068 & WLP300 & OYL-021 & G01 & GY017 & Fermentis SafAle WB-06 \\ \midrule
3333-PC & WLP380 & OYL-022 & n/a & n/a & Fermentis SafAle WB-06 \\ \midrule
3522 & WLP550 & n/a & B45 & n/a & n/a \\ \midrule
3724 & WLP565 & OYL-027 & n/a & GY027 & Fermentis SafAle T-58 / Lallemand Belle Saison Belgian
Saison-Style \\ \midrule
3787 & WLP530 & n/a & B48 & n/a & n/a \\ \midrule
3944 & WLP400 & OYL-030 & B44 & n/a & Mangrove Jack's M21 Belgian Wit \\ \midrule
n/a & WLP039 & n/a & n/a & n/a & Lallemand Nottingham High Performance Ale \\ \midrule
n/a & WLP095 & n/a & A04 & GY054 & n/a \\ \midrule
\end{tabu}

\small
Sources: Kristen England's "Yeast Strain Comparison Chart", Salt City Brew Suplly's
"Yeast Comparison Chart" and Jan Brückelmaier's "Bier brauen: Grundlagen,
Rohstoffe, Brauprozess".
\normalsize

\end{recipeblock}


\backmatter
\clearpage
\pdfbookmark[part]{Recipe Index}{index}
\printindex

\end{document}
