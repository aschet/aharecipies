\part{\styleamericanipa}

\chapter*{Bell's Two Hearted Ale Clone}

\begin{aboutblock}
Bell's Brewery of Kalamazoo, Mich. brews a little beer called Two Hearted Ale. Maybe you've
heard of it? This India pale ale is bursting with hop aromas ranging from pine to grapefruit
thanks to the use of 100 percent Centennial hops. This recipe was created by David Curtis and
Ryan Kramer of Bell's General Store.
\end{aboutblock}

\specifications{\styleamericanipa}{\galtol{5}}{1.063}{1.012}{6.7~\%}{55}{\srmtoebc{10}}{75~min}

\begin{methodandtiming}
 
\begin{mashsteps}
\mashstep{\ftoc{150}}{45~min}
\mashstep{\ftoc{170}}{raise to over 15~min}
\mashstep{\ftoc{170}}{10~min}
\end{mashsteps}

\begin{directions}
Use carbon filtered water, adjust with 4 g gypsum. Ferment warm (ale temperature).
Dry hop one week into fermentation. Allow two hearted clone to stay warm with hops for a week.
Rack beer, crash cool, and cold age for a week.
\end{directions}

\end{methodandtiming}

\pagebreak

\begin{ingredientsblock}

\begin{malts}
\malt{Briess Brewers}{\lbtokg{10}}
\malt{Briess Pale Ale}{\lbtokg{2.83}}
\malt{Briess Caramel 40 L}{\oztokg{8}}
\end{malts}

\begin{hops}
\hop{\hopcentennial}{9.1~\%}{45~min}{\oztog{1.2}}
\hop{\hopcentennial}{9.1~\%}{30~min}{\oztog{1.2}}
\hop{\hopcentennial}{9.1~\%}{\dryh{}{}}{\oztog{3.5}}
\end{hops}

\begin{yeasts}
\yeast{White Labs WLP001 / White Labs WLP051}
\end{yeasts}

\end{ingredientsblock}

\chapter*{Bissel Brothers Brewing The Substance New England IPA}

\begin{aboutblock}
The Substance from Bissel Brothers Brewing Co. flirts with the new world IPA style in a way
that intrigues and compels, adding complexity and not detracting from the beer. It does have
notes of tropical citrus, but it is still first and foremost "dank" with a perceived bitterness
that contributes to an overall balanced experience.
\end{aboutblock}

\specifications{\styleamericanipa}{\galtol{5.16}}{1.061}{1.011}{}{}{}{}

\begin{methodandtiming}
 
\begin{mashsteps}
\mashstep{\ftoc{150}}{}
\end{mashsteps}

\begin{fermentationsteps}
\fermentationstep{\ftoc{68}}{--}
\fermentationstep{\ftoc{71}}{raise to on second day}
\end{fermentationsteps}

\end{methodandtiming}

\pagebreak

\begin{ingredientsblock}

\begin{malts}
\malt{Pale Two-row}{\lbtokg{10}}
\malt{Flaked Wheat}{\lbtokg{1.18}}
\malt{Crystal 20 L}{\oztokg{5.6}}
\malt{Flaked Oats}{\oztokg{3.8}}
\end{malts}

\begin{hops}
\hop{\hopapollo}{}{Start}{\oztog{1}}
\hop{\hopfalconersflight}{}{\foh{}}{\oztog{1}}
\hop{\hopcentennial}{}{\foh{}}{\oztog{1}}
\hop{\hopfalconersflight}{}{\dryh{}{}}{\oztog{3}}
\hop{\hopcentennial}{}{\dryh{}{}}{\oztog{1.5}}
\hop{\hopeureka}{}{\dryh{}{}}{\oztog{1}}
\hop{\hopapollo}{}{\dryh{}{}}{\oztog{1}}
\hop{\hopchinook}{}{\dryh{}{}}{\oztog{1}}

\end{hops}

\begin{yeasts}
\yeast{Wyeast 2112 / White Labs WLP810}
\end{yeasts}

\end{ingredientsblock}

\chapter*{Corridor Brewery Wizard Fight American IPA}

\begin{aboutblock}
This flagship beer from Corridor Brewery features a plethora of cool kid hops
including Mosaic, Citra, and El Dorado create a citrus and tropical paradise.
\end{aboutblock}

\specifications{\styleamericanipa}{\galtol{5}}{1.059}{1.010}{6.5~\%}{60}{}{90~min}

\begin{methodandtiming}
 
\begin{mashsteps}
\mashstep{\ftoc{152}}{60~min}
\end{mashsteps}

\begin{fermentationsteps}
\fermentationstep{\ftoc{68}}{}
\end{fermentationsteps}

\begin{directions}
Addition of 0.5 g \ce{CaSO4} and 0.5 g \ce{CaCl2} at mash in. Cold crash one
day after dry hopping.
\end{directions}

\end{methodandtiming}

\pagebreak

\begin{ingredientsblock}

\begin{malts}
\malt{Two-row}{\lbtokg{8}}
\malt{Bonlander Munich}{\oztokg{12}}
\malt{Carapils}{\oztokg{12}}
\malt{Flaked Oats}{\oztokg{4}}
\end{malts}

\begin{hops}
\hop{\hopwarrior}{16~\%}{90~min}{\oztog{0.75}}
\hop{\hopchinook}{13~\%}{15~min}{\oztog{0.5}}
\hop{\hopcitra}{13~\%}{\foh{}}{\oztog{0.64}}
\hop{\hopeldorado}{15~\%}{\foh{}}{\oztog{0.64}}
\hop{\hopmosaic}{11~\%}{\foh{}}{\oztog{0.64}}
\hop{\hopmosaic}{}{\dryh{}{5~days}}{\oztog{0.5}}
\hop{\hopcitra}{}{\dryh{}{5~days}}{\oztog{0.5}}
\hop{\hopeldorado}{}{\dryh{}{5~days}}{\oztog{0.5}}
\end{hops}

\begin{yeasts}
\yeast{Brewing Science Institute A-18}
\end{yeasts}

\end{ingredientsblock}

\chapter*{Deschutes' Fresh Squeezed IPA Clone}

\begin{aboutblock}
The name Fresh Squeezed IPA gives you an idea of what's in store when you brew this clone. First and
foremost, you should drink this beer fresh. This hop-centric IPA has big, piney hop aroma that's full
of fruit and peppery notes. It drips with juicy citrus and grapefruit flavor thanks to the Citra hops,
while the Mosaic hops present soft, fruit flavors like honeydew. A mild malt profile of pale, Munich
and crystal take a back seat to the hops, making this easy to drink IPA.
\end{aboutblock}

\specifications{\styleamericanipa}{\galtol{5.5}}{1.066}{1.014}{7~\%}{60}{\srmtoebc{10}}{90~min}

\begin{methodandtiming}
 
\begin{mashsteps}
\mashstep{\ftoc{150}}{60~min}
\end{mashsteps}

\begin{fermentationsteps}
\fermentationstep{\ftoc{70}}{}
\end{fermentationsteps}

\begin{directions}
To brew this Fresh Squeezed IPA clone, use 1 g/gal gypsum to treat distilled or reverse osmosis water. Drink it as fresh as possible (2--3 weeks after packaging) for maximum late hop character.
\end{directions}

\end{methodandtiming}

\pagebreak

\begin{ingredientsblock}

\begin{malts}
\malt{Pale Two-row}{\lbtokg{11}}
\malt{Munich}{\lbtokg{1.75}}
\malt{Crystal 75 L}{\lbtokg{0.75}}
\end{malts}

\begin{hops}
\hop{\hopnugget}{13~\%}{60~min}{\oztog{0.5}}
\hop{\hopcitra}{12~\%}{15~min}{\oztog{1}}
\hop{\hopmosaic}{12~\%}{15~min}{\oztog{1}}
\hop{Whirlfloc Tablet}{}{10~min}{1}
\hop{\hopcitra}{12~\%}{\whirl{}{10~min}}{\oztog{1}}
\hop{\hopcitra}{12~\%}{\dryh{}{5~days}}{\oztog{1}}
\hop{\hopmosaic}{12~\%}{\dryh{}{5~days}}{\oztog{1}}

\end{hops}

\begin{yeasts}
\yeast{White Labs WLP001}
\end{yeasts}

\end{ingredientsblock}

\chapter*{Fargo Brewing Company Wood Chipper IPA}

\begin{aboutblock}
This classic American IPA from Fargo Brewing Company showcases aromatic and bold
hop flavors. Horizon hops and oats provide a sleek, velvety body and balanced bitterness
while pounds per barrel of Cascade, Centennial, Chinook and Simcoe hops give this IPA
waves of citrus and pine flavors. That's one delicious beer eh? Oh yeah, you betcha!
\end{aboutblock}

\specifications{\styleamericanipa}{\galtol{5}}{1.062}{}{6.5~\%}{59}{\srmtoebc{4.5}}{60~min}

\begin{methodandtiming}
 
\begin{mashsteps}
\mashstep{\ftoc{155}}{}
\end{mashsteps}

\end{methodandtiming}

\pagebreak

\begin{ingredientsblock}

\begin{malts}
\malt{Pale}{\lbtokg{10.5}}
\malt{Weyermann Munich I}{\lbtokg{1}}
\malt{Rice Hulls}{--}
\end{malts}

\begin{hops}
\hop{\hopcentennial}{5.8~\%}{FWH}{\oztog{0.5}}
\hop{\hophorizon}{11.3~\%}{60~min}{\oztog{0.75}}
\hop{Whirlfloc Tablet}{}{15~min}{1}
\hop{\hopcascade}{7.55~\%}{10~min}{\oztog{0.75}}
\hop{\hopchinook}{11.1~\%}{10~min}{\oztog{0.5}}
\hop{Servomyces}{}{10~min}{1}
\hop{\hopsimcoe}{}{\dryh{}{}}{\oztog{1}}
\hop{\hopchinook}{}{\dryh{}{}}{\oztog{1}}
\hop{\hopcentennial}{}{\dryh{}{}}{\oztog{1}}
\hop{\hopcascade}{}{\dryh{}{}}{\oztog{1}}
\end{hops}

\begin{yeasts}
\yeast{American Ale}
\end{yeasts}

\end{ingredientsblock}

\chapter*{Focal Point (Inspired by The Alchemist's Focal Banger)}

\begin{aboutblock}
The Alchemist's Focal Banger recipe is a closely guarded secret, but this American IPA is
inspired by famous hoppy ale. This recipe was formulated by the AHA with a few tips from
John Kimmich, head brewer and owner of the Stow, Vt.-based brewery.
\end{aboutblock}

\specifications{\styleamericanipa}{\galtol{5.5}}{1.064}{1.012}{7~\%}{80}{\srmtoebc{5}}{60~min}

\begin{methodandtiming}

\begin{mashsteps}
\mashstep{\ftoc{150}}{75~min}
\end{mashsteps}

\begin{directions}
If you don't have the equipment to conduct a whirlpool at the end of the boil, simply conduct a
hop stand by steeping the final addition of Mosaic in the hot wort for 10 minutes before chilling.
Add dry hops for 3 days, prior to packaging. 
\end{directions}

\end{methodandtiming}

\pagebreak

\begin{ingredientsblock}

\begin{malts}
\malt{Pale}{\lbtokg{9}}
\malt{Pilsner}{\lbtokg{4.8}}
\end{malts}

\begin{hops}
\hop{Hop Extract}{}{60~min}{6~ml}
\hop{\hopmosaic}{12.25~\%}{10~min}{\oztog{1}}
\hop{\hopmosaic}{12.25~\%}{5~min}{\oztog{1}}
\hop{\hopmosaic}{12.25~\%}{\whirl{}{}}{\oztog{1}}
\hop{\hopcitra}{12.25~\%}{\dryh{}{3~days}}{\oztog{4}}
\end{hops}

\begin{yeasts}
\yeast{White Labs WLP095 / Omega Yeast OLY-052 / GigaYeast GY054 / The Yeast Bay WLP4000 / Imperial Yeast A04}
\end{yeasts}

\end{ingredientsblock}

\chapter*{Good Word Brewing Never Sleep New England IPA}

\begin{aboutblock}
This is one of Good Word Brewing's most popular beers, described as "juicy" without being overly
sweet thanks to Vic Secret and Citra hops. Pilsner malt dominates alongside English pale malt and
oats for a little more body and mouthfeel.
\end{aboutblock}

\specifications{\styleamericanipa}{\galtol{5}}{1.065}{1.013}{7~\%}{45}{\srmtoebc{3.8}}{90~min}

\begin{methodandtiming}
 
\begin{mashsteps}
\mashstep{\ftoc{146}}{15~min}
\mashstep{\ftoc{156}}{30~min}
\mashstep{\ftoc{168}}{10~min}
\end{mashsteps}

\begin{fermentationsteps}
\fermentationstep{\ftoc{68}}{2~days}
\fermentationstep{\ftoc{70}}{2~days}
\fermentationstep{\ftoc{72}}{--}
\end{fermentationsteps}

\begin{directions}
After boil, add whirlpool hops once wort is below \ftoc{180} to prevent isomerization of hops.
Dry hop for 3 days when final gravity is within 0.5--1.0 °P. Complete a diacetyl rest before
cold crashing. Do this by taking a \oztog{2} sample that can be capped. Place sample in
\ftoc{140} water for 20 minutes. Allow sample to come down to room temperature and test for
diacetyl by smell and taste. If still present wait another 24--48 hours and retest.
Only cold crash after the sample has passed the test. Crash at \ftoc{32} for 4--6 days.
\end{directions}

\end{methodandtiming}

\pagebreak

\begin{ingredientsblock}

\begin{malts}
\malt{Pilsner}{\lbtokg{6.4}}
\malt{Pale}{\lbtokg{3.25}}
\malt{Oat}{\lbtokg{1.5}}
\end{malts}

\begin{hops}
\hop{Dextrose}{}{}{\oztog{14}}
\hop{\hopvicsecret}{}{10~min}{\oztog{1}}
\hop{\hopvicsecret}{}{5~min}{\oztog{1.5}}
\hop{\hopcitra}{}{5~min}{\oztog{1.5}}
\hop{\hopcitra}{}{\whirl{}{}}{\oztog{5}}
\hop{\hopvicsecret}{}{\whirl{}{}}{\oztog{5}}
\hop{\hopvicsecret}{}{\dryh{}{3~days}}{\oztog{8}}
\hop{\hopcitra}{}{\dryh{}{3~days}}{\oztog{8}}
\end{hops}

\begin{yeasts}
\yeast{Unspecified}
\end{yeasts}

\end{ingredientsblock}

\chapter*{Great South Bay Brewery Massive IPA}

\begin{aboutblock}
Let's just say this Massive IPA from New York's Great South Bay Brewery doesn't skimp on
the hops! With a hefty dose of flameout hops, Massive IPA is packed full of American hop
flavor and aroma.
\end{aboutblock}

\specifications{\styleamericanipa}{\galtol{5}}{1.065}{1.011}{6.7~\%}{55}{\srmtoebc{10}}{60~min}

\begin{methodandtiming}
 
\begin{mashsteps}
\mashstep{\ftoc{151}}{}
\end{mashsteps}

\begin{directions}
Adjust water with \tsptog{1} \ce{CaSO4}. Dry hop at high krausen and carbonate to 2.5 volumes \ce{CO2}.
\end{directions}

\end{methodandtiming}

\pagebreak

\begin{ingredientsblock}

\begin{malts}
\malt{Two-row}{\lbtokg{10.5}}
\malt{Weyermann CARAAMBER}{\lbtokg{0.75}}
\malt{Crystal 30 L}{\lbtokg{0.5}}
\malt{Carapils}{\lbtokg{0.6}}
\malt{Pre-gelatinized Flaked Oats}{\lbtokg{0.5}}
\end{malts}

\begin{hops}
\hop{\hopchinook}{}{60~min}{\oztog{0.75}}
\hop{\hopcentennial}{}{1~min}{\oztog{0.5}}
\hop{\hopcascade}{}{1~min}{\oztog{1}}
\hop{\hopsimcoe}{}{1~min}{\oztog{0.5}}
\hop{\hopcascade}{}{\foh{}}{\oztog{0.25}}
\hop{\hopcentennial}{}{\foh{}}{\oztog{0.25}}
\hop{\hopsimcoe}{}{\foh{}}{\oztog{0.5}}
\hop{\hopcascade}{}{\dryh{}{}}{\oztog{0.5}}
\hop{\hopsimcoe}{}{\dryh{}{}}{\oztog{0.5}}
\end{hops}

\begin{yeasts}
\yeast{American Ale}
\end{yeasts}

\end{ingredientsblock}

\chapter*{Green Flash West Coast IPA}

\begin{aboutblock}
The plethora of hops creates a layered drinking experience of bitterness and hop-forward flavor
and aroma. The additions of Simcoe and Centennial give off tropical grapefruit and piney notes
respectively, while the Cascade hops add some floral quality to compliment the spicy citrus notes
of the Columbus and Amarillo hops.
With an assertive flavor and grapefruit bitterness held up by crystal malt, it's a beer of extremes -- overdoing things just because they wanted to. And you should to!
\end{aboutblock}

\specifications{\styleamericanipa}{\galtol{5.5}}{1.075}{1.018}{7.48~\%}{90}{\srmtoebc{7}}{90~min}

\begin{methodandtiming}
 
\begin{mashsteps}
\mashstep{\ftoc{152}}{60~min}
\mashstep{\ftoc{165}}{10~min}
\end{mashsteps}

\begin{fermentationsteps}
\fermentationstep{\ftoc{70}}{}
\end{fermentationsteps}

\begin{directions}
To brew Green Flash West Coast IPA, the water profile should be similar to Burton-Upon-Trent's,
but with roughly half the mineral content. When fermentation is complete, dry hop in primary
fermenter for seven days. Drink it fresh for maximum late hop character.
\end{directions}

\end{methodandtiming}

\pagebreak

\begin{ingredientsblock}

\begin{malts}
\malt{Pale Two-row}{\lbtokg{12.5}}
\malt{Bairds Carastan}{\lbtokg{1.25}}
\malt{Briess Dextrin}{\lbtokg{1.25}}
\end{malts}

\begin{hops}
\hop{\hopsimcoe}{13~\%}{90~min}{\oztog{0.5}}
\hop{\hopcolumbus}{14~\%}{90~min}{\oztog{0.25}}
\hop{\hopsimcoe}{13~\%}{60~min}{\oztog{0.25}}
\hop{\hopcolumbus}{14~\%}{60~min}{\oztog{0.25}}
\hop{\hopsimcoe}{13~\%}{30~min}{\oztog{0.25}}
\hop{\hopcolumbus}{14~\%}{30~min}{\oztog{0.25}}
\hop{\hopsimcoe}{13~\%}{15~min}{\oztog{0.75}}
\hop{\hopcolumbus}{14~\%}{15~min}{\oztog{0.75}}
\hop{Whirlfloc Tablet}{}{10~min}{1}
\hop{\hopsimcoe}{13~\%}{\whirl{}{10~min}}{\oztog{0.5}}
\hop{\hopcolumbus}{14~\%}{\whirl{}{10~min}}{\oztog{0.5}}
\hop{\hopcascade}{5.75~\%}{\whirl{}{10~min}}{\oztog{1}}
\hop{\hopsimcoe}{13~\%}{\dryh{}{7~days}}{\oztog{0.5}}
\hop{\hopcascade}{5.75~\%}{\dryh{}{7~days}}{\oztog{0.5}}
\hop{\hopamarillo}{10~\%}{\dryh{}{7~days}}{\oztog{0.5}}
\hop{\hopcentennial}{10.5~\%}{\dryh{}{7~days}}{\oztog{0.5}}
\end{hops}

\begin{yeasts}
\yeast{White Labs WLP001}
\end{yeasts}

\end{ingredientsblock}

\chapter*{Lazy Magnolia Brewery Southern Hops'pitality IPA}

\begin{aboutblock}
This session IPA from Lazy Magnolia Brewing in Kiln, Miss. is meant for sharing with
family and friends. It's described as having a bold citrus burst on the front end, with
hints of tropical fruits such as grapefruit, orange, and mango in the finish.
\end{aboutblock}

\specifications{\styleamericanipa}{\galtol{5.5}}{1.054}{1.010}{5.8~\%}{}{}{90~min}

\begin{methodandtiming}
 
\begin{mashsteps}
\mashstep{\ftoc{152}}{60~min}
\end{mashsteps}

\begin{fermentationsteps}
\fermentationstep{\ftoc{65}}{}
\end{fermentationsteps}

\begin{directions}
Use water with Calcium Chloride and food-grade acid added to increase hardness and reduce alkalinity
if necessary. Our water is extremely alkaline with zero hardness and calcium. Unless your water is
extremely soft, mineral additions may not be required. Different water profiles will result in
different hop flavors, so have fun with it.
\end{directions}

\end{methodandtiming}

\pagebreak

\begin{ingredientsblock}

\begin{malts}
\malt{Two-row}{\lbtokg{10}}
\malt{Carapils}{\lbtokg{0.5}}
\malt{Crystal 40 L}{\lbtokg{0.5}}
\end{malts}

\begin{hops}
\hop{\hopnugget}{}{60~min}{\oztog{0.25}}
\hop{\hopcentennial}{}{10~min}{\oztog{0.5}}
\hop{\hopcitra}{}{10~min}{\oztog{0.5}}
\hop{\hopsimcoe}{}{10~min}{\oztog{1}}
\hop{Whirlfloc Tablet}{}{10~min}{1}
\hop{\hopcentennial}{}{5~min}{\oztog{0.6}}
\hop{\hopcitra}{}{5~min}{\oztog{0.6}}
\hop{\hopsimcoe}{}{5~min}{\oztog{0.6}}
\hop{\hopcitra}{}{\whirl{}{}}{\oztog{0.75}}
\hop{\hopsummit}{}{\whirl{}{}}{\oztog{0.6}}
\hop{\hopsimcoe}{}{\whirl{}{}}{\oztog{0.6}}
\hop{\hopfalconersflight}{}{\whirl{}{}}{\oztog{0.6}}
\hop{\hopcitra}{}{\dryh{}{}}{\oztog{1.5}}
\end{hops}

\begin{yeasts}
\yeast{Fermentis SafAle US-05 / White Labs WLP001}
\end{yeasts}

\end{ingredientsblock}

\chapter*{Lupulin Brewing Straight Hash Homie IPA}

\begin{aboutblock}
This appropriately named IPA from Lupulin Brewing is made with pure lupulin powder,
paying homage to the brewery's name. The bursting tropical flavors and soft bitterness
may fool you, but no pellet or hop touched this beer. Brew your own and see what you think!
\end{aboutblock}

\specifications{\styleamericanipa}{\galtol{6}}{1.076}{1.018}{7.75~\%}{60}{\srmtoebc{5.5}}{60~min}

\begin{methodandtiming}
 
\begin{mashsteps}
\mashstep{\ftoc{152}}{}
\end{mashsteps}

\begin{fermentationsteps}
\fermentationstep{\ftoc{66}}{--}
\fermentationstep{\ftoc{72}}{raise to over 1 week}
\end{fermentationsteps}

\end{methodandtiming}

\pagebreak

\begin{ingredientsblock}

\begin{malts}
\malt{Two-row}{\lbtokg{10.5}}
\malt{Caramalt}{\lbtokg{1}}
\malt{Flaked Rye}{\lbtokg{1}}
\malt{Vienna}{\lbtokg{1}}
\end{malts}

\begin{hops}
\hop{Dextrose}{}{60~min}{\oztog{12}}
\hop{\hopcitra ~Hash}{}{\whirl{}{}}{\oztog{2}}
\hop{\hopsimcoe ~Hash}{}{\whirl{}{}}{\oztog{2}}
\hop{\hopcitra ~Hash}{}{\dryh{}{3~days}}{\oztog{3}}
\hop{\hopmosaic ~Hash}{}{\dryh{}{3~days}}{\oztog{2}}
\end{hops}

\begin{yeasts}
\yeast{Omega Yeast OYL-052}
\end{yeasts}

\end{ingredientsblock}

\chapter*{Odell IPA}

\begin{aboutblock}
Odell Brewing Co. of Fort Collins, Colorado makes one tasty IPA, and the 2007 Great American
Beer Festival judges agreed when they awarded Odell IPA with a gold medal in the
"American-style IPA" category.
\end{aboutblock}

\specifications{\styleamericanipa}{\galtol{5.5}}{1.067}{}{}{47}{\srmtoebc{7}}{90~min}

\begin{methodandtiming}
 
\begin{mashsteps}
\mashstep{\ftoc{154}}{60~min}
\mashstep{\ftoc{168}}{10~min}
\end{mashsteps}

\begin{fermentationsteps}
\fermentationstep{\ftoc{68}}{--}
\fermentationstep{\ftoc{60}}{1 week}
\end{fermentationsteps}

\begin{directions}
Use a hopback at runoff for \oztog{1} Simcoe and Chinook additions, or steep whole flowers
at flameout for 10 minutes.
\end{directions}

\end{methodandtiming}

\pagebreak

\begin{ingredientsblock}

\begin{malts}
\malt{Gambrinus ESB Pale}{\lbtokg{6}}
\malt{Pale}{\lbtokg{5}}
\malt{Vienna}{\lbtokg{2}}
\malt{Thomas Fawcett Caramalt}{\oztokg{10}}
\malt{Weyermann CARAFOAM}{\oztokg{8}}
\end{malts}

\begin{hops}
\hop{\hophorizon}{13~\%}{90~min}{\oztog{0.75}}
\hop{\hopsimcoe}{13~\%}{90~min}{\oztog{0.5}}
\hop{\hopcolumbus}{15~\%}{Hopback}{\oztog{1}}
\hop{\hopchinook}{13~\%}{Hopback}{\oztog{1}}
\hop{\hopsimcoe}{13~\%}{\dryh{}{}}{\oztog{0.5}}
\hop{\hophorizon}{13~\%}{\dryh{}{}}{\oztog{0.5}}
\hop{\hopamarillo}{13~\%}{\dryh{}{}}{\oztog{0.5}}
\hop{\hopcentennial}{13~\%}{\dryh{}{}}{\oztog{0.5}}
\end{hops}

\begin{yeasts}
\yeast{Nottingham Ale}
\end{yeasts}

\end{ingredientsblock}

\chapter*{Perfect Plain Brewing Co. Holy Spin American IPA}

\begin{aboutblock}
Dubbed the Holy Spin, the third spin of a vinyl record is known to be when the tunes
are at their best. This IPA recipe from Perfect Plain Brewing Co. was the third turn
of their brewhouse and is dry-hopped, abundantly so, with citra hops to the tune
of 3.3 lb/bbl!
\end{aboutblock}

\specifications{\styleamericanipa}{\galtol{5}}{1.061}{1.010}{6.5~\%}{43}{\srmtoebc{5}}{90~min}

\begin{methodandtiming}
 
\begin{mashsteps}
\mashstep{\ftoc{150}}{}
\end{mashsteps}

\begin{directions}
Targeting a mash pH between 5.2 and 5.4. Chill wort below \ftoc{170} or below (if possible)
before adding whirlpool hops for 15 minutes. Dry hop with \oztog{2} of Citra hops 48 hours
after the start of primary fermentation, being careful to prevent as much oxygen pickup as
possible. Once final gravity is reached, add the remaining \oztog{5} of Citra dry hops for
5 days. Crash and carbonate to 2.5 volumes of \ce{CO2}.
\end{directions}

\end{methodandtiming}

\pagebreak

\begin{ingredientsblock}

\begin{malts}
\malt{Pale Two-row}{\lbtokg{8.5}}
\malt{Unmalted White Wheat}{\lbtokg{2.1}}
\malt{Flaked Oats}{\oztokg{10}}
\end{malts}

\begin{hops}
\hop{\hopnugget}{12.5~\%}{60~min}{\oztog{1}}
\hop{\hopcitra}{}{\whirl{}{}}{\oztog{2}}
\hop{\hopcitra}{}{\dryhbt{}{}}{\oztog{2}}
\hop{\hopcitra}{}{\dryh{}{5~days}}{\oztog{5}}
\end{hops}

\begin{yeasts}
\yeast{Fermentis SafAle US-05}
\end{yeasts}

\end{ingredientsblock}

\chapter*{Providence Brewing Company Battlecow Galacticose New England IPA}

\begin{aboutblock}
This deceptively strong beer from Providence Brewing Company is brewed with Two-row pale
malt, Cara\-foam, rolled oats, milk sugar, and intensely double dry-hopped with Citra and
Mosaic giving it a decidedly dank aroma. Bursting with mango, orange and pineapple flavors,
this is a juicy milkshake New England IPA that'll have you begging for another sip.
\end{aboutblock}

\specifications{\styleamericanipa}{\galtol{5.5}}{1.071}{1.019}{8.1~\%}{70}{\srmtoebc{3.6}}{60~min}

\begin{methodandtiming}
 
\begin{mashsteps}
\mashstep{\ftoc{155}}{60~min}
\end{mashsteps}

\begin{fermentationsteps}
\fermentationstep{\ftoc{70}}{}
\end{fermentationsteps}

\begin{directions}
Turn off heat and let temperature drop to \ftoc{198}, then add the whirlpool hops and
whirlpool for 30 minutes. Two days after pitching the yeast, add the first dry hop
additions. After active fermentation has stopped, typically 5 days after pitching the
yeast, add the second round of dry hops. A day later, burp your fermenter (if possible)
with a 30 second burst of \ce{CO2} to rouse the hops. Add 2 oz Citra and 2 oz Mosaic pellet hops 8-10 days later. Just before kegging your beer, bag and add last dry hop additions to the keg or bottling bucket.
\end{directions}

\end{methodandtiming}

\pagebreak

\begin{ingredientsblock}

\begin{malts}
\malt{Two-row}{\lbtokg{10}}
\malt{Flaked Oats}{\lbtokg{2}}
\malt{Weyermann CARAFOAM}{\lbtokg{1.5}}
\malt{White Wheat}{\lbtokg{1.5}}
\end{malts}

\begin{hops}
\hop{\hopcolumbus ~Cones}{14~\%}{FWH}{\oztog{2}}
\hop{\hopcolumbus}{14~\%}{60~min}{\oztog{2}}
\hop{Lactose}{}{10~min}{\lbtokg{2}}
\hop{\hopcitra}{12~\%}{\whirl{}{30~min}}{\oztog{2}}
\hop{\hopmosaic}{12.25~\%}{\whirl{}{30~min}}{\oztog{2}}
\hop{\hopcitra}{12~\%}{\dryhbt{}{}}{\oztog{1}}
\hop{\hopmosaic}{12.25~\%}{\dryhbt{}{}}{\oztog{1}}
\hop{\hopcitra}{12~\%}{\dryh{1}{4~days}}{\oztog{1}}
\hop{\hopmosaic}{12.25~\%}{\dryh{1}{4~days}}{\oztog{1}}
\hop{\hopcitra}{12~\%}{\dryh{2}{10~days}}{\oztog{2}}
\hop{\hopmosaic}{12.25~\%}{\dryh{2}{10~days}}{\oztog{2}}
\hop{\hopcitra ~Cryo}{26~\%}{\dryh{3}{kegging}}{\oztog{1}}
\hop{\hopmosaic ~Cryo}{26~\%}{\dryh{3}{kegging}}{\oztog{1}}
\end{hops}

\begin{yeasts}
\yeast{The Yeast Bay WLP4042}
\end{yeasts}

\end{ingredientsblock}

\chapter*{Red Door Brewing Company New England IPA}

\begin{aboutblock}
This juicy and hazy India pale ale from Red Door Brewing Co. features an intense tropical
fruit and floral nose. This is a perfect choice for warm weather.
\end{aboutblock}

\specifications{\styleamericanipa}{\galtol{5}}{1.066}{1.014}{7.1~\%}{69}{\srmtoebc{4.3}}{60~min}

\begin{methodandtiming}
 
\begin{mashsteps}
\mashstep{\ftoc{154}}{60~min}
\end{mashsteps}

\begin{fermentationsteps}
\fermentationstep{\ftoc{68}}{}
\end{fermentationsteps}

\begin{directions}
Mash with a girst ratio of 2.5 qt/lb. Add the first dry hops and recirculate. After 2 days,
add the second dry hops and recirculate. Cold crash the next day to \ftoc{33}. Leave cold
crashed for 5 days.
\end{directions}

\end{methodandtiming}

\pagebreak

\begin{ingredientsblock}

\begin{malts}
\malt{Two-row}{\lbtokg{7.45}}
\malt{Malted White Wheat}{\lbtokg{3.77}}
\malt{Flaked Oats}{\lbtokg{2.18}}
\malt{Calcium Chloride}{\oztog{0.32}}
\malt{Phosphoric Acid (85\%)}{2.21~ml}
\end{malts}

\begin{hops}
\hop{Calcium Sulfate}{}{}{\oztog{0.32}}
\hop{\hopcitra}{13.7~\%}{\fwh}{\oztog{0.2}}
\hop{\hopmosaic}{10.7~\%}{\fwh}{\oztog{0.2}}
\hop{\hopeldorado}{15~\%}{\fwh}{\oztog{0.2}}
\hop{Zinc}{}{15~min}{16~ml}
\hop{\hopcitra}{13.7~\%}{\whirl{}{15~min}}{\oztog{1.3}}
\hop{\hopmosaic}{10.7~\%}{\whirl{}{15~min}}{\oztog{1.3}}
\hop{\hopeldorado}{15~\%}{\whirl{}{15~min}}{\oztog{1.3}}
\hop{\hopcitra}{13.7~\%}{\dryh{1}{2~days}}{\oztog{1.3}}
\hop{\hopmosaic}{10.7~\%}{\dryh{1}{2~days}}{\oztog{1.3}}
\hop{\hopeldorado}{15~\%}{\dryh{1}{2~days}}{\oztog{1.3}}
\hop{\hopcitra}{13.7~\%}{\dryh{2}{5~days}}{\oztog{1.3}}
\hop{\hopmosaic}{10.7~\%}{\dryh{2}{5~days}}{\oztog{1.3}}
\hop{\hopeldorado}{15~\%}{\dryh{2}{5~days}}{\oztog{1.3}}
\end{hops}

\begin{yeasts}
\yeast{Wyeast 1318}
\end{yeasts}

\end{ingredientsblock}

\chapter*{Russian River Blind Pig IPA Clone}

\begin{aboutblock}
Vinnie Cilurzo says about the groundbreaking Blind Pig IPA: "Blind Pig IPA was first brewed
in Temecula, Calif. at my first brewery, Blind Pig Brewing Company, in 1994. This version
was 92 bittering units and had very little malt with a very forward hop character. In December
1996, I left the brewery and my former business partner continued the brewery for a few years.
After the original brewery closed and I was at Russian River Brewing Company, we were able to
trademark the name and start making Blind Pig IPA again. The recipe has changed, in that we have
added a couple of new hop varieties that were not in existence when the brewery in Temecula was 
open."
\end{aboutblock}

\specifications{\styleamericanipa}{\galtol{5}}{1.057}{1.013}{6.1~\%}{62}{}{90~min}

\begin{methodandtiming}
 
\begin{mashsteps}
\mashstep{\ftoc{153}}{60~min}
\end{mashsteps}

\begin{fermentationsteps}
\fermentationstep{\ftoc{68}}{}
\end{fermentationsteps}

\end{methodandtiming}

\pagebreak

\begin{ingredientsblock}

\begin{malts}
\malt{Pale Two-row}{\lbtokg{9.8}}
\malt{Crystal 40 L}{\oztokg{6.5}}
\malt{Dextrin}{\oztokg{5}}
\end{malts}

\begin{hops}
\hop{\hopcolumbus}{16~\%}{90~min}{\oztog{0.25}}
\hop{\hopchinook}{13~\%}{90~min}{\oztog{0.5}}
\hop{\hopamarillo}{7.5~\%}{30~min}{\oztog{0.5}}
\hop{\hopcascade}{5.75~\%}{\foh{}}{\oztog{0.5}}
\hop{\hopamarillo}{7.5~\%}{\foh{}}{\oztog{0.5}}
\hop{\hopcentennial}{10.5~\%}{\foh{}}{\oztog{0.5}}
\hop{\hopsimcoe}{10.5~\%}{\foh{}}{\oztog{0.5}}
\hop{\hopcascade}{5.75~\%}{\dryh{}{10~days}}{\oztog{0.5}}
\hop{\hopamarillo}{7.5~\%}{\dryh{}{10~days}}{\oztog{0.5}}
\hop{\hopcolumbus}{16~\%}{\dryh{}{10~days}}{\oztog{0.5}}
\end{hops}

\begin{yeasts}
\yeast{White Labs WLP001 / Wyeast 1056}
\end{yeasts}

\end{ingredientsblock}

\chapter*{Spice Trade Brewing Sun Temple IPA}

\begin{aboutblock}
Recipe courtesy Jeff Tyler, Spice Trade Brewing Co., Arvada, Colo. Tyler says:
Don't use any cold-side clarifying agents! Haze means there are hop polyphenols in
solution, which promote magical, juicy, fruit-forward flavor!
\end{aboutblock}

\specifications{\styleamericanipa}{\galtol{5.5}}{1.064}{1.010}{7.1~\%}{75}{\srmtoebc{7}}{60~min}

\begin{methodandtiming}
 
\begin{mashsteps}
\mashstep{\ftoc{148}}{60~min}
\end{mashsteps}

\begin{fermentationsteps}
\fermentationstep{\ftoc{68}}{}
\end{fermentationsteps}

\begin{directions}
Minerals are important for this style. If you dabble in chemistry, shoot for 100 ppm
chloride and 200 ppm sulfate. Specific salt additions depend on the base water profile
you start with. After flameout, add whirlpool hops and stir wort for 30 minutes to
create a whirlpool and precipitate out the trub. Bottle or keg with 2.6 volumes (5.2 g/L)
of \ce{CO2}.
\end{directions}

\end{methodandtiming}

\pagebreak

\begin{ingredientsblock}

\begin{malts}
\malt{Pale}{\lbtokg{10.63}}
\malt{Weyermann CARAAMER}{\oztokg{6}}
\malt{Dingemans Special B}{\oztokg{3}}
\end{malts}

\begin{hops}
\hop{\hopmagnum}{12.3~\%}{60~min}{\oztog{0.6}}
\hop{\hopeldorado}{9~\%}{20~min}{\oztog{1.9}}
\hop{Yeast Nutrient}{}{15~min}{\tsptog{0.25}}
\hop{\hopcitra}{14.1~\%}{10~min}{\oztog{0.8}}
\hop{Whirlfloc Tablet}{}{10~min}{1}
\hop{Dextrose}{}{10~min}{\lbtokg{1}}
\hop{\hopeldorado}{9~\%}{FO}{\oztog{0.8}}
\hop{\hopsimcoe}{12.3~\%}{FO}{\oztog{0.3}}
\hop{\hopeldorado}{9~\%}{\whirl{}{30~min}}{\oztog{0.75}}
\hop{\hopsimcoe}{12.3~\%}{\whirl{}{30~min}}{\oztog{1.25}}
\hop{\hopeldorado}{9~\%}{\dryh{1}{4~days}}{\oztog{1.2}}
\hop{\hopsimcoe}{12.3~\%}{\dryh{1}{4~days}}{\oztog{0.7}}
\hop{\hopcitra}{14.1~\%}{\dryh{1}{4~days}}{\oztog{0.7}}
\hop{\hopeldorado}{9~\%}{\dryh{2}{3~days}}{\oztog{1.2}}
\hop{\hopsimcoe}{12.3~\%}{\dryh{2}{3~days}}{\oztog{0.7}}
\hop{\hopcitra}{14.1~\%}{\dryh{2}{3~days}}{\oztog{0.7}}
\hop{\hopeldorado}{9~\%}{\dryh{3}{}}{\oztog{1.2}}
\hop{\hopsimcoe}{12.3~\%}{\dryh{3}{}}{\oztog{0.7}}
\hop{\hopcitra}{14.1~\%}{\dryh{3}{}}{\oztog{0.8}}
\end{hops}

\begin{yeasts}
\yeast{GigaYeast GY054 / Inland Island Yeast Lab. INIS-003 / The Yeast Bay WLP 4000}
\end{yeasts}

\end{ingredientsblock}

\chapter*{Uinta Brewing Co. Hop Nosh Tangerine}

\begin{aboutblock}
Hop Nosh Tangerine is a play on Uinta Brewing’s (Salt Lake City, Utah) flagship IPA. This
homebrew recipe features an aromatic menagerie of tropical hops and tangerine zest and wraps
up with a crisp, bitter finish.
\end{aboutblock}

\specifications{\styleamericanipa}{\galtol{5}}{1.065}{}{6.7~\%}{}{}{60~min}

\begin{methodandtiming}
 
\begin{mashsteps}
\mashstep{\ftoc{154}}{}
\end{mashsteps}

\begin{fermentationsteps}
\fermentationstep{\ftoc{66}}{}
\end{fermentationsteps}

\end{methodandtiming}

\pagebreak

\begin{ingredientsblock}

\begin{malts}
\malt{Pale Two-row}{\lbtokg{11}}
\malt{Munich}{\lbtokg{1}}
\malt{Crystal 40 L}{\lbtokg{0.5}}
\end{malts}

\begin{hops}
\hop{\hopchinook}{}{60~min}{\oztog{1}}
\hop{\hopcascade}{}{30~min}{\oztog{1.5}}
\hop{\hopbravo}{}{5~min}{\oztog{1}}
\hop{\hopcascade}{}{5~min}{\oztog{1}}
\hop{Tangerine Concentrate}{}{1~min}{1 L}
\hop{\hopbravo}{}{\whirl{}{}}{\oztog{0.5}}
\hop{\hopcitra}{}{\whirl{}{}}{\oztog{0.5}}
\hop{\hopgalaxy}{}{\whirl{}{}}{\oztog{1}}
\hop{\hopgalaxy}{}{\dryh{}{}}{\oztog{1}}
\hop{\hopcitra}{}{\dryh{}{}}{\oztog{0.75}}
\hop{\hopchinook}{}{\dryh{}{}}{\oztog{0.75}}
\end{hops}

\begin{yeasts}
\yeast{Wyeast 1007}
\end{yeasts}

\end{ingredientsblock}

\chapter*{Von Ebert Brewing Sabrage Brut IPA}

\begin{aboutblock}
Sabrage is the technique for opening a champagne bottle with a saber, a fitting name for
this beer recipe from Von Ebert Brewing. This recipe is dry and effervescent, with minimal
bitterness and hop character that's dominated by grapefruit-heavy Citra and a touch of
resinous Chinook. You don't need a ceremonial occasion to drink this beer!
\end{aboutblock}

\specifications{\styleamericanipa}{\galtol{5.5}}{1.050}{}{6.6~\%}{10}{}{90~min}

\begin{methodandtiming}
 
\begin{mashsteps}
\mashstep{\ftoc{142}}{60~min}
\mashstep{\ftoc{168}}{--}
\end{mashsteps}

\begin{fermentationsteps}
\fermentationstep{\ftoc{65}}{}
\end{fermentationsteps}

\begin{directions}
Force carbonate or bottle condition to 3.0 volumes of \ce{CO2}. For a special version of
this recipe, try incorporating a grape varietal into the fermentation.
\end{directions}

\end{methodandtiming}

\pagebreak

\begin{ingredientsblock}

\begin{malts}
\malt{Pilsner}{\lbtokg{8}}
\malt{Flaked Rice}{\lbtokg{2}}
\end{malts}

\begin{hops}
\hop{\hopcitra}{}{Mash}{\oztog{2}}
\hop{\hopcitra}{}{\fwh}{\oztog{2}}
\hop{\hopcitra}{}{\whirl{}{}}{\oztog{2}}
\hop{\hopcitra}{}{\dryh{}{3~days}}{\oztog{6}}
\hop{\hopchinook}{}{\dryh{}{3~days}}{\oztog{2}}
\end{hops}

\begin{yeasts}
\yeast{Neutral Ale}
\end{yeasts}

\end{ingredientsblock}

\chapter*{WeldWerks Brewing Juicy Bits NEIPA}

\begin{aboutblock}
From WeldWerks: "Our version of a New England-style IPA featuring a huge citrus and tropical
fruit character from the Mosaic, Citra, and El Dorado hops, a softer, fluffier mouthfeel from
the lower attenuation, and the characteristic New England hop haze. The end result is a beer
reminiscent of citrus juice with extra pulp, thus the name Juicy Bits."
\end{aboutblock}

\specifications{\styleamericanipa}{\galtol{5}}{1.062}{1.012}{}{45}{\srmtoebc{4.5}}{90~min}

\begin{methodandtiming}
 
\begin{mashsteps}
\mashstep{\ftoc{149}}{45~min}
\end{mashsteps}

\begin{fermentationsteps}
\fermentationstep{\ftoc{67}}{}
\end{fermentationsteps}

\begin{directions}
If desired, use Epsom salt (\ce{MgSO4}) and calcium chloride for water adjustments, adding
half at mash and half at sparge, targeting about 250 ppm chloride and 80 ppm sulfate.
Add the first dry hop addition when the beer has fermented to about 2--3 °P from final gravity,
and then add the last two dry hop additions after terminal gravity has been reached.
\end{directions}

\end{methodandtiming}

\pagebreak

\begin{ingredientsblock}

\begin{malts}
\malt{Pale Two-row}{\lbtokg{4}}
\malt{Pilsner}{\lbtokg{4}}
\malt{Dextrin}{\lbtokg{1}}
\malt{Pale Wheat}{\lbtokg{1}}
\malt{Flaked Oats}{\oztokg{12}}
\malt{Flaked Wheat}{\oztokg{12}}
\malt{Wheat}{\oztokg{12}}
\end{malts}

\begin{hops}
\hop{\hopmagnum}{14~\%}{FWH}{\oztog{0.33}}
\hop{Dextrose}{}{90~min}{\oztog{6}}
\hop{\hopcitra}{12.5~\%}{\whirl{1}{10~min}}{\oztog{0.33}}
\hop{\hopeldorado}{15.7~\%}{\whirl{1}{10~min}}{\oztog{0.33}}
\hop{\hopmosaic}{13.1~\%}{\whirl{1}{10~min}}{\oztog{0.33}}
\hop{\hopcitra}{12.5~\%}{\whirl{2}{10~min}}{\oztog{0.66}}
\hop{\hopeldorado}{15.7~\%}{\whirl{2}{10~min}}{\oztog{0.66}}
\hop{\hopmosaic}{13.1~\%}{\whirl{2}{10~min}}{\oztog{0.66}}
\hop{\hopcitra}{12.5~\%}{\whirl{3}{20~min}}{\oztog{1}}
\hop{\hopeldorado}{15.7~\%}{\whirl{3}{20~min}}{\oztog{1}}
\hop{\hopmosaic}{13.1~\%}{\whirl{3}{20~min}}{\oztog{1}}
\hop{\hopcitra}{12.5~\%}{\dryhbt{}{}}{\oztog{0.5}}
\hop{\hopeldorado}{15.7~\%}{\dryhbt{}{}}{\oztog{0.5}}
\hop{\hopmosaic}{13.1~\%}{\dryhbt{}{}}{\oztog{0.5}}
\hop{\hopcitra}{12.5~\%}{\dryh{1}{3~days}}{\oztog{0.5}}
\hop{\hopeldorado}{15.7~\%}{\dryh{1}{3~days}}{\oztog{0.5}}
\hop{\hopmosaic}{13.1~\%}{\dryh{1}{3~days}}{\oztog{0.5}}
\hop{\hopcitra}{12.5~\%}{\dryh{2}{3~days}}{\oztog{1}}
\hop{\hopeldorado}{15.7~\%}{\dryh{2}{3~days}}{\oztog{1}}
\hop{\hopmosaic}{13.1~\%}{\dryh{2}{3~days}}{\oztog{1}}
\end{hops}

\end{ingredientsblock}

\pagebreak

\begin{ingredientsblock}

\begin{yeasts}
\yeast{Wyeast 1318}
\end{yeasts}

\end{ingredientsblock}


\chapter*{Whetstone Station Brewery Whetstoner Session IPA}

\begin{aboutblock}
This bright and delicious session IPA from Whetstone Craft Beers features Simcoe,
Amarillo and Citra hops. While it's hazy, aromatic and full of flavor, at just 4.5~\% ABV
this crisp beer is perfect for when you've got thirst that needs quenching.
\end{aboutblock}

\specifications{\styleamericanipa}{\galtol{5}}{1.045}{1.010}{4.6~\%}{36}{\srmtoebc{4.8}}{90~min}

\begin{methodandtiming}
 
\begin{mashsteps}
\mashstep{\ftoc{150}}{}
\mashstep{\ftoc{168}}{}
\end{mashsteps}

\begin{fermentationsteps}
\fermentationstep{\ftoc{65}}{}
\end{fermentationsteps}

\begin{directions}
Carbonate to 2.3 volumes of \ce{CO2}.
\end{directions}

\end{methodandtiming}

\pagebreak

\begin{ingredientsblock}

\begin{malts}
\malt{Pale Two-row}{\lbtokg{7}}
\malt{White Wheat}{\lbtokg{2}}
\malt{Crystal 30 L}{\lbtokg{0.5}}
\end{malts}

\begin{hops}
\hop{\hopamarillo}{9.2~\%}{\whirl{}{}}{\oztog{2.25}}
\hop{\hopsimcoe}{13~\%}{\whirl{}{}}{\oztog{1.5}}
\hop{\hopamarillo}{9.2~\%}{\dryh{}{2~days}}{\oztog{0.75}}
\hop{\hopcitra}{13.4~\%}{\dryh{}{2~days}}{\oztog{0.75}}
\hop{\hopsimcoe}{13~\%}{\dryh{}{2~days}}{\oztog{0.75}}
\end{hops}

\begin{yeasts}
\yeast{American Ale}
\end{yeasts}

\end{ingredientsblock}

\chapter*{Zipline Brewing Co. NZ IPA}

\begin{aboutblock}
Zipline's India pale ale is brewed in Lincoln, Neb. and is packed with New Zealand hop
varieties like Pacific Jade, Rakau, and Wakatu, known for their exotic fruity flavors and
aromas.
\end{aboutblock}

\specifications{\styleamericanipa}{\galtol{5.25}}{1.062}{}{}{}{}{60~min}

\begin{methodandtiming}
 
\begin{mashsteps}
\mashstep{\ftoc{152}}{}
\end{mashsteps}

\begin{directions}
Target mash pH is 5.3. Add dry hops after 9 days in primary fermentation.
\end{directions}

\end{methodandtiming}

\pagebreak

\begin{ingredientsblock}

\begin{malts}
\malt{Briess Pilsner}{\lbtokg{9.15}}
\malt{Weyermann Vienna}{\lbtokg{0.9}}
\malt{Weyermann CARAMUNICH II}{\lbtokg{0.65}}
\malt{Briess Victory}{\lbtokg{0.48}}
\malt{Flaked White Wheat}{\lbtokg{0.8}}
\end{malts}

\begin{hops}
\hop{\hoppacificjade}{}{60~min}{11~g}
\hop{\hoppacificgem}{}{30~min}{11~g}
\hop{\hoprakau}{}{30~min}{11~g}
\hop{\hopwakatu}{}{10~min}{10~g}
\hop{\hoprakau}{}{10~min}{11~g}
\hop{\hopmotueka}{}{10~min}{10~g}
\hop{\hoppacifica}{}{10~min}{11~g}
\hop{\hoppacificgem}{}{\whirl{}{}}{14~g}
\hop{\hopwakatu}{}{\whirl{}{}}{14~g}
\hop{\hoprakau}{}{\whirl{}{}}{14~g}
\hop{\hopmotueka}{}{\whirl{}{}}{14~g}
\hop{\hoppacifica}{}{\whirl{}{}}{14~g}
\hop{\hoppacificgem}{}{\dryh{}{}}{14~g}
\hop{\hopwakatu}{}{\dryh{}{}}{14~g}
\hop{\hoprakau}{}{\dryh{}{}}{14~g}
\hop{\hopmotueka}{}{\dryh{}{}}{14~g}
\hop{\hoppacifica}{}{\dryh{}{}}{14~g}
\end{hops}

\begin{yeasts}
\yeast{IPA}
\end{yeasts}

\end{ingredientsblock}

