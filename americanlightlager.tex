\stylesection{\styleamericanlightlager}

% -----------------------------------------------------------------------------
\begin{recipe}{"Barstool Blues" American Lite Lager} % rechecked
% -----------------------------------------------------------------------------

\begin{aboutblock}
Recipe by Derrick Flippin of Rural Hall, NC. Gold medal in Category 1: Light Lager
during the 2016 National Homebrew Competition in Baltimore, MD.
\sourceaha
\end{aboutblock}

\specifications{\styleamericanlightlager}{\galtol{5.5}}{1.047}{1.010}{4.86~\%}{}{}{60~min}{3}

\begin{methodandtiming}

\begin{mashsteps}
\mashstep{\ftoc{148}}{90~min}
\mashstep{\ftoc{168}}{10~min}

\end{mashsteps}

\begin{fermentationsteps}
\fermentationstep{\ftoc{48}}{21~days}
\fermentationstep{\ftoc{33}}{Transfer to secondary; 14~days}
\end{fermentationsteps}

\end{methodandtiming}

\recipebreak

\begin{ingredientsblock}

\begin{malts}
\malt{Six-row}{\lbtokg{8.56}}
\malt{Flaked Maize}{\lbtokg{2.12}}
\end{malts}

\begin{hops}
\hop{\hopmthood}{5.5~\%}{60~min}{\oztog{0.5}}
\hop{\hopmthood}{5.5~\%}{5~min}{\oztog{0.5}}
\end{hops}

\singleyeast{Wyeast 2042-PC}

\end{ingredientsblock}

\end{recipe}

% -----------------------------------------------------------------------------
\begin{recipe}{Joker's \#1 Light Lager} % rechecked
% -----------------------------------------------------------------------------

\begin{aboutblock}
Recipe by Brooks Conley of Stevenson Ranch, CA. Gold medal in Category 1: Pale
American Beer with an American Light Lager during the 2019 National Homebrew
Competition in Providence, RI.
\sourceaha
\end{aboutblock}

\specifications{\styleamericanlightlager}{\galtol{12}}{1.035}{1.003}{4.2~\%}{12}{\srmtoebc{2}}{90~min}{2.8}

\begin{methodandtiming}

\begin{mashsteps}
\mashstep{\ftoc{149}}{90~min}
\mashstep{\ftoc{168}}{Mash out}
\end{mashsteps}

\begin{fermentationsteps}
\fermentationstep{\ftoc{51}}{10~days}
\fermentationstep{\ftoc{62}}{Free raise; 5~days}
\fermentationstep{\ftoc{32}}{Slow reduction over 7~days; 21~days}
\end{fermentationsteps}

\begin{directions}
Water adjustment: \waterprofile{44}{5}{10}{80}{50}{}, mash pH of 5.2.
\end{directions}

\end{methodandtiming}

\recipebreak

\begin{ingredientsblock}

\begin{malts}
\malt{Great Western Pure California}{\lbtokg{10.5}}
\malt{Flaked Rice}{\lbtokg{2.8}}
\end{malts}

\begin{hops}
\hop{\hophersbrucker}{4~\%}{60~min}{\oztog{1.3}}
\hop{\hopsaaz}{3.5~\%}{30~min}{\oztog{1}}
\end{hops}

\singleyeast{White Labs WLP840}

\end{ingredientsblock}

\end{recipe}

% -----------------------------------------------------------------------------
\begin{recipe}{Kein Bier Light Lager} % rechecked
% -----------------------------------------------------------------------------

\begin{aboutblock}
Recipe by Gregory Strawser and Ryan Metzger of Rochester, NY. Silver medal in
Category 1: Pale American Lager during the 2018 National Homebrew Competition in
Portland, OR.
\sourceaha
\end{aboutblock}

\specifications{\styleamericanlightlager}{\galtol{5}}{1.034}{1.005}{3.8~\%}{13}{\srmtoebc{3}}{60~min}{}

\begin{methodandtiming}

\begin{mashsteps}
\mashstep{\ftoc{148}}{60~min}
\mashstep{\ftoc{168}}{Mash out}
\end{mashsteps}

\begin{fermentationsteps}
\fermentationstep{\ftoc{50}}{10~days}
\fermentationstep{\ftoc{65}}{Slow raise; 3~days}
\end{fermentationsteps}

\begin{directions}
Cold crash for 2 days.
\end{directions}

\end{methodandtiming}

\recipebreak

\begin{ingredientsblock}

\begin{malts}
\malt{Pale}{\lbtokg{6}}
\malt{Flaked Rice}{\lbtokg{1}}
\malt{Weyermann CARAHELL}{\lbtokg{0.25}}
\end{malts}

\begin{hops}
\hop{\hophallertaumittelfruh}{5~\%}{60~min}{\oztog{0.75}}
\end{hops}

\singleyeast{White Labs WLP840 / White Labs WLP838}

\end{ingredientsblock}

% -----------------------------------------------------------------------------
\begin{recipe}{Pisswasser Lite} % rechecked
% -----------------------------------------------------------------------------

\begin{aboutblock}
Recipe by Marie-Annick Scott. Award-winning recipe from the 2017 ALES Open-Canada's
largest homebrew competition in Regina, SK. This recipe is brewed using a process
similar to that used by large breweries.
\sourcezymurgy{July / August 2017}
\end{aboutblock}

\specifications{\styleamericanlightlager}{\galtol{11}}{1.070}{1.011}{4~\%}{6}{\srmtoebc{2}}{60~min}{2.5}

\begin{methodandtiming}

\begin{mashsteps}
\mashstep{\ftoc{100}}{20~min}
\mashstep{\ftoc{122}}{30~min}
\mashdecoctthick{}
\mashstep{\ftoc{156}}{15~min}
\mashdecoctreturn{\ftoc{146}}{120~min}
\end{mashsteps}

\begin{fermentationsteps}
\fermentationstep{\ftoc{50}}{Fermentation slowdown}
\fermentationstep{\ftoc{65}}{Diacetyl Rest}
\fermentationstep{\ftoc{14}}{Cold crash; 12~hours}
\end{fermentationsteps}

\begin{directions}
Water adjustment: perform a Munich decarbonation. This recipe is brewed for \galtol{5.5}
and then diluted with reverse osmosis water to \galtol{11}. For the decoction separate
mash into thick and thin parts and perform no boil. Add fining agents during cold crash
after yeast has flocculated and chill haze has formed.
\end{directions}

\end{methodandtiming}

\recipebreak

\begin{ingredientsblock}

\begin{malts}
\malt{Six-row}{\lbtokg{8}}
\malt{Flaked Rice}{\lbtokg{6.8}}
\end{malts}

\begin{hops}
\hop{\hopwillamette}{5.5~\%}{60~min}{\oztog{0.5}}
\hop{\hopcascade}{5.5~\%}{60~min}{\oztog{0.13}}
\end{hops}

\singleyeast{Wyeast 2035-PC}

\end{ingredientsblock}

\end{recipe}
