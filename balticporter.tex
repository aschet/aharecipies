\stylesection{\stylebalticporter}

\begin{recipie}{Continental-Style Baltic Porter}

\begin{aboutblock}
In the mid-1800s, Baltic porter as a style relieved a formative transformation:
lager yeast. Cold shipping of the first strong porter exports had mellowed the
English-brewed porters, but beers brewed in the Baltic countries needed yeast
adapted to ferment cool, not just condition cold. Lager yeast therefore became
the Baltic brewery standard; ale yeasts were unsuitable, and strong porters
lost much of their ale yeast-derived ester and phenols, gaining signature
lager smoothness.
\end{aboutblock}

\specifications{\stylebalticporter}{\galtol{5.5}}{1.077}{1.012}{8.6~\%}{29}{\srmtoebc{29}}{120~min}

\begin{methodandtiming}
 
\begin{mashsteps}
\mashstep{\ftoc{148}}{60~min}
\mashstep{\ftoc{168}}{Mashout}
\end{mashsteps}

\begin{fermentationsteps}
\fermentationstep{\ftoc{48}}{36~hours / fermentation start}
\fermentationstep{\ftoc{50}}{2~weeks}
\fermentationstep{\ftoc{55}}{Free-raise to}
\fermentationstep{\ftoc{55}}{Until fermentation slowndown}
\fermentationstep{\ftoc{60}}{7--14~days until final gravity}
\end{fermentationsteps}

\begin{directions}
Cold condition for at least 1 month at \ftoc{35} before packaging, although
3 months is better.
\end{directions}

\end{methodandtiming}

\begin{ingredientsblock}

\begin{malts}
\malt{Pilsner}{\lbtokg{5}}
\malt{Munich}{\lbtokg{4.25}}
\malt{Vienna}{\lbtokg{4}}
\malt{Weyermann CARAMUNICH I}{\lbtokg{1}}
\malt{Dingemans Special B}{\oztokg{8}}
\malt{Briess Extra Special}{\oztokg{8}}
\malt{Weyermann CARAFA II}{\oztokg{0.5}}
\end{malts}

\begin{hops}
\hop{\hopmarynka}{10.5~\%}{60~min}{\oztog{0.5}}
\hop{\hoplubelska}{5~\%}{60~min}{\oztog{2}}
\hop{Whirlfloc Tablet}{}{15~min}{1}
\end{hops}

\begin{yeasts}
\yeast{White Labs WLP802}
\end{yeasts}

\end{ingredientsblock}

\end{recipie}