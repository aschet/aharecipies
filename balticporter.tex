\stylesection{\stylebalticporter}

% -----------------------------------------------------------------------------
\begin{recipie}{Continental-Style Baltic Porter}
% -----------------------------------------------------------------------------

\begin{aboutblock}
In the mid-1800s, Baltic porter as a style relieved a formative transformation:
lager yeast. Cold shipping of the first strong porter exports had mellowed the
English-brewed porters, but beers brewed in the Baltic countries needed yeast
adapted to ferment cool, not just condition cold. Lager yeast therefore became
the Baltic brewery standard; ale yeasts were unsuitable, and strong porters
lost much of their ale yeast-derived ester and phenols, gaining signature
lager smoothness. \sourcezymurgy{September/October 2017}
\end{aboutblock}

\specifications{\stylebalticporter}{\galtol{5.5}}{1.077}{1.012}{8.6~\%}{29}{\srmtoebc{29}}{120~min}{}

\begin{methodandtiming}
 
\begin{mashsteps}
\mashstep{\ftoc{148}}{60~min}
\mashstep{\ftoc{168}}{Mashout}
\end{mashsteps}

\begin{fermentationsteps}
\fermentationstep{\ftoc{48}}{36~hours / fermentation start}
\fermentationstep{\ftoc{50}}{2~weeks}
\fermentationstep{\ftoc{55}}{Free raise to; until fermentation slowndown}
\fermentationstep{\ftoc{60}}{1--2~weeks until fully attenuated}
\end{fermentationsteps}

\begin{directions}
Cold condition for at least 1 month at \ftoc{35} before packaging, although
3 months is better.
\end{directions}

\end{methodandtiming}

\pagebreak

\begin{ingredientsblock}

\begin{malts}
\malt{Pilsner}{\lbtokg{5}}
\malt{Munich}{\lbtokg{4.25}}
\malt{Vienna}{\lbtokg{4}}
\malt{Weyermann CARAMUNICH I}{\lbtokg{1}}
\malt{Dingemans Special B}{\oztokg{8}}
\malt{Briess Extra Special}{\oztokg{8}}
\malt{Weyermann CARAFA II}{\oztokg{0.5}}
\end{malts}

\begin{hops}
\hop{\hopmarynka}{10.5~\%}{60~min}{\oztog{0.5}}
\hop{\hoplubelska}{5~\%}{60~min}{\oztog{2}}
\hop{Whirlfloc Tablet}{}{15~min}{1}
\end{hops}

\begin{yeasts}
\yeast{White Labs WLP802}
\end{yeasts}

\end{ingredientsblock}

% -----------------------------------------------------------------------------
\begin{recipie}{Odin's Beard Baltic Porter}
% -----------------------------------------------------------------------------

\begin{aboutblock}
Justin McClenahan of Silver Spring, MD, member of the The Brewing Network, won a
gold medal in Category \#6: Strong European Lager with a Baltic Porter during the
2019 National Homebrew Competition Final Round in Providence, RI. McClenahan's
Strong European Lager was chosen as the best among 269 entries in the category.
\end{aboutblock}

\specifications{\stylebalticporter}{\galtol{12}}{1.086}{1.016}{8.7~\%}{43}{\srmtoebc{30}}{90~min}{2.6}

\begin{methodandtiming}
 
\begin{mashsteps}
\mashstep{\ftoc{156}}{60~min}
\mashstep{\ftoc{168}}{Mashout}
\end{mashsteps}

\begin{fermentationsteps}
\fermentationstep{\ftoc{54}}{--}
\fermentationstep{\ftoc{64}}{Diacetyl rest, 3~days}
\fermentationstep{\ftoc{38}}{Slowly reduce to \ftoc{38}}
\end{fermentationsteps}

\begin{directions}
Water adjustment: \ce{Ca} 50~ppm, \ce{Mg} 10~ppm, \ce{Na} 33~ppm, \ce{SO4} 57~ppm,
\ce{Cl} 44~ppm, \ce{HCO3} 142~ppm. Tansfer to secondary after diacetyl rest and
dump yeast and slowly reduce temperature to \ftoc{38}. Lager for 30 days.
\end{directions}

\end{methodandtiming}

\pagebreak

\begin{ingredientsblock}

\begin{malts}
\malt{Swaen Pilsner}{\lbtokg{25}}
\malt{BEST Munich Dark}{\lbtokg{6}}
\malt{Simpsons Golden Naked Oats}{\lbtokg{2.5}}
\malt{Pale Chocolate}{\lbtokg{1.5}}
\malt{Weyermann CARAMUNICH I}{\lbtokg{1}}
\malt{Thomas Fawcett Roasted Barley}{\oztog{12}}
\end{malts}

\begin{hops}
\hop{\hopmagnum}{14~\%}{\fwh}{\oztog{1.5}}
\hop{\hopmagnum}{12~\%}{40~min}{\oztog{0.5}}
\hop{Light Brown Sugar}{}{15~min}{\lbtokg{2}}
\hop{\hopsterling}{7.5~\%}{\whirl{}{20~min}}{\oztog{2}}
\end{hops}

\begin{yeasts}
\yeast{White Labs WLP833}
\yeast{White Labs WLP838 (Secondary)}
\end{yeasts}

\end{ingredientsblock}

\end{recipie}

\end{recipie}