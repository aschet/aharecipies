\stylesection{\styleblondeale}

% -----------------------------------------------------------------------------
\begin{recipe}{All-American Blonde Ale}
% -----------------------------------------------------------------------------

\begin{aboutblock}
Recipe courtesy of the Aspen Ridge Brew Crew. \sourceaha
\end{aboutblock}

\specifications{\styleblondeale}{\galtol{11}}{1.048}{1.012}{4.6~\%}{25}{\srmtoebc{3}}{60~min}{2.3}

\begin{methodandtiming}
 
\begin{mashsteps}
\mashstep{\ftoc{150}}{75~min}
\mashstep{\ftoc{168}}{Mashout}
\end{mashsteps}

\begin{fermentationsteps}
\fermentationstep{64}{}
\end{fermentationsteps}

\end{methodandtiming}

\recipebreak

\begin{ingredientsblock}

\begin{malts}
\malt{Pale}{\lbtokg{20}}
\end{malts}

\begin{hops}
\hop{\hopcascade}{5.5~\%}{60~min}{\oztog{2}}
\hop{\hopcascade}{5.5~\%}{15~min}{\oztog{1}}
\hop{Whirlfloc Tablet}{}{15~min}{2}
\hop{\hopcascade}{5.5~\%}{2~min}{\oztog{2}}
\end{hops}

\singleyeast{Wyeast 1099}

\begin{twists}
\twist{Yeast Nutrient}{Primary}{\tsptog{2}}
\end{twists}

\end{ingredientsblock}

\end{recipe}


% -----------------------------------------------------------------------------
\begin{recipe}{Trans-Atlantic Blonde Ale}
% -----------------------------------------------------------------------------

\begin{aboutblock}
Every ingredient in a beer should serve a purpose, and sometimes it only takes a
little of this and a bit of that to create a well-balanced, focused beer. Take
this easy-drinking blonde ale recipe, for example. Simplicity is the name of the
game, using just two types of malt, one type of hop, and yeast. Brewers have
options to increase the strength through a sugar addition and to experiment with
yeast character. A great recipe for newbies and pros alike! This beer recipe is
featured in Simple Homebrewing by Denny Conn and Drew Beechum. \sourceaha
\end{aboutblock}

\specifications{\styleblondeale}{\galtol{5.5}}{1.048}{1.011}{4.8~\%}{24}{\srmtoebc{4}}{60~min}{2.5}

\begin{methodandtiming}
 
\begin{mashsteps}
\mashstep{\ftoc{154}}{60~min}
\mashstep{\ftoc{168}}{Mashout}
\end{mashsteps}

\begin{fermentationsteps}
\fermentationstep{\ftoc{62}}{1~week}
\end{fermentationsteps}

\begin{directions}
Optional: transfer to secondary fermenter and age for 10--14 days at
\ftoc{65}. For 7~\% ABV \lbtokg{1.5} of dextrose could be added during
the boil.
\end{directions}

\end{methodandtiming}

\recipebreak

\begin{ingredientsblock}

\begin{malts}
\malt{Pilsner}{\lbtokg{9}}
\malt{Carapils / Dextrin}{\lbtokg{1}}
\end{malts}

\begin{hops}
\hop{\hopmagnum}{12~\%}{60~min}{\oztog{0.5}}
\hop{\hopwillamette}{5.5~\%}{\whirl{}{20~min}}{\oztog{0.5}}
\end{hops}

\singleyeast{Wyeast 1272 / Wyeast 1450 / Wyeast 1214 / White Labs WLP550}

\end{ingredientsblock}

\end{recipe}
