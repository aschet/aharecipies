\stylesection{\stylecreamale}

% -----------------------------------------------------------------------------
\begin{recipe}{Aspen Ridge Cream Ale}
% -----------------------------------------------------------------------------

\begin{aboutblock}
Recipe by Aspen Ridge Brew Crew. \sourceaha
\end{aboutblock}

\specifications{\stylecreamale}{\galtol{10}}{1.051}{1.011}{5.4~\%}{20}{\srmtoebc{4}}{60~min}{2.3}

\begin{methodandtiming}
 
\begin{mashsteps}
\mashstep{\ftoc{152}}{75~min}
\mashstep{\ftoc{168}}{Mash out}
\end{mashsteps}

\begin{fermentationsteps}
\fermentationstep{67}{}
\end{fermentationsteps}

\end{methodandtiming}

\recipebreak

\begin{ingredientsblock}
  
\begin{malts}
\malt{Pale}{\lbtokg{7}}
\malt{Pilsner}{\lbtokg{7}}
\malt{Carapils / Dextrin}{\lbtokg{2}}
\malt{Flaked Maize}{\lbtokg{2}}
\malt{Flaked Rice}{\lbtokg{1.5}}
\malt{Gambrinus Honey}{\oztog{12}}
\end{malts}

\begin{hops}
\hop{\hopliberty}{4.3~\%}{60~min}{\oztog{1}}
\hop{\hopliberty}{4.3~\%}{30~min}{\oztog{1}}
\hop{Whirlfloc Tablet}{}{15~min}{2}
\hop{\hopmthood}{6~\%}{5~min}{\oztog{2}}
\end{hops}

\singleyeast{Wyeast 2565}

\end{ingredientsblock}

\end{recipe}

% -----------------------------------------------------------------------------
\begin{recipe}{BlytheStone Light Cream Ale}
% -----------------------------------------------------------------------------

\begin{aboutblock}
Recipe by Kevin Nanzer of Sacramento, CA. Bronze medal in Category 1: Pale American
Lager during the 2018 National Homebrew Competition in Portland, OR. \sourceaha
\end{aboutblock}

\specifications{\stylecreamale}{\galtol{20}}{1.051}{1.009}{5.5~\%}{12.8}{\srmtoebc{3.6}}{90~min}{2.5}

\begin{methodandtiming}

\begin{mashsteps}
\mashstep{\ftoc{152}}{60~min}
\end{mashsteps}

\begin{fermentationsteps}
\fermentationstep{\ftoc{72}}{10~days}
\end{fermentationsteps}

\begin{directions}
Water adjustment: use carbon filtered Sacramento City tap water with 10~g
calcium sulfate, 17~g calcium chloride and 23~ml lactic acid in the mash.
Lager at \ftoc{33} for 3 weeks.
\end{directions}

\end{methodandtiming}

\recipebreak

\begin{ingredientsblock}

\begin{malts}
\malt{Briess Pilsen}{\lbtokg{24}}
\malt{Rahr Pale Ale}{\lbtokg{14}}
\malt{Flaked Rice}{\lbtokg{5}}
\malt{Flaked Oats}{\lbtokg{2}}
\malt{Flaked Maize}{\lbtokg{1}}
\malt{Melanoidin}{\oztog{4}}
\end{malts}

\begin{hops}
\hop{\hopmagnum}{11.9~\%}{60~min}{\oztog{1}}
\hop{\hophallertaumittelfruh}{3.8~\%}{20~min}{\oztog{2}}
\end{hops}

\singleyeast{Fermentis SafLager W-34/70}

\end{ingredientsblock}

\end{recipe}

% -----------------------------------------------------------------------------
\begin{recipe}{"Cream Corn" Cream Ale}
% -----------------------------------------------------------------------------

\begin{aboutblock}
Recipe by Richard Romanko of East Pittsburgh, PA. Silver medal in Category
\#1: Pale American Beer during the 2019 National Homebrew Competition
in Providence, RI. \sourceaha
\end{aboutblock}
 
\specifications{\stylecreamale}{\galtol{11}}{1.045}{1.004}{5.38~\%}{12.7}{\srmtoebc{2.9}}{60~min}{}

\begin{methodandtiming}
 
\begin{mashsteps}
\mashstep{\ftoc{145}}{75~min}
\end{mashsteps}

\begin{directions}
Use reverse osmosis water treated with 7.5~g of calcium sulfate and 4.5~g of calcium
chloride. Target mash pH of 5.35. Cold crash when fermentation is complete and
try to lager for a couple of weeks prior to serving.
\end{directions}

\end{methodandtiming}

\recipebreak

\begin{ingredientsblock}

\begin{malts}
\malt{BEST Heidelberg}{\lbtokg{7}}
\malt{Pale}{\lbtokg{6}}
\malt{Flaked Maize}{\lbtokg{4}}
\malt{Flaked Barley}{\lbtokg{1}}
\malt{Acidulated}{\oztog{5}}
\end{malts}

\begin{hops}
\hop{\hopsterling}{3.2~\%}{60~min}{\oztog{0.6}}
\hop{Whirlfloc Tablet}{}{10~min}{1}
\end{hops}

\singleyeast{White Labs WLP080}

\end{ingredientsblock}

\end{recipe}

% -----------------------------------------------------------------------------
\begin{recipe}{Lao Kang's Cream Ale}
% -----------------------------------------------------------------------------

\begin{aboutblock}
Recipe by Chris "Pacman" Ingermann of Muncie, IN. Gold medal in the 2002 National
Homebrew Competition. Slightly more elevated hop flavor than traditional cream ales,
with some fruitiness and sweet corn notes. \sourceaha
\end{aboutblock}

\specifications{\stylecreamale}{\galtol{12.5}}{1.053}{1.015}{4.9~\%}{}{}{70~min}{2.1}

\begin{methodandtiming}

\begin{mashsteps}
\mashstep{\ftoc{122}}{20~min}
\mashstep{\ftoc{134}}{20~min}
\mashstep{\ftoc{154}}{60~min}
\mashstep{\ftoc{168}}{20~min}
\end{mashsteps}

\begin{fermentationsteps}
\fermentationstep{\ftoc{65}--\ftoc{70}}{}
\end{fermentationsteps}

\end{methodandtiming}

\recipebreak

\begin{ingredientsblock}

\begin{malts}
\malt{Six-row}{\lbtokg{16}}
\malt{Flaked Maize}{\lbtokg{4}}
\malt{Gambrinus Honey}{\lbtokg{0.5}}
\end{malts}

\begin{hops}
\hop{\hopliberty ~Cones}{4.8~\%}{60~min}{\oztog{3}}
\hop{\hopliberty ~Cones}{4.8~\%}{5~min}{\oztog{1}}
\end{hops}

\singleyeast{White Labs WLP810}

\end{ingredientsblock}

\end{recipe}

% -----------------------------------------------------------------------------
\begin{recipe}{Roadhouse Brewing Co. Family Vacation Cream Ale Clone}
% -----------------------------------------------------------------------------

\begin{aboutblock}
\sourceaha
\end{aboutblock}

\specifications{\stylecreamale}{\galtol{15}}{1.044}{1.007}{4.9~\%}{11}{\srmtoebc{3}}{60~min}{2.5}

\begin{methodandtiming}

\begin{mashsteps}
\mashstep{\ftoc{152}}{60~min}
\end{mashsteps}

\begin{fermentationsteps}
\fermentationstep{\ftoc{69}}{}
\end{fermentationsteps}

\begin{directions}
Steep cooling to \ftoc{32} for 3--7 days after fermentation will reduce chill haze.
Add fining agent of your choice.
\end{directions}

\end{methodandtiming}

\recipebreak

\begin{ingredientsblock}

\begin{malts}
\malt{Pilsner}{\lbtokg{26}}
\malt{Flaked Barley}{\lbtokg{3}}
\end{malts}

\begin{hops}
\hop{\hopsaaz}{3.4~\%}{60~min}{\oztog{1}}
\hop{\hopsaaz}{3.4~\%}{30~min}{\oztog{1}}
\hop{Whirlfloc Tablet}{}{10~min}{--}
\hop{Yeast Nutrient}{}{10~min}{--}
\hop{\hopcomet}{9.5~\%}{5~min}{\oztog{1}}
\hop{\hopsaaz}{3.4~\%}{5~min}{\oztog{1}}
\end{hops}

\singleyeast{American Ale}

\end{ingredientsblock}

\end{recipe}
