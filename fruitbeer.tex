\stylesection{\stylefruitbeer}

% -----------------------------------------------------------------------------
\begin{recipie}{Sour Blonde Ale with Blackberries and Raspberries}
% -----------------------------------------------------------------------------

\begin{aboutblock}
Damien Jones of Petaluma, CA, won a silver medal in Category \#28: American Wild
Ale with a Soured Blonde Ale with raspberry, blackberry during the 2019 National
Homebrew Competition Final Round in Providence, RI. Jones's Soured Blonde Ale
with raspberry, blackberry was chosen as the runner-up entry among 334 entries in
the category. \sourceaha
\end{aboutblock}

\specifications{\stylefruitbeer}{\galtol{5.5}}{1.046}{1.010}{5.1~\%}{}{\srmtoebc{14.6}}{60~min}{2.5}

\begin{methodandtiming}
 
\begin{mashsteps}
\mashstep{\ftoc{156}}{45~min}
\end{mashsteps}

\begin{fermentationsteps}
\fermentationstep{\ftoc{60}}{Pitch}
\fermentationstep{\ftoc{75}}{Free raise to; until fully attenuated}
\end{fermentationsteps}

\begin{directions}
Adjust hopping rate based on your mixed culture hop tolerance. After airlock
activity has subsided, transfer to a purged vessel and minimize head space.
Keep airlock topped up. Condition for 4--6 months at cellar temperatures.
Transfer to purged vessel on top of fruit puree. After secondary fermentation
is complete, condition for several months on the fruit. Cold crash and transfer
to a purged keg leaving the fruit behind.
\end{directions}

\end{methodandtiming}

\pagebreak

\begin{ingredientsblock}

\begin{malts}
\malt{Pilsner}{\lbtokg{7.5}}
\malt{Wheat}{\lbtokg{1.5}}
\malt{Aromatic}{\oztog{8}}
\malt{Chateau Spelt}{\oztog{8}}
\malt{Thomas Fawcett Crystal Malt II}{\oztog{8}}
\malt{Flaked Oats}{\oztog{8}}
\malt{Weyermann CARAFA Special III}{\oztog{4}}
\end{malts}

\begin{hops}
\hop{\hopstrisselspalt}{4.0~\%}{60~min}{\oztog{0.1}}
\end{hops}

\begin{yeasts}
\yeast{Mixed Culture}
\end{yeasts}

\begin{twists}
\twist{Blackberry Puree}{Secondary}{\oztokg{48}}
\twist{Raspberry Puree}{Secondary}{\oztokg{48}}
\end{twists}

\end{ingredientsblock}

\end{recipie}
