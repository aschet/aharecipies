\stylesection{\styleoktoberfest}

% -----------------------------------------------------------------------------
\begin{recipie}{Pickelhaube Festbier}
% -----------------------------------------------------------------------------

\begin{aboutblock}
In the 1970s, Munich Breweries like Paulaner began to notice that paler
versions of their Oktoberfestbier were selling better than the classic,
full-bodied, copper-colored brew. So Paualaner initiated a return to the
more Vienna-style Oktoberfest, which was lighter in body and color, in
an an attempt to make it still taste great and yet be less filling. The
result was festbier, which was adopted in 1990 os Oktoberfest's official
beer.  Festiber provides a wonderful opportunity to fine-tune the balance
between malty flavor, clean lager character, crisp though not quite bitter
noble hop accents, and drinkability.
\end{aboutblock}

\specifications{\styleoktoberfest}{\galtol{5.5}}{1.055}{1.010}{5.9~\%}{21}{\srmtoebc{4.1}}{90~min}{}

\begin{methodandtiming}
 
\begin{mashsteps}
\mashstep{\ftoc{122}}{20~min}
\fermentationstep{\ftoc{122}}{9~l decoction with 15~min boil}
\mashstep{\ftoc{150}}{60~min}
\mashstep{\ftoc{168}}{10~min}
\end{mashsteps}

\begin{fermentationsteps}
\fermentationstep{\ftoc{48}}{Until fermentation slows}
\fermentationstep{\ftoc{55}}{3~days}
\fermentationstep{\ftoc{35}}{When gravity reaches \sgtop{1.012}}
\end{fermentationsteps}

\begin{directions}
Water adjustment: reverse osmosis water with 1~g/gal calcium cloride. Lager the beer
for two or three months.
\end{directions}

\end{methodandtiming}

\begin{ingredientsblock}

\begin{malts}
\malt{Pilsner}{\lbtokg{7}}
\malt{Vienna}{\lbtokg{4}}
\end{malts}

\begin{hops}
\hop{\hophallertaumittelfruh}{4.8~\%}{\fwh}{\oztog{1}}
\hop{\hopspalt}{4.5~\%}{10~min}{\oztog{0.5}}
\end{hops}

\begin{yeasts}
\yeast{Bavarian / Munich Lager}
\end{yeasts}

\end{ingredientsblock}

\end{recipie}
