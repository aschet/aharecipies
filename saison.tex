\stylesection{\stylesaison}

% -----------------------------------------------------------------------------
\begin{recipie}{Saison de Liège 1851}
% -----------------------------------------------------------------------------

\begin{aboutblock}
Recipe by Roel Mulder. An interpretation of the recipe given by Georges Lacambre
in his reference work Traité complet de la fabrication des bières et de la
distillation. I have replaced the 6 to 8 hours of boiling with an addition of
some dark malt. Little is known of the original yeast profile: the historical 60
to 63 percent attenuation seems poor to our eyes, but for its day, this beer was
relatively well attenuated. A saison- or Brett-like profile seems appropriate,
but to get that historical flavor, try to keep attenuation low by creating more
unfermentable sugars in the mash. This helps simulate the fairly uncontrolled
conditions that would have been found in a traditional Belgian brewery.
\end{aboutblock}

\specifications{\stylesaison}{\galtol{4.98}}{1.048}{1.018}{4~\%}{40}{\srmtoebc{8}}{60~min}{2.6}

\begin{methodandtiming}
 
\begin{mashsteps}
\mashstep{\ftoc{158}}{60~min}
\end{mashsteps}

\begin{fermentationsteps}
\fermentationstep{\ftoc{63}}{}
\end{fermentationsteps}

\begin{directions}
Filtering might be a problem, so use rice hulls or wheat husks as a filter aid. Another
option is the traditional way: push in a basket, let the wort flow in, and just spoon
it out. Condition for 4 to 6 months before bottling or kegging. This beer should be ready
to enjoy within 16 to 24 weeks after brew day.
\end{directions}

\end{methodandtiming}

\pagebreak

\begin{ingredientsblock}

\begin{malts}
\malt{Spelt}{\lbtokg{4.2}}
\malt{Unmalted Wheat}{\lbtokg{4.2}}
\malt{Weyermann CARAMUNICH III}{\oztog{14}}
\end{malts}

\begin{hops}
\hop{\hophallertaumittelfruh}{4~\%}{60~min}{\oztog{2.8}}
\end{hops}

\begin{yeasts}
\yeast{Fermentis SafAle S-33}
\end{yeasts}

\end{ingredientsblock}

\end{recipie}

% -----------------------------------------------------------------------------
\begin{recipie}{Zon Saison}
% -----------------------------------------------------------------------------

\begin{aboutblock}
Mike Vandervoort of Toronto, CAN, member of the GTA Brews, won a silver medal
in Category \#20: Saison during the 2019 National Homebrew Competition Final Round
in Providence, RI. Vandervoort’s Saison was chosen as the runner-up entry among 253
entries in the category.
\end{aboutblock}

\specifications{\stylesaison}{\galtol{5}}{1.049}{1.007}{5.5~\%}{27}{\srmtoebc{3.2}}{60~min}{}

\begin{methodandtiming}
 
\begin{mashsteps}
\mashstep{\ftoc{135}--\ftoc{140}}{Only unmalted wheat, 30--45~min}
\mashstep{\ftoc{142}}{30~min}
\mashstep{\ftoc{156}}{30~min}
\end{mashsteps}

\begin{fermentationsteps}
\fermentationstep{\ftoc{71}--\ftoc{80}}{}
\end{fermentationsteps}

\begin{directions}
Target 5.4 for pH.
\end{directions}

\end{methodandtiming}

\pagebreak

\begin{ingredientsblock}

\begin{malts}
\malt{Pilsner}{\lbtokg{5.75}}
\malt{Weyermann CARAFOAM}{\lbtokg{0.625}}
\malt{Unmalted Wheat}{\lbtokg{1.875}}
\end{malts}

\begin{hops}
\hop{\hopsaaz}{2.8~\%}{60~min}{\oztog{0.75}}
\hop{\hopcitra}{13.8~\%}{20~min}{\oztog{0.4}}
\hop{\hopmandarinabavaria}{9.6~\%}{5~min}{\oztog{0.75}}
\hop{\hopmosaic}{12.5~\%}{\whirl{}{20~min}}{\oztog{0.4}}
\end{hops}

\begin{yeasts}
\yeast{Escarpment Labs Ontario Farmhouse Ale Blend}
\end{yeasts}

\end{ingredientsblock}

\end{recipie}

