\stylesection{\stylesaison}

% -----------------------------------------------------------------------------
\begin{recipe}{Birra Corina Saison}
% -----------------------------------------------------------------------------

\begin{aboutblock}
Luis Cuellar of Croton-On-Hudson, NY, member of the Homebrewers of Western
Loudoun (HOWL), won a gold medal in Category \#20: Saison during the 2019
National Homebrew Competition Final Round in Providence, RI. Cuellar's Saison
was chosen as the best among 253 entries in the category. \sourceaha
\end{aboutblock}

\specifications{\stylesaison}{\galtol{3}}{1.043}{1.008}{4.6~\%}{31}{\srmtoebc{3.7}}{90~min}{2.5}

\begin{methodandtiming}
 
\begin{mashsteps}
\mashstep{\ftoc{156}}{60~min}
\mashstep{\ftoc{170}}{Mashout}
\end{mashsteps}

\begin{fermentationsteps}
\fermentationstep{\ftoc{64}}{5~days}
\fermentationstep{\ftoc{80}}{Free raise to; 20~days}
\end{fermentationsteps}

\begin{directions}
When using Belgian bottles carbonate to 2.5 volumes of \ce{CO2}. Keep bottles
at \ftoc{75} for two weeks, then cold condition for at least two weeks before
consumption.
\end{directions}

\end{methodandtiming}

\recipebreak

\begin{ingredientsblock}

\begin{malts}
\malt{Avangard Pilsner}{\lbtokg{3.42}}
\malt{Briess Rye}{\lbtokg{1}}
\malt{Briess Red Wheat}{\lbtokg{1}}
\end{malts}

\begin{hops}
\hop{\hopapollo}{17~\%}{60~min}{\oztog{0.28}}
\hop{\hopperle}{4.9~\%}{15~min}{\oztog{0.18}}
\hop{Irish Moss}{}{15~min}{\tsptog{0.5}}
\hop{Yeast Nutrient}{}{15~min}{\tsptog{0.5}}
\hop{\hopcalypso}{15.4~\%}{\foh{}{}}{\oztog{0.21}}
\hop{\hopkohatu}{6.4~\%}{\foh{}{}}{\oztog{0.14}}
\hop{Cracked Coriander Seeds}{}{\foh{}{}}{\oztog{0.07}}
\end{hops}

\singleyeast{White Labs WLP565}

\end{ingredientsblock}

\end{recipe}

% -----------------------------------------------------------------------------
\begin{recipe}{David Allen's Saison}
% -----------------------------------------------------------------------------

\begin{aboutblock}
David Allen of Mill Creek, WA, member of the Greater Everett Brewers League, won
a bronze medal in Category \#20: Saison during the 2019 National Homebrew
Competition Final Round in Providence, RI. Allen's standard Saison was chosen as
a top three entry among 253 entries in the category. \sourceaha
\end{aboutblock}

\specifications{\stylesaison}{\galtol{5.5}}{1.048}{1.006}{5.5~\%}{23}{\srmtoebc{8}}{45~min}{3.5}

\begin{methodandtiming}
 
\begin{mashsteps}
\mashstep{\ftoc{148}}{}
\end{mashsteps}

\begin{fermentationsteps}
\fermentationstep{\ftoc{70}}{}
\end{fermentationsteps}

\begin{directions}
Re-ferment in cork/cage Belgian bottles which will safely hold the carbonation level.
Boil a 0.5~l of water with \oztog{6} of dextrose for 10 minutes; let cool. Blend with
your beer and add \tbsptog{6} of yeast slurry harvested from the batch.
Again, you want bottles that can hold the pressure. Otherwise, carbonate to 3.5 volumes
of \ce{CO2}.
\end{directions}

\end{methodandtiming}

\recipebreak

\begin{ingredientsblock}

\begin{malts}
\malt{Dingemans Pilsen}{\lbtokg{10.25}}
\malt{Castle Chateu Pale Ale}{\lbtokg{1.5}}
\malt{Rahr White Wheat}{\lbtokg{0.5}}
\malt{Rahr Standard Six-Row}{\lbtokg{0.25}}
\end{malts}

\begin{hops}
\hop{\hopaurora}{7.5~\%}{45~min}{\oztog{0.65}}
\hop{\hoploral}{10.2~\%}{15~min}{\oztog{0.25}}
\hop{\hopceleia}{3.7~\%}{5~min}{\oztog{1}}
\hop{\hopceleia}{3.7~\%}{\foh{}{}}{\oztog{1}}
\hop{\hoploral}{10.2~\%}{\foh{}{}}{\oztog{0.25}}
\hop{Whirlfloc Tablet}{}{}{1}
\hop{Wyeast Beer Nutrient Blend}{}{}{--}
\end{hops}

\begin{yeastsx}
\yeastx{Wyeast 3724}{Primary}
\yeastx{The Yeast Bay WLP4021}{Primary}
\end{yeastsx}

\end{ingredientsblock}

\end{recipe}

% -----------------------------------------------------------------------------
\begin{recipe}{Ladyface Ale Companie La Grisette Clone}
% -----------------------------------------------------------------------------

\begin{aboutblock}
Ladyface Ale Companie brews their award-winning Belgian, French and American
ales in the Agoura Hills of California. La Grisette is a thirst-quenching
Belgian farmhouse ale akin to saison, which took silver in the 2018 Las Angeles
International Beer Competition. \sourceaha
\end{aboutblock}

\specifications{\stylesaison}{\galtol{5}}{1.051}{}{5~\%}{28}{\srmtoebc{4}}{60~min}{}

\begin{methodandtiming}
 
\begin{mashsteps}
\mashstep{\ftoc{150}}{}
\end{mashsteps}

\end{methodandtiming}

\recipebreak

\begin{ingredientsblock}

\begin{malts}
\malt{Pale}{\lbtokg{3.8}}
\malt{Pale Wheat}{\lbtokg{3.8}}
\malt{Acidulated}{\lbtokg{1.25}}
\malt{Flaked Oats}{\lbtokg{1.12}}
\end{malts}

\begin{hops}
\hop{\hopapollo}{18.5~\%}{60~min}{\oztog{0.1}}
\hop{\hopapollo}{18.5~\%}{20~min}{\oztog{0.5}}
\end{hops}

\singleyeast{Brewing Science Institute B-22}

\end{ingredientsblock}

\end{recipe}

% -----------------------------------------------------------------------------
\begin{recipe}{Saison de Liège 1851}
% -----------------------------------------------------------------------------

\begin{aboutblock}
Recipe by Roel Mulder. An interpretation of the recipe given by Georges Lacambre
in his reference work Traité complet de la fabrication des bières et de la
distillation. I have replaced the 6 to 8 hours of boiling with an addition of
some dark malt. Little is known of the original yeast profile: the historical 60
to 63 percent attenuation seems poor to our eyes, but for its day, this beer was
relatively well attenuated. A saison- or Brett-like profile seems appropriate,
but to get that historical flavor, try to keep attenuation low by creating more
unfermentable sugars in the mash. This helps simulate the fairly uncontrolled
conditions that would have been found in a traditional Belgian brewery.
\sourcezymurgy{September / October 2019}
\end{aboutblock}

\specifications{\stylesaison}{\galtol{4.98}}{1.048}{1.018}{4~\%}{40}{\srmtoebc{8}}{60~min}{2.6}

\begin{methodandtiming}
 
\begin{mashsteps}
\mashstep{\ftoc{158}}{60~min}
\end{mashsteps}

\begin{fermentationsteps}
\fermentationstep{\ftoc{63}}{}
\end{fermentationsteps}

\begin{directions}
Filtering might be a problem, so use rice hulls or wheat husks as a filter aid. Another
option is the traditional way: push in a basket, let the wort flow in, and just spoon
it out. Condition for 4 to 6 months before bottling or kegging. This beer should be ready
to enjoy within 16 to 24 weeks after brew day.
\end{directions}

\end{methodandtiming}

\recipebreak

\begin{ingredientsblock}

\begin{malts}
\malt{Spelt}{\lbtokg{4.2}}
\malt{Unmalted Wheat}{\lbtokg{4.2}}
\malt{Weyermann CARAMUNICH III}{\oztog{14}}
\end{malts}

\begin{hops}
\hop{\hophallertaumittelfruh}{4~\%}{60~min}{\oztog{2.8}}
\end{hops}

\singleyeast{Fermentis SafAle S-33}

\end{ingredientsblock}

\end{recipe}

% -----------------------------------------------------------------------------
\begin{recipe}{Zon Saison}
% -----------------------------------------------------------------------------

\begin{aboutblock}
Mike Vandervoort of Toronto, CAN, member of the GTA Brews, won a silver medal
in Category \#20: Saison during the 2019 National Homebrew Competition Final Round
in Providence, RI. Vandervoort’s Saison was chosen as the runner-up entry among 253
entries in the category. \sourceaha
\end{aboutblock}

\specifications{\stylesaison}{\galtol{5}}{1.049}{1.007}{5.5~\%}{27}{\srmtoebc{3.2}}{60~min}{}

\begin{methodandtiming}
 
\begin{mashsteps}
\mashstep{\ftoc{135}--\ftoc{140}}{Only unmalted wheat, 30--45~min}
\mashstep{\ftoc{142}}{30~min}
\mashstep{\ftoc{156}}{30~min}
\end{mashsteps}

\begin{fermentationsteps}
\fermentationstep{\ftoc{71}--\ftoc{80}}{}
\end{fermentationsteps}

\begin{directions}
Target 5.4 for pH.
\end{directions}

\end{methodandtiming}

\recipebreak

\begin{ingredientsblock}

\begin{malts}
\malt{Pilsner}{\lbtokg{5.75}}
\malt{Weyermann CARAFOAM}{\lbtokg{0.625}}
\malt{Unmalted Wheat}{\lbtokg{1.875}}
\end{malts}

\begin{hops}
\hop{\hopsaaz}{2.8~\%}{60~min}{\oztog{0.75}}
\hop{\hopcitra}{13.8~\%}{20~min}{\oztog{0.4}}
\hop{\hopmandarinabavaria}{9.6~\%}{5~min}{\oztog{0.75}}
\hop{\hopmosaic}{12.5~\%}{\whirl{}{20~min}}{\oztog{0.4}}
\end{hops}

\singleyeast{Escarpment Labs Ontario Farmhouse Ale Blend}

\end{ingredientsblock}

\end{recipe}

