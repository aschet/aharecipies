\stylesection{\styleschwarzbier}

% -----------------------------------------------------------------------------
\begin{recipie}{Red Rock Brewing Black Bier}
% -----------------------------------------------------------------------------

\begin{aboutblock}
This beer recipe was taken from the book Session Beers: Brewing for Flavor and
Balance by Jennifer Talley. This German-style dark lager from Red Rock Brewery
is a classic schwarzbier that won a gold medal at the 2010 Great American Beer
Festival. It's unusually dark color is gained from four different malts and very
light hopping that make it a sessionable black lager.
\end{aboutblock}

\specifications{\styleschwarzbier}{\galtol{5}}{1.040}{1.009}{4~\%}{28}{}{60~min}

\begin{methodandtiming}
 
\begin{mashsteps}
\mashstep{\ftoc{148}}{}
\end{mashsteps}

\begin{fermentationsteps}
\fermentationstep{\ftoc{48}}{}
\end{fermentationsteps}

\begin{directions}
Use reverse osmosis water pH with lactic acid, no calcium needed.
Age in the coldest place you have, \ftoc{33}--\ftoc{36} is optimal. Since it's
only 4~\% ABV do not cool below \ftoc{33}. Carbonate to 2.5 volumes \ce{CO2}.
\end{directions}

\end{methodandtiming}

\pagebreak

\begin{ingredientsblock}

\begin{malts}
\malt{Pilsner}{\lbtokg{7}}
\malt{Munich}{\oztokg{12}}
\malt{Dehusked Black}{\oztokg{5}}
\malt{Roast}{\oztokg{1.5}}
\end{malts}

\begin{hops}
\hop{\hoptettnang}{4~\%}{60~min}{\oztog{0.75}}
\hop{\hoptettnang}{4~\%}{30~min}{\oztog{0.75}}
\hop{\hophallertauermittelfruh}{4~\%}{\whirl{}{}}{\oztog{1.5}}
\end{hops}

\begin{yeasts}
\yeast{Wyeast 2124}
\end{yeasts}

\end{ingredientsblock}

\end{recipie}
