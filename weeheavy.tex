\stylesection{\styleweeheavy}

% -----------------------------------------------------------------------------
\begin{recipe}{Albanach Láidir}
% -----------------------------------------------------------------------------

\begin{aboutblock}
Recipe by Mark Schoppe of Austin, TX. Gold medal in Category \#9: Scottish and Irish
Ale during the 2012 National Homebrew Competition in Seattle, WA.
\sourceaha
\end{aboutblock}

\specifications{\styleweeheavy}{\galtol{6}}{1.085}{1.023}{8.14~\%}{22}{\srmtoebc{18}}{75~min}{}

\begin{methodandtiming}

\begin{mashsteps}
\mashstep{\ftoc{155}}{}
\end{mashsteps}

\begin{fermentationsteps}
\fermentationstep{\ftoc{65}}{3~weeks}
\end{fermentationsteps}

\begin{directions}
Collect \galtol{1} of first runnings, boil until condensed to a syrup and add to kettle. 
Condition for 2 weeks at room temperature.
\end{directions}

\end{methodandtiming}

\recipebreak

\begin{ingredientsblock}

\begin{malts}
\malt{Rahr Pale Ale}{\lbtokg{13.75}}
\malt{Carapils / Dextrin}{\lbtokg{2.4}}
\malt{Weyermann Beech Smoked Barley Malt}{\oztog{9.5}}
\malt{Dingemans Biscuit}{\oztog{9.5}}
\malt{Briess Victory}{\oztog{9.5}}
\malt{Caramel / Crystal 60 L}{\oztog{9.5}}
\malt{Roasted Barley}{\oztog{5}}
\end{malts}

\begin{hops}
\hop{\hopsummit}{18.5~\%}{75~min}{\oztog{0.6}}
\hop{\hopeastkentgolding}{5~\%}{15~min}{\oztog{0.25}}
\hop{\hopeastkentgolding}{5~\%}{5~min}{\oztog{0.25}}
\end{hops}

\singleyeast{White Labs WLP028}

\end{ingredientsblock}

\end{recipe}

% -----------------------------------------------------------------------------
\begin{recipe}{Russell's Scottish 80}
% -----------------------------------------------------------------------------

\begin{aboutblock}
Recipe by Russell Brunner of Tamarac, FL. Gold medal in Category \#9: Scottish
and Irish Ale during the 2013 National Homebrew Competition in Philadelphia, PA.
\sourceaha
\end{aboutblock}

\specifications{\styleweeheavy}{\galtol{7}}{1.069}{1.019}{6.7}{14.4}{}{60~min}{}

\begin{methodandtiming}

\begin{mashsteps}
\mashstep{\ftoc{158}}{60~min}
\mashstep{\ftoc{168}}{Mash out}
\end{mashsteps}

\begin{fermentationsteps}
\fermentationstep{\ftoc{67}}{2~weeks}
\end{fermentationsteps}

\begin{directions}
Water adjustment: use reverse osmosis water with 4.2~g calcium chloride.
\end{directions}

\end{methodandtiming}

\recipebreak

\begin{ingredientsblock}

\begin{malts}
\malt{Maris Otter}{\lbtokg{13.5}}
\malt{Caramel / Crystal 40 L}{\lbtokg{1.38}}
\malt{Gambrinus Honey}{\lbtokg{0.69}}
\malt{Munich}{\lbtokg{0.69}}
\malt{Caramel / Crystal 120 L}{\lbtokg{0.38}}
\malt{Pale Chocolate}{\oztog{4}}
\malt{Rice Hulls}{\lbtokg{1}}
\end{malts}

\begin{hops}
\hop{\hopeastkentgolding}{6.4~\%}{120~min}{\oztog{1}}
\end{hops}

\singleyeast{White Labs WLP001}

\end{ingredientsblock}

\end{recipe}

% -----------------------------------------------------------------------------
\begin{recipe}{Southern Highlander Wee Heavy}
% -----------------------------------------------------------------------------

\begin{aboutblock}
Recipe by Drew Beechum. Taste of a dark, nutty, caramel flavor with a hint of smoke.
Mimics a pecan pie's brown sugar syrup with the dose of maple syrup during secondary
fermentation. \sourcezymurgy{March / April 2010}.
\end{aboutblock}

\specifications{\styleweeheavy}{\galtol{5.5}}{1.084}{}{}{19}{\srmtoebc{21}}{180~min}{}

\begin{methodandtiming}

\begin{mashsteps}
\mashstep{\ftoc{155}}{60~min}
\end{mashsteps}

\begin{directions}
Take \qttol{2.5} of first runnings and boil vigorously to reduce volume
\cuptoml{3}. The result should be thick and syrupy.
\end{directions}

\end{methodandtiming}

\recipebreak

\begin{ingredientsblock}

\begin{malts}
\malt{Maris Otter}{\lbtokg{15.5}}
\malt{Weyermann CARAAROMA}{\lbtokg{0.5}}
\malt{Double Pecan-Smoked}{\lbtokg{0.5}}
\malt{Roasted Barley}{\lbtokg{0.25}}
\end{malts}

\begin{hops}
\hop{\hopmagnum}{12.9~\%}{90~min}{\oztog{0.45}}
\end{hops}

\singleyeast{Wyeast 1728}

\begin{twists}
\twist{Grade B Maple Syrup}{Secondary}{\oztoml{8}}
\end{twists}

\end{ingredientsblock}

\end{recipe}

