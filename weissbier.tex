\stylesection{\styleweissbier}

% -----------------------------------------------------------------------------
\begin{recipe}{"Barb's Hef" Weissbier}
% -----------------------------------------------------------------------------

\begin{aboutblock}
Recipe by Nick Corona of San Marcos, CA. Gold medal in Category \#18: German
Wheat and Rye Beer during the 2016 National Homebrew Competition in
Baltimore, MD. \sourceaha
\end{aboutblock}

\specifications{\styleweissbier}{\galtol{5}}{1.049}{1.010}{}{}{}{90~min}{3.5}

\begin{methodandtiming}
 
\begin{mashsteps}
\mashstep{\ftoc{115}}{10~min}
\mashstep{\ftoc{127}}{10~min}
\mashstep{\ftoc{149}}{60~min}
\mashstep{\ftoc{168}}{Mash out}
\end{mashsteps}

\begin{fermentationsteps}
\fermentationstep{\ftoc{62}}{3~days}
\fermentationstep{\ftoc{65}}{3~days}
\fermentationstep{\ftoc{68}}{5~days}
\end{fermentationsteps}

\end{methodandtiming}

\recipebreak

\begin{ingredientsblock}

\begin{malts}
\malt{Pilsner}{\lbtokg{4.25}}
\malt{Wheat}{\lbtokg{4.25}}
\malt{Rice Hulls}{\lbtokg{0.5}}
\end{malts}

\begin{hops}
\hop{\hophallertaumittelfruh}{3.75~\%}{90~min}{\oztog{0.25}}
\hop{\hophallertaumittelfruh}{3.75~\%}{60~min}{\oztog{0.75}}
\end{hops}

\singleyeast{White Labs WLP380}

\end{ingredientsblock}

\end{recipe}

% -----------------------------------------------------------------------------
\begin{recipe}{Bigfoot's (D)elight Weissbier}
% -----------------------------------------------------------------------------

\begin{aboutblock}
Recipe by Sean Manrique of San Lorenzo, CA. Gold medal in Category \#7: German
Wheat Beer during the 2019 National Homebrew Competition in Providence, RI.
\sourceaha
\end{aboutblock}

\specifications{\styleweissbier}{\galtol{11}}{1.046}{1.010}{5.7~\%}{15}{\srmtoebc{4}}{60~min}{}

\begin{methodandtiming}
 
\begin{mashsteps}
\mashstep{\ftoc{154}}{60~min}
\mashstep{\ftoc{168}}{Mash out}
\end{mashsteps}

\begin{fermentationsteps}
\fermentationstep{\ftoc{60}}{2~weeks}
\fermentationstep{\ftoc{35}}{7~days}
\end{fermentationsteps}

\end{methodandtiming}

\recipebreak

\begin{ingredientsblock}

\begin{malts}
\malt{White Wheat}{\lbtokg{13}}
\malt{Pilsner}{\lbtokg{10}}
\malt{Rice Hulls}{\lbtokg{8}}
\end{malts}

\begin{hops}
\hop{\hophallertaumittelfruh}{4.8~\%}{60~min}{\oztog{2}}
\end{hops}

\singleyeast{White Labs WLP300}

\end{ingredientsblock}

\end{recipe}

% -----------------------------------------------------------------------------
\begin{recipe}{Cane Toad Weisse}
% -----------------------------------------------------------------------------

\begin{aboutblock}
An attempt at Redback from Matilda Bay Brewing in Australia. Somewhat drier,
but with high carbonation, low hopping, and subtle hints of clove and banana.
\sourcezymurgy{March / April 2019}
\end{aboutblock}

\specifications{\styleweissbier}{\galtol{5.5}}{1.050}{1.007}{5.7~\%}{13}{\srmtoebc{3}}{60~min}{3.5}

\begin{methodandtiming}
 
\begin{mashsteps}
\mashstep{\ftoc{125}}{20~min}
\mashstep{\ftoc{140}}{30~min}
\mashstep{\ftoc{152}}{40~min}
\mashstep{\ftoc{168}}{Mash out}
\end{mashsteps}

\begin{fermentationsteps}
\fermentationstep{\ftoc{64}}{3~days}
\fermentationstep{\ftoc{68}}{Free raise to; until fully attenuated}
\end{fermentationsteps}

\begin{directions}
Water adjustment: use \gpgaltogpl{1} calcium cloride added to reverse osmosis water. 
Keep bottles at \ftoc{70} for a week or until they begin to clear. Store at cellar
temperatures for 2 weeks.
\end{directions}

\end{methodandtiming}

\recipebreak

\begin{ingredientsblock}
    
\begin{malts}
\malt{Wheat}{\lbtokg{4.5}}
\malt{Briess Brewers}{\lbtokg{4.5}}
\end{malts}

\begin{hops}
\hop{\hopsterling}{2.3~\%}{60~min}{\oztog{2}}
\hop{Raw Organic Cane Sugar}{}{--}{\oztog{12}}

\end{hops}

\singleyeast{White Labs WLP380}

\end{ingredientsblock}

\end{recipe}

\stylesection{Workbench}

% -----------------------------------------------------------------------------
\begin{recipe}{Kevin's Mom}
% -----------------------------------------------------------------------------

\begin{aboutblock}
Recipe by Chris Colby. \sourcezymurgy{January / February 2018}
\end{aboutblock}

\specifications{\styleweissbier}{\galtol{5}}{1.052}{1.012}{5.1~\%}{19}{\srmtoebc{3.9 }}{90~min}{4}

\begin{methodandtiming}
 
\begin{mashsteps}
\mashstep{\ftoc{104}}{Mash in}
\mashstep{\ftoc{113}}{5~min}
\mashstep{\ftoc{122}}{15~min}
\mashdecoctthick{with 40~\% of mash}
\mashstep{\ftoc{158}}{Until converted}
\mashdecoctboil{20~min}
\mashdecoctreturn{\ftoc{149}}{15~min}
\mashstep{\ftoc{158}}{Until converted}
\mashstep{\ftoc{168}}{Mash out}
\end{mashsteps}

\begin{fermentationsteps}
\fermentationstep{\ftoc{54}}{Pitch}
\fermentationstep{\ftoc{64}}{Free raise to}
\end{fermentationsteps}

\begin{directions}
Use \qttol{3.6} of wort for bottling. Take the lid off the fermenter for the
1 to 2 days when the fermentation is at it's most vigorous. Bottle condition
at room temperature for 2 weeks.
\end{directions}

\end{methodandtiming}

\recipebreak

\begin{ingredientsblock}

\begin{malts}
\malt{Red Wheat}{\lbtokg{6.75}}
\malt{Undermodified Pilsner}{\lbtokg{3}}
\end{malts}

\begin{hops}
\hop{\hophallertaumittelfruh}{4~\%}{60~min}{\oztog{1.3}}
\end{hops}

\singleyeast{Wyeast 3068 / White Labs WLP300}

\begin{twists}
\twist{Wort / Speise}{Bottling}{\qttol{3.6}}
\twist{Lager Yeast}{Bottling}{\oztog{1.2}}
\end{twists}

\end{ingredientsblock}

\end{recipe}

% -----------------------------------------------------------------------------
\begin{recipe}{Hackysack Superstar}
% -----------------------------------------------------------------------------

\begin{aboutblock}
Recipe by Rob Knighton of Columbia, PA. Gold medal in Category \#19: German
Wheat \& Rye Beer during the 2017 National Homebrew Competition in Minneapolis,
MN. \sourceaha
\end{aboutblock}

\specifications{\styleweissbier}{\galtol{6}}{1.048}{1.010}{}{}{}{60~min}{3}

\begin{methodandtiming}
 
\begin{mashsteps}
\mashstep{\ftoc{96}}{Mash in}
\mashstep{\ftoc{115}}{Raise to over 10~min; 10~min}
\mashstep{\ftoc{127}}{Raise to over 10~min; 10~min}
\mashstep{\ftoc{149}}{Raise to over 15~min; 45~min}
\mashstep{\ftoc{168}}{Mash out}
\end{mashsteps}

\begin{fermentationsteps}
\fermentationstep{\ftoc{62}}{7~days}
\fermentationstep{Transfer to secondary; \ftoc{62}}{3~days}
\end{fermentationsteps}

\begin{directions}
Ferment only \galtol{4.25} of wort. Use \qttol{1} for krausening. Ferment open, or
with only tin foil as cover. Apply airlock on the fourth day. Use closed fermenter
in secondary.
\end{directions}

\end{methodandtiming}

\recipebreak

\begin{ingredientsblock}

\begin{malts}
\malt{Briess Red Wheat}{\lbtokg{5}}
\malt{Weyermann Pilsner}{\lbtokg{4.5}}
\malt{Acidulated}{\oztog{6}}
\malt{Weyermann Melanoidin}{\oztog{2}}
\end{malts}

\begin{hops}
\hop{\hopmagnum}{12.4~\%}{60~min}{\oztog{0.25}}
\end{hops}

\singleyeast{White Labs WLP380}

\begin{twists}
\twist{Wort / Speise}{Secondary}{\qttol{1}}
\end{twists}

\end{ingredientsblock}

\end{recipe}

% -----------------------------------------------------------------------------
\begin{recipe}{Honolulu BeerWorks CocoWeizen Clone}
% -----------------------------------------------------------------------------

\begin{aboutblock}
\sourceaha
\end{aboutblock}

\specifications{\styleweissbier}{\galtol{5}}{1.048}{1.006}{5.5~\%}{14}{\srmtoebc{5.5}}{60~min}{}

\begin{methodandtiming}
 
\begin{mashsteps}
\mashstep{\ftoc{150}}{60~min}
\end{mashsteps}

\begin{fermentationsteps}
\fermentationstep{\ftoc{72}}{10~days}
\end{fermentationsteps}

\begin{directions}
Toast coconut until very dark, but not burnt.
\end{directions}

\end{methodandtiming}

\recipebreak

\begin{ingredientsblock}

\begin{malts}
\malt{White Wheat}{\lbtokg{8}}
\malt{Two-row}{\lbtokg{2.5}}
\end{malts}

\begin{hops}
\hop{\hopcrystal}{3.9~\%}{60~min}{\oztog{1.2}}
\end{hops}

\singleyeast{Wyeast 3068}

\begin{twists}
\twist{Toasted and Shredded Coconut}{Secondary (2~days)}{\lbtokg{1}}
\end{twists}

\end{ingredientsblock}

\end{recipe}

% -----------------------------------------------------------------------------
\begin{recipe}{Spicy Nana Weissbier}
% -----------------------------------------------------------------------------

\begin{aboutblock}
Recipe by Dennis Pike of Chapel Hill, NC. Silver medal in Category 7: German Wheat
Beer during the 2018 National Homebrew Competition in Portland, OR. \sourceaha
\end{aboutblock}

\specifications{\styleweissbier}{\galtol{8}}{1.048}{1.010}{5~\%}{13}{\srmtoebc{3.4}}{90~min}{}

\begin{methodandtiming}
 
\begin{mashsteps}
\mashstep{\ftoc{105}}{Mash in}
\mashstep{\ftoc{112}}{10~min}
\mashstep{\ftoc{126}}{10~min}
\mashstep{\ftoc{149}}{5~min}
\mashdecoctthick{with 1/3 of mash}
\mashstep{\ftoc{158}}{20~min}
\mashdecoctboil{10~min}
\mashdecoctreturn{\ftoc{158}}{10~min}
\mashstep{\ftoc{169}}{10~min}
\end{mashsteps}

\begin{fermentationsteps}
\fermentationstep{\ftoc{58}}{Pitch}
\fermentationstep{\ftoc{63}}{Raise to over 1~day; 1~week}
\fermentationstep{\ftoc{68}}{Raise to over 1~day; 2~weeks}
\end{fermentationsteps}

\begin{directions}
Water adjustment: target 100~ppm Ca and 5.4 pH. Use \qttol{2} of wort for
bottling. Bottle condition at room temperature for 2 weeks.
\end{directions}

\end{methodandtiming}

\recipebreak

\begin{ingredientsblock}

\begin{malts}
\malt{Durst Wheat}{\lbtokg{8.75}}
\malt{Durst Pilsner}{\lbtokg{5.75}}
\end{malts}

\begin{hops}
\hop{\hophallertaumittelfruh}{4.1~\%}{90~min}{\oztog{0.2}}
\hop{\hophallertaumittelfruh}{4.1~\%}{60~min}{\oztog{1}}
\hop{Wyeast Beer Nutrient Blend}{}{}{--}
\end{hops}

\singleyeast{White Labs WLP300}

\begin{twists}
\twist{Wort / Speise}{Bottling}{\qttol{2}}
\end{twists}

\end{ingredientsblock}

\end{recipe}

% -----------------------------------------------------------------------------
\begin{recipe}{Stew's Brew Hefeweizen}
% -----------------------------------------------------------------------------

\begin{aboutblock}
Recipe by Zach Gelfand. Won first prize, production competition.
\sourcezymurgy{May / June 2018}
\end{aboutblock}

\specifications{\styleweissbier}{\galtol{5.5}}{1.046}{1.010}{4.7~\%}{9}{\srmtoebc{4}}{90~min}{}

\begin{methodandtiming}
 
\begin{mashsteps}
\mashstep{\ftoc{122}}{20~min}
\mashstep{\ftoc{150}}{60~min}
\mashstep{\ftoc{168}}{Mash out}
\end{mashsteps}

\begin{fermentationsteps}
\fermentationstep{\ftoc{65}}{1~week}
\end{fermentationsteps}

\end{methodandtiming}

\recipebreak

\begin{ingredientsblock}

\begin{malts}
\malt{Weyermann Pale Wheat}{\lbtokg{4.75}}
\malt{Pilsner}{\lbtokg{3.75}}
\malt{Weyermann Melanoidin}{\oztog{8}}
\end{malts}

\begin{hops}
\hop{\hopeastkentgolding}{3.9~\%}{60~min}{\oztog{0.67}}
\end{hops}

\singleyeast{White Labs WLP300}

\end{ingredientsblock}

\end{recipe}

% -----------------------------------------------------------------------------
\begin{recipe}{The Bizarro Jerry Weissbier}
% -----------------------------------------------------------------------------

\begin{aboutblock}
Recipe by Tyler Cipriani and Blazey Heier of Longmont, CO. Bronze medal in 
Category 7: German Wheat Beer during the 2018 National Homebrew Competition
in Portland, OR. \sourceaha
\end{aboutblock}

\specifications{\styleweissbier}{\galtol{6}}{1.050}{}{}{12.3}{\srmtoebc{4.8}}{60~min}{}

\begin{methodandtiming}
 
\begin{mashsteps}
\mashstep{\ftoc{148}}{During decoction}
\mashdecoctthick{with 1/3 of mash}
\mashstep{\ftoc{160}}{15~min}
\mashdecoctboil{20~min}
\mashdecoctreturn{\ftoc{158}}{25~min}
\mashstep{\ftoc{170}}{10~min}
\end{mashsteps}

\begin{fermentationsteps}
\fermentationstep{\ftoc{56}}{Pitch}
\fermentationstep{\ftoc{64}}{Raise to}
\fermentationstep{\ftoc{68}}{Free raise to over 3~days}
\end{fermentationsteps}

\begin{directions}
Water adjustment: use carbon filtered Longmont, Colorado water with \tsptog{2} of
calcium sulfate in the mash.
\end{directions}

\end{methodandtiming}

\recipebreak

\begin{ingredientsblock}

\begin{malts}
\malt{Weyermann Pale Wheat}{\lbtokg{8}}
\malt{Root Shoot Odyssey Pilsner}{\lbtokg{4.75}}
\malt{Weyermann CARAHELL}{\oztog{12}}
\end{malts}

\begin{hops}
\hop{\hophallertaumittelfruh}{4.5~\%}{60~min}{\oztog{1}}
\hop{Wyeast Beer Nutrient Blend}{}{10~min}{\tsptog{1}}
\hop{Whirlfloc Tablet}{}{2~min}{1}
\end{hops}

\singleyeast{White Labs WLP300}

\end{ingredientsblock}

\end{recipe}

% -----------------------------------------------------------------------------
\begin{recipe}{Where's Fluffy Weissbier}
% -----------------------------------------------------------------------------

\begin{aboutblock}
Recipe by Paul Brown of Pinole, CA. Bronze medal in Category \#7: German Wheat
Beer during the 2019 National Homebrew Competition Final Round in Providence, RI. \sourceaha
\end{aboutblock}

\specifications{\styleweissbier}{\galtol{5.5}}{1.044}{1.006}{6.5~\%}{9.8}{\srmtoebc{5}}{90~min}{3}

\begin{methodandtiming}
 
\begin{mashsteps}
\mashstep{\ftoc{122}}{30~min}
\mashstep{\ftoc{148}}{30~min}
\mashstep{\ftoc{168}}{Mash out}
\end{mashsteps}

\begin{fermentationsteps}
\fermentationstep{\ftoc{64}}{5~days}
\fermentationstep{\ftoc{66}}{Until fully attenuated}
\end{fermentationsteps}

\end{methodandtiming}

\recipebreak

\begin{ingredientsblock}

\begin{malts}
\malt{Rahr Red Wheat}{\lbtokg{6}}
\malt{Weyermann Pilsner}{\lbtokg{5}}
\malt{Weyermann Melanoidin}{\oztog{12}}
\malt{Weyermann Acidulated}{\oztog{8}}
\end{malts}

\begin{hops}
\hop{\hopmagnum}{12~\%}{\fwh}{\oztog{0.3}}
\hop{Yeast Nutrient}{}{}{--}
\end{hops}

\singleyeast{White Labs WLP300}

\end{ingredientsblock}

\end{recipe}
