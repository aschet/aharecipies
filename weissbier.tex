\stylesection{\styleweissbier}

% -----------------------------------------------------------------------------
\begin{recipe}{Bigfoot's (D)elight Weissbier}
% -----------------------------------------------------------------------------

\begin{aboutblock}
Sean Manrique of San Lorenzo, CA, won a gold medal in Category \#7: German Wheat
Beer with a Weissbier during the 2019 National Homebrew Competition Final Round
in Providence, RI, with the help of Indy Montoya. Manrique and Montoya's German
Wheat Beer was chosen as the best among 200 entries in the category.
\sourceaha
\end{aboutblock}

\specifications{\styleweissbier}{\galtol{11}}{1.046}{1.010}{5.7~\%}{15}{\srmtoebc{4}}{60~min}{}

\begin{methodandtiming}
 
\begin{mashsteps}
\mashstep{\ftoc{154}}{60~min}
\mashstep{\ftoc{168}}{Mashout}
\end{mashsteps}

\begin{fermentationsteps}
\fermentationstep{\ftoc{60}}{2~weeks}
\fermentationstep{\ftoc{35}}{7~days}
\end{fermentationsteps}

\end{methodandtiming}

\recipebreak

\begin{ingredientsblock}

\begin{malts}
\malt{White Wheat}{\lbtokg{13}}
\malt{Pilsner}{\lbtokg{10}}
\malt{Rice Hulls}{\lbtokg{8}}
\end{malts}

\begin{hops}
\hop{\hophallertaumittelfruh}{4.8~\%}{60~min}{\oztog{2}}
\end{hops}

\singleyeast{White Labs WLP300}

\end{ingredientsblock}

\end{recipe}

% -----------------------------------------------------------------------------
\begin{recipe}{Cane Toad Weisse}
% -----------------------------------------------------------------------------

\begin{aboutblock}
This beer was initially conceived as an attempt at one of my first and favorite
craft-style ales from way back in the 1980s: Redback from Matilda Bay Brewing
in Australia. I have fond memories of quaffing bottles of Redback during my
undergraduate years at Cal Poly, San Luis Obispo. That beer did not contain
sugar but was instead a down-under version of Bavarian wheat ale. My recipe is
somewhat drier due to the cane sugar, but with high carbonation, low hopping,
and Bavarian wheat yeast to add subtle hints of clove and banana, it is exceptionally
refreshing. \sourcezymurgy{March / April 2019}
\end{aboutblock}

\specifications{\styleweissbier}{\galtol{5.5}}{1.050}{1.007}{5.7~\%}{13}{\srmtoebc{3}}{60~min}{3.5}

\begin{methodandtiming}
 
\begin{mashsteps}
\mashstep{\ftoc{125}}{20~min}
\mashstep{\ftoc{140}}{30~min}
\mashstep{\ftoc{152}}{40~min}
\mashstep{\ftoc{168}}{Mashout}
\end{mashsteps}

\begin{fermentationsteps}
\fermentationstep{\ftoc{64}}{3~days}
\fermentationstep{\ftoc{68}}{Free raise to; until fully attenuated}
\end{fermentationsteps}

\begin{directions}
Water adjustment: use \gpgaltogpl{1} calcium cloride added to reverse osmosis water. 
Keep bottles at \ftoc{70} for a week or until they begin to clear. Store at cellar
temperatures for 2 weeks.
\end{directions}

\end{methodandtiming}

\recipebreak

\begin{ingredientsblock}
    
\begin{malts}
\malt{Wheat}{\lbtokg{4.5}}
\malt{Briess Brewers}{\lbtokg{4.5}}
\end{malts}

\begin{hops}
\hop{\hopsterling}{2.3~\%}{60~min}{\oztog{2}}
\hop{Raw Organic Cane Sugar}{}{--}{\oztog{12}}

\end{hops}

\singleyeast{White Labs WLP380}

\end{ingredientsblock}

\end{recipe}

% -----------------------------------------------------------------------------
\begin{recipe}{Hackysack Superstar}
% -----------------------------------------------------------------------------

\begin{aboutblock}
Rob Knighton of Columbia, PA won a gold medal in Category \#19: German Wheat \& Rye
Beer during the 2017 National Homebrew Competition Final Round in Minneapolis, MN.
Knighton's German Wheat \& Rye Beer was chosen as the best among 192 entries in
the category. \sourceaha
\end{aboutblock}

\specifications{\styleweissbier}{\galtol{6}}{1.048}{1.010}{}{}{}{60~min}{3}

\begin{methodandtiming}
 
\begin{mashsteps}
\mashstep{\ftoc{96}}{Dough in}
\mashstep{\ftoc{115}}{Raise to over 10~min; 10~min}
\mashstep{\ftoc{127}}{Raise to over 10~min; 10~min}
\mashstep{\ftoc{149}}{Raise to over 15~min; 45~min}
\mashstep{\ftoc{168}}{Mashout}
\end{mashsteps}

\begin{fermentationsteps}
\fermentationstep{\ftoc{62}}{7~days}
\fermentationstep{Transfer to secondary; \ftoc{62}}{3~days}
\end{fermentationsteps}

\begin{directions}
Ferment only \galtol{4.25} of wort. Use \qttol{1} for krausening. Ferment open, or
with only tin foil as cover. Apply airlock on the fourth day. Use closed fermenter
in secondary.
\end{directions}

\end{methodandtiming}

\recipebreak

\begin{ingredientsblock}

\begin{malts}
\malt{Briess Red Wheat}{\lbtokg{5}}
\malt{Weyermann Pilsner}{\lbtokg{4.5}}
\malt{Acidulated}{\oztog{6}}
\malt{Weyermann Melanoidin}{\oztog{2}}
\end{malts}

\begin{hops}
\hop{\hopmagnum}{12.4~\%}{60~min}{\oztog{0.25}}
\end{hops}

\singleyeast{White Labs WLP380}

\begin{twists}
\twist{Wort / Speise}{Secondary}{\qttol{1}}
\end{twists}

\end{ingredientsblock}

\end{recipe}

% -----------------------------------------------------------------------------
\begin{recipe}{Honolulu BeerWorks CocoWeizen Clone}
% -----------------------------------------------------------------------------

\begin{aboutblock}
Instead of a commonly thought of coconut dark beer, this German-style coconut
hefeweizen is a lighter-bodied beer inspired by one Honolulu Beer Works brewer's wife.
Honolulu BeerWorks CocoWeizen took home a gold medal at Japan's International
Beer Cup 2018. \sourceaha
\end{aboutblock}

\specifications{\styleweissbier}{\galtol{5}}{1.048}{1.006}{5.5~\%}{14}{\srmtoebc{5.5}}{60~min}{}

\begin{methodandtiming}
 
\begin{mashsteps}
\mashstep{\ftoc{150}}{60~min}
\end{mashsteps}

\begin{fermentationsteps}
\fermentationstep{\ftoc{72}}{10~days}
\end{fermentationsteps}

\begin{directions}
Toast coconut until very dark, but not burnt.
\end{directions}

\end{methodandtiming}

\recipebreak

\begin{ingredientsblock}

\begin{malts}
\malt{White Wheat}{\lbtokg{8}}
\malt{Two-row}{\lbtokg{2.5}}
\end{malts}

\begin{hops}
\hop{\hopcrystal}{3.9~\%}{60~min}{\oztog{1.2}}
\end{hops}

\singleyeast{Wyeast 3068}

\begin{twists}
\twist{Toasted and Shredded Coconut}{Secondary (2~days)}{\lbtokg{1}}
\end{twists}

\end{ingredientsblock}

\end{recipe}

% -----------------------------------------------------------------------------
\begin{recipe}{Stew's Brew Hefeweizen}
% -----------------------------------------------------------------------------

\begin{aboutblock}
Bavarian Weissbier. Zach Gelfand, first prize, production competition.
\sourcezymurgy{May / June 2018}
\end{aboutblock}

\specifications{\styleweissbier}{\galtol{5.5}}{1.046}{1.010}{4.7~\%}{9}{\srmtoebc{4}}{90~min}{}

\begin{methodandtiming}
 
\begin{mashsteps}
\mashstep{\ftoc{122}}{20~min}
\mashstep{\ftoc{150}}{60~min}
\mashstep{\ftoc{168}}{Mashout}
\end{mashsteps}

\begin{fermentationsteps}
\fermentationstep{\ftoc{65}}{1~week}
\end{fermentationsteps}

\end{methodandtiming}

\recipebreak

\begin{ingredientsblock}

\begin{malts}
\malt{Weyermann Pale Wheat}{\lbtokg{4.75}}
\malt{Pilsner}{\lbtokg{3.75}}
\malt{Weyermann Melanoidin}{\oztog{8}}
\end{malts}

\begin{hops}
\hop{\hopeastkentgolding}{3.9~\%}{60~min}{\oztog{0.67}}
\end{hops}

\singleyeast{White Labs WLP300}

\end{ingredientsblock}

\end{recipe}

% -----------------------------------------------------------------------------
\begin{recipe}{The Bizarro Jerry Weissbier}
% -----------------------------------------------------------------------------

\begin{aboutblock}
Tyler Cipriani and Blazey Heier of Longmont, CO won a bronze medal in Category 7:
German Wheat Beer during the 2018 National Homebrew Competition Final Round in
Portland, OR. Cipriani and Heier's weissbier earned 3rd place among 200 entries
in the category. \sourceaha
\end{aboutblock}

\specifications{\styleweissbier}{\galtol{6}}{1.050}{}{}{12.3}{\srmtoebc{4.8}}{60~min}{}

\begin{methodandtiming}
 
\begin{mashsteps}
\mashstep{\ftoc{148}}{During decoction}
\mashdecoctthick{with 1/3 of mash}
\mashstep{\ftoc{160}}{15~min}
\mashdecoctboil{20~min}
\mashdecoctreturn{\ftoc{158}}{25~min}
\mashstep{\ftoc{170}}{10~min}
\end{mashsteps}

\begin{fermentationsteps}
\fermentationstep{\ftoc{56}}{Pitch}
\fermentationstep{\ftoc{64}}{Raise to}
\fermentationstep{\ftoc{68}}{Free raise to over 3~days}
\end{fermentationsteps}

\begin{directions}
Water adjustment: use carbon filtered Longmont, Colorado water with \tsptog{2} of
calcium sulfate in the mash.
\end{directions}

\end{methodandtiming}

\recipebreak

\begin{ingredientsblock}

\begin{malts}
\malt{Weyermann Pale Wheat}{\lbtokg{8}}
\malt{Root Shoot Odyssey Pilsner}{\lbtokg{4.75}}
\malt{Weyermann CARAHELL}{\oztog{12}}
\end{malts}

\begin{hops}
\hop{\hophallertaumittelfruh}{4.5~\%}{60~min}{\oztog{1}}
\hop{Wyeast Beer Nutrient Blend}{}{10~min}{\tsptog{1}}
\hop{Whirlfloc Tablet}{}{2~min}{1}
\end{hops}

\singleyeast{White Labs WLP300}

\end{ingredientsblock}

\end{recipe}

% -----------------------------------------------------------------------------
\begin{recipe}{Where's Fluffy Weissbier}
% -----------------------------------------------------------------------------

\begin{aboutblock}
Paul Brown of Pinole, CA, member of the Diablo Order of Zymiracle Enthusiasts (DOZE),
won a bronze medal in Category \#7: German Wheat Beer with a Weissbier during the 2019
National Homebrew Competition Final Round in Providence, RI. Brown's Weissbier was
chosen as a top three entry among 200 entries in the category. \sourceaha
\end{aboutblock}

\specifications{\styleweissbier}{\galtol{5.5}}{1.044}{1.006}{6.5~\%}{9.8}{\srmtoebc{5}}{90~min}{3}

\begin{methodandtiming}
 
\begin{mashsteps}
\mashstep{\ftoc{122}}{30~min}
\mashstep{\ftoc{148}}{30~min}
\mashstep{\ftoc{168}}{Mashout}
\end{mashsteps}

\begin{fermentationsteps}
\fermentationstep{\ftoc{64}}{5~days}
\fermentationstep{\ftoc{66}}{Until fully attenuated}
\end{fermentationsteps}

\end{methodandtiming}

\recipebreak

\begin{ingredientsblock}

\begin{malts}
\malt{Rahr Red Wheat}{\lbtokg{6}}
\malt{Weyermann Pilsner}{\lbtokg{5}}
\malt{Weyermann Melanoidin}{\oztog{12}}
\malt{Weyermann Acidulated}{\oztog{8}}
\end{malts}

\begin{hops}
\hop{\hopmagnum}{12~\%}{\fwh}{\oztog{0.3}}
\hop{Yeast Nutrient}{}{}{--}
\end{hops}

\singleyeast{White Labs WLP300}

\end{ingredientsblock}

\end{recipe}
