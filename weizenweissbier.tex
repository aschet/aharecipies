\stylesection{\styleweizenweissbier}

% -----------------------------------------------------------------------------
\begin{recipie}{Cane Toad Weisse}
% -----------------------------------------------------------------------------

\begin{aboutblock}
This beer was initially conceived as an attempt at one of my first and favorite
craft-style ales from way back in the 1980s: Redback from Matilda Bay Brewing
in Australia. I have fond memories of quaffing bottles of Redback during my
undergraduate years at Cal Poly, San Luis Obispo. That beer did not contain
sugar but was instead a down-under version of Bavarian wheat ale. My recipe is
somewhat drier due to the cane sugar, but with high carbonation, low hopping,
and Bavarian wheat yeast to add subtle hints of clove and banana, it is exceptionally
refreshing.
\end{aboutblock}

\specifications{\styleweizenweissbier}{\galtol{5.5}}{1.050}{1.007}{5.7~\%}{13}{\srmtoebc{3}}{60~min}{3.5}

\begin{methodandtiming}
 
\begin{mashsteps}
\mashstep{\ftoc{125}}{20~min}
\mashstep{\ftoc{140}}{30~min}
\mashstep{\ftoc{152}}{40~min}
\mashstep{\ftoc{168}}{Mashout}
\end{mashsteps}

\begin{fermentationsteps}
\fermentationstep{\ftoc{64}}{3~days}
\fermentationstep{\ftoc{68}}{Free-raise to}
\end{fermentationsteps}

\begin{directions}
Water adjustment: use 1 g/gal calcium cloride added to reverse osmosis water. 
Keep bottles at \ftoc{70} for a week or until they begin to clear. Store at cellar
temperatures for two weeks.
\end{directions}

\end{methodandtiming}

\begin{ingredientsblock}
    
\begin{malts}
\malt{Wheat}{\lbtokg{4.5}}
\malt{Briess two-row pale malt}{\lbtokg{4.5}}
\end{malts}

\begin{hops}
\hop{\hopsterling}{2.3~\%}{60~min}{\oztog{2}}
\hop{Raw Organic Cane Sugar}{}{--}{\oztog{12}}

\end{hops}

\begin{yeasts}
\yeast{White Labs WLP380}
\end{yeasts}

\end{ingredientsblock}

\end{recipie}

% -----------------------------------------------------------------------------
\begin{recipie}{Where's Fluffy Weissbier}
% -----------------------------------------------------------------------------

\begin{aboutblock}
Paul Brown of Pinole, CA, member of the Diablo Order of Zymiracle Enthusiasts (DOZE),
won a bronze medal in Category \#7: German Wheat Beer with a Weissbier during the 2019
National Homebrew Competition Final Round in Providence, RI. Brown's Weissbier was
chosen as a top three entry among 200 entries in the category.
\end{aboutblock}

\specifications{\styleweizenweissbier}{\galtol{5.5}}{1.044}{1.006}{6.5~\%}{9.8}{\srmtoebc{5}}{90~min}{3}

\begin{methodandtiming}
 
\begin{mashsteps}
\mashstep{\ftoc{122}}{30~min}
\mashstep{\ftoc{148}}{30~min}
\mashstep{\ftoc{168}}{Mashout}
\end{mashsteps}

\begin{fermentationsteps}
\fermentationstep{\ftoc{64}}{5~days}
\fermentationstep{\ftoc{66}}{--}
\end{fermentationsteps}

\end{methodandtiming}

\pagebreak

\begin{ingredientsblock}

\begin{malts}
\malt{Rahr Red Wheat}{\lbtokg{6}}
\malt{Weyermann Pilsner}{\lbtokg{5}}
\malt{Weyermann Melanoidin}{\oztog{12}}
\malt{Weyermann Acidulated}{\oztog{8}}
\end{malts}

\begin{hops}
\hop{\hopmagnum}{12~\%}{\fwh}{\oztog{0.3}}
\hop{Yeast Nutrient}{}{}{--}
\end{hops}

\begin{yeasts}
\yeast{White Labs WLP300}
\end{yeasts}

\end{ingredientsblock}

\end{recipie}
