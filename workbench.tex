\documentclass[10pt,oneside]{scrbook}

\usepackage{scrlayer-scrpage}
\usepackage[utf8]{inputenc}
\usepackage[american]{babel}
\usepackage[pass]{geometry}
\usepackage[regular,condensed,sfdefault]{roboto}
\usepackage[T1]{fontenc}
\usepackage{fp}
\usepackage{xcolor}
\usepackage{mdframed}
\usepackage{colortbl}
\usepackage{graphicx}
\usepackage{tabu}
\usepackage{booktabs}
\usepackage[version=4]{mhchem}
\usepackage{xifthen}
\usepackage{imakeidx}
\usepackage[hidelinks,pdfencoding=auto,pdftex,
  pdfauthor={Thomas Ascher},
  pdfusetitle,
  pdfkeywords={beer, brewing, recipes}]{hyperref}
\usepackage{multicol}
\usepackage{microtype}

\KOMAoptions{toc=flat,toc=listof}

\addtokomafont{part}{\Huge}
\addtokomafont{chapter}{\Huge}
\addtokomafont{section}{\Huge}

\definecolor{softgray}{HTML}{F1F1F1}
\definecolor{hardgray}{HTML}{6A6A6A}
\chead{}
\setlength{\parindent}{0mm} 
\RedeclareSectionCommand[tocnumwidth=0.85cm]{part}

\newmdenv[linewidth=3pt,
linecolor=black,
backgroundcolor=softgray,
fontcolor=hardgray,
rightline=false,
leftline=false,
bottomline=false,
skipabove=1mm,
skipbelow=1.5mm]{recipeframe}

\newcommand{\stylecategory}[1]{\part{#1}}
\newcommand{\stylesection}[1]{\chapter{#1} \clearpage}

\newenvironment{recipe}[2][]{\begin{multicols}{2}[\section*{#2}]\ifthenelse{\equal{#1}{}}{\index{#2}}{\index{#1@#2}}}{\end{multicols} \newpage}

\newcommand{\recipebreak}{\vfill\null \columnbreak}

\newenvironment{recipeblock}[1]
{\uppercase{\textbf{#1}} \begin{recipeframe}}{\end{recipeframe}}

\newenvironment{aboutblock}
{\begin{recipeblock}{About This Recipe}}{\end{recipeblock}}

\newenvironment{ingredientsblock}
{\begin{recipeblock}{Ingredients}}{\end{recipeblock}}

\newcommand{\recipesection}[2]{
\begin{minipage}[t][1.2cm]{\textwidth}
	\begin{minipage}[c][1.2cm][c]{0.95cm}
		\includegraphics[width=0.8cm]{#1}
	\end{minipage}
	\begin{minipage}[c][1.2cm][c]{5cm}
		\large{\textcolor{black}{\uppercase{\textbf{#2}}}}
	\end{minipage}
	\hfill
\end{minipage}
}

\newcommand{\ibutoibu}[1]{\FPeval\amountg{round((#1):0)}\FPprint\amountg ~IBU}
\newcommand{\ftoc}[1]{\FPeval\amountg{round(((#1 - 32.0) / 1.8):0)}\FPprint\amountg ~°C}
\newcommand{\gpgaltogpl}[1]{\FPeval\amountg{round((#1 * 0.264172):1)}\FPprint\amountg ~g/l}
\newcommand{\ozpgaltogpl}[1]{\FPeval\amountg{round((#1 * 7.48915):1)}\FPprint\amountg ~g/l}
\newcommand{\tsptog}[1]{\FPeval\amountg{round((#1 * 4.92892):0)}\FPprint\amountg ~g}
\newcommand{\tbsptog}[1]{\FPeval\amountg{round((#1 * 15.0):0)}\FPprint\amountg ~g}
\newcommand{\oztog}[1]{\FPeval\amountg{round((#1 * 28.34952):0)}\FPprint\amountg ~g}
\newcommand{\oztokg}[1]{\FPeval\amountg{round((#1 * 0.02834952):2)}\FPprint\amountg ~kg}
\newcommand{\lbtokg}[1]{\FPeval\amountg{round((#1 * 0.4535924):2)}\FPprint\amountg ~kg}
\newcommand{\galtol}[1]{\FPeval\amountg{round((#1 * 3.78541):1)}\FPprint\amountg ~l}
\newcommand{\qttol}[1]{\FPeval\amountg{round((#1 * 0.94635295):1)}\FPprint\amountg ~l}
\newcommand{\cuptoml}[1]{\FPeval\amountg{round((#1 * 240.0):0)}\FPprint\amountg ~ml}
\newcommand{\tsptoml}[1]{\FPeval\amountg{round((#1 * 4.92892):0)}\FPprint\amountg ~ml}
\newcommand{\srmtoebc}[1]{\FPeval\amountg{round((#1 * 1.97):0)}\FPprint\amountg ~EBC}
\newcommand{\sgtop}[1]{\FPeval\amountg{round((135.997 * pow(3,#1) - 630.272 * pow(2,#1) + 1111.14 * #1 - 616.868):1)}\FPprint\amountg ~°P}
\newcommand{\voltogl}[1]{\FPeval\amountg{round((#1 * 1.96):1)}\FPprint\amountg ~g/l}
\newcommand{\lboztokg}[2]{\FPeval\amountg{round((#1 * 0.4535924 + #2 * 28.34952 / 1000.0):2)}\FPprint\amountg ~kg}


\newcommand{\specitem}[1]{\textcolor{black}{\uppercase{\textbf{#1}}}}

\newenvironment{malts} {\recipesection{images/malt.pdf}{Malt / Mash Additions}
\begin{tabu} to \textwidth {Xr}
\textbf{Name} & \textbf{Amount} \\ \midrule}{\end{tabu}}
\newcommand{\malt}[2]{#1 & #2 \\ \midrule}

\newenvironment{hops} {\recipesection{images/hop.pdf}{Hops / Boil Additions}
\begin{tabu} to \textwidth {Xcr}
\textbf{Name} & \textbf{Addition} & \textbf{Amount} \\ \midrule}{\end{tabu}
{\centering\small{FWP = first wort hopping, FO = flameout, WP = whirlpool, DHBT =
biotransformation, DH = dry hopping} \par}}
\newcommand{\hop}[4]{#1 \ifthenelse{\isempty{#2}}{}{(#2)} & \ifthenelse{\isempty{#3}}{--}{#3} & #4 \\ \midrule}

\newenvironment{yeastsx} {\recipesection{images/yeast.pdf}{Yeast}
\begin{tabu} to \textwidth {Xr}
\textbf{Name} & \textbf{Addition} \\ \midrule}{\end{tabu}}
\newcommand{\yeastx}[2]{#1 & #2 \\ \midrule}

\newenvironment{yeasts} {\recipesection{images/yeast.pdf}{Yeast}
\begin{tabu} to \textwidth {X}}{\end{tabu}}
\newcommand{\yeast}[1]{#1 \\ \midrule}

\newcommand{\singleyeast}[1]{
\begin{yeasts}
\yeast{#1}
\end{yeasts}
}

\newenvironment{twists} {\recipesection{images/star.pdf}{Fermentation Additons}
\begin{tabu} to \textwidth {Xcr}
\textbf{Name} & \textbf{Addition} & \textbf{Amount} \\ \midrule}{\end{tabu}}
\newcommand{\twist}[3]{#1 & \ifthenelse{\isempty{#2}}{--}{#2} & #3 \\ \midrule}

\newenvironment{mashsteps} {\recipesection{images/mash.pdf}{Mash}
\begin{tabu} to \textwidth {Xr}}{\end{tabu}}
\newcommand{\mashstep}[2]{#1 & #2 \\ \midrule}
\newcommand{\mashdecoctthin}[1]{\mashstep{}{Thin decoction #1}}
\newcommand{\mashdecoctthick}[1]{\mashstep{}{Thick decoction #1}}
\newcommand{\mashdecoctboil}[1]{\mashstep{\ftoc{212}}{Boil #1}}
\newcommand{\mashdecoctreturn}[2]{\mashstep{#1}{Return to main mash\ifthenelse{\isempty{#2}}{}{; #2}}}

\newenvironment{fermentationsteps} {\recipesection{images/ferment.pdf}{Fermentation} 
\begin{tabu} to \textwidth {Xr}}{\end{tabu}}
\newcommand{\fermentationstep}[2]{#1 & #2 \\ \midrule}

\newenvironment{directions} {\recipesection{images/bulp.pdf}{Directions}}{}

\newenvironment{methodandtiming} {\begin{recipeblock}{Method / Timings}}{\end{recipeblock}}

\newenvironment{yeastinfos}[1] {\begin{recipeblock}{#1} 
\begin{tabu} to \textwidth {lX}}{\end{tabu}
\end{recipeblock}}
\newcommand{\yeastinfo}[2]{#1 & #2 \\ \midrule}

\newcommand{\dryhbt}[2]{DHBT\ifthenelse{\isempty{#1}}{}{\textsubscript{#1}}\ifthenelse{\isempty{#2}}{}{ (#2)}}
\newcommand{\dryh}[2]{DH\ifthenelse{\isempty{#1}}{}{\textsubscript{#1}}\ifthenelse{\isempty{#2}}{}{ (#2)}}
\newcommand{\whirl}[2]{WP\ifthenelse{\isempty{#1}}{}{\textsubscript{#1}}\ifthenelse{\isempty{#2}}{}{ (#2)}}
\newcommand{\fwh}{FWH}
\newcommand{\foh}[1]{FO\ifthenelse{\isempty{#1}}{}{ (#1)}}

\newcommand{\hopahtanum}{Ahtanum}
\newcommand{\hopamarillo}{Amarillo}
\newcommand{\hopapollo}{Apollo}
\newcommand{\hopaurora}{Aurora}
\newcommand{\hopbravo}{Bravo}
\newcommand{\hopcalypso}{Calypso}
\newcommand{\hopcascade}{Cascade}
\newcommand{\hopceleia}{Celeia}
\newcommand{\hopcentennial}{Centennial}
\newcommand{\hopchallenger}{Challenger}
\newcommand{\hopchinook}{Chinook}
\newcommand{\hopcitra}{Citra}
\newcommand{\hopcolumbus}{Columbus}
\newcommand{\hopcomet}{Comet}
\newcommand{\hopcrystal}{Crystal}
\newcommand{\hopeastkentgolding}{East Kent Golding}
\newcommand{\hopeldorado}{El Dorado}
\newcommand{\hopeureka}{Eureka}
\newcommand{\hopfalconersflight}{Falconer's Flight}
\newcommand{\hopfuggle}{Fuggle}
\newcommand{\hopgalaxy}{Galaxy}
\newcommand{\hopgolding}{Golding}
\newcommand{\hophallertaumittelfruh}{Hallertau Mittelfrüh}
\newcommand{\hophallertautradition}{Hallertau Tradition}
\newcommand{\hopherkules}{Herkules}
\newcommand{\hophersbrucker}{Hersbrucker}
\newcommand{\hophorizon}{Horizon}
\newcommand{\hopidahoseven}{Idaho 7}
\newcommand{\hopkohatu}{Kohatu}
\newcommand{\hoplemondrop}{Lemondrop}
\newcommand{\hopliberty}{Liberty}
\newcommand{\hoploral}{Loral}
\newcommand{\hoplubelska}{Lubelska}
\newcommand{\hopmagnum}{Magnum}
\newcommand{\hopmandarinabavaria}{Mandarina Bavaria}
\newcommand{\hopmarynka}{Marynka}
\newcommand{\hopmosaic}{Mosaic}
\newcommand{\hopmotueka}{Motueka}
\newcommand{\hopmthood}{Mt. Hood}
\newcommand{\hopnelsonsauvin}{Nelson Sauvin}
\newcommand{\hopnorthernbrewer}{Northern Brewer}
\newcommand{\hopnugget}{Nugget}
\newcommand{\hoppacifica}{Pacifica}
\newcommand{\hoppacificgem}{Pacific Gem}
\newcommand{\hoppacificjade}{Pacific Jade}
\newcommand{\hoppalisade}{Palisade}
\newcommand{\hopperle}{Perle}
\newcommand{\hoppolaris}{Polaris}
\newcommand{\hoprakau}{Rakau}
\newcommand{\hopsaaz}{Saaz}
\newcommand{\hopsantiam}{Santiam}
\newcommand{\hopsimcoe}{Simcoe}
\newcommand{\hopspalt}{Spalt}
\newcommand{\hopspaltselect}{Spalt Select}
\newcommand{\hopsterling}{Sterling}
\newcommand{\hopstrisselspalt}{Strisselspalt}
\newcommand{\hopstyriangolding}{Styrian Golding}
\newcommand{\hopsummit}{Summit}
\newcommand{\hoptarget}{Target}
\newcommand{\hoptettnang}{Tettnang}
\newcommand{\hopvanguard}{Vanguard}
\newcommand{\hopvicsecret}{Vic Secret}
\newcommand{\hopwakatu}{Wakatu}
\newcommand{\hopwarrior}{Warrior}
\newcommand{\hopwillamette}{Willamette}
\newcommand{\hopzythos}{Zythos}

% Light Lager
\newcommand{\styleamericanlager}{American Lager}
\newcommand{\stylemunichhelles}{Munich Helles}
\newcommand{\stylefestbier}{Festbier}

% Pilsner
\newcommand{\stylegermanpils}{German Pils}
\newcommand{\styleczechpremiumpalelager}{Czech Premium Pale Lager}

% European Amber Lager
\newcommand{\styleviennalager}{Vienna Lager}

% Bock
\newcommand{\stylehellesbock}{Helles Bock}
\newcommand{\styledopplebock}{Dopplebock}

% Dark Lager
\newcommand{\stylemunichdunkel}{Munich Dunkel}
\newcommand{\styleschwarzbier}{Schwarzbier}

% Light Hybrid Beer
\newcommand{\stylecreamale}{Cream Ale}
\newcommand{\styleblondeale}{Blonde Ale}
\newcommand{\styleamericanwheat}{American Wheat}

% Amber Hybrid Beer
\newcommand{\styleinternationalamberlager}{International Amber Lager}
\newcommand{\stylecaliforniacommon}{California Common}

% English Pale Ale
\newcommand{\styleordinarybitter}{Ordinary Bitter}
\newcommand{\stylestrongbitter}{Strong Bitter}

% Scottish & Irish Ale
\newcommand{\stylescottishlight}{Scottish Light}
\newcommand{\stylescottishheavy}{Scottish Heavy}
\newcommand{\styleirishredale}{Irish Red Ale}

% American Pale Ale
\newcommand{\styleamericanpaleale}{American Pale Ale}

% Other American Ale
\newcommand{\styleamericanamberale}{American Amber Ale}

% English Brown Ale
\newcommand{\styledarkmild}{Dark Mild}
\newcommand{\stylebritishbrownale}{British Brown Ale}

% Porter
\newcommand{\styleenglishporter}{English Porter}
\newcommand{\styleamericanporter}{American Porter}
\newcommand{\stylebalticporter}{Baltic Porter}

% Stout
\newcommand{\styleoatmealstout}{Oatmeal Stout}
\newcommand{\stylesweetstout}{Sweet Stout}

% Strong Stout
\newcommand{\styleamericanstout}{American Stout}
\newcommand{\styleimperialstout}{Imperial Stout}

% American IPA
\newcommand{\styleamericanipa}{American IPA}

% India Pale Ale
\newcommand{\styleindiapaleale}{India Pale Ale}
\newcommand{\styleryeipa}{Specialty IPA: Rye IPA}

% German Wheat Rye Beer
\newcommand{\styleweissbier}{Weissbier}
\newcommand{\styleweizenbock}{Weizenbock}

% Belgian & French Al
\newcommand{\stylewitbier}{Witbier}
\newcommand{\stylebelgianpaleale}{Belgian Pale Ale}
\newcommand{\stylesaison}{Saison}

% Belgian Strong Ale
\newcommand{\stylebelgianblondale}{Belgian Blond Ale}
\newcommand{\stylebelgiandubbel}{Belgian Dubbel}
\newcommand{\stylebelgiangoldenstrongale}{Belgian Golden Strong Ale}
\newcommand{\stylebelgiandarkstrongale}{Belgian Dark Strong Ale}

% Strong Ale
\newcommand{\styleoldale}{Old Ale}
\newcommand{\styleenglishbarleywine}{English Barleywine}
\newcommand{\styleamericanbarleywine}{American Barleywine}

% Fruit Beer
\newcommand{\stylefruitbeer}{Fruit Beer}

% Spice / Herb / Vegetable Beer
\newcommand{\stylewinterseasonalbeer}{Winter Seasonal Beer}

% Specialty Beer
\newcommand{\stylealternativesugarbeer}{Alternative Sugar Beer}

\newcommand{\sourceaha}{Source: American Homebrewers Association.}
\newcommand{\sourcezymurgy}[1]{Source: Zymurgy #1.}
\newcommand{\waterprofile}[6]{\ce{Ca} #1~ppm, \ce{Mg} #2~ppm, \ce{SO4} #3~ppm, \ce{Na} #4~ppm, \ce{Cl} #5~ppm\ifthenelse{\isempty{#6}}{}{, \ce{HCO3} #6~ppm}}

\newcommand{\specifications}[9]{
\begin{recipeblock}{Specifications}
\begin{tabu} to \textwidth {Xr}
\specitem{Style} & #1 \\ \midrule
\specitem{Volume} & #2 \\ \midrule
\specitem{Original Gravity} & \ifthenelse{\isempty{#3}}{--}{\sgtop{#3} / #3} \\ \midrule
\specitem{Final Gravity} & \ifthenelse{\isempty{#4}}{--}{\sgtop{#4} / #4} \\ \midrule
\specitem{ABV} & \ifthenelse{\isempty{#5}}{--}{#5} \\ \midrule
\specitem{Bitterness} & \ifthenelse{\isempty{#6}}{--}{\ibutoibu{#6}} \\ \midrule
\specitem{Color} & \ifthenelse{\isempty{#7}}{--}{#7} \\ \midrule
\specitem{Boil Time} & \ifthenelse{\isempty{#8}}{--}{#8} \\ \midrule
\specitem{Carbonation} & \ifthenelse{\isempty{#9}}{--}{\voltogl{#9} / #9~vol} \\ \midrule
\end{tabu}
\end{recipeblock}
}

\newgeometry{left=2.5cm,right=2.5cm,top=2.5cm,bottom=3cm}

\RedeclareSectionCommand[beforeskip=0mm,afterindent=false,afterskip=3mm]{chapter}
\RedeclareSectionCommand[afterindent=false,afterskip=0.5mm]{section}
\renewcommand*{\partpagestyle}{empty} 


\begin{document}

\mainmatter
\twocolumn

\part{Workbench}

% -----------------------------------------------------------------------------
\begin{recipie}{Crow Peak Brewing Co. Pile 'O Dirt Porter}
% -----------------------------------------------------------------------------

\begin{aboutblock}
This beer was named Pile 'O Dirt because of the ridiculous amount of dirt Crow
Peak Brewing had to build their original brewery on to get them out of the flood
plain. This porter is very dark in color with a nice tan head and complexity
due to the variety of specialty malts used.
\end{aboutblock}

\specifications{\stylerobustporter}{\galtol{5.5}}{1.062}{1.015}{5.5~\%}{20}{\srmtoebc{35}}{60~min}{}

\begin{methodandtiming}
 
\begin{mashsteps}
\mashstep{\ftoc{156}}{60~min}
\end{mashsteps}

\begin{fermentationsteps}
\fermentationstep{\ftoc{63}}{}
\end{fermentationsteps}

\end{methodandtiming}

\pagebreak

\begin{ingredientsblock}

\begin{malts}
\malt{Briess Pale Ale}{\lbtokg{9}}
\malt{Briess Bonlander Munich 10 L}{\lbtokg{1.5}}
\malt{Briess Chocolate}{\lbtokg{0.5}}
\malt{Briess Carapils}{\lbtokg{0.5}}
\malt{Briess Extra Special}{\lbtokg{0.5}}
\malt{Briess Blackprinz}{\lbtokg{0.5}}
\end{malts}

\begin{hops}
\hop{\hopperle}{8.2~\%}{\fwh}{\oztog{0.5}}
\hop{\hopwillamette}{4.5~\%}{10~min}{\oztog{0.3}}
\hop{\hopwillamette}{4.5~\%}{\whirl{}{}}{\oztog{0.3}}
\end{hops}

\begin{yeasts}
\yeast{White Labs WLP041}
\end{yeasts}

\end{ingredientsblock}

\end{recipie}

\backmatter

\appendix

\twocolumn

\part{Appendix}

\chapter{Malt}

\begin{maltinfos}{Vendor Neutral}
\maltinfo{Black}{\ltoebc{470}}{\ltoebc{620}}
\maltinfo{Brown}{\ltoebc{50}}{\ltoebc{75}}
\maltinfo{Caramel / Crystal 10 L}{\ltoebc{10}}{\ltoebc{10}}
\maltinfo{Caramel / Crystal 15 L}{\ltoebc{15}}{\ltoebc{15}}
\maltinfo{Caramel / Crystal 20 L}{\ltoebc{20}}{\ltoebc{20}}
\maltinfo{Caramel / Crystal 30 L}{\ltoebc{30}}{\ltoebc{30}}
\maltinfo{Caramel / Crystal 40 L}{\ltoebc{40}}{\ltoebc{40}}
\maltinfo{Caramel / Crystal 45 L}{\ltoebc{45}}{\ltoebc{45}}
\maltinfo{Caramel / Crystal 55 L}{\ltoebc{55}}{\ltoebc{55}}
\maltinfo{Caramel / Crystal 60 L}{\ltoebc{60}}{\ltoebc{60}}
\maltinfo{Caramel / Crystal 65 L}{\ltoebc{65}}{\ltoebc{65}}
\maltinfo{Caramel / Crystal 75 L}{\ltoebc{75}}{\ltoebc{75}}
\maltinfo{Caramel / Crystal 80 L}{\ltoebc{80}}{\ltoebc{80}}
\maltinfo{Caramel / Crystal 90 L}{\ltoebc{90}}{\ltoebc{90}}
\maltinfo{Caramel / Crystal 120 L}{\ltoebc{120}}{\ltoebc{120}}
\maltinfo{Chocolate}{\ltoebc{400}}{\ltoebc{500}}
\maltinfo{Pale Chocolate}{\ltoebc{200}}{\ltoebc{300}}
\end{maltinfos}

\begin{maltinfos}{Admiral}
\maltinfo{Feldblume}{\srmtoebc{2}}{\srmtoebc{3}}
\end{maltinfos}

\begin{maltinfos}{Avangard}
\maltinfo{Pilsner}{\ebctoebc{3}}{\ebctoebc{3.5}}
\maltinfo{Vienna}{\ebctoebc{8}}{\ebctoebc{15}}
\end{maltinfos}

\begin{maltinfos}{Bairds}
\maltinfo{Brown}{\ebctoebc{110}}{\ebctoebc{130}}
\maltinfo{Caramalt Medium}{\ebctoebc{55}}{\ebctoebc{75}}
\maltinfo{Crystal Medium}{\ebctoebc{140}}{\ebctoebc{160}}
\maltinfo{Maris Otter Pale Ale}{\ebctoebc{5}}{\ebctoebc{7}}
\end{maltinfos}

\pagebreak

\begin{maltinfos}{BEST}
\maltinfo{Caramel Aromatic}{\ebctoebc{41}}{\ebctoebc{60}}
\maltinfo{Chit}{\ebctoebc{2}}{\ebctoebc{3}}
\maltinfo{Heidelberg}{\ebctoebc{2.9}}{\ebctoebc{2.9}}
\maltinfo{Munich}{\ebctoebc{11}}{\ebctoebc{20}}
\maltinfo{Munich Dark}{\ebctoebc{21}}{\ebctoebc{35}}
\maltinfo{Pilsen}{\ebctoebc{3}}{\ebctoebc{4.9}}
\maltinfo{Vienna}{\ebctoebc{8}}{\ebctoebc{10}}
\end{maltinfos}

\begin{maltinfos}{Briess}
\maltinfo{Aromatic Munich 20 L}{\ltoebc{20}}{\ltoebc{20}}
\maltinfo{Blackprinz}{\ltoebc{500}}{\ltoebc{500}}
\maltinfo{Blonde RoastOat}{\ltoebc{4}}{\ltoebc{4}}
\maltinfo{Bonlander Munich 10 L}{\ltoebc{10}}{\ltoebc{10}}
\maltinfo{Brewers}{\ltoebc{1.8}}{\ltoebc{1.8}}
\maltinfo{Caramel 20 L}{\ltoebc{20}}{\ltoebc{20}}
\maltinfo{Caramel 30 L}{\ltoebc{30}}{\ltoebc{30}}
\maltinfo{Caramel 40 L}{\ltoebc{40}}{\ltoebc{40}}
\maltinfo{Caramel 60 L}{\ltoebc{60}}{\ltoebc{60}}
\maltinfo{Caramel 80 L}{\ltoebc{80}}{\ltoebc{80}}
\maltinfo{Caramel Vienne 20 L}{\ltoebc{20}}{\ltoebc{20}}
\maltinfo{Carapils}{\ltoebc{1.5}}{\ltoebc{1.5}}
\maltinfo{Chocolate}{\ltoebc{350}}{\ltoebc{350}}
\maltinfo{Extra Special}{\ltoebc{130}}{\ltoebc{130}}
\maltinfo{Midnight Wheat}{\ltoebc{550}}{\ltoebc{550}}
\maltinfo{Organic Caramel 60 L}{\ltoebc{60}}{\ltoebc{60}}
\maltinfo{Organic Chocolate}{\ltoebc{350}}{\ltoebc{350}}
\maltinfo{Roasted Barley}{\ltoebc{300}}{\ltoebc{300}}
\maltinfo{Pale Ale}{\ltoebc{3.5}}{\ltoebc{3.5}}
\maltinfo{Pilsen}{\ltoebc{1.2}}{\ltoebc{1.2}}
\maltinfo{Red Wheat}{\ltoebc{3}}{\ltoebc{3}}
\maltinfo{Rye}{\ltoebc{3.7}}{\ltoebc{3.7}}
\maltinfo{Special Roast}{\ltoebc{40}}{\ltoebc{40}}
\maltinfo{Synergy Select Pilsen}{\ltoebc{1.8}}{\ltoebc{1.8}}
\maltinfo{Victory}{\ltoebc{28}}{\ltoebc{28}}
\maltinfo{White Wheat}{\ltoebc{2.8}}{\ltoebc{2.8}}
\end{maltinfos}

\pagebreak

\begin{maltinfos}{Cargill}
\maltinfo{Pauls Mild Ale (Dextrin)}{\ebctoebc{8}}{\ebctoebc{10}}
\end{maltinfos}

\begin{maltinfos}{Castle}
\maltinfo{Chateau Abbey}{\ebctoebc{45}}{\ebctoebc{45}}
\maltinfo{Chateau Black Nature}{\ebctoebc{1150}}{\ebctoebc{1400}}
\maltinfo{Chateau Pale Ale}{\ebctoebc{7}}{\ebctoebc{10}}
\maltinfo{Chateau Special Belgium}{\ebctoebc{260}}{\ebctoebc{320}}
\maltinfo{Chateau Spelt}{\ebctoebc{3}}{\ebctoebc{7}}
\end{maltinfos}

\begin{maltinfos}{Crisp}
\maltinfo{Amber}{\ebctoebc{60}}{\ebctoebc{80}}
\maltinfo{Brown}{\ebctoebc{120}}{\ebctoebc{150}}
\maltinfo{Cara}{\ebctoebc{25}}{\ebctoebc{35}}
\maltinfo{Chocolate}{\ebctoebc{900}}{\ebctoebc{1100}}
\maltinfo{Dextrin}{\ebctoebc{2.5}}{\ebctoebc{3.5}}
\maltinfo{Finest Maris Otter Ale}{\ebctoebc{5.5}}{\ebctoebc{7.5}}
\maltinfo{Light Crystal}{\ebctoebc{160}}{\ebctoebc{180}}
\maltinfo{Medium Crystal}{\ebctoebc{250}}{\ebctoebc{290}}
\maltinfo{Roast Barley}{\ebctoebc{1250}}{\ebctoebc{1450}}
\end{maltinfos}

\begin{maltinfos}{Dingemans}
\maltinfo{Aromatic / Amber}{\ebctoebc{40}}{\ebctoebc{60}}
\maltinfo{Biscuit}{\ebctoebc{50}}{\ebctoebc{70}}
\maltinfo{Cara 120}{\ebctoebc{100}}{\ebctoebc{140}}
\maltinfo{Mroost 900 (Chocolate)}{\ebctoebc{800}}{\ebctoebc{1000}}
\maltinfo{Organic Biscuit}{\ebctoebc{50}}{\ebctoebc{70}}
\maltinfo{Pilsen}{\ebctoebc{2.5}}{\ebctoebc{3.5}}
\maltinfo{Special B}{\ebctoebc{300}}{\ebctoebc{350}}
\end{maltinfos}

\begin{maltinfos}{Durst}
\maltinfo{Pilsner}{\ebctoebc{3}}{\ebctoebc{4}}
\maltinfo{Wheat}{\ebctoebc{3.2}}{\ebctoebc{4}}
\end{maltinfos}

\pagebreak

\begin{maltinfos}{Fawcett}
\maltinfo{Brown}{\ltoebc{66}}{\ltoebc{76}}
\maltinfo{Caramalt}{\ltoebc{10}}{\ltoebc{12}}
\maltinfo{Chocolate}{\ltoebc{390}}{\ltoebc{450}}
\maltinfo{Crystal Malt II}{\ltoebc{60}}{\ltoebc{70}}
\maltinfo{Dark Crystal I}{\ltoebc{80}}{\ltoebc{90}}
\maltinfo{Dark Crystal II}{\ltoebc{110}}{\ltoebc{130}}
\maltinfo{Golden Promise Pale Ale}{\ltoebc{2.3}}{\ltoebc{3}}
\maltinfo{Pale Chocolate}{\ltoebc{200}}{\ltoebc{260}}
\maltinfo{Pearl Pale Ale}{\ltoebc{2.3}}{\ltoebc{3}}
\maltinfo{Roasted Barley}{\ltoebc{490}}{\ltoebc{600}}
\maltinfo{Torrified Wheat}{\ltoebc{1.5}}{\ltoebc{3}}
\end{maltinfos}

\begin{maltinfos}{Gambrinus}
\maltinfo{ESB Pale}{\ltoebc{2.5}}{\ltoebc{3.5}}
\maltinfo{Honey}{\ebctoebc{17}}{\ebctoebc{25}}
\end{maltinfos}

\begin{maltinfos}{Great Western}
\maltinfo{Crystal 40}{\ltoebc{40}}{\ltoebc{40}}
\maltinfo{Crystal 60}{\ltoebc{60}}{\ltoebc{60}}
\maltinfo{Crystal 75}{\ltoebc{75}}{\ltoebc{75}}
\maltinfo{Crystal 120}{\ltoebc{120}}{\ltoebc{120}}
\maltinfo{Organic Premium Two-row}{\ltoebc{2}}{\ltoebc{2}}
\maltinfo{Premium Two-row}{\ltoebc{1.6}}{\ltoebc{1.6}}
\maltinfo{Pure California}{\ltoebc{2}}{\ltoebc{2}}
\maltinfo{Superior Pilsen}{\ltoebc{1.6}}{\ltoebc{1.6}}
\end{maltinfos}

\begin{maltinfos}{Malteurop}
\maltinfo{Pilsen}{\ebctoebc{3}}{\ebctoebc{4.5}}
\end{maltinfos}

\begin{maltinfos}{Mecca Grade}
\maltinfo{Pelton}{\srmtoebc{1.6}}{\srmtoebc{1.8}}
\end{maltinfos}

\begin{maltinfos}{Patagonia}
\maltinfo{Black Pearl}{\ebctoebc{800}}{\ebctoebc{1000}}
\end{maltinfos}

\pagebreak

\begin{maltinfos}{Rahr}
\maltinfo{Pale Ale}{\ltoebc{3}}{\ltoebc{4}}
\maltinfo{Premium Pilsner}{\ltoebc{1.5}}{\ltoebc{2.0}}
\maltinfo{Red Wheat}{\ltoebc{3}}{\ltoebc{3.5}}
\maltinfo{Standard Six-row}{\ltoebc{2.1}}{\ltoebc{2.5}}
\maltinfo{Standard Two-row}{\ltoebc{1.7}}{\ltoebc{2}}
\maltinfo{White Wheat}{\ltoebc{3}}{\ltoebc{3.5}}
\end{maltinfos}

\begin{maltinfos}{Root Shoot}
\maltinfo{Odyssey Pilsner}{\ltoebc{2}}{\ltoebc{2}}
\end{maltinfos}

\begin{maltinfos}{Simpsons}
\maltinfo{Amber}{\ebctoebc{54}}{\ebctoebc{71}}
\maltinfo{Brown}{\ebctoebc{162}}{\ebctoebc{226}}
\maltinfo{Chocolate}{\ebctoebc{1067}}{\ebctoebc{1300}}
\maltinfo{Crystal Dark}{\ebctoebc{250}}{\ebctoebc{285}}
\maltinfo{Crystal Medium}{\ebctoebc{167}}{\ebctoebc{190}}
\maltinfo{Crystal T50}{\ebctoebc{130}}{\ebctoebc{145}}
\maltinfo{DRC}{\ebctoebc{280}}{\ebctoebc{320}}
\maltinfo{Finest Pale Ale Golden Promise}{\ebctoebc{4.5}}{\ebctoebc{6.5}}
\maltinfo{Finest Pale Ale Maris Otter}{\ebctoebc{4.5}}{\ebctoebc{6.5}}
\maltinfo{Golden Naked Oats}{\ebctoebc{12}}{\ebctoebc{25}}
\maltinfo{Roasted Barley}{\ebctoebc{1300}}{\ebctoebc{1900}}
\end{maltinfos}

\begin{maltinfos}{The Swaen}
\maltinfo{Pilsner}{\ebctoebc{3}}{\ebctoebc{4.5}}
\end{maltinfos}

\pagebreak

\begin{maltinfos}{Weyermann}
\maltinfo{Barke Pilsner}{\ebctoebc{2.5}}{\ebctoebc{4.5}}
\maltinfo{Beech Smoked Barley Malt}{\ebctoebc{4}}{\ebctoebc{8}}
\maltinfo{Bohemian Pilsner}{\ebctoebc{3}}{\ebctoebc{5}}
\maltinfo{CARAAMBER}{\ebctoebc{60}}{\ebctoebc{80}}
\maltinfo{CARAAROMA}{\ebctoebc{350}}{\ebctoebc{450}}
\maltinfo{CARABOHEMIAN}{\ebctoebc{170}}{\ebctoebc{220}}
\maltinfo{CARAFA I}{\ebctoebc{800}}{\ebctoebc{1000}}
\maltinfo{CARAFA II}{\ebctoebc{1100}}{\ebctoebc{1200}}
\maltinfo{CARAFA SPECIAL II}{\ebctoebc{1100}}{\ebctoebc{1200}}
\maltinfo{CARAFA SPECIAL III}{\ebctoebc{1300}}{\ebctoebc{1500}}
\maltinfo{CARAHELL}{\ebctoebc{20}}{\ebctoebc{30}}
\maltinfo{CARAMUNICH I}{\ebctoebc{80}}{\ebctoebc{100}}
\maltinfo{CARAMUNICH II}{\ebctoebc{110}}{\ebctoebc{130}}
\maltinfo{CARAMUNICH III}{\ebctoebc{140}}{\ebctoebc{160}}
\maltinfo{CARAPILS}{\ebctoebc{2.5}}{\ebctoebc{6.5}}
\maltinfo{CARARED}{\ebctoebc{40}}{\ebctoebc{60}}
\maltinfo{Chocolate Wheat}{\ebctoebc{900}}{\ebctoebc{1200}}
\maltinfo{Melanoidin}{\ebctoebc{60}}{\ebctoebc{80}}
\maltinfo{Munich I}{\ebctoebc{12}}{\ebctoebc{18}}
\maltinfo{Munich II}{\ebctoebc{20}}{\ebctoebc{25}}
\maltinfo{Pale Wheat}{\ebctoebc{3}}{\ebctoebc{5}}
\maltinfo{Pilsner}{\ebctoebc{2.5}}{\ebctoebc{4.5}}
\maltinfo{Vienna}{\ebctoebc{6}}{\ebctoebc{9}}
\end{maltinfos}

\chapter{Yeast}

\begin{yeastinfos}{Brewing Science Institute}
\yeastinfo{A-18}{London Ale III}
\yeastinfo{B-22}{LaChouffe}
\yeastinfo{S-26}{Farmhouse Ale}
\end{yeastinfos}

\begin{yeastinfos}{GigaYeast}
\yeastinfo{GY001}{Norcal Ale \#1}
\yeastinfo{GY002}{Czech Pilsner}
\yeastinfo{GY005}{Golden Gate Lager}
\yeastinfo{GY014}{Belgian Abbey Ale}
\yeastinfo{GY017}{Bavarian Hefe}
\yeastinfo{GY020}{Portland Hefe}
\yeastinfo{GY021}{Kölsch Bier}
\yeastinfo{GY027}{Saison \#2}
\yeastinfo{GY031}{British Ale \#2}
\yeastinfo{GY044}{Scotch Ale \#1}
\yeastinfo{GY045}{German Lager}
\yeastinfo{GY054}{Vermont IPA}
\yeastinfo{GY080}{Irish Stout}
\end{yeastinfos}

\begin{yeastinfos}{Imperial Yeast}
\yeastinfo{A01}{House}
\yeastinfo{A04}{Barbarian}
\yeastinfo{A07}{Flagship}
\yeastinfo{A09}{Pub}
\yeastinfo{A10}{Darkness}
\yeastinfo{A15}{Independence}
\yeastinfo{A24}{Dry Hop}
\yeastinfo{A31}{Tartan}
\yeastinfo{B44}{Whiteout}
\yeastinfo{B45}{Gnome}
\yeastinfo{B48}{Triple Double}
\yeastinfo{B63}{Monastic}
\yeastinfo{G01}{Stefon}
\yeastinfo{G02}{Kaiser}
\yeastinfo{G03}{Dieter}
\yeastinfo{L05}{Cablecar}
\yeastinfo{L13}{Global}
\yeastinfo{L17}{Harvest}
\end{yeastinfos}

\pagebreak

\begin{yeastinfos}{Inland Island Yeast Labs}
\yeastinfo{INIS-003}{Colorado IPA}
\end{yeastinfos}

\begin{yeastinfos}{Omega Yeast}
\yeastinfo{OYL-001}{Alt}
\yeastinfo{OYL-002}{American Wheat}
\yeastinfo{OYL-003}{London Ale}
\yeastinfo{OYL-004}{West Coast Ale I}
\yeastinfo{OYL-005}{Irish Ale}
\yeastinfo{OYL-006}{British Ale I}
\yeastinfo{OYL-007}{British Ale II}
\yeastinfo{OYL-009}{West Coast Ale II}
\yeastinfo{OYL-010}{British Ale IV}
\yeastinfo{OYL-012}{British Ale V}
\yeastinfo{OYL-014}{British Ale VII}
\yeastinfo{OYL-015}{Scottish Ale}
\yeastinfo{OYL-016}{British Ale VIII}
\yeastinfo{OYL-017}{Kolsch}
\yeastinfo{OYL-018}{Abbey Ale C}
\yeastinfo{OYL-019}{Belgian Ale D}
\yeastinfo{OYL-021}{Hefeweizen Ale I}
\yeastinfo{OYL-022}{Hefeweizen Ale II}
\yeastinfo{OYL-027}{Belgian Saison I}
\yeastinfo{OYL-030}{Wit}
\yeastinfo{OLY-041}{CL-50 Ale}
\yeastinfo{OLY-052}{DIPA Ale}
\yeastinfo{OLY-101}{Pilsner I}
\yeastinfo{OLY-105}{West Coast Lager}
\yeastinfo{OLY-106}{German Lager I}
\yeastinfo{OLY-107}{Oktoberfest}
\yeastinfo{OLY-108}{Pilsner II}
\yeastinfo{OLY-109}{German Lager II}
\yeastinfo{OYL-500}{Saisonstein's Monster}
\end{yeastinfos}

\begin{yeastinfos}{RVA Yeast Labs}
\yeastinfo{RVA 132}{Manchester Ale}
\end{yeastinfos}

\pagebreak

\begin{yeastinfos}{The Yeast Bay}
\yeastinfo{WLP4000}{Vermont Ale}
\yeastinfo{WLP4021}{Saison Blend II}
\yeastinfo{WLP4042}{Hazy Daze}
\yeastinfo{WLP4627}{Funktown Pale Ale}
\end{yeastinfos}

\pagebreak

\begin{yeastinfos}{White Labs}
\yeastinfo{WLP001}{California Ale}
\yeastinfo{WLP002}{English Ale}
\yeastinfo{WLP004}{Irish Ale}
\yeastinfo{WLP007}{Dry English Ale}
\yeastinfo{WLP013}{London Ale}
\yeastinfo{WLP023}{Burton Ale}
\yeastinfo{WLP025}{Southwold Ale}
\yeastinfo{WLP028}{Edinburgh Scottish Ale}
\yeastinfo{WLP029}{German Ale / Kölsch}
\yeastinfo{WLP036}{Düsseldorf Alt Ale}
\yeastinfo{WLP039}{East Midlands Ale}
\yeastinfo{WLP041}{Pacific Ale}
\yeastinfo{WLP051}{California V Ale}
\yeastinfo{WLP080}{Cream Ale Yeast Blend}
\yeastinfo{WLP090}{San Diego Super}
\yeastinfo{WLP095}{Burlington Ale}
\yeastinfo{WLP300}{Hefeweizen Ale}
\yeastinfo{WLP320}{American Hefeweizen Ale}
\yeastinfo{WLP380}{Hefeweizen IV Ale}
\yeastinfo{WLP400}{Belgian Wit Ale}
\yeastinfo{WLP500}{Monastery Ale}
\yeastinfo{WLP510}{Bastogne Belgian Ale}
\yeastinfo{WLP530}{Abbey Ale}
\yeastinfo{WLP550}{Belgian Ale}
\yeastinfo{WLP565}{Belgian Saison I Ale}
\yeastinfo{WLP570}{Belgian Golden Ale}
\yeastinfo{WLP800}{Pilsner Lager}
\yeastinfo{WLP802}{Czech Budejovice Lager}
\yeastinfo{WLP810}{San Francisco Lager}
\yeastinfo{WLP820}{Oktoberfest/Märzen Lager}
\yeastinfo{WLP830}{German Lager}
\yeastinfo{WLP833}{German Bock Lager}
\yeastinfo{WLP838}{Southern German Lager}
\yeastinfo{WLP840}{American Lager Yeast}
\yeastinfo{WLP920}{Old Bavarian Lager}
\yeastinfo{WLP940}{Mexican Lager}
\yeastinfo{WLP1983}{Charlie's Fist Bump}
\end{yeastinfos}

\pagebreak

\begin{yeastinfos}{Wyeast}
\yeastinfo{1007}{German Ale}
\yeastinfo{1010}{American Wheat}
\yeastinfo{1028}{London Ale}
\yeastinfo{1056}{American Ale}
\yeastinfo{1084}{Irish Ale}
\yeastinfo{1098}{British Ale}
\yeastinfo{1099}{Whitbread Ale}
\yeastinfo{1187}{Ringwood Ale}
\yeastinfo{1214}{Belgian Abbey Style Ale}
\yeastinfo{1272}{American Ale II}
\yeastinfo{1275}{Thames Valley Ale}
\yeastinfo{1318}{London Ale III}
\yeastinfo{1335}{British Ale II}
\yeastinfo{1388}{Belgian Strong Ale}
\yeastinfo{1450}{Denny's Favorite 50 Ale}
\yeastinfo{1469}{West Yorkshire Ale}
\yeastinfo{1728}{Scottish Ale}
\yeastinfo{1764-PC}{ROGUE Pacman}
\yeastinfo{1968}{London ESB Ale}
\yeastinfo{2001}{Pilsner Urquell H-Strain}
\yeastinfo{2035-PC}{American Lager}
\yeastinfo{2042-PC}{Danish Lager}
\yeastinfo{2112}{California Lager}
\yeastinfo{2124}{Bohemian Lager}
\yeastinfo{2206}{Bavarian Lager}
\yeastinfo{2247-PC}{European Lager}
\yeastinfo{2278}{Czech Pils}
\yeastinfo{2308}{Munich Lager}
\yeastinfo{2352}{Munich Lager II}
\yeastinfo{2565}{Kölsch}
\yeastinfo{2633}{Octoberfest Lager Blend}
\yeastinfo{3068}{Weihenstephan Weizen}
\yeastinfo{3333-PC}{German Wheat}
\yeastinfo{3522}{Belgian Ardennes}
\yeastinfo{3711}{French Saison}
\yeastinfo{3724}{Belgian Saison}
\yeastinfo{3726}{Farmhouse Ale}
\yeastinfo{3787}{Trappist Style High Gravity}
\yeastinfo{3944}{Belgian Witbier}
\end{yeastinfos}

\onecolumn

\chapter{Yeast Substitution}

\begin{recipeblock}{Substitution Table} 

\begin{tabu} to \textwidth {cccccX}

\textbf{Wyeast} & \textbf{White Labs} & \textbf{Omega} & \textbf{Imperial} & \textbf{GigaYeast} & \textbf{Dry} \\ \midrule

1007 & WLP036 & OYL-001 & G02 & n/a & Fermentis SafeAle K-97 \\ \midrule
1010 & WLP320 & OYL-002 & n/a & GY020 & n/a \\ \midrule
1028 & WLP013 & OYL-003 & n/a & n/a & Lallemand Windstor British-Style Beer \\ \midrule
1056 & WLP001 & OYL-004	& A07 & GY001 & Fermentis SafAle US-05 / Lallemand BRY-97 American West Coast Ale \\ \midrule
1084 & WLP004 & OYL-005 & A10 & GY080 & n/a \\ \midrule
1098 & WLP007 & OYL-006 & A01 & GY031 & Fermentis SafAle S-04 \\ \midrule
1099 & n/a & OYL-007 & n/a & n/a & n/a \\ \midrule
1214 & WLP500 & OYL-018 & B63 & GY014 & Fermentis SafAle T-58 / Mangrove Jack's M31 Belgian Tripel \\ \midrule
1272 & WLP051 & OYL-009 & A15 & n/a & n/a \\ \midrule
1275 & WLP023 & OYL-010 & n/a & n/a & Lallemand Windstor British-Style Beer \\ \midrule
1318 & n/a & OYL-011 & n/a & n/a & n/a \\ \midrule
1335 & WLP025 & n/a & n/a & n/a & n/a \\ \midrule
1388 & WLP570 & OYL-019 & n/a & GY048 & Mangrove Jack's M31 Belgian Tripel \\ \midrule
1450 & n/a & OYL-041 & n/a & n/a & n/a \\ \midrule
1469 & n/a & OYL-014 & n/a & n/a & n/a \\ \midrule
1728 & WLP028 & OYL-015 & A31 & GY044 & n/a \\ \midrule
1764-PC & n/a & n/a & n/a & n/a & Mangrove Jack's M44 US West Coast \\ \midrule
1968 & WLP002 & OYL-016 & A09 & n/a & Lallemand London ESB English-Style Ale / Mangrove Jack's M15 Empire Ale \\ \midrule
2001 & WLP800 & OYL-101 & n/a & GY002 & n/a \\ \midrule
2112 & WLP810 & OYL-105 & L05 & GY005 & Mangrove Jack's M54 Californian Lager \\ \midrule
2124 & WLP830 & OYL-106 & L13 & GY045 & Fermentis SafLager W-34/70 / Mangrove Jack's M84 Bohemian Lager \\ \midrule
2206 & WLP820 & OYL-107 & n/a & n/a & Fermentis SafLager S-23 \\ \midrule
2247-PC	& WLP920 & n/a & n/a & n/a & n/a \\ \midrule
2278 & n/a & OYL-108 & n/a & n/a & n/a \\ \midrule
2308 & WLP838 & OYL-109 & n/a & n/a & n/a \\ \midrule
2487-PC & WLP833 & n/a & n/a & n/a & n/a \\ \midrule
2565 & WLP029 & OYL-017 & G03 & GY021 & Fermentis SafAle K-97 \\ \midrule
3068 & WLP300 & OYL-021 & G01 & GY017 & Fermentis SafAle WB-06 \\ \midrule
3333-PC & WLP380 & OYL-022 & n/a & n/a & Fermentis SafAle WB-06 \\ \midrule
3522 & WLP550 & n/a & B45 & n/a & n/a \\ \midrule
3724 & WLP565 & OYL-027 & n/a & GY027 & Fermentis SafAle T-58 / Lallemand Belle Saison Belgian
Saison-Style \\ \midrule
3787 & WLP530 & n/a & B48 & n/a & n/a \\ \midrule
3944 & WLP400 & OYL-030 & B44 & n/a & Mangrove Jack's M21 Belgian Wit \\ \midrule
n/a & WLP039 & n/a & n/a & n/a & Lallemand Nottingham High Performance Ale \\ \midrule
n/a & WLP095 & n/a & A04 & GY054 & n/a \\ \midrule
\end{tabu}

\small
Sources: Kristen England's "Yeast Strain Comparison Chart", Salt City Brew Suplly's
"Yeast Comparison Chart" and Jan Brückelmaier's "Bier brauen: Grundlagen,
Rohstoffe, Brauprozess".
\normalsize

\end{recipeblock}


\end{document}